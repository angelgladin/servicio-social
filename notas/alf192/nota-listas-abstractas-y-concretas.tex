\documentclass[letterpaper,11pt]{article}
\usepackage[includeheadfoot,margin=1.3in]{geometry}

\usepackage[utf8]{inputenc}
\usepackage[english,spanish]{babel}
\usepackage{amssymb,amsmath,mathrsfs}

\usepackage{import}
\usepackage{todonotes} % Para hacer anotaciones, remover cuando ya no tengan anotaciones.

\input{macrosAyLF}

\title{Aut\'omatas y Lenguajes Formales 2019-II \\ 
%Posgrado en Ciencia e Ingeniería de la Computación UNAM \\ 
Facultad de Ciencias UNAM\footnote{Material elaborado en el marco del proyecto
  PAPIME PE102117} \\ 
Nota de Clase X: Listas abtractas y concretas} 
\author{Favio E. Miranda Perea \and A. Liliana Reyes Cabello \and
Lourdes Gonz\'alez Huesca}
\date{-1 de diciembre de 2020}

\begin{document}
\maketitle

%% Traducción del libro
%% Formal Languages and Compilations
%% Stefano Crespi Reghizzi
%% Páginas 25-30
\section{Abstracción lingüística}

Si uno recuerda un lenguaje de programación con el que se está familiarizado, se puede observar que, 
aunque superficialmente diferentes en sus \textit{palabras reservada} y \textit{separadores}, son
usualmente bastante similares en un nivel más profundo.
Cambiando el enfoque de sintaxis concreta a sintaxis abstracta, podemos reducir la desconcertante
variedad de construcciones del lenguaje a algunas estructuras esenciales. El verbo ``abstraer''
significa
\begin{quotation}
considerar un concepto sin pensar en un ejemplo específico.
\end{quotation}

Abstrayendo de los caracteres concretos que representan una construcción del lenguaje se hace la
abstracción lingüistica. Esto es la transformación del lenguaje que remplaza los caracteres
terminales de un lenguaje concreto con otros caracteres tomados de una alfabeto abstracto.
Los caracteres abstractos deben ser más simples y adecuados para representar construcciones 
similares de diferntes lenguajes artificiales\footnote{
La idea de la abstracción del lenguaje se inspira en la investigación en lingüística con el objetivo
de descubrir las similitudes subyacentes de los lenguajes humanos, sin tener en cuenta las
diferencias morfológicas y sintácticas.}.

Mediante este enfoque, las estructuras de sintaxis abstractas de lenguajes artificiales existentes
son facilmente descritas como composición de algunos paradigmas básicos, mediante operaciones
estándares del lenguaje: unión, iteración y sustitución. Empezando de un lenguaje abstracto, un
lenguaje concreto o existente es obtenido por la transformación de inversa, matefóricamente llamada
recubrimiendo con azúcar sintáctica.

Fabricando un lenguaje en su sintaxis abstracta y sintaxis concreta se compensa en varias maneras.
Cuando se estudian diferentes lenguajes se obtiene mucho ahorro conceptual. Cuando se diseñan
compiladores, la abstracción ayuda para la portabilidad entre diferentes lengajes, si las funciones
del compilador están diseñadas para procesar lo abtracto, en vez de lo concreto, el lenguaje
construye. Por lo tanto partes de, digamos, el compilador de C puede ser reusado para lenguajes 
similares como FORTRAN o Pascal.


\section{Listas abstractas y concretas}

Una \textit{lista abstracta} contienen un número no acotado de elementos $e$ del mismo tipo. Está
definido por la expresión regular $e^+$ o $e^\star$, si pueden faltar elementos.

Un elemento por el momento puede ser visto como un caracter terminal, pero en refinamientos
posteriores, el elemento se puede convertir en una cadena de otro lenguaje formal: pensemos, por
ejemplo, en una lista de números.
\begin{quotation}
Lista con separadores y símbolos que abren/cierran.
\end{quotation}

En muchos casos concretos, los elementos adyacentes deben ser separados por cadenas llamadas
\textit{separadores}, $s$ en sintaxis abstracta. Por lo tanto en una lista de números, un separador
debe delimitar el final de un número y el principio del siguiente.

Una \textit{lista con separadores} es definida por la expresión regular $e(se)^\star$, diciendo que
el primer elemento puede ser seguido por cero o más pares $se$. La definición equivalente
$(es)^\star e$ difiere dando evidencia al último elemento.

En muchos casos concretos hay otro requisito, destinado a la legibilidad o el procesamiento por
computadora: hacer el inicio y final de una lista facilmente reconocible agregando algunos signos
especiales a sufijos y prefijos: en abstracto, el caracter inicial o la marca de inicio $i$, y el
caracter final o la marca final $f$.

Listas con separadores y marcas inicial/final son definidas como
\[
    ie(se)^*f
\]

\paragraph{Ejemplo:} Algunas listas concretas

Las listas están en todos lados en los lenguajes, como es mostrado por ejemplos típicos.

Bloque de instrucciones: $\qquad begin\; instr_1;\; instr_2;\; \ldots\; instr_n\; end$

donde $instr$ posiblemente está ahí para asignaciones, go to, sentencia \texttt{if}, sentencia
\texttt{write}, etc. Los términos abstractos y concretos correspondientes son:

\begin{table}[ht]
\centering
    \begin{tabular}{c|c}
    alfabeto abstracto & alfabeto concreto \\ \hline
    $i$                & $begin$           \\
    $e$                & $instr$           \\
    $s$                & $;$               \\
    $f$                & $end$              
    \end{tabular}
\end{table}

Parámetros de procedimientos: como en
\[
    \underbrace{procedure\; WRITE(}_{i}
    \underbrace{par_1}_{e}
    \underbrace{,}_{s},
    par_2,
    \ldots,
    par_n
    \underbrace{)}_{f}
\]

Una lista vacía de parámetros debe ser legal, como por ejempo $procedure\; WRITE()$, la expresión
regular se convierte en $i[e(se^\star)]f$.

Definición de arreglo: 
$
    \qquad
    \underbrace{array\; MATRIX'['}_{i}
    \underbrace{int_1}_{e}
    \underbrace{,}_{s},
    int_2,
    \ldots,
    int_n
    \underbrace{']'}_{f}
$

donde cada $int$ es un intervalo como 10...50.


\subsection{Sustitución}

Los ejemplos de arriba muestran la transformación de símbolos concretos a abtractos. Los diseñadores
de lenguajes encuentran útil trabajar con refinamientos paso a paso, como es hecho en cualquier
rama de ingeniería, cuando un sistema complejo es dividido en sus componentes, atómico o de otros
tipos. Para tal fin, se presentan una nueva operación del lenguaje de sustitución, que remplaza un
caracter terminal del lenguaje denominado como \textit{fuente}, con una sentencia de otro lenguaje
llamada \textit{objetivo}. Como siempre $\Sigma$ es el alfabeto fuente y $L \subseteq \Sigma^\star$
el lenguaje fuente. Consideremos una sentencia de $L$ conteniendo una o más ocurrencias de un
caracter fuente $b$: 
\[
    x = a_1 a_2 \ldots a_n
    \qquad
    \text{donde para alguna $i$, $a_i = b$}
\]

Sea $\Delta$ otro alfabeto, llamado objetivo, y $L_b \subseteq\Delta^\star$ sea el
\textit{lenguaje imagen} de $b$. La \textit{sustitución} del lenguaje $L_b$ para $b$ en una cadena
$x$ produce un conjunto de cadenas, que es, una lenguaje del alfabeto
$(\Sigma \setminus \{ b \}) \cup \Delta$, definido como
\[
    \{ y \;|\; y = y_1 y_2 \ldots y_n
    \;\land\;
    (\texttt{if } a_i \neq b \texttt{ then } y_i = a_i \texttt{ else } y_i \in L_b)\}
\]

Notemos que todos los caracteres que no sean $b$ no cambian. La sustitución puede ser definida
en todo el lenguaje fuente, aplicando la operación a cada sentencia objetivo.

\paragraph{Ejemplo:} El ejemplo anterior se retomará

Regresando al caso de lista de parámetros, la sintaxis abstracta es
\[
    ie(se)^\star f
\]
y las sustituciones a ser aplicadas se muestran a continuación:

\begin{table}[ht]
    \centering
    \begin{tabular}{c|c}
    caracter abstracto & cadena                                                          \\ \hline
    $i$                & $L_i = procedure \langle \text{procedure identifier} \rangle ($ \\
    $e$                & $L_e = \langle \text{parameeter identifier} \rangle$            \\
    $s$                & $L_s = ,$                                                       \\
    $f$                & $L_f = )$                                                      
    \end{tabular}
\end{table}

Por ejempo, la marca de inicio de $i$ es remplazada con una cadena del lenguaje $L_i$, donde
\texttt{procedure identifier} tiene que estar de acuerdo con las reglas del lenguaje técnico.

Claramente los lenguajes objetivos de la sustitución deependern en la azucar sintáctica del
lenguaje concreto al que está destinado.

Cabe notar que las cuatro sustituciones son independientes y pueden ser aplicadas en cualquier
orden.

\paragraph{Ejemplo:} Identificadores con guiones bajo.

En ciertos lenguajes de programación, los nombres de identificadores nmemoteícos largos pueden ser
construidos añadiendo cadenas alfanuméricas separas por un guión bajo:
de ahí que \texttt{LOOP3\_OF\_35} es un identificador válido. Más precisamente la primera cadena
debe iniciar con una letra, los otros deben contener letras y dígitos, y los guiones adyacentes
están prohibidos, así como guiones finales.

A primera vista el lenguaje es una lista no vacía de cadenas $s$, separadas por un guión:
\[
    s(\_s)^\star 
\]

Sin embargo, la primera cadea debe ser diferente de las otras y debe ser tomada para ser la marca
de inicio de una posible lista vacía:
\[
    i(\_s)^\star 
\]

Sustituyendo a $i$ el lenguaje $(A \ldots Z)(A \ldots Z | 0 \ldots 9)^\star$, y a $s$ el lenguaje\\
$(A \ldots Z)(A \ldots Z | 0 \ldots 9)^+$, la expresión regular final es obtenida.

Este es un ejemplo demasiado simple de diseño de sintaxis por abstracción y refinamiento paso a
paso, un método que será utilizado ahora y después después de la introducción de gramáticas.


\subsection{Listas de precedencia o jerárquicas}

Una constrtucción recurrente es una lista, tal que cada elemento que está en turno es una lista de
una de tipo diferente. La primera línea está en el nivel 1, la segunda al nivel 2, y así
sucesivamente. La estructura abtracta presente, llamada lista jerárquica, está resrtringida a
listas con un número acotado de niveles. El caso en el que los niveles no están acotados es
estudiado después usando gramáticas, bajo el nombre de estrtucturas anidadas.

Una lista jerárquica es también llamada \textit{lista con precedencias}, porque una lista a nivel
$k$ limita sus elementos más que a una lista a nivel $k-1$; en otras palabras los elementos en un
más alto nivel deben ser juntados en una lista, y cada uno se convierte en un elemento al
siguiente nivel más bajo.

Cada nivel debe tener una marca que abra/cierre y un separador; tales delimitadores son usualmente
distintos nivel por nivel, para evitar confusiones.

La estructura de la lista jerárquica de niveles $k \geq 2$ es

\begin{align*}
    list_1 &= i_1 list_2 (s_1 list_2)^\star f_1\\
    list_2 &= i_2 list_3 (s_2 list_3)^\star f_2\\
    \ldots\\
    list_k &= i_k e_k (s_k e_k)^\star f_k
\end{align*}

Nótese que sólo el último nivel puede contener elementos atómicos. Pero una variación común permite
que a cualquier nivel $k$ los elementos atómicos $e_k$ ocurran al lado de otro con listas de
nivel $k+1$. A continuación se presentas algunos ejemplos conretos.


\paragraph{Ejemplo:} Dos listas jerárquicas

Bloque de instrucciones \texttt{print}, $WRITE(var_1, var_2, \ldots, var_n)$, es decir, una lista.
Hay dos niveles:
\begin{enumerate}
    \item[Nivel 1:] Lista de instrucciones $instr$, abierta por $begin$, separados por un punto y
    coma y cerrado por $end$.  
    \item[Nivel 2:] Lista de variables $var$ separados por una coma, con $i_2 = WRITE($ y $f_2=)$.
\end{enumerate}

Las expresiones aritméticas no usan paréntesis: los niveles de precedencia determinan de los
operadores determinan cuantos niveles hay en la lista. Por ejemplo, los operadores $\div$,
$\times$ y $+$, $-$ están etiquetados en dos niveles y la cadena
\[
    3 + \underbrace{5 \times 7 \times 4}_{term_1} - \underbrace{8 \times 2 \div 5}_{term_2} + 8 + 3
\]

es una lista de dos niveles, sin marcas de apertura ni de cierre. En el nivel uno encontramos una
lista de términos ($e_1$ = término) separados por los signos $+$ y $-$, es decir, por operadores de
menor precedencia. En el nivel dos vemos una lista de números, separados por signos de mayor
precedencia. 

Se puede ir más allá e introducir un tercer nivel con el signo de exponenciación ``$\ast\ast$''
como separador.

Por supuesto, las estructuras jerárquicas también son omnipresentes en los lenguajes naturales.
Piense en una lista de sustantivos
\[ \mathbf{padre,\; madre,\; hijo,\; e\; hija} \]

Aquí podemos observar una diferencia con respecto al paradigma abstracto: el penúltimo elemento
tiene un separador distinto, posiblemente para advertir al oyente de que ya vamos a acabar de
hablar. Además, los elementos de la lista pueden enriquecerse con calificadores de segundo nivel,
como la lista de adjetivos.

En todo tipo de documentos y medios escritos, las listas jerárquicas son extremadamente comunes.
Por ejemplo, un libro es una lista de capítulos, separados por páginas en blanco, entre la portada
y la contraportada. Un capítulo es una lista de secciones; una sección, una lista de párrafos, y
así sucesivamente.

\end{document}
