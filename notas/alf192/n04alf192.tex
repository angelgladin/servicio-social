\documentclass[letterpaper,11pt]{article}
\usepackage[includeheadfoot,margin=1.3in]{geometry}
\usepackage{anysize}

\usepackage[utf8]{inputenc}
% \usepackage[latin1]{inputenc}
\usepackage[english,spanish]{babel}
%\usepackage{lmodern}   % font shapes...
\usepackage[T1]{fontenc} % join the compound symbols as a single symbol

\usepackage{amssymb,amsmath,amsthm}
\usepackage{mathrsfs}
\usepackage{epsfig}

\usepackage{fancyhdr}
\usepackage{hyperref}
\usepackage{todonotes} % Para hacer anotaciones, remover cuando ya no tengan anotaciones.
\usepackage{url}

\usepackage{import}
\usepackage{comment}
\usepackage[autostyle=true,spanish=mexican]{csquotes}

\usepackage{url}

\usepackage{pgf}
\usepackage{tikz}
\usetikzlibrary{automata,arrows}
\usetikzlibrary{babel}
\usetikzlibrary{arrows,automata,positioning, calc}
\tikzset{node distance=2.5cm, % Minimum distance between two nodes. Change if necessary.
        every state/.style={ % Sets the properties for each state
            semithick,
            fill=gray!10},
        initial text={}, % No label on start arrow
        double distance=2pt, % Adjust appearance of accept states
        every edge/.style={ % Sets the properties for each transition
            draw,
            ->,>=stealth', % Makes edges directed with bold arrowheads
            auto,
            semithick}
        }

\newcommand{\TransitionTableAndAutomataSideBySide}[2]{
\[ \renewcommand{\arraystretch}{1.3} #1 \qquad\qquad\raisebox{-0.45\totalheight}{#2} \]
}

%\pagestyle{fancyplain}
\input{macrosAyLF}

\title{Aut\'omatas y Lenguajes Formales 2019-II \\ 
% Posgrado en Ciencia e Ingeniería de la Computación UNAM \\ 
Facultad de Ciencias UNAM\footnote{Material elaborado en el marco del proyecto
  PAPIME PE102117} \\ 
Nota de Clase 4} 
\author{Favio E. Miranda Perea \and A. Liliana Reyes Cabello \and
Lourdes Gonz\'alez Huesca}
\date{12 de marzo de 2019}

\begin{document}
\maketitle
\section{Aut\'omatas No-Deterministas}

El determinismo de un aut\'omata, deseable desde el punto de vista te\'orico, 
puede provocar complicaciones en la práctica.
Veamos las diferencias entre estos conceptos:
\bi
 \item \textbf{Determinismo:} dado un estado~$q$ y un s\'imbolo~$a$ existe una 
  \'unica transici\'on~$\delta(q,a)=p$, es decir $\delta$ es una 
  funci\'on total.
 \item \textbf{No-determinismo:} no hay una transici\'on \'unica al leer un
  s\'imbolo~$a$ en un estado dado~$q$.
\ei

El no-determinismo se traduce en que hay m\'as de una transici\'on al leer un 
s\'imbolo, es decir, $\delta(q,a)$ deja de ser función. 
O bien no hay transici\'on, es decir, $\delta(q,a)$ no est\'a definida 
($\delta$ se vuelve funci\'on parcial).
Sin embargo la m\'aquina funciona \'unicamente al leer un s\'imbolo. % y podemos 
% decir que existe el no-determinismo sin lectura de s\'imbolos.

Veamos c\'omo se puede agregar el no-determinismo a la definici\'on de 
aut\'omata que vimos anteriormente:
\defin{
Un aut\'omata finito \textbf{no} determinista (AFN) es una quintupla
$M=\pt{Q,\Sigma,\delta,q_0,F}$ donde
\bi
 \item $Q$ es un conjunto finito de estados.
 \item $\Sigma$ es el alfabeto de entrada.
 \item $\delta:Q\times\Sigma\imp \Pe(Q)$ es la funci\'on de 
  transici\'on.
 \item $q_0\in Q$ es el estado inicial.
 \item $F\inc Q$ es el conjunto de estados finales.
\ei
}
Obs\'ervese que la imagen de~$\delta$ es ahora un elemento de~$\Pe(Q)$, es 
decir es un subconjunto de estados de $Q$.
Adem\'as~$\delta(q,a)=\{q_1,q_2,\ldots,q_n\}$ indica que al leer el 
s\'imbolo~$a$ en el estado~$q$ la m\'aquina puede pasar a cualquiera de los
estados $q_1,\ldots,q_n$.
Si $\delta(q,a)=\varnothing$ entonces no hay transici\'on posible desde 
el estado~$q$ al leer~$a$, es decir, la m\'aquina est\'a bloqueada.

\noindent Veamos como las nociones vistas para los AFD se modifican:

\defin{
Sea $M=\pt{Q,\Sigma,\delta,q_0,F}$ un AFN. La funci\'on de 
transici\'on~$\delta$ se extiende a cadenas mediante una funci\'on 
$\dest:Q\times\sest \imp \Pe(Q)$ definida recursivamente como sigue:
\bi
 \item $\dest(q,\vacia) = q $
 \item $\dest(q, wa) = \bigcup_{p\in\dest(q,w)}\delta(p,a)$ 
\item  Alternativamente: $\dest(q,aw) = \bigcup_{p\in \delta(q,a)}\dest(p,w)$
\ei
}

\noindent El lenguaje de aceptaci\'on se define mediante~$\dest$ como sigue:
$$ L(M)=\{w\in\sest\;|\;\dest(q_0,w)\cap F\neq\vacio\} $$
Es decir, $w\in L(M)$ si y s\'olo si existe al menos un c\'omputo o un 
procesamiento de~$w$ que conduce a un estado final al iniciar la m\'aquina en 
$q_0$.

\todo{[START] Ejemplos}
% TODO 1. Pon 3 ejemplos de AFN, la def formal, la tabla y el diagrama.

\paragraph{Ejemplos de AFN:}
\be
\item Sea un $M = \pt{\{q_0, q_1, q_2\}, \{0,1\}, \delta, q_0, \{q_2\}}$ un AFN que acepta todas las cadenas que
terminan en \texttt{01}.

\TransitionTableAndAutomataSideBySide {
  \begin{array}[c]{|c|c|c|}\hline 
    \delta & 0            & 1       \\\hline 
    q_0    & \{q_0, q_1\} & \{q_1\} \\\hline 
    q_1    & \vacio       & \{q_2\} \\\hline 
    *q_2   & \vacio       & \vacio  \\\hline
  \end{array}
} {
  \begin{tikzpicture}
    \node[state, initial] (q0) {$q_0$};
    \node[state] (q1) [right of=q0] {$q_1$};
    \node[state, accepting] (q2) [right of=q1] {$q_2$};
  
    \draw
        (q0)
            edge [loop above] node {0,1} ( )
            edge node {0} (q1)
        (q1)
            edge node {1} (q2);
  \end{tikzpicture}
}

\item Sea un $M = \pt{\{q_0, q_1, q_2, q_3\}, \{0,1\}, \delta, q_0, \{q_3\}}$ un AFN acepta todas las cadenas que
contienen \texttt{00} o \texttt{11} como subcadena.

\TransitionTableAndAutomataSideBySide {
  \begin{array}[c]{|c|c|c|}\hline 
    \delta & 0            & 1            \\\hline 
    q_0    & \{q_0, q_1\} & \{q_0, q_2\} \\\hline 
    q_1    & \{q_3\}      & \vacio       \\\hline 
    q_2    & \vacio       & \{q_3\}      \\\hline
    q_3    & \{q_3\}      & \{q_3\}      \\\hline
  \end{array}
} {
  \begin{tikzpicture}
    \node[state, initial] (q0) {$q_0$};
    \node[state] (q1) [right of=q0] {$q_1$};
    \node[state] (q2) [below of=q0] {$q_2$};
    \node[state, accepting] (q3) [below of=q1] {$q_3$};
  
    \draw
        (q0)
            edge [loop above] node {0,1} ( )
            edge [bend left] node {0} (q1)
            edge [bend right] node [left] {1} (q2)
        (q1)
            edge [bend left] node {0} (q3)
        (q2)
            edge [bend right] node [below] {1} (q3)
        (q3)
            edge [loop below] node {0,1} ( );
  \end{tikzpicture}
}

\item Sea un $M = \pt{\{q_0, q_1, q_2, q_3\}, \{0,1\}, \delta, q_0, \{q_3\}}$ un AFN que acepta todas las cadenas que
terminan en \texttt{01}.

\TransitionTableAndAutomataSideBySide {
  \begin{array}[c]{|c|c|c|}\hline 
    \delta & 0            & 1            \\\hline 
    q_0    & \{q_0, q_1\} & \{q_0\} \\\hline 
    q_1    & \{q_2\}      & \vacio       \\\hline 
    q_2    & \vacio       & \{q_3\}      \\\hline
    q_3    & \{q_3\}      & \{q_3\}      \\\hline
  \end{array}
} {
  \begin{tikzpicture}
    \node[state, initial] (q0) {$q_0$};
    \node[state] (q1) [right of=q0] {$q_1$};
    \node[state] (q2) [right of=q1] {$q_2$};
    \node[state, accepting] (q3) [right of=q2] {$q_3$};
  
    \draw
        (q0)
            edge [loop above] node {0,1} ( )
            edge node {0} (q1)
        (q1)
            edge node {0} (q2)
        (q2)
            edge node {1} (q3)
        (q3)
            edge [loop above] node {0,1} ( );
  \end{tikzpicture}
}

\ee
\todo{[END] Ejemplos}

\todo{[START] Ejecución}
% TODO 2. Pon ejemplos de ejecución de la delta estrella, usando los automatas anteriores, dos que acepten y dos que no acepten.
\paragraph{Ejemplos de ejecución de $\dest$ para el AFN del inciso 1:}

\begin{itemize}
  \item Cadena aceptada, con la entrada \texttt{00101}.
  \be
    \item $\dest(q_0, \vacia) = \{q_0\}$
    \item $\dest(q_0, 0) = \delta(q_0, 0) = \{q_0, q_1\}$
    \item $\dest(q_0, 00) = \delta(q_0, 0) \cup \delta(q_1, 0) = \{q_0, q_1\} \cup \vacio = \{q_0, q_1\}$
    \item $\dest(q_0, 001) = \delta(q_0, 1) \cup \delta(q_1, 1) = \{q_0\} \cup \{q_2\}= \{q_0, q_2\}$
    \item $\dest(q_0, 0010) = \delta(q_0, 0) \cup \delta(q_2, 0) = \{q_0, q_1\} \cup \vacio = \{q_0, q_1\}$
    \item $\dest(q_0, 00101) = \delta(q_0, 1) \cup \delta(q_1, 1) = \{q_0\} \cup \{q_2\} = \{q_0, q_2\}$
  \ee

  En la línea (1) se aplica la regla básica. Se obtiene la línea (2) aplicando $\delta$ al estado $q_0$, que está en el
  conjunto anterior, y se obtiene $\{q_0, q_1\}$ como resultado. En la línea (3) se obtiene de tomar la unión  sobre
  los dos estados en el conjunto anterior de lo que se obtuvo cuando les aplicamos $\delta$ con la entrada \texttt{0}.
  Que es, $\delta(q_0, 0) = \{q_0, q_1\}$, mientra que $\delta(q_1, 0) = \vacio$. Para la línea (4), tomamos la unión de
  $\delta(q_0, 1) = \{q_0\}$ y $\delta(q_1, 1) = \{q_2\}$. Las líneas (5) y (6) son similares a las líneas (3) y (4).

  \item Cadena no aceptada, con la entrada \texttt{110}.  
  \be
    \item $\dest(q_0, \vacia) = \{q_0\}$
    \item $\dest(q_0, 1) = \delta(q_0, 1) = \{q_0\}$
    \item $\dest(q_0, 11) = \delta(q_0, 1) = \{q_0\}$
    \item $\dest(q_0, 110) = \delta(q_0, 0) = \{q_0, q_1\}$
  \ee

  Como $\{q_0, q_1\} \cap F = \vacio$, entonces no es aceptado.
\end{itemize}

\paragraph{Ejemplos de ejecución de $\dest$ para el AFN del inciso 2:}

\begin{itemize}
  \item Cadena aceptada, con la entrada \texttt{}.
  
  \be
    \item $\dest$
  \ee

  \item Cadena no aceptada, con la entrada \texttt{}.
  
  \be
    \item $\dest$
  \ee
\end{itemize}
\todo{[END] Ejecución}




\subsection{Eliminaci\'on del no-determinismo}
\label{ssec:elimNdet}

% Una pregunta natural es comparar dos aut\'omatas, uno determinista y el otro 
% no. As\'i tambi\'en buscar una forma de eliminar el no-determinismo y conservar 
% una m\'aquina que acepte un lenguaje dado. 

Es claro que todo AFD es un AFN, con la particularidad de que~$\delta(p,a)$ consta 
de un \'unico estado. El recíproco también es cierto, dando como resultado la equivalencia entre ambos modelos de reconocimiento de cadenas.

La idea para transformar un AFN en un AFD es considerar a cada conjunto de 
estados~$\delta(p,a)$ del AFN como un \'unico estado del nuevo AFD.
A este m\'etodo se conoce como la \textit{construcci\'on de subconjuntos}, cuyos detalles técnicos damos a continuación.

\defin{
Dado un AFN~$M=\pt{Q,\Sigma,\delta,q_0,F}$ definimos un AFD
$M^D=\pt{Q^D,\Sigma,\delta_D,q^D_0,F^D}$ como sigue:
\bi
 \item $Q^D =_{def} \Pe(Q)$
 \item $\delta_D(S,a) =_{def}\bigcup_{q\in S}\delta(q,a)$
 \item $q^d_0 =_{def} \{q_0\}$
 \item $F^d =_{def} \{S\inc Q\;|\;S\cap F\neq\vacio\}$
\ei  
}


A continuación mostramos que ambos autómatas son equivalentes.

\todo{[START] Ejemplos}
3. Pon 3 ejemplos de la transformacion a AFD, uno de los anteriores y otros dos diferentes.
\todo{[END] Ejemplos}


\begin{lemma}
  Sea $M$ un autómata finito no determinista. Se cumple lo siguiente
\[
\fa w\in\sest(\delta^\star(q_0,w)=\delta_D^\star(\{q_0\},w))
\]
\end{lemma}
\begin{proof}
  Inducción sobre $w$.
\end{proof}

\begin{proposition}
 Sea $M$ un autómata finito no determinista. Se cumple que  $L(M)=L(M^D)$.
\end{proposition}
\begin{proof}
  La demostración se sigue del lema anterior.
\end{proof}


\section{AFN con \texorpdfstring{$\vacia$}--transiciones}

Otra de las m\'aquinas que son \'utiles para procesar cadenas son las que 
permiten procesar cadenas vac\'ias, no s\'olo al tener como estado final a 
$q_0$ sino que permiten procesar (sub)cadenas vac\'ias en una parte intermedia.
De esta forma se definen los aut\'omatas con transiciones etiquetadas con la 
cadena $\vacia$:
\defin{ 
Un aut\'omata finito \textbf{no} determinista con $\vacia$-transiciones 
(AFN$\cv$) es una quintupla $ M = \pt{Q,\Sigma,\delta,q_0,F}$ donde
\bi
 \item $Q$ es un conjunto finito de estados.
 \item $\Sigma$ es el alfabeto de entrada.
 \item $\delta:Q\times(\Sigma\cup\{\cv\})\imp \Pe(Q)$ es la funci\'on de 
  transici\'on.
 \item $q_0\in Q$ es el estado inicial.
 \item $F\inc Q$ es el conjunto de estados finales.
\ei
}

De la definici\'on de $\delta:Q\times(\Sigma\cup\{\cv\})\imp \Pe(Q)$ se observa 
que la \'unica diferencia est\'a en el dominio de $\delta$.
Es decir que las transiciones de la forma $\delta(q,\cv)=a$ est\'an permitidas 
e indican que la m\'aquina puede cambiar de estado \textit{sin} leer ning\'un 
símbolo.
% Esto causa tambi\'en un no-determinismo, que es aun m\'as complicado de        
% modelar matem\'aticamente pero tiene grandes ventajas:
\medskip
Algunas observaciones importantes acerca de los AFN$\cv$ son: 
\bi
 \item Se permiten m\'ultiples c\'omputos para una cadena de entrada.
 \item Pueden existir c\'omputos bloqueados.
 \item A diferencia de los AFD y AFN simples, pueden existir c\'omputos
  infinitos, es decir, surge la \textbf{no-terminaci\'on}.
 \item La presencia de $\vacia$-transiciones permite mayor libertad y modularidad en el 
  dise\~no.
\ei

Para extender la definici\'on de transiciones a procesamiento de cadenas es 
necesario introducir un concepto previo que considera los estados a los que la 
m\'aquina puede llegar al incluir las cadenas vac\'ias.

\defin{
Dado un estado~$q$, definimos la~$\vacia$-cerradura de $q$ como el conjunto 
de estados alcanzables desde $q$ mediante cero o m\'as $\vacia$-transiciones.
Es decir 
\[
Cl_{\vacia} (q) =
\{s\in Q \mid \exists p_1,\ldots,p_n \text{ con }
  p_1 = q,\; p_n = s, \; p_i\in\delta(p_{i-1},\vacia)\}
\]
Recursivamente:
\bi
 \item $q\in Cl_{\vacia}(q)$
 \item Si $r\in Cl_{\vacia}(q)$ y $\delta(r,\vacia) = s$ 
  entonces $s\in Cl_{\vacia}(q)$
\ei
}

La definición anterior puede darse mediante reglas de inferencia como sigue:
\[
\frac{}{q\in Cl_{\vacia}(q)}(Cl1)\;\;\;\;\;\;\;\;\frac{r\in Cl_{\vacia}(q)\;\;\;\;\;\delta(r,\vacia) = s}{s\in Cl_{\vacia}(q)}(Cl2)
\]


La operación de $\cv$-cerradura se extiende a conjuntos de estados como sigue:
$$ Cl_{\vacia}(S)=\bigcup_{q\in S} Cl_{\vacia}(q)$$







Con la $\vacia$-cerradura de estados se puede definir la función de transición extendida 
$\delta^\star$:

\defin{
Sea $M=\pt{Q,\Sigma,\delta,q_0,F}$ un AFN$\vacia$. La funci\'on de 
transici\'on~$\delta$ se extiende a cadenas mediante una funci\'on
$$\dest:Q\times(\sest\cup\{\vacia\})\imp \Pe(Q) $$
definida recursivamente como sigue:
\bi
 \item $\dest(q,\vacia) = Cl_{\vacia}(q)$
 \item $\dest(q,w\sigma) = 
    Cl_{\vacia}\Big(\bigcup_{p\in\dest(q,w)}\delta(p,\sigma)\Big)$
 % \item $\dest(q,aw) =  
 %    Cl_{\vacia}\Big(\bigcup_{q'\in\dest(q,w)}\delta(q',a)\Big)$
\ei
}

\subsection{Eliminaci\'on de \texorpdfstring{$\cv$}--transiciones}

Estudiaremos el proceso de eliminaci\'on de transiciones $\vacia$. Esta 
eliminaci\'on implica que un AFN es tambi\'en un AFN$\vacia$.
Es decir que cualquier AFN$\vacia$ es inmediatamente un AFN con la 
particularidad de que no existen $\vacia$-transiciones.

\defin{
Dado un AFN$\vacia$, $M=\pt{Q,\Sigma,\delta,q_{0},F}$, definimos el $AFN$
asociado a $M$, denotado $M_N$, como \[M_N=\pt{Q,\Sigma,\delta_N,q_0,F_N},\] donde:
\bi
 % \item $Q:=Q_1$
 % \item $p_0:=q_{0}$
 \item $\delta_N(q,\sigma)=_{def}\delta^\star(q,\sigma) % = 
         % Cl_{\vacia}\Big(\bigcup_{p\in Cl_{\vacia}(q)}\delta_1(p,a)\Big)
      $
 \item $F_N=_{def}F\cup\{q_{0}\}$ si $Cl_{\vacia}(q_{0})\cap F\neq\vacio$ o bien
$F_N=_{def}F$ en caso contrario.
\ei
}

La equivalencia entre $M$ y $M_N$ se bosqueja a continuación

\begin{lemma}
  Sean $M$ un $AFN\vacia$, $q\in Q$ y $w,v\in\sest$. Se cumple que
\[
\dest(q,wv)=\bigcup_{p\in\dest(q,w)}\dest(p,v)
\] 
\end{lemma}
\begin{proof}
  Inducción sobre $v$
\end{proof}

\begin{lemma}
  Sean $M$ un $AFN\vacia$, $M_N$ el AFN asociado a $M$, $q\in Q$ y $w\in\sest$. Si $w\neq\vacia$ entonces
\[
%w\neq\vacia \imp 
\dest_N(q,w)=\dest(q,w)
\]
\end{lemma}
\begin{proof}
  Inducción sobre $w$. El lema anterior será de utilidad.
\end{proof}

\begin{proposition}
 Sea $M$ un AFN$\vacia$. Los autómatas $M$ y $M_N$ son equivalentes, es decir
$L(M)=L(M_N)$
\end{proposition}
\begin{proof}
  Es consecuencia directa del lema anterior y de la definición de $F_N$.
\end{proof}
\section{Equivalencias}
Nuevamente surgen preguntas como la equivalencia entre aut\'omatas 
deterministas, no deterministas y con transiciones $\vacia$. 

En la subsecci\'on~\ref{ssec:elimNdet} se estudi\'o la forma de eliminar el 
no-determinismo de un aut\'omata, creando un AFD. 
Es decir AFN $\Imp$ AFD. 

En la subsecci\'on pasada se revis\'o la forma de eliminar 
$\vacia$-transiciones y de esta forma obtener un AFN. As\'i tambi\'en se 
observ\'o que un AFN es exactamente un AFN$\vacia$ ya que incluye el 
no-determinismo pero sin transiciones de la cadena vac\'ia.

Esta equivalencia, AFN $\;\Iff\;$ AFN$\vacia$, cierra el ciclo de equivalencias 
de aut\'omatas finitos. 
Cualquier tipo de aut\'omata finito puede convertirse en un AFD y viceversa, 
es decir:
\begin{center}
  AFD $\Iff\;$ AFN $\Iff\;$ AFN$\vacia$
\end{center}

\vspace*{15pt}

Nuestra siguiente meta es probar el Teorema de Kleene, el cual es uno de los 
resultados m\'as importantes en la Teor\'ia de la Computaci\'on pues asegura la 
equivalencia entre dos de nuestros tres conceptos fundamentales: los 
aut\'omatas finitos y los lenguajes regulares.

\end{document}
