\documentclass[letterpaper,11pt]{article}
\usepackage[includeheadfoot,margin=1.3in]{geometry}
\usepackage{anysize}

\usepackage[utf8]{inputenc}
% \usepackage[latin1]{inputenc}
\usepackage[english,spanish]{babel}
%\usepackage{lmodern}   % font shapes...
\usepackage[T1]{fontenc} % join the compound symbols as a single symbol

\usepackage{amssymb,amsmath,amsthm}
\usepackage{mathrsfs}
\usepackage{epsfig}

\usepackage{fancyhdr}
\usepackage{hyperref}
\usepackage{url}

\usepackage{import}
\usepackage{comment}
\usepackage[autostyle=true,spanish=mexican]{csquotes}

\usepackage{alltt}
\usepackage[section]{placeins}

\usepackage{pgf}
\usepackage{tikz}
\usetikzlibrary{automata,arrows}
\usetikzlibrary{babel}
\usetikzlibrary{arrows,automata,positioning, calc}
\tikzset{node distance=2.5cm, % Minimum distance between two nodes. Change if necessary.
        every state/.style={ % Sets the properties for each state
            semithick,
            fill=gray!10},
        initial text={}, % No label on start arrow
        double distance=2pt, % Adjust appearance of accept states
        every edge/.style={ % Sets the properties for each transition
            draw,
            ->,>=stealth', % Makes edges directed with bold arrowheads
            auto,
            semithick}
        }
\usepackage{todonotes} % Para hacer anotaciones, remover cuando ya no tengan anotaciones.

%\pagestyle{fancyplain}
\input{macrosAyLF}

\title{Aut\'omatas y Lenguajes Formales 2019-II \\ 
% Posgrado en Ciencia e Ingeniería de la Computación UNAM \\ 
Facultad de Ciencias UNAM\footnote{Material elaborado en el marco del proyecto
  PAPIME PE102117} \\ 
Nota de Clase 5: El Teorema de Kleene} 
\author{Favio E. Miranda Perea \and A. Liliana Reyes Cabello \and
Lourdes Gonz\'alez Huesca}
\date{16 de marzo de 2019}

\begin{document}
\maketitle

Como hemos visto, la relaci\'on entre aut\'omatas finitos y lenguajes 
regulares parece de equivalencia, las nociones detr\'as de ambos conceptos la 
suguieren. En esta nota demostraremos que ese es el caso utilizando el m\'etodo 
propuesto por Kleene~\footnote{Stephen Cole Kleene fue un matem\'atico 
estadounidense, alumno de Alonzo Church. Tambi\'en es conocido por iniciar la 
teor\'ia de la recursi\'on que fue usada para los fundamentos de la Teor\'ia de 
la Computaci\'on, como la noci\'on de computabilidad.} 


\section{Teorema de Kleene}

\teo{ Un lenguaje es regular si y s\'olo si es aceptado por un aut\'omata 
finito.
}
\vspace*{-20pt}
\begin{proof}
 La prueba es en dos partes:
 \bi
  \item[I] S\'intesis: Dado un lenguaje regular~$L$, existe un aut\'omata 
    finito~$M$ tal que $L=L(M)$.
  \item[II] An\'alisis: Dado un aut\'omata finito~$M$, existe una      
    expresi\'on regular~$\alpha$ tal que $L(M)=L(\alpha)$. 
    Es decir, $L(M)$ es regular.
 \ei
\end{proof}

A continuaci\'on se abordar\'an dos teoremas que demuestran al anterior. 
Para ello recordemos que las expresiones regulares est\'an en correspondencia 
con los lenguajes regulares y de esta forma podremos empatar tambi\'en a las 
expresiones regulares con los aut\'omatas finitos. 


\subsection{Teorema de S\'intesis de Kleene}

En esta secci\'on se demostrar\'a una parte de la doble implicaci\'on del 
teorema de Kleene: se pasar\'a de expresiones regulares a aut\'omatas finitos.

Este teorema se denomina de s\'intesis ya que se proporcionar\'a una m\'aquina 
para reconocer un lenguaje regular dado. 
Es decir, se sintetizar\'a un aut\'omata finito analizando la forma de una 
expresi\'on regular que genera al lenguaje dado.

\newpage

\teo{
Dada una expresi\'on regular~$\alpha$ existe un aut\'omata 
finito~$M$ tal que $L(\alpha) = L(M)$.
}
%\vspace*{-20pt}
\begin{proof}
La demostraci\'on es \textit{constructiva} y se har\'a mediante 
inducci\'on sobre las expresiones regulares, es decir proporcionando un 
aut\'omata que \textit{reconzca} cada caso de una expresi\'on regular. 
\begin{description}
\item{\textbf{Base de la Inducción}} \hfill\\
El caso en que $\alpha = \vacio$, el siguiente autómata reconoce a $L(\alpha)$: 
\[
\begin{tikzpicture}[node distance=2cm,every node/.style={scale=0.8},semithick]
  \node[state,initial,initial text=] (q) {$q_0$};
\end{tikzpicture}
\]
Caso en que $\alpha=\vacia$. Entonces el siguiente aut\'omata reconoce a 
$L(\alpha)$: 
%\hspace*{10pt}
\[
\begin{tikzpicture}[node distance=2cm,every node/.style={scale=0.8},semithick]
  \node[state,initial,initial text=,accepting by double] (q) {$q_0$};
\end{tikzpicture}
\]
Para $\alpha = a$, con $a\in \Sigma$ se tiene el siguiente aut\'omata que 
reconoce a $L(\alpha)$:
%\hspace*{7pt}
\[
\begin{tikzpicture}[node distance=2cm,every node/.style={scale=0.8},semithick]
   \node[state,initial,initial text=,] (q) {$q_0$};
   \node[state,accepting by double] (p) [right of=q] {$q_1$};
   \path[->] (q) edge [bend left] node [above] {a} (p);
\end{tikzpicture}    
\]
\item{\textbf{Hip\'otesis de inducci\'on}}\\\hfill
  Sean $M_1,\,M_2$ dos aut\'omatas que reconocen a los lenguajes $L(\alpha_1)$ y 
  $L(\alpha_2)$ respectivamente.

\item{\textbf{Paso Inductivo}} \\\hfill
 Caso en que $\alpha=\alpha_1+\alpha_2$.
 El siguiente aut\'omata reconoce a $L(\alpha)$:
% \hspace{10pt}
\[ 
\begin{tikzpicture}[node distance=2cm,every node/.style={scale=0.8},semithick]
   \node[state,initial,initial text=] (q) {$q$};
   \node[state] (p1) [rectangle,above right of=q] {$M_1$};
   \node[state] (p2) [rectangle,below right of=q] {$M_2$};
   \path[->] (q) edge [bend left] node [above] {$\vacia$} (p1);
   \path[->] (q) edge [bend right] node [below] {$\vacia$} (p2);
 \end{tikzpicture}
\] 
donde $M_1,\,M_2$ son aut\'omatas dados por la hip\'otesis de inducci\'on y las 
transiciones $\vacia$ van hacia los estados iniciales de cada 
uno de ellos. 



Para el caso en que $\alpha=\alpha_1\alpha_2$ entonces el siguiente aut\'omata 
reconoce a $L(\alpha)$: 
\begin{center}
\begin{tikzpicture}[node distance=3cm,every node/.style={scale=0.8},semithick]
   \node[state,initial,initial text=] (p1) [rectangle] {$M_1$};
   \node[state] (p2) [rectangle,right of=p1] {$M_2$};
   \path[->] (p1) edge [bend left] node [above] {$\vacia$} (p2);
   \path[->] (p1) edge [left] node [above] {$\vdots$} (p2);
   \path[->] (p1) edge [bend right] node [above] {$\vacia$} (p2);
\end{tikzpicture}    
\end{center}
\noindent donde $M_1,\,M_2$ son aut\'omatas que reconocen a 
$L(\alpha_1),L(\alpha_2)$ dados por la hip\'otesis de inducci\'on, el estado 
inicial es el de $M_1$ y las transiciones $\vacia$ van de los estados finales de 
$M_1$ hacia el inicial en $M_2$.

Finalmente el caso $\alpha=\alpha_1^\star$ tiene al siguiente aut\'omata que 
reconoce a $L(\alpha)$:
\begin{center}
\begin{tikzpicture}[node distance=3cm,every node/.style={scale=0.8},semithick]
   \node[state,initial,initial text=,accepting by double] (q) {$q$};
   \node[state] (p1) [rectangle,right of=q] {$M$};
   \path[<-] (q) edge [bend left] node [above] {$\vacia$} (p1);
   \path[->] (p1) edge [bend left] node [above] {$\vacia$} (q);
\end{tikzpicture}   
\end{center}
\noindent 
donde $M_1$ es $M$ más $q_0$ (el estado inicial de $M_1$) y $M_1$ es un
autómata que reconoce a $L(\alpha_1)$ dado por la hip\'otesis de inducci\'on y
las transiciones $\vacia$ conectan los estados finales de $M_1$ con el estado
inicial $q_0$, el cúal se vuelve final, si no lo era en $M_1$.
\end{description}
Se deja como ejercicio en todos los casos verificar que el autómata recien
construido efectivamente acepta al lenguaje deseado.
\end{proof}

\todo{[START] Ejemplo}
\paragraph{Ejemplo}
Crear un AFN con $\vacia$-transiciones para la siguiente expresión regular:
$$ (00+1)^\star(10)^\star $$

\begin{enumerate}
\item $(\color{blue}0\color{black}0+1)^\star(10)^\star$

\begin{tikzpicture}[node distance=2cm, every node/.style={scale=0.8}]
  \node[state, initial] (q0) {};
  \node[state, accepting] (q1) [right of=q0] {};

  \draw
      (q0)
          edge node {$0$} (q1);
\end{tikzpicture}

\item $(\color{blue}00\color{black}+1)^\star(10)^\star$

\begin{tikzpicture}[node distance=2cm, every node/.style={scale=0.8}]
  \node[state, initial] (q0) {};
  \node[state] (q1) [right of=q0] {};
  \node[state] (q2) [right of=q1] {};
  \node[state, accepting] (q3) [right of=q2] {};

  \draw
      (q0)
          edge node {$0$} (q1)
      (q1)
          edge node {$\vacia$} (q2)
      (q2)
          edge node {$0$} (q3);
\end{tikzpicture}

\item $(\color{blue}00+1\color{black})^\star(10)^\star$

\begin{tikzpicture}[node distance=2cm, every node/.style={scale=0.8}]
  \node[state, initial] (q0) {};
  \node[state] (q1) [right of=q0] {};
  \node[state, accepting] (q2) [right of=q1] {};
  \node[state] (q3) [below of=q1] {};
  \node[state] (q4) [right of=q3] {};
  \node[state] (q5) [right of=q4] {};
  \node[state, accepting] (q6) [right of=q5] {};

  \draw
      (q0)
          edge node {$\vacia$} (q1)
          edge node {$\vacia$} (q3)
      (q1)
          edge node {$1$} (q2)
      (q3)
          edge node {$0$} (q4)
      (q4)
          edge node {$\vacia$} (q5)
      (q5)
          edge node {$0$} (q6);
\end{tikzpicture}

\item $\color{blue}(00+1^\star)\color{black}(10)^\star$

\begin{tikzpicture}[node distance=2cm, every node/.style={scale=0.8}]
  \node[state, initial, accepting] (q7) {};
  \node[state] (q0) [right of=q7] {};
  \node[state] (q1) [right of=q0] {};
  \node[state, accepting] (q2) [right of=q1] {};
  \node[state] (q3) [below of=q1] {};
  \node[state] (q4) [right of=q3] {};
  \node[state] (q5) [right of=q4] {};
  \node[state, accepting] (q6) [right of=q5] {};
  

  \draw
      (q0)
          edge node {$\vacia$} (q1)
          edge node {$\vacia$} (q3)
      (q1)
          edge node {$1$} (q2)
      (q2)
          edge [bend right] node [above] {$\vacia$} (q7)
      (q3)
          edge node {$0$} (q4)
      (q4)
          edge node {$\vacia$} (q5)
      (q5)
          edge node {$0$} (q6)
      (q6)
          edge [bend left=40] node {$\vacia$} (q7)
      (q7)
        edge node {$\vacia$} (q0);
\end{tikzpicture}

\item $(00+1)^\star(\color{blue}10\color{black})^\star$

\begin{tikzpicture}[node distance=2cm, every node/.style={scale=0.8}]
  \node[state, initial] (q0) {};
  \node[state] (q1) [right of=q0] {};
  \node[state] (q2) [right of=q1] {};
  \node[state, accepting] (q3) [right of=q2] {};

  \draw
      (q0)
          edge node {$1$} (q1)
      (q1)
          edge node {$\vacia$} (q2)
      (q2)
          edge node {$0$} (q3);
\end{tikzpicture}

\item $(00+1)^\star\color{blue}(10)^\star$

\begin{tikzpicture}[node distance=2cm, every node/.style={scale=0.8}]
  \node[state, initial] (q4) {};
  \node[state] (q0) [right of=q4] {};
  \node[state] (q1) [right of=q0] {};
  \node[state] (q2) [right of=q1] {};
  \node[state] (q3) [right of=q2] {};

  \draw
      (q0)
          edge node {$1$} (q1)
      (q1)
          edge node {$\vacia$} (q2)
      (q2)
          edge node {$0$} (q3)
      (q3)
          edge [bend left] node {$\vacia$} (q4)
      (q4)
          edge node {$\vacia$} (q0);
\end{tikzpicture}

\item $\color{blue}(00+1)^\star(10)^\star$

\begin{tikzpicture}[node distance=2cm, every node/.style={scale=0.8}]
  \node[state, initial] (q0) {};
  \node[state] (q1) [right of=q0] {};
  \node[state] (q2) [right of=q1] {};
  \node[state] (q3) [right of=q2] {};
  \node[state] (q4) [below of=q2] {};
  \node[state] (q5) [right of=q4] {};
  \node[state] (q6) [right of=q5] {};
  \node[state] (q7) [right of=q6] {};
  
  \node[state, accepting] (q8) [below of=q4] {};
  \node[state] (q9) [right of=q8] {};
  \node[state] (q10) [right of=q9] {};
  \node[state] (q11) [right of=q10] {};
  \node[state] (q12) [right of=q11] {};

  \draw
      (q0)
          edge node {$\vacia$} (q1)
          edge [bend right] node [left] {$\vacia$} (q8)
      (q1)
          edge node {$\vacia$} (q2)
          edge node {$\vacia$} (q4)
      (q2)
          edge node {$1$} (q3)
      (q3)
          edge [bend right] node [above] {$\vacia$} (q0)
      (q4)
          edge node {$0$} (q5)
      (q5)
          edge node {$\vacia$} (q6)
      (q6)
          edge node {$0$} (q7)
      (q7)
          edge [bend left=40] node {$\vacia$} (q0)
    
      (q8)
          edge node {$\vacia$} (q9)
      (q9)
          edge node {$1$} (q10)
      (q10)
          edge node {$\vacia$} (q11)
      (q11)
          edge node {$0$} (q12)
      (q12)
          edge [bend left] node {$\vacia$} (q8);
\end{tikzpicture}

\end{enumerate}

\todo{[END] Ejemplo}

\subsection{Teorema de An\'alisis de Kleene}

Ahora demostraremos la segunda parte en donde un aut\'omata finito implica  
una expresi\'on regular, la noci\'on detr\'as de esta parte es que analizaremos 
los lenguajes acumulados de cada estado para generar una expresi\'on regular.

\teo{
Dado un aut\'omata finito~$M$ existe una expresi\'on regular~$\alpha$ tal que 
$L(M)=L(\al)$. Es decir, $L(M)$ es regular.
}
%\vspace*{-20pt}
\begin{proof}
Existen diversas demostraciones, nosotros usaremos el m\'etodo de ecuaciones 
caracter\'isticas usando el Lema de Arden.
\end{proof}

\subsubsection{Lema de Arden}
Este lema extrae un conjunto de ecuaciones para determinar el lenguaje de 
aceptaci\'on de una m\'aquina.
Primero veamos la definic\'on de dichas ecuaciones y despu\'es el m\'etodo para 
obtener las ecuaciones dado un aut\'omata. 

\defin{
Sean $A,B\inc \sest$ y $X$ una variable:
\bi
 \item Una ecuaci\'on lineal derecha para $X$ es una expresi\'on de la forma:
  $$X = AX+B$$
 \item An\'alogamente, una ecuaci\'on lineal izquierda es una expresi\'on de 
  la forma:
  $$X = XA+B $$
  Donde el s\'imbolo $+$ denota a la uni\'on de lenguajes.
\ei
}

\lema{\textbf{\!\!: Lema de Arden}
Sean $A,B\inc\sest$ dos lenguajes y $X=AX+B$ una ecuaci\'on lineal derecha. 
Entonces
\be
 \item $A^\star B$ es una soluci\'on de la ecuaci\'on, es decir, 
  $A^\star B=A(A^\star B)+B$.
 \item Si $L$ es una soluci\'on entonces $A^\star B\inc L$, es decir, 
  $A^\star B$ es la soluci\'on m\'inima.
 \item Si $\cv\notin A$ entonces $A^\star B$ es la \'unica soluci\'on.
 \item Si $\cv\in A$ entonces existe un infinidad de soluciones.
\ee 
}
\begin{proof}$\;$\medskip
  \begin{enumerate}
  \item Por propiedades de lenguajes regulares se tiene que:
\[
A(A^\star B) + B = (AA^\star)B + \vacia B = (AA^\star+ \vacia) B = A^\star B
\]
\item Sea $L$ una solución, es decir, $L=AL + B$. Sea $w\in A^\star B$ y demostremos
  que $w\in L$. Como $w\in A^\star B$, existen $u\in A^\star$ y $v\in B$ tales
  que $w=uv$. Veamos que $uv\in L$ mediante inducción sobre $u\in A^\star$
  \begin{itemize}
  \item Base, $u=\vacia$. En este caso $w=v\in B\inc AL+B = L\;\; \therefore
    w\in L$.
  \item H. I. Si $u_2\in A^\star$ entonces $u_2v\in L$.
  \item Paso Inductivo: sea $u=u_1u_2$ donde $u_1\in A$ y $u_2\in
    A^\star$. Queremos demostrar que $w=uv=u_1u_2v\in L$. Por la H.I. tenemos
    que $u_2v\in L$, así $u_1(u_2v)\in AL\inc AL + B = L\;\; \therefore
    w=uv=(u_1u_2)v\in L$.
  \end{itemize}
\item Supongamos que $\vacia\notin A$. Sea $L=AL+B$ una solución, veamos que
  $L=A^\star B$. Por la parte 2 se tiene que $A^\star B\inc L$. Si la contención no es propia hemos
  terminado. Por lo tanto supongamos que $A^\star B\subset L$. En tal caso
  existe un $C$ tal que $L=A^\star B+C$ y $A^\star B\cap C=\varnothing$.
  Afirmamos que $C=\varnothing$. Como $L=AL+B$ entonces $A^\star B+C = A^\star
  B+AC$ de donde al intersectar con $C$ en ambos lados y simplificando se
  sigue que $(A^\star B\cap C)+C=(A^\star B\cap C)+(AC\cap C)$. Luego
  entonces $C=AC\cap C$, puesto que $A^\star B\cap C=\varnothing$. De aquí se
  sigue que $C\inc AC$, pero como $\vacia\notin A$, esto sólo puede
  pasar\footnote{Tómese una cadena $w\in C$ de longitud mínima, entonces
   $w\in AC$ de donde  $w=uv$ con $u\in A$ y $v\in C$, pero como $u\neq\vacia$
 entonces $v\in C$ tendría menor longitud que $w$ contradiciendo la
 minimalidad de $w$.} si $C=\varnothing$.
 Así que $L=A^\star B+C = A^\star B+\varnothing = A^\star B$. Luego entonces
 $A^\star B$ es la única solución.
\item Se deja como ejercicio.
  \end{enumerate}
\end{proof}

Esta parte mostrar\'a que un aut\'omata finito implica una expresi\'on regular, 
esto se har\'a por medio de un sistema de ecuaciones a partir de un 
AFN, para ello se abstraer\'a la noci\'on del lenguaje aceptado desde un estado 
particular del aut\'omata y no necesariamente del inicial.

\defin{
Dado un AFN~$M=\pt{Q,\Sigma,\delta,q_0,F}$ tal que $Q=\{q_0,\ldots, q_n\}$.
Definimos los siguientes conjuntos:
\bi
 \item El conjunto de cadenas que se aceptan desde el estado $q_i$,
   para cualquier $1\leq i\leq n$:
   $$ L_i=\{w\in\sest \mid \dest(q_i,w)\cap F\neq\vacio\}$$
 \item $L_0$ es el lenguaje aceptado por $M$, es decir, $L_0=L(M)$.
\ei
}

En general no es sencillo calcular directamente los conjuntos $L_i$. Para 
obtener una expresi\'on regular completa respecto al aut\'omata, se 
obtendr\'a un sistema de ecuaciones a partir de un AFN. Al resolverlo, $L_0$ 
ser\'a el lenguaje que reconoce el aut\'omata como se mencion\'o arriba. 

\noindent El sistema de ecuaciones se define usando:
\be
 \item El conjunto de s\'imbolos de $\Sigma$ tal que existe una transici\'on 
  del estado $q_i$ al estado $q_j$, para cualesquiera $1\leq i,j\leq n$, con 
  $n$ el total de estados:
      $$ X_{i,j}=\{a\in\Sigma \mid q_j\in\delta(q_i,a)\}$$
 \item El conjunto auxiliar $Y_i$ que indica si $\vacia$ es aceptada desde $q_i$
  \[
   Y_i = \left\{\ba{cl}
          \{\vacia\} & \mbox{si}\;q_i\in F \\
          \vacio & \mbox{en otro caso}
         \ea
         \right.
   \]
\ee
  
A partir de estos conjuntos podemos hallar una caracterización de 
cada lenguaje $L_i$ mediante ecuaciones % de los lenguajes de cada estado est\'an dadas 
% por la siguiente propiedad que es fácil de demostrar para cualquier
%$1\leq i\leq n$:
$$ L_i=\sum_{j=0}^n X_{i,j}L_j + Y_i$$
\noindent Ésta es la llamada ecuación característica para $L_i$, dichas
ecuaciones general el llamado sistema de ecuaciones 
caracter\'isticas para un AFN.

\vspace*{20pt}

Finalmente, el Lema de Arden nos indica c\'omo calcular los conjuntos $L_i$. 
Esta ser\'a la idea a seguir para demostrar el Teorema de Análisis de Kleene:
\be
 \item Dado el aut\'omata~$M$ construir los conjuntos $ X_{i,j},\;Y_i $ y
   plantear las ecuaciones características.
 \item Resolver el sistema de ecuaciones caracter\'isticas, empleando el Lema 
  de Arden donde se requiera.
 \item La soluci\'on para $L_0$ genera una expresi\'on regular para $L(M)$.
\ee

Es importante remarcar que la solución del sistema de ecuaciones raramente es
única, puesto que éste puede resolverse de distintas maneras. Por supuesto que
todas las soluciones resultan equivalentes.







\end{document}
