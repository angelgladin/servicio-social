\documentclass[letterpaper,11pt]{article}
\usepackage[includeheadfoot,margin=1.3in]{geometry}
\usepackage{anysize}

\usepackage[utf8]{inputenc}
\usepackage[english,spanish]{babel}
\usepackage{amssymb,amsmath}
\usepackage{mathrsfs}
\usepackage{epsfig}

\usepackage{fancyhdr}
\usepackage{hyperref}
\usepackage{url}

\usepackage{import}
\usepackage{comment}
\usepackage[autostyle=true,spanish=mexican]{csquotes}

\usepackage{todonotes} % Para hacer anotaciones, remover cuando ya no tengan anotaciones.
\usepackage{url}

\usepackage{pgf}
\usepackage{tikz}
\usetikzlibrary{automata,arrows}
\usetikzlibrary{babel}

\usetikzlibrary{arrows,automata,positioning, calc}
\tikzset{node distance=2.5cm, % Minimum distance between two nodes. Change if necessary.
        every state/.style={ % Sets the properties for each state
            semithick,
            fill=gray!10},
        initial text={}, % No label on start arrow
        double distance=2pt, % Adjust appearance of accept states
        every edge/.style={ % Sets the properties for each transition
            draw,
            ->,>=stealth', % Makes edges directed with bold arrowheads
            auto,
            semithick}
        }
\input{macrosAyLF}

\title{Aut\'omatas y Lenguajes Formales 2019-II \\ 
%Posgrado en Ciencia e Ingeniería de la Computación UNAM \\ 
Facultad de Ciencias UNAM\footnote{Material elaborado en el marco del proyecto
  PAPIME PE102117} \\ 
Nota de Clase X: Autómatas de divisibilidad} 
\author{Favio E. Miranda Perea \and A. Liliana Reyes Cabello \and
Lourdes Gonz\'alez Huesca}
\date{-1 de diciembre de 2020}

\begin{document}
\maketitle

%%% 2
\begin{center}
  \begin{tikzpicture}
    \node[state, accepting, initial] (q0) {$q_0$};
    \node[state] (q1) [right of=q0] {$q_1$};

    \draw
      (q0)
          edge [loop below] node {0} ( )
          edge [bend left] node {1} (q1)
      (q1)
          edge [bend left] node {0} (q0)
          edge [loop below] node [below] {1} ( );
  \end{tikzpicture}
\end{center}

%%% 3
\begin{center}
  \begin{tikzpicture}
    \node[state, accepting, initial] (q0) {$q_0$};
    \node[state] (q1) [right of=q0] {$q_1$};
    \node[state] (q2) [right of=q1] {$q_2$};

    \draw
      (q0)
          edge [loop below] node {0} ( )
          edge [bend left] node {1} (q1)
      (q1)
          edge [bend left] node {0} (q2)
          edge [bend left] node {1} (q0)
      (q2)
          edge [bend left] node {0} (q1)
          edge [loop below] node {1} ( );
  \end{tikzpicture}
\end{center}

%%% 4
\begin{center}
  \begin{tikzpicture}
    \node[state, accepting, initial] (q0) {$q_0$};
    \node[state] (q1) [right of=q0] {$q_1$};
    \node[state] (q2) [below of=q0] {$q_2$};
    \node[state] (q3) [below of=q1] {$q_3$};

    \draw
      (q0)
          edge [loop above] node {0} ( )
          edge [bend left] node {1} (q1)
      (q1)
          edge [bend left] node {0} (q2)
          edge [bend left] node {1} (q3)
      (q2)
          edge [bend left] node {0} (q0)
          edge [bend left] node {1} (q1)
      (q3)
          edge [bend left] node {0} (q2)
          edge [loop below] node {1} ( );
  \end{tikzpicture}
\end{center}

%%% 5
\begin{center}
  \begin{tikzpicture}
    \node[state, accepting, initial] (q0) at (0, 0) {$q_0$};
    \node[state] (q1) at (0, 4) {$q_1$};
    \node[state] (q2) at (2, 2) {$q_2$};
    \node[state] (q3) at (4, 4) {$q_3$};
    \node[state] (q4) at (4, 0) {$q_4$};

    \draw
      (q0)
          edge [loop below] node {0} ( )
          edge node {1} (q1)
      (q1)
          edge node [below] {0} (q2)
          edge [bend left] node {1} (q3)
      (q2)
          edge node {0} (q4)
          edge node [above] {1} (q0)
      (q3)
          edge [bend left] node [above] {0} (q1)
          edge node {1} (q2)
      (q4)
          edge node [right] {0} (q3)
          edge [loop below] node {1} ( );
  \end{tikzpicture}
\end{center}

%%% 6
\begin{center}
  \begin{tikzpicture}
    \node[state, accepting, initial] (q0) at (0, 2) {$q_0$};
    \node[state] (q1) at (3, 4) {$q_1$};
    \node[state] (q2) at (6, 4) {$q_2$};
    \node[state] (q3) at (3, 0) {$q_3$};
    \node[state] (q4) at (6, 0) {$q_4$};
    \node[state] (q5) at (9, 2) {$q_5$};

    \draw
      (q0)
          edge [loop below] node {0} ( )
          edge node {1} (q1)
      (q1)
          edge node {0} (q2)
          edge [bend left] node {1} (q3)
      (q2)
          edge [bend left] node {0} (q4)
          edge node {1} (q5)
      (q3)
          edge node {0} (q0)
          edge [bend left] node {1} (q1)
      (q4)
          edge [bend left] node {0} (q2)
          edge node {1} (q3)
      (q5)
          edge node {0} (q4)
          edge [loop below] node {1} ( );
  \end{tikzpicture}
\end{center}

%%% 7
\begin{center}
    \begin{tikzpicture}
      \node[state, accepting, initial] (q0) at (0, 0){$q_0$};
      \node[state] (q1) at (0, 3) {$q_1$};
      \node[state] (q2) at (3, 6) {$q_2$};
      \node[state] (q3) at (3, 0) {$q_3$};
      \node[state] (q4) at (3, 3) {$q_4$};
      \node[state] (q5) at (6, 3) {$q_5$};
      \node[state] (q6) at (6, 0) {$q_6$};
  
      \draw
        (q0)
            edge [loop below] node {0} ( )
            edge node {1} (q1)
        (q1)
            edge node {0} (q2)
            edge node {1} (q3)
        (q2)
            edge [bend left] node {0} (q4)
            edge node {1} (q5)
        (q3)
            edge node [below] {0} (q6)
            edge node {1} (q0)
        (q4)
            edge node {0} (q1)
            edge [bend left] node {1} (q2)
        (q5)
            edge node [above] {0} (q3)
            edge node {1} (q4)
        (q6)
            edge node [right] {0} (q5)
            edge [loop below] node {1} ( );
  \end{tikzpicture}
\end{center}

%%% 8
\begin{center}
  \begin{tikzpicture}
    \node[state, accepting, initial] (q0) at (0, 2) {$q_0$};
    \node[state] (q1) at (2, 4) {$q_1$};
    \node[state] (q2) at (5, 2) {$q_2$};
    \node[state] (q3) at (11, 4) {$q_3$};
    \node[state] (q4) at (2, 0) {$q_4$};
    \node[state] (q5) at (8, 2) {$q_5$};
    \node[state] (q6) at (11, 0) {$q_6$};
    \node[state] (q7) at (13, 2) {$q_7$};

    \draw
      (q0)
          edge [loop below] node {0} ( )
          edge node {1} (q1)
      (q1)
          edge node {0} (q2)
          edge node {1} (q3)
      (q2)
          edge node {0} (q4)
          edge [bend left] node {1} (q5)
      (q3)
          edge node {0} (q6)
          edge node {1} (q7)
      (q4)
          edge node {0} (q0)
          edge node {1} (q1)
      (q5)
          edge [bend left] node {0} (q2)
          edge node {1} (q3)
      (q6)
          edge node {0} (q4)
          edge node {1} (q5)
      (q7)
          edge node {0} (q6)
          edge [loop below] node {1} ( );
  \end{tikzpicture}
\end{center}

%%% 9
\begin{center}
  \begin{tikzpicture}
    \node[state, accepting, initial] (q0) at (4, 10) {$q_0$};
    \node[state] (q1) at (8, 8) {$q_1$};
    \node[state] (q2) at (4, 6) {$q_2$};
    \node[state] (q3) at (8, 0) {$q_3$};
    \node[state] (q4) at (0, 8) {$q_4$};
    \node[state] (q5) at (6, 4) {$q_5$};
    \node[state] (q6) at (0, 0) {$q_6$};
    \node[state] (q7) at (4, 2) {$q_7$};
    \node[state] (q8) at (2, 4) {$q_8$};

    \draw
      (q0)
          edge [loop below] node {0} ( )
          edge node {1} (q1)
      (q1)
          edge node {0} (q2)
          edge node {1} (q3)
      (q2)
          edge node {0} (q4)
          edge [bend left] node {1} (q5)
      (q3)
          edge node {0} (q6)
          edge node {1} (q7)
      (q4)
          edge node {0} (q8)
          edge node {1} (q0)
      (q5)
          edge node {0} (q1)
          edge [bend left] node {1} (q2)
      (q6)
          edge [bend right] node [below] {0} (q3)
          edge node {1} (q4)
      (q7)
          edge node {0} (q5)
          edge node {1} (q6)
      (q8)
          edge node {0} (q7)
          edge [loop below] node {1} ( );
  \end{tikzpicture}
\end{center}


\end{document}
