\documentclass[letterpaper,11pt]{article}
\usepackage[includeheadfoot,margin=1.3in]{geometry}
\usepackage{anysize}

\usepackage[utf8]{inputenc}
\usepackage[english,spanish]{babel}
\usepackage{amssymb,amsmath}
\usepackage{mathrsfs}

\usepackage{import}
\usepackage[autostyle=true,spanish=mexican]{csquotes}

\usepackage{minted}

\usepackage{tikz}
\usetikzlibrary{automata,arrows}
\usetikzlibrary{babel}
\usetikzlibrary{arrows,automata,positioning, calc}
\tikzset{node distance=2.5cm, % Minimum distance between two nodes. Change if necessary.
        every state/.style={ % Sets the properties for each state
            semithick,
            fill=gray!10},
        initial text={}, % No label on start arrow
        double distance=2pt, % Adjust appearance of accept states
        every edge/.style={ % Sets the properties for each transition
            draw,
            ->,>=stealth', % Makes edges directed with bold arrowheads
            auto,
            semithick}
        }
\input{macrosAyLF}

\title{Aut\'omatas y Lenguajes Formales 2019-II \\ 
%Posgrado en Ciencia e Ingeniería de la Computación UNAM \\ 
Facultad de Ciencias UNAM\footnote{Material elaborado en el marco del proyecto
  PAPIME PE102117} \\ 
Nota de Clase: Autómatas de divisibilidad} 
\author{Favio E. Miranda Perea \and A. Liliana Reyes Cabello \and
Lourdes Gonz\'alez Huesca \and
Ángel Iván Gladín García
}
\date{\today}

\begin{document}
\maketitle

En el estudio de autómatas finitos deterministas (AFD), una de sus muchas aplicaciones es el
verificar si dado un número $n$ es divisible o no por un número $k$.

Para mayor comodidad, se usará el número $n$ y $k$ codificado en sistema binario (aunque también
se podría hacer cualquier base $b$). Se construirá un DFA con $k$ estados, donde $n$ es el número
por el cual se quiere ver si es divisible o no, en donde cada estado tendrá la función de
transición definida para 0 ó 1.

Asumamos que cada número es ``leído'' de izquierda a derecha, es decir, el bit más significativo
(``msb'' \textit{most significative bit} por sus siglas en inglés) es leído primero.

Sea $x$ sea el valor del número binario en algún punto procesando la cadena y sea su residuo $r$
módulo $d$. Esto significa que $x$ puede ser expresado como combinación lineal como $x = md + r$
para algún entero $m$. El residuo puede tener valores $0 \leq r < d-1$. Cada unos de estos residuos
corresponden a los estados del autómata.

Se restringirá el dominio a solo aceptar cadenas no vacías para mayor practicidad.

Para calcular la función de transición verificamos lo siguiente; si se está en un estado $q_i$,
se ha leído $q_i$. Si se lee \texttt{0}, el nuevo número a leer
será $2 * q_i$. Si se lee \texttt{1}, el nuevo número se convierte en $2 * q_i + 1$. El nuevo estado puede
ser obtenido restando $k$ (en caso de ser mayor que $q_i$) de esos valores
($2 * q_i$ ó $2 * q_i + 1$) donde $0 \leq q_i < k$.

El estado inicial será $q_0$ y para verficar que si el número ees divisible su estado final (y el
único) debe ser $q_0$ porque significa que el número ya procesado y su residuo fue 0.

Tal autómata (AFD) queda formalmente definido como:
\begin{itemize}
  \item $Q = \{ q_i \;|\; 0 \leq i < k \}$
  \item $\Sigma = \{ 0, 1 \}$
  \item $\delta (q_i, 0) = q_{2i \mod k}$ y $\delta (q_i, 1) = q_{(2i + 1) \mod k}$
  \item $F = \{ q_0 \}$
\end{itemize}

Veamos a través de un ejemplo su construcción y después la verificación de un caso en específico:

\paragraph{Ejemplo:} Verificar si dado un número $n$ es divisible por 3 o no.

Cualquier número puede escrito de la forma $n = 3*x + y$ donde $x$ es el cociente y $y$ el residuo.
Para 3, hay tres estado en el AFD, cada uno correspondiendo al residuo 0, 1 ó 2. Y para cada estado
dos transiciones correpondientes a 0 y 1. La función de transición $\delta(q_i, a)$ nos dice que
leyendo del alfabeto $a$, nos movemos del estado $q_i$ a $q_{i'}$. El estado inicial siempre será
$q_0$, y el estado final indica en residuo, si el estado final es $q_0$, el número es divisible.

Cabe notar que el último estado es el residuo.

A continuación se muestra el autómata construido,
\begin{center}
  \begin{tikzpicture}
    \node[state, accepting, initial] (q0) {$q_0$};
    \node[state] (q1) [right of=q0] {$q_1$};
    \node[state] (q2) [right of=q1] {$q_2$};

    \draw
      (q0)
          edge [loop below] node {0} ( )
          edge [bend left] node {1} (q1)
      (q1)
          edge [bend left] node {0} (q2)
          edge [bend left] node {1} (q0)
      (q2)
          edge [bend left] node {0} (q1)
          edge [loop below] node {1} ( );
  \end{tikzpicture}
\end{center}

Analicemos por casos;
\begin{itemize}
  \item Se está en el estado $q_0$ y se lee \texttt{0}, nos quedamos en el estado $q_0$.
  \item Se está en el estado $q_0$ y se lee \texttt{1}, nos movemos al estado $q_1$ porque el número
  formado que es \texttt{1} y en decimal da de residuo 1.
  \item Se está en el estado $q_1$ y se lee \texttt{0}, nos movemos al estado $q_2$ porque el número
  formado que es \texttt{10} y en decimal da de residuo 2.
  \item Se está en el estado $q_1$ y se lee \texttt{1}, nos movemos al estado $q_0$ porque el número
  formado que es \texttt{11} y en decimal da de residuo 0.
  \item Se está en el estado $q_2$ y se lee \texttt{0}, nos movemos al estado $q_1$ porque el número
  formado que es \texttt{100} y en decimal da de residuo 1.
  \item Se está en el estado $q_2$ y se lee \texttt{1}, nos quedamos al estado $q_2$ porque el
  número formado que es \texttt{101} y en decimal da de residuo 2.
\end{itemize}

Quedando la función de transición como:
\[
  \begin{array}[c]{c|c|c} 
    \delta &  0 & 1\\\hline 
    q_{0} & q_{0} & q_{1}\\\hline 
    q_{1} & q_{2} & q_{0}\\\hline 
    q_{2} & q_{1} & q_{2}\\
  \end{array}  
\]

Verifiquemos si 4 es divisible por 3: La representación binaria de 6 es \texttt{100}, empezando en
$q_0$ se sigue que:
\begin{itemize}
  \item Estado: $q_0$, se lee \texttt{1}, nuevo estado $q_1$.
  \item Estado: $q_1$, se lee \texttt{0}, nuevo estado $q_2$.
  \item Estado: $q_2$, se lee \texttt{0}, nuevo estado $q_1$.
\end{itemize}
Como el último estado es $q_1$, por tanto el número 4 tiene de residuo 1 y no es divisible por 3.


\paragraph{Autómata para verificar si un número es divisible por 2.}
\begin{center}
  \begin{tikzpicture}
    \node[state, accepting, initial] (q0) {$q_0$};
    \node[state] (q1) [right of=q0] {$q_1$};

    \draw
      (q0)
          edge [loop below] node {0} ( )
          edge [bend left] node {1} (q1)
      (q1)
          edge [bend left] node {0} (q0)
          edge [loop below] node [below] {1} ( );
  \end{tikzpicture}
\end{center}

\paragraph{Autómata para verificar si un número es divisible por 4.}
\begin{center}
  \begin{tikzpicture}
    \node[state, accepting, initial] (q0) {$q_0$};
    \node[state] (q1) [right of=q0] {$q_1$};
    \node[state] (q2) [below of=q0] {$q_2$};
    \node[state] (q3) [below of=q1] {$q_3$};

    \draw
      (q0)
          edge [loop above] node {0} ( )
          edge [bend left] node {1} (q1)
      (q1)
          edge [bend left] node {0} (q2)
          edge [bend left] node {1} (q3)
      (q2)
          edge [bend left] node {0} (q0)
          edge [bend left] node {1} (q1)
      (q3)
          edge [bend left] node {0} (q2)
          edge [loop below] node {1} ( );
  \end{tikzpicture}
\end{center}

\paragraph{Autómata para verificar si un número es divisible por 5.}
\begin{center}
  \begin{tikzpicture}
    \node[state, accepting, initial] (q0) at (0, 0) {$q_0$};
    \node[state] (q1) at (0, 4) {$q_1$};
    \node[state] (q2) at (2, 2) {$q_2$};
    \node[state] (q3) at (4, 4) {$q_3$};
    \node[state] (q4) at (4, 0) {$q_4$};

    \draw
      (q0)
          edge [loop below] node {0} ( )
          edge node {1} (q1)
      (q1)
          edge node [below] {0} (q2)
          edge [bend left] node {1} (q3)
      (q2)
          edge node {0} (q4)
          edge node [above] {1} (q0)
      (q3)
          edge [bend left] node [above] {0} (q1)
          edge node {1} (q2)
      (q4)
          edge node [right] {0} (q3)
          edge [loop below] node {1} ( );
  \end{tikzpicture}
\end{center}

\paragraph{Autómata para verificar si un número es divisible por 6.}
\begin{center}
  \begin{tikzpicture}
    \node[state, accepting, initial] (q0) at (0, 2) {$q_0$};
    \node[state] (q1) at (3, 4) {$q_1$};
    \node[state] (q2) at (6, 4) {$q_2$};
    \node[state] (q3) at (3, 0) {$q_3$};
    \node[state] (q4) at (6, 0) {$q_4$};
    \node[state] (q5) at (9, 2) {$q_5$};

    \draw
      (q0)
          edge [loop below] node {0} ( )
          edge node {1} (q1)
      (q1)
          edge node {0} (q2)
          edge [bend left] node {1} (q3)
      (q2)
          edge [bend left] node {0} (q4)
          edge node {1} (q5)
      (q3)
          edge node {0} (q0)
          edge [bend left] node {1} (q1)
      (q4)
          edge [bend left] node {0} (q2)
          edge node {1} (q3)
      (q5)
          edge node {0} (q4)
          edge [loop below] node {1} ( );
  \end{tikzpicture}
\end{center}

\paragraph{Autómata para verificar si un número es divisible por 7.}
\begin{center}
    \begin{tikzpicture}
      \node[state, accepting, initial] (q0) at (0, 0){$q_0$};
      \node[state] (q1) at (0, 3) {$q_1$};
      \node[state] (q2) at (3, 6) {$q_2$};
      \node[state] (q3) at (3, 0) {$q_3$};
      \node[state] (q4) at (3, 3) {$q_4$};
      \node[state] (q5) at (6, 3) {$q_5$};
      \node[state] (q6) at (6, 0) {$q_6$};
  
      \draw
        (q0)
            edge [loop below] node {0} ( )
            edge node {1} (q1)
        (q1)
            edge node {0} (q2)
            edge node {1} (q3)
        (q2)
            edge [bend left] node {0} (q4)
            edge node {1} (q5)
        (q3)
            edge node [below] {0} (q6)
            edge node {1} (q0)
        (q4)
            edge node {0} (q1)
            edge [bend left] node {1} (q2)
        (q5)
            edge node [above] {0} (q3)
            edge node {1} (q4)
        (q6)
            edge node [right] {0} (q5)
            edge [loop below] node {1} ( );
  \end{tikzpicture}
\end{center}

\paragraph{Autómata para verificar si un número es divisible por 8.}
\begin{center}
  \begin{tikzpicture}
    \node[state, accepting, initial] (q0) at (0, 2) {$q_0$};
    \node[state] (q1) at (2, 4) {$q_1$};
    \node[state] (q2) at (5, 2) {$q_2$};
    \node[state] (q3) at (11, 4) {$q_3$};
    \node[state] (q4) at (2, 0) {$q_4$};
    \node[state] (q5) at (8, 2) {$q_5$};
    \node[state] (q6) at (11, 0) {$q_6$};
    \node[state] (q7) at (13, 2) {$q_7$};

    \draw
      (q0)
          edge [loop below] node {0} ( )
          edge node {1} (q1)
      (q1)
          edge node {0} (q2)
          edge node {1} (q3)
      (q2)
          edge node {0} (q4)
          edge [bend left] node {1} (q5)
      (q3)
          edge node {0} (q6)
          edge node {1} (q7)
      (q4)
          edge node {0} (q0)
          edge node {1} (q1)
      (q5)
          edge [bend left] node {0} (q2)
          edge node {1} (q3)
      (q6)
          edge node {0} (q4)
          edge node {1} (q5)
      (q7)
          edge node {0} (q6)
          edge [loop below] node {1} ( );
  \end{tikzpicture}
\end{center}

\paragraph{Autómata para verificar si un número es divisible por 9.}
\begin{center}
  \begin{tikzpicture}
    \node[state, accepting, initial] (q0) at (4, 10) {$q_0$};
    \node[state] (q1) at (8, 8) {$q_1$};
    \node[state] (q2) at (4, 6) {$q_2$};
    \node[state] (q3) at (8, 0) {$q_3$};
    \node[state] (q4) at (0, 8) {$q_4$};
    \node[state] (q5) at (6, 4) {$q_5$};
    \node[state] (q6) at (0, 0) {$q_6$};
    \node[state] (q7) at (4, 2) {$q_7$};
    \node[state] (q8) at (2, 4) {$q_8$};

    \draw
      (q0)
          edge [loop below] node {0} ( )
          edge node {1} (q1)
      (q1)
          edge node {0} (q2)
          edge node {1} (q3)
      (q2)
          edge node {0} (q4)
          edge [bend left] node {1} (q5)
      (q3)
          edge node {0} (q6)
          edge node {1} (q7)
      (q4)
          edge node {0} (q8)
          edge node {1} (q0)
      (q5)
          edge node {0} (q1)
          edge [bend left] node {1} (q2)
      (q6)
          edge [bend right] node [below] {0} (q3)
          edge node {1} (q4)
      (q7)
          edge node {0} (q5)
          edge node {1} (q6)
      (q8)
          edge node {0} (q7)
          edge [loop below] node {1} ( );
  \end{tikzpicture}
\end{center}

La pregunta aquí es, ¿se podría hace un programa que construya un autómata que verifique si un
número $n$ es divisible por un número $k$? Claro que sí.

El siguiente programa es mostrado a continuación\footnote{
El siguiente programa fue escrito usando \textit{Python} en su versión \texttt{3.8.1}, aunque con
que sea cualquier posterior a la versión 3 debe de funcionar.
Su ejecución es de la siguiente manera una vez ya guardado el archivo:\\
\texttt{\$ python3 dfa-division.py}

Una vez ejecutado, se introducira linea por linea cada número. Primero el número $n$ y acto seguido
el número $k$. Un ejemplo de dicha ejecución es el siguiente:\\
\texttt{\$ python3 dfa-division.py}\\
\texttt{10}\\
\texttt{6}\\
Dado como resultado:\\
\texttt{10 is not divisible by 6 and the remainder is 4}
}

\pagebreak

\inputminted[linenos]{python}{dfa-division.py}

\end{document}
