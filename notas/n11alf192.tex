\documentclass[letterpaper,11pt]{article}
\usepackage[includeheadfoot,margin=1.3in]{geometry}
\usepackage{anysize}

\usepackage[utf8]{inputenc}
% \usepackage[latin1]{inputenc}
\usepackage[english,spanish]{babel}
\usepackage{lmodern}   % font shapes...
\usepackage[T1]{fontenc} % join the compound symbols as a single symbol

\usepackage{amssymb,amsmath}
\usepackage{mathrsfs}
\usepackage{epsfig}

\usepackage{fancyhdr}
\usepackage{hyperref}
\usepackage{url}

\usepackage{import}
\usepackage{comment}
\usepackage[autostyle=true,spanish=mexican]{csquotes}

\usepackage{alltt}
\usepackage[section]{placeins}

\usepackage{url}
\usepackage{pgf}
\usepackage{tikz}
\usetikzlibrary{automata,arrows,trees}
\usetikzlibrary{babel}
%\pagestyle{fancyplain}
\input{macrosAyLF}

\title{Aut\'omatas y Lenguajes Formales 2019-II \\ 
% Posgrado en Ciencia e Ingeniería de la Computación UNAM \\ 
Facultad de Ciencias UNAM \\ 
Nota de Clase 11, Ambigüedad, gramáticas en lenguajes de programación} 
\author{Favio E. Miranda Perea \and A. Liliana Reyes Cabello \and
Lourdes Gonz\'alez Huesca}
\date{\today}

\begin{document}
\maketitle

\section{Ambig\"{u}edad}
Puede ser que dos derivaciones distintas tengan un mismo \'arbol de 
derivaci\'on y tambi\'en puede suceder que haya m\'as de un \'arbol de 
derivaci\'on para una cadena. Lo ideal es que cada cadena tenga s\'olo un 
\'arbol asociado. Si sucede lo anterior, implica que el lenguaje 
\textbf{no es ambiguo}. Desafortunadamente existen lenguajes ambiguos.
Veamos una definici\'on formal de este fen\'omeno:

\defin{
Una gram\'atica se dice \textbf{ambigua} si existe una palabra~$w$ con dos o 
m\'as \'arboles de derivaci\'on distintos.
En general una palabra puede tener m\'as de una derivaci\'on, pero  
\textnormal{un s\'olo \'arbol} y en tal caso no hay ambig\"{u}edad.
}

Algunas veces se puede suprimir la ambig\"{u}edad directamente. Sin embargo 
\textbf{no} hay un algoritmo para remover ambig\"{u}edad. Es decir el problema
de eliminar la ambig\"{u}edad es indecidible.

\vspace*{-7pt}

\paragraph{Ejemplo:}
Considere la siguiente gram\'atica: 
$$ \qquad \qquad S\imp AA \quad A\imp aSa \mid a $$

\noindent La palabra $a^5$ tiene las siguientes derivaciones diferentes y por 
la izquierda:
\bi
 \item[] $S\imp AA\imp aA \imp aaSa\imp aaAAa\imp aaaAa\imp aaaaa$
 \item[] $S\imp AA\imp aSaA \imp aAAaA\imp aaAaA\imp aaaaA\imp aaaaa $
\ei
\noindent Las dos derivaciones generan \'arboles distintos:


\begin{small}
\begin{minipage}{.4\textwidth}
  \centering
  \begin{tikzpicture}[level distance=12mm]
  \tikzstyle{level 1}=[sibling distance=20mm]
  \tikzstyle{level 2}=[sibling distance=10mm]
 
  \node {$S$}
    child {node {$A$}
      child {node {$a$}}
     }
      child {node {$A$}
        child {node {$aSa$}}
        child {node {$AA$}
	  child {node {$a$}}
	  child {node {$a$}}
	}
    };
 \end{tikzpicture}
 \end{minipage}
\begin{minipage}{.4\textwidth}
 \centering
 \begin{tikzpicture}[level distance=12mm]
  \tikzstyle{level 1}=[sibling distance=20mm]
  \tikzstyle{level 2}=[sibling distance=10mm]
 
  \node {$S$}
    child {node {$A$}
      child {node {$aSa$}
	child {node {$AA$}
	  child {node {$a$}}
	  child {node {$a$}}
	}
     }
    }
      child {node {$A$}
        child {node {$a$}}
    };
 \end{tikzpicture}
 \end{minipage}
\end{small}


%\subsection{Lenguajes Ambiguos}

\defin{
Un lenguaje~$L$ es ambiguo si existe una gram\'atica ambigua $G$ que 
genera a $L$.\\
Un lenguaje es \textbf{inherentemente} ambiguo si \textit{todas} las 
gram\'aticas que lo generan son ambiguas.
}

\paragraph{Ejemplo:}
El lenguaje~$L = \{a^{2+3i} \mid i\geq 0\}$ es ambiguo por ser generado por la 
siguiente gram\'atica ambigua:
$$ S\imp AA \qquad \qquad A\imp aSa \mid a $$
Sin embargo este lenguaje tambi\'en es generado por una gram\'atica no ambigua:
$$ S\imp aa \mid aaU \qquad \qquad U\imp aaaU \mid aaa $$
En este caso la derivaci\'on de $a^5$ es:
$$S\imp aaU\imp aaaaa$$
y por lo tanto $L$ no es un lenguaje inherentemente ambiguo.

\paragraph{Ejemplo:}
El lenguaje 
$$L = \{a^nb^nc^md^m \mid n,m\geq 1\}\cup\{a^nb^mc^md^n \mid n,m\geq 1\}$$
es generado por la gram\'atica:
\[
 \ba{lll}
  S\imp AB \mid C & \qquad A\imp aAb \mid ab &\qquad B\imp cBd \mid cd \\
  C\imp aCd \mid aDd &\qquad  D\imp bDc \mid bc & 
 \ea
\]
La cadena~$aabbccdd$ tiene dos derivaciones por la izquierda:
\bi
\item[] $S\imp AB\imp aAbB\imp aabbB\imp aabbcBd\imp aabbccdd$
\item[] $S\imp C\imp aCd\imp aaDdd\imp aabDcdd\imp aabbccdd$
\ei

Como se coment\'o antes, la ambig\"{u}edad no puede ser eliminada con un 
m\'etodo y en general, probar la ambig\"{u}edad inherente es complicado.  

Veamos ahora algunos ejemplos de gramáticas ambigüas en relación a lenguajes
de programación

\begin{example} Lenguaje de paréntesis balanceados:
\beqs
S \to \cv\;|\;(S)\;|\;SS
\eeqs
\bi
\item Ambigüedad: $()()()$ tiene dos árboles de derivación %es ambigua pero equivalente a la siguientes gramática no ambigua:
\item Gramática no ambigua equivalente:
\beqs
S \to \cv\;|\;(S)S
\eeqs
\ei
\end{example}


\begin{example} Expresiones aritméticas: % este lenguaje es generado por la siguiente
  % gramática ambigua
\beqs
S \to S + S\;|\;S*S\;|\;a
\eeqs
donde $a$ es un terminal que representa a los identificadores y constantes.
\bi
\item Ambigüedad: a + a * a

\item Gramática no ambigua equivalente:  se obtiene modelando la precedencia de
operadores como sigue:
\beqs
\ba{rll}
S & \to & S+T\;|\;T \\
T & \to & T*F\;|\;F\\
F & \to & (S)\;|\;a
\ea
\eeqs
\ei
\end{example}


\section{Gramáticas libres de contexto en lenguajes de programación}
\bi
\item  El estudio formal de los lenguajes de programación se divide en sintaxis, pragmática y semántica.
\item La semántica se encarga de definir el significado de las expresiones, enunciados
y unidades de programa.
\item La pragmática define la implementación del lenguaje basada en la metodología y estrategias de programación deseadas
\item La sintaxis se encarga de definir la forma de
las expresiones y enunciados de un lenguaje y se sirve fundamentalmente de los
conceptos y herramientas de nuestro curso. 
\item Antes del proceso de evaluación, un compilador e intérprete necesita realizar
los procesos de análisis léxico y sintáctico, describitos a
continuación a grandes rasgos.
\ei

\subsection{Análisis léxico y sintáctico}
\bi
\item Análisis léxico: se encarga de transformar el programa fuente en una
  lista de unidades sintácticas de bajo nivel llamadas lexemas, los cuales se
  clasifican en distintas categorías llamadas {\em tokens}, como pueden ser
  identificadores, constantes, separadores, etc. 
\item El análisis léxico se sirve
  fundamentalmente de expresiones regulares para su definición y reconocimiento.
\item Análisis sintáctico: se encarga de transformar la lista de lexemas en un
  programa objeto, el cual es una 
  expresión válida de la llamada sintaxis abstracta del lenguaje. Este
  programa es esencialmente un árbol de derivación dictado por
  una gramática libre de contexto que define al lenguaje de programación. 
\item Por
  lo tanto el análisis sintáctico es esencialmente una forma del problema de
  la pertenencia en gramáticas libres de contexto.
\ei

\subsection{Forma de Backus-Naur}  
\bi
\item Las gramáticas libres de contexto para lenguajes de programación suelen
escribirse en la forma de Backus-Naur o BNF.
\item Este método de definición de gramáticas fue introducido por John Backus para el lenguaje {\sc Algol} 58 en 1959 y fue mejorado por Peter Naur para la definición de {\sc Algol} 60.
\item Este sistema notacional para definir lenguajes libres de contexto 
 sigue las siguientes convenciones:

\bi
\item El símbolo de reescritura $\to$ se reemplaza con $::=$.\medskip
\item El símbolo $\mid$ significa {\em o} y abreviar la definición de producciones de una misma variable.\medskip
\item Las variables se escriben entre paréntesis triangulares y por lo general
  utilizan nombres largos que ayuden a la descripción de \\ las categorías del lenguaje.
\ei
\ei



\begin{example}
%\titulos{Gramáticas en forma de Backus-Naur}{Ejemplos}
Algunos ejemplos de gramáticas en FBN son:
\bi
\item Lenguaje de paréntesis balanceados
\beqs
\ba{rll}
<parent\_balanc> & ::= & \cv \;|\; \\
                 & & \big (<parent\_balanc>\big )\;|\; \\
                 & & <parent\_balanc><parent\_balanc> 
\ea
\eeqs
\item Expresiones aritméticas:
\beqs
\ba{rll}
<expr> & ::= & <expr>\,<op>\,<expr>\;|\; \\
       & & (<expr>)\;|\;<id>\\
<op> & := & +\;|\;-\;|\;*\;|\;/\\
<id> & := & a\;|\;b\;|\;c
\ea
\eeqs

\item Bloques de asignación de expresiones aritméticas:
% {{\tt begin}\;a {\tt :=} b/c\; ; \; b {\tt :=} a*(b+c)\;{\tt end}}
% %{Gramáticas en forma de Backus-Naur}
\beqs
\ba{rll}
<programa> & ::= & {\tt begin}\;<sec\_enunc>\;{\tt end}\\ \\
<sec\_enunc> & ::= & <enunc>\;|\;<enunc>\,;\,<sec\_enunc>\\ \\
<enunc> & ::= & <id>\;{\tt :=}\;<expr>\\ \\
<expr> & ::= & <expr>\,<op>\,<expr>\;|\; \\ \\
       & & (<expr>)\;|\;<id>\\ \\
<op> & := & +\;|\;-\;|\;*\;|\;/\\ \\
<id> & := & a\;|\;b\;|\;c
\ea
\eeqs

\ei
\end{example}




% \begin{frame}
% \titulos{Bloques de asignación de expresiones aritméticas:}
% {{\tt begin}\;a {\tt :=} b/c\; ; \; b {\tt :=} a*(b+c)\;{\tt end}}
% %{Gramáticas en forma de Backus-Naur}
% \beqs
% \ba{rll}
% <programa> & ::= & {\tt begin}\;<sec\_enunc>\;{\tt end}\\ \\
% <sec\_enunc> & ::= & <enunc>\;|\;<enunc>\,;\,<sec\_enunc>\\ \\
% <enunc> & ::= & <id>\;{\tt :=}\;<expr>\\ \\
% <expr> & ::= & <expr>\,<op>\,<expr>\;|\; \\ \\
%        & & (<expr>)\;|\;<id>\\ \\
% <op> & := & +\;|\;-\;|\;*\;|\;/\\ \\
% <id> & := & a\;|\;b\;|\;c
% \ea
% \eeqs


% \end{frame}

% Veamos ahora tres ejemplos importantes de gramáticas ambiguas en lenguajes de programación

Terminamos con un ejemplo importante en la historia de los lenguajes de programación:

% \begin{frame}[fragile]
% \titulos{Gramáticas en forma de Backus-Naur}{Ejemplos}
\bi
\item Expresiones condicionales:
\beqs
\ba{rll}
<enunc> & ::= & <condicional>\;|\;<otro>\\
<condicional> & := & {\tt if}\;<expr>\;{\tt then}\;<enunc> \;|\; \\
              &    & {\tt if}\;<expr>\;{\tt then}\;<enunc> \\
              &    & \;{\tt            else}\;<enunc>
\ea
\eeqs
\item Problema del {\tt else} colgante ({\em dangling else}):
\begin{verbatim}
if false then if false then 0 else 1
\end{verbatim}
\item Dos significados:
\begin{verbatim}
if false then (if false then 0) else 1
if false then (if false then 0 else 1)
\end{verbatim}
%\ec
\item Una gramática equivalente no ambigua es:
\beqs
\ba{rll}
<enunc> & ::= \; <condicional>\;|\;<otro>\\ \\
<condicional> & := \;<cond\_doble>\;|\;<cond\_final>& \\ \\
<cond\_doble> & :=  \; {\tt if}\;<expr>\;{\tt then}\;<cond\_doble>\;{\tt
  else}\;<cond\_doble> \\ \\
<cond\_final> & :=  {\tt if}\;<expr>\;{\tt then}\;<enunc>\;|\;\\
             &     \;\;\;\;\;{\tt if}\;<expr>\;{\tt then}\;<cond\_simple>\;{\tt
               else}\;<cond\_final> \\
% \; {\tt if}\;<expr>\;{\tt then}\;<cond-doble>\;{\tt
%   else}\;<cond_doble> \\
%& {\tt if}\;<expr>\;{\tt then}\;<enunc> \;|\; \\
% S & \to  C\;|\;O\\
% C & \to  C1\;|\;C2\\
% C1 & \to  {\tt if}\;E\;{\tt then}\;C1\;{\tt else}\;C1  \\
% C2 & \to    {\tt if}\;E\;{\tt then}\;S\;|\; %\\
% %   & & 
% {\tt if}\;E\;{\tt then}\;C1\;{\tt else}\;C2
\ea
\eeqs
\item Idea: $C1$ genera condicionales dobles (if-then-else) balanceados; $C2$ representa condicionales
simples (if-then) y condicionales dobles pero de forma que un if-then sólo figura colgando \\ al final (en el else). %único lugar donde puede quedar un condicional sencillo es en la condición else.

\ei
% \end{frame}

% \begin{frame}
% \titulos{Paréntesis balanceados}{Eliminación de la ambigüedad}
%   \bi 


% \begin{frame}
% \titulos{Expresiones aritméticas}{Eliminación de la ambigüedad}
% \bi
% \item Expresiones aritméticas: % este lenguaje es generado por la siguiente
%   % gramática ambigua
% \beqs
% S \to S + S\;|\;S*S\;|\;a
% \eeqs
% donde $a$ es un terminal que representa a los identificadores y constantes.
% \item Ambigüedad: a + a * a

% \item Gramática no ambigua equivalente:  se obtiene modelando la precedencia de
% operadores como sigue:
% \beqs
% \ba{rll}
% S & \to & S+T\;|\;T \\
% T & \to & T*F\;|\;F\\
% F & \to & (S)\;|\;a
% \ea
% \eeqs
% \ei
% \end{frame}
% \begin{frame}
% \titulos{Expresiones condicionales}{Eliminación de la ambigüedad}
% \bi
% \item Expresiones condicionales:
% \beqs
% \ba{rll}
% S & \to & C\;|\;O\\
% C & \to & {\tt if}\;E\;{\tt then}\;S \;|\; %\\
%               %&    & 
% {\tt if}\;E\;{\tt then}\;S\;{\tt else}\;S
% \ea
% \eeqs
% \item Una gramática equivalente no ambigua es:
% \beqs
% \ba{rll}
% S & \to & C\;|\;O\\
% C & \to & C1\;|\;C2\\
% C1 & \to & {\tt if}\;E\;{\tt then}\;C1\;{\tt else}\;C1  \\
% C2 & \to   & {\tt if}\;E\;{\tt then}\;S\;|\; %\\
% %   & & 
% {\tt if}\;E\;{\tt then}\;C1\;{\tt else}\;C2
% \ea
% \eeqs
% \item Idea: $C1$ genera condicionales dobles (if-then-else) balanceados; $C2$ representa condicionales
% simples (if-then) y condicionales dobles pero de forma que un if-then sólo figura colgando \\ al final (en el else). %único lugar donde puede quedar un condicional sencillo es en la condición else.
% \ei
% \end{frame}



\end{document}

%%% Local Variables: 
%%% mode: latex
%%% TeX-master: t
%%% End: 
