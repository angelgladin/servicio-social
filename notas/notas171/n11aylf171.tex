
\documentclass[letterpaper,11pt]{article}
\usepackage[includeheadfoot,margin=1.3in]{geometry}
\usepackage{anysize}

\usepackage[utf8]{inputenc}
% \usepackage[latin1]{inputenc}
\usepackage[english,spanish]{babel}
\usepackage{lmodern}   % font shapes...
\usepackage[T1]{fontenc} % join the compound symbols as a single symbol

\usepackage{amssymb,amsmath}
\usepackage{mathrsfs}
\usepackage{epsfig}

\usepackage{fancyhdr}
\usepackage{hyperref}
\usepackage{url}

\usepackage{import}
\usepackage{comment}
\usepackage[autostyle=true,spanish=mexican]{csquotes}

\usepackage{alltt}

\usepackage[section]{placeins}
\usepackage{wrapfig}
\usepackage{multicol}

\usepackage{pgf}
\usepackage{tikz}
\usetikzlibrary{automata,arrows,trees}
\usetikzlibrary{babel}
%\pagestyle{fancyplain}
\input{../macrosAyLF}

\title{Aut\'omatas y Lenguajes Formales 2017-1 \\ 
% Posgrado en Ciencia e Ingeniería de la Computación UNAM \\ 
Facultad de Ciencias UNAM \\ 
Nota de Clase 11} 
\author{Favio E. Miranda Perea \and A. Liliana Reyes Cabello \and
Lourdes Gonz\'alez Huesca}
\date{\today}

\begin{document}
\maketitle

\vspace*{-30pt}

\section{Introducci\'on}

Los lenguajes regulares como lenguajes sencillos de reconocer por una m\'aquina 
encuentran su mancuerna con los aut\'omatas finitos, los cuales son 
dispositivos cuya memoria es limitada: s\'olo son \'utiles al recordar estado 
por estado alguna propiedad sin poder hacerlo a lo largo de todo una secuencia 
de estados. 

Para poder reconocer lenguajes m\'as expresivos, como los lenguajes libres de 
contexto podemos pensar en un alguna generalizaci\'on de los aut\'omatas 
finitos, la puede realizarse de diversos modos:
\bi
 \item permitir que el aut\'omata \textit{lea} una entrada un n\'umero 
  arbitrario de veces;
%   lo cual requiere que la cabeza de lectura se mueva
%    (m\'aquinas bidireccionales con el mismo poder que AF).
 \item a\~nadir un dispositivo de conteo (m\'aquinas contadoras);
 \item a\~nadir dispositivos de salida (traductores, aut\'omatas de Mealy y 
  Moore);
 \item agregar un dispositivo de memoria, por ejemplo mediante una pila para 
  almacenar datos.
\ei
De las generalizaciones anteriores, la \'ultima permite abstraer 
caracter\'isticas m\'as complejas de un lenguaje al tener m\'as memoria que 
almacene.
Los aut\'omatas que ser\'an de utilidad para reconocer lenguajes libres de 
contexto son los que generalizan a los aut\'omatas finitos al incluir una pila.

\vspace*{-10pt}

\section{Aut\'omatas de pila}
Los aut\'omatas de pila son una generalizaci\'on de los aut\'omatas finitos, el 
dispositivo de memoria es una pila que una estructura de datos para almacenar 
objetos. 

\vspace*{7pt}

\begin{wrapfigure}{r}{0.45\textwidth}
 \begin{center}
  \vspace*{-30pt}
  \includegraphics[scale=0.20]{pila.jpeg}
 \end{center}
\end{wrapfigure}
Una pila es una estructura que puede almacenar elementos utilizando una
funci\'on llamada \verb=PUSH= y se puede acceder a los datos almacenados 
mediante dos funciones: obtener un elemento al \enquote{sacarlo} mediante la 
funci\'on \verb=POP= o s\'olamente \enquote{ver} (\verb-PEEK-) un elemento. 
Estas funciones s\'olo acceden al tope de la pila.
El tope de la pila es exactamente el \'ultimo elemento en ser almacenado y si 
no hay elementos almacenados decimos que la pila est\'a vac\'ia.

Esta caracter\'istica le da otro nombre a las pilas \textbf{\enquote{Last In 
First Out} (LIFO)} es decir que el \'ultimo elemento en ser almacenado ser\'a 
el primero en salir de la pila.\\
El manejo de pilas en los aut\'omatas considera algunas variantes:
\bi 
 \item cada pila tiene un alfabeto que contiene los s\'imbolos permitidos para 
  almacenarse en la pila. Este alfabeto debe incluir el s\'imbolo inicial de la 
  pila o \enquote{fondo} de la pila;
 \item la pila ser\'a tratada como una cadena de s\'imbolos del alfabeto 
  mencionado en donde el tope ser\'a el s\'imbolo m\'as a la izquierda de la 
  cadena y el tope ser\'a el s\'imbolo m\'as a la derecha;
 \item diremos que una pila est\'a vac\'ia si est\'a representada por la cadena 
  vac\'ia.
\ei

\vspace*{10pt}

Una visi\'on panor\'amica de los aut\'omatas de pila distingue tres 
par\'ametros: el estado actual, el s\'imbolo que se est\'a procesando adem\'as 
del s\'imbolo que se encuentra en el tope de la pila. 
Tambi\'en caracteriza su funcionamiento central: la m\'aquina es capaz de leer 
el s\'imbolo a procesar de la cadena de entrada, cambiar de estado y sustituir 
el s\'imbolo en el tope de la pila por una secuencia de s\'imbolos.

Al igual que los aut\'omatas finitos, se puede incluir el no-determinismo, se 
har\'a \'enfasis en la diferencia entre determinismo y no-determinismo dentro 
de estas m\'aquinas m\'as adelante. Ahora se presentan formalmente los 
aut\'omatas no deterministas as\'i como los conceptos que detallan su 
funcionamiento y el procesamiento de cadenas.

\defin{
Un aut\'omata de pila \textbf{no determinista} es una s\'eptupla:
$$ M = \pt{Q,\Sigma,\Gamma,\delta,q_0,Z_0,F} $$ 
donde:
\begin{multicols}{2}
\bi
 \item $Q\neq\vacio$ es un conjunto finito de estados.
 \item $\Sigma$ es el alfabeto de entrada.
 \item $\Gamma$ es el alfabeto de la pila.
 \item 
 $\delta : 
Q\times(\Sigma\cup\{\vacia\})\times\Gamma\imp\Pe(Q\times\Gamma^\star)$ 
%   es la funci\'on de transici\'on.
 \item $q_0\in Q$ es el estado inicial.
 \item $Z_0\in\Gamma$ es el s\'imbolo inicial de la pila.
 \item $F\inc Q$ es el conjunto de estados finales.
\ei
\end{multicols}
}
Analicemos con detalle la funci\'on de transici\'on: 
\[
\delta : Q\times(\Sigma\cup\{\vacia\})\times\Gamma\imp\Pe(Q\times\Gamma^\star)
\]
esta funci\'on recibe tres argumentos: un estado, un s\'imbolo o la cadena 
vac\'ia y el s\'imbolo al tope de la pila; devuelve un par con el nuevo estado 
y los s\'imbolos que ser\'an el nuevo tope de la pila.

% \newpage
Las transiciones son los pares que devuelve la funci\'on $\delta$:
si $(p,\gamma)\in \delta(q,a,s)$ entonces estando en el estado $q$ y leyendo 
$a$ con $s$ en el tope de la pila, el aut\'omata hace lo siguiente:
\bi
 \item consume el s\'imbolo $a$
 \item elimina $s$ del tope de la pila, es decir \verb=POP=($s$)
 \item escribe $\gamma$ en el tope de la pila, es decir \verb=PUSH=($\ga$) 
 \item cambia al estado $p$.
\ei
Es posible consumir, leer o escribir la cadena vac\'ia $\vacia$, de ah\'i que 
estos aut\'omatas sean no-deterministas.


\paragraph{Transiciones especiales}
Los siguientes tipos de transiciones son de especial inter\'es:
\bi
 \item $(q',s)\in\delta(q,a,s)$\\
  en este caso el contenido de la pila no se ha alterado.
 \item $(q',\vacia)\in\delta(q,a,s)$\\
  el s\'imbolo $s$ se borra y el aut\'omata tiene un nuevo tope en la 
  pila, es decir, el s\'imbolo colocado inmediatamente abajo de $s$.
 \item $(q',\gamma)\in\delta(q,\vacia,s)$\\
  esta es una $\vacia$-transici\'on, el s\'imbolo de entrada no se 
  procesa, se cambi\'o de estado y adem\'as el tope de la pila $s$ se 
  reemplaza por $\gamma$.
\ei
    
\subsection{Aceptaci\'on de palabras por un aut\'omata}
\defin{
Una configuraci\'on o descripci\'on instant\'anea de un aut\'omata de pila es 
una terna  $$  \pt{q,aw,s\beta} $$
que representa lo siguiente:
\bi
 \item el aut\'omata est\'a en el estado $q$
 \item $aw$ es la parte a\'un \textbf{no} procesada de la cadena de entrada,
  siendo $a$ el siguiente s\'imbolo a leer
 \item $s\beta$ es el contenido total de la pila, siendo $s$ el        
  s\'imbolo colocado en el tope.
\ei
}

La noci\'on informal de c\'omputo se formaliza mediante una 
relaci\'on~$\vdash_M$ entre configuraciones, llamada \textbf{paso de
computaci\'on} tal que 
\[
\pt{q,aw,s\beta}\vdash_M \pt{p,w,\gamma\beta}\;\;\text{ si y s\'olo si } \;\;
(p,\gamma)\in\delta(q,a,s)
\]
\bi
 \item $q$ es el estado actual de la m\'aquina $M$
 \item $aw$ es la parte de la cadena de entrada que falta por procesar con $a$ 
  el siguiente s\'imbolo a procesar
 \item $q$ es el nuevo estado de la m\'aquina
 \item $w$ es la nueva cadena a procesar
 \item $\gamma \beta$ es el contenido de la pila en donde $\gamma$ est\'a en el 
    tope
\ei  

\noindent Existen diferentes tipos de configuraciones que debemos enfatizar:
\bi
 \item Configuraci\'on inicial: $\pt{q_0,w,Z_0}$ para cualquier $w\in\sest$.
 \item Configuraci\'on de aceptaci\'on con estados finales: 
  $\pt{q_f,\vacia,\beta}$ con $q_f\in F$.
 \item Configuraci\'on de aceptaci\'on por pila vac\'ia:          
   $\pt{q_f,\vacia,\vacia}$.
\ei
Las configuraciones descritas son importantes y las dos \'ultimas distinguen 
dos formas de aceptar cadenas, una verificando el estado de la m\'aquina y la 
otra considerando el contenido de la pila. Estas formas ser\'an m\'as claras 
despu\'es de introducir el concepto de procesamiento de cadenas.

     

El procesamiento de una cadena en un aut\'omata de pila no-determinista puede 
hacerse mediante un \'arbol que analice los posibles c\'omputos. La ra\'iz del 
\'arbol ser\'a la configuraci\'on inicial $\pt{q_0,w,Z_0}$ y sus descendientes 
los pasos de computaci\'on generados por la m\'aquina. 
Las hojas del \'arbol pueden ser c\'omputos bloqueados o alguna de las 
configuraciones de aceptaci\'on descritas antes.
        
\subsection{Lenguaje aceptado por un aut\'omata de pila}
Con la noci\'on de paso de computaci\'on se formaliza la noci\'on de lenguaje 
aceptado por un aut\'omata por estados finales.
% de manera an\'aloga al caso de los au\'tomatas finitos:
La relaci\'on $\vdash^\star$ se define de la manera usual, es decir es la 
extensi\'on a varios pasos de la noci\'on de c\'omputo.

Se define el lenguaje de aceptaci\'on de un aut\'omata de pila 
como se hizo con los aut\'omatas finitos: al procesar una cadena, 
el aut\'omata ha alcanzado un estado final
\[ 
L(M) = \{w\in\sest \mid 
  \pt{q_0,w,Z_0}\vdash_M^\star\pt{q_f,\vacia,\beta} \text{ y } q_f\in F\}
\]
Obs\'ervese que el contenido de la pila $\beta$ es irrelevante y la cadena ha 
sido aceptada ya que se termin\'o de procesar llegando al estado final $q_f$.

\vspace*{10pt}

Otra forma de aceptaci\'on para los aut\'omatas de pila es mediante la pila 
vac\'ia:
\[  
V(M) = \{w\in\sest \mid \pt{q_0,w,Z_0}\vdash_M^\star\pt{p,\vacia,\vacia} \} 
\]
En este caso, obs\'ervese que el estado $p$ es irrelevante pero lo importante 
es procesar por completo la cadena y que la pila est\'e vac\'ia.

\vspace*{10pt}

Estos dos tipos de aceptaci\'on son equivalentes como veremos m\'as adelante, 
ahora veamos ejemplos de estas m\'aquinas.
\paragraph{Ejemplo:}
Considere el lenguaje $L_1 = \{a^ib^i \mid i\geq 0\}$, el siguente aut\'omata 
de pila reconoce a $L_1$: 
$$ M_1 =\pt{\{q_0,q_1,q_2\},\{a,b\},\{X,Z_0\},\delta,q_0,Z_0,\{q_2\}} $$
donde:
\[
 \begin{array}{rclcrcl}
  \delta(q_0,a,Z_0) & = &\{(q_0,XZ_0)\} & \qquad \qquad 
    & \delta(q_0,a,X) & = & \{(q_0,XX)\} \\
  \delta(q_0,\vacia,Z_0) & =& \{(q_2,\vacia)\} & \qquad \qquad
    & \delta(q_0,b,X) & = & \{(q_1,\vacia)\}  \\
  \delta(q_1,b,X) & = & \{(q_1,\vacia)\} & \qquad \qquad 
    & \delta(q_1,\vacia,Z_0) & = &\{(q_2,\vacia)\}
 \end{array}
\]
La aceptaci\'on de cadenas es por medio de estados finales.

\paragraph{Ejemplo:}
Ahora veamos un aut\'omata que reconoce a $L_1$ por medio de la pila vac\'ia:
$$ M_1'=\pt{\{q_0,q_1\},\{a,b\},\{X,Z_0\},\delta,q_0,Z_0,\vacio} $$
\[
 \begin{array}{rclcrcl}
  \delta(q_0,a,Z_0) & = & \{(q_0,XZ_0)\}\qquad \qquad 
   &\delta(q_0,a,X) & = & \{(q_0,XX)\} \\
  \delta(q_0,\vacia,Z_0) & = & \{(q_0,\vacia)\} \qquad \qquad 
   &\delta(q_0,b,X)& = &\{(q_1,\vacia)\}\\
  \delta (q_1,b,X) & = & \{(q_1,\vacia)\}  \qquad \qquad  
    & \delta(q_1,\vacia,Z_0) & = &\{(q_1,\vacia)\}
 \end{array}
\]

\paragraph{Ejemplo:}
Sea $L_2=\{a^ib^j \mid i\geq j\geq 0\}$, el siguiente aut\'omata acepta por 
pila vac\'ia el lenguaje:
$$  M_2=\pt{\{q_0,q_1\},\{a,b\},\{X,Z_0\},\delta,q_0,Z_0,\vacio}$$
\[
 \begin{array}{rclcrcl}
  \delta(q_0,a,Z_0) & = & \{(q_0,XZ_0)\} \qquad \qquad 
   & \delta(q_0,a,X) & = & \{(q_0,XX)\}\\
  \delta(q_0,\vacia,Z_0) & = &\{(q_0,\vacia)\}\qquad \qquad 
   & \delta(q_0,b,X)& = & \{(q_1,\vacia)\}\\
  \delta(q_1,b,X) & = & \{(q_1,\vacia)\} \qquad \qquad 
   & \delta(q_1,\vacia,Z_0) & = & \{(q_1,\vacia)\} \\
  \delta(q_1,\vacia,X) & = & \{(q_1,\vacia)\} & & &
 \end{array}
\]

Al igual que los aut\'omatas finitos, los aut\'omatas de pila tienen una 
representaci\'on gr\'afica usando v\'ertices o nodos y transiciones entre 
ellos.
Las transiciones estar\'an etiquetadas por los pares que representan el 
s\'imbolo a procesar y el \textit{antes/despu\'es} del tope en la pila. Veamos 
un ejemplo:

\paragraph{Ejemplo:}
El lenguaje $L_3 = \{wcw^R \mid w\in\{a,b\}^\star\}$ es aceptado por el 
aut\'omata 
$$ M_3 = \pt{\{q_0,q_1\},\{a,b,c\},\{A,B,Z_0\},\delta,q_0,Z_0,\vacio}$$
\[
 \begin{array}{rclcrcl}
 \delta(q_0,a,Z_0) & = & \{(q_0,AZ_0)\} & \qquad \qquad 
  & \delta(q_0,b,Z_0)&  =& \{(q_0,BZ_0)\}\\
 \delta(q_0,a,A)& = & \{(q_0,AA)\} & \qquad \qquad
  & \delta(q_0,b,A) & =& \{(q_0,BA)\}\\
 \delta(q_0,a,B) & = & \{(q_0,AB)\} &\qquad \qquad 
   & \delta(q_0,b,B)& = &\{(q_0,BB)\}\\
 \delta(q_0,c,Z_0) & = & \{(q_1,Z_0)\} &\qquad \qquad 
  & \delta(q_0,c,A) & = &\{(q_1,A)\}\\
 \delta(q_0,c,B)& = &\{(q_1,B)\} & \qquad \qquad 
  & \delta(q_1,a,A)& =& \{(q_1,\vacia)\}\\
 \delta(q_1,b,B) & = & \{(q_1,\vacia)\} & \qquad \qquad 
  & \delta(q_1,\vacia,Z_0) & = & \{(q_1,\vacia)\}
 \end{array}
\]

\vspace*{10pt}

\noindent Tambi\'en es aceptado por el siguiente aut\'omata:
\begin{center}
\begin{tikzpicture}[node distance=6cm,every node/.style={scale=0.8},semithick]
    \node[state,initial,initial text=] (q0) {$q_{0}$};
    \node[state] (q1) [right of=q0] {$q_1$};
    \node[state,accepting by double] (q2) [right of=q1] {$q_2$};
    \path[->] (q0) edge [in=160,out=130,loop] node [left] {$a,\,Z_0/aZ_0$} 
    (q0);
    \path[->] (q0) edge [in=110,out=80,loop] node [above] {$b,\,Z_0/bZ_0$} (q0);
    \path[->] (q0) edge [in=30,out=60,loop] node [right] {$a,\,a/aa$} (q0);
    \path[->] (q0) edge [in=200,out=230,loop] node [left] {$a,\,b/ab$} (q0);
    \path[->] (q0) edge [in=260,out=290,loop] node [below] {$b,\,a/ba$} (q0);
    \path[->] (q0) edge [in=310,out=340,loop] node [right] {$b,\,b/bb$} (q0);
    \path[->] (q0) edge node [above] {$\qquad\qquad c,\,a/a\quad c,\,Z_0/Z_0$} 
      node [below] {$c,\,b/b$} (q1);
    \path[->] (q1) edge [loop above] node [above] {$a,\,a/\vacia$} (q1);
    \path[->] (q1) edge [loop below] node [below] {$b,\,b/\vacia$} (q1);
    \path[->] (q1) edge node [above] {$\vacia,\, Z_0/Z_0$} (q2);
 \end{tikzpicture}
\end{center}

\subsection{Equivalencia de la aceptaci\'on}
Veamos las demostraciones para justificar que los dos tipos de aceptaci\'on 
en un aut\'omata de pila, por estados finales o por pila vac\'ia, son 
equivalentes.

\paragraph{De estados finales a pila vac\'ia}
\lema{
Dado un lenguaje $L$ tal que $L=L(M)$ para alg\'un aut\'omata de pila~$M$, 
existe un aut\'omata de pila $M'$ tal que $L=V(M')$. Es decir, todo lenguaje 
aceptado por estados finales es aceptado por pila vac\'ia.}
La prueba consiste en construir un nuevo aut\'omata que acepta el mismo 
lenguaje con la pila vac\'ia.\\ 
Sea $M=\pt{Q,\Sigma,\Gamma,\delta,q_0,Z_0,F}$ entonces definimos a $M'$ como 
sigue:
$$ M'=\pt{Q\cup\{p_0,p\},\Sigma,\Gamma\cup\{N_0\},\delta,p_0,N_0,F} $$
en donde
\bi
 \item se agregan dos estados $p_0,\;p$ siendo $p_0$ el nuevo estado inicial;
 \item se agrega un s\'imbolo $N_0$ a $\Gamma$, el cual ser\'a el nuevo
    s\'imbolo inicial de la pila;
 \item la funci\'on de transici\'on conserva todas las transiciones de~$M$ 
  adem\'as de las siguientes:
  \bi
   \item[] $\delta(p_0,\vacia,N_0)=\{(q_0,Z_0N_0)\}$
   \item[] $(p,s)\in\delta(q_f,\vacia,s)$, para todo 
   $q_f\in F,\;s\in\Gamma\cup\{N_0\}$   \item[] 
$\delta(p,\vacia,s)=\{(p,\vacia)\}$
  \ei
\ei      
 
\paragraph{De pila vac\'ia a estados finales}
\lema{
Dado un lenguaje $L$ tal que $L=V(M)$ para un aut\'omata de pila~$M$, existe un 
aut\'omata de pila $M'$ tal que $L=L(M')$. Es decir, todo lenguaje aceptado 
por pila vac\'ia es aceptado por estados finales.}
Esta prueba consiste en construir los estados finales del aut\'omata.\\
Sea $M=\pt{Q,\Sigma,\Gamma,\delta,q_0,Z_0,F}$ entonces definimos a $M'$ como 
sigue:
  $$M'=\pt{Q\cup\{p_0,p_f\},\Sigma,\Gamma\cup\{N_0\},\delta,p_0,N_0,\{p_f\}}$$
es decir
\bi
 \item se agregan dos estados, $p_0,p_f$ siendo $p_0$ el nuevo estado
   inicial y $p_f$ el \textbf{\'unico} estado final;
 \item se agrega un s\'imbolo $N_0$ a $\Gamma$, el cual ser\'a el nuevo
   s\'imbolo inicial de la pila;
 \item la funci\'on de transici\'on conserva todas las transiciones de~$M$ y 
  se agregan:
  \bi
   \item[] $\delta(p_0,\vacia,N_0)=\{(q_0,Z_0N_0)\}$
   \item[] $(p_f,s)\in\delta(q,\vacia,N_0)$, para todo $q\in Q$
  \ei  
\ei
  
\subsection{Aut\'omatas de pila deterministas}

Los aut\'omatas presentados hasta ahora son no-deterministas. La capacidad de 
ser determinista est\'a dada por que s\'olo hay un estado en la transici\'on 
de cualquier estado leyendo un s\'imbolo del alfabeto.

En los aut\'omatas de pila se logra al modificar la funci\'on de transici\'on:
\[
 \delta : Q\times(\Sigma\cup\{\vacia\})\times\Gamma\imp Q\times\Gamma^\star
\]
es decir hay a lo m\'as una transici\'on en un estado y s\'imbolo dados: 
\bi
 \item $\delta(q,a,s)=(p,\gamma)$ para $p,\gamma$ \'unicos 
 \item o bien $\delta(q,a,s)$ no est\'a definida, es decir, $\delta$ es una 
  funci\'on parcial
\ei
  
\noindent Al permitirse las $\vacia$-transiciones necesitamos la siguiente 
convenci\'on para mantener el determinismo cuando son retiradas:
\bi
 \item[] Si $\delta(q,a,s)$ est\'a definida entonces $\delta(q,\vacia,s)$ 
  est\'a indefinida y viceversa. Es decir, ambas transiciones no pueden estar 
  definidas al mismo tiempo.
\ei

\subsection{Determinismo vs. No-determinismo}
El poder expresivo que proporciona el no-determinismo en una m\'aquina puede 
ser valioso pero tambi\'en puede agregar complejidades que no son deseadas al 
momento de una implementaci\'on seria.

Obviamente un aut\'omata de pila determinista es un caso particular de un 
aut\'omata de pila no-determinista. 
A diferencia de los aut\'omatas finitos, en el caso de los aut\'omatas de 
pila no hay equivalencia entre los aut\'omatas deterministas y los 
no-deterministas.

Existen lenguajes que s\'olo pueden ser reconocidos mediante un aut\'omata de 
pila no-determinista.
Por ejemplo, $L = \{w\in\{a,b\}^\star \mid w=w^R\}$ el lenguaje de 
los pal\'indromos. 
Esto implica que hay lenguajes libres de contexto que no pueden ser 
reconocidos por un aut\'omata de pila determinista.

En lo siguiente estudiaremos a fondo la equivalencia entre estas m\'aquinas 
con memoria y los lenguajes libres de contexto.

\section{Equivalencia con Lenguajes Libres de Contexto}

\teo{ Un lenguaje es libre de contexto si y s\'olo si es aceptado 
por un aut\'omata de pila (no-determinista).
}
\vspace*{-20pt}
\begin{proof}
 La prueba es en dos partes:
 \bi
  \item[I] S\'intesis: Dado un lenguaje libre de contexto~$L$ existe un 
    aut\'omata de pila~$M$ tal que $L=L(M)$.
  \item[II] An\'alisis: Dado un aut\'omata de pila~$M$ existe una gram\'atica
    libre de contexto~$G$ tal que $L(M)=L(G)$. Es decir, $L(M)$ es libre 
  de contexto.
 \ei
\end{proof}


\subsection{De una Gram\'atica Libre de Contexto a un Aut\'omata de Pila}
Como vimos anteriormente, un lenguaje libre de contexto est\'a determinado 
por una gram\'atica libre de contexto que lo genere. 
Sea $G=\pt{V,T,S,P}$ una gram\'atica libre de contexto, definimos un  
aut\'omata de pila~$M_G$ como sigue:
\bi
 \item $Q=\{q_0,q_1,q_2\}$
 \item $\Sigma = T, \; \Gamma = V \cup \{Z_0\}$
 \item $F=\{q_2\}$
 \item $\delta$ se define mediante:
  \[
   \begin{array}{rcl}
     \delta(q_0,\vacia,Z_0) & = & \{(q_1,SZ_0)\} \\
     \delta(q_1,\vacia,A) & = & \{(q_1,\alpha) \mid A\imp\alpha\in P\} \\
     \delta(q_1,a,a) & = & \{(q_1,\vacia)\} \;\;\fa a\in\Sigma \\
     \delta(q_1,\vacia,Z_0) & = & \{(q_2,Z_0)\}
   \end{array}
  \]
\ei
Se cumple $L(G)=L(M_G)$.

\paragraph{Ejemplo:}
Dada $G$ mediante 
$$  S\imp aSb \mid  cSbS \mid  a $$
el aut\'omata de pila correspondiente es:
\[
 \begin{array}{rclcrcl}
  \delta(q_0,\vacia,Z_0) & = & \{(q_1,SZ_0)\} & \qquad \qquad &
  \delta(q_1,\vacia,S)& =&\{(q_1,aSb),\;(q_1,cSb),\;(q_1,a)\} \\
  \delta(q_1,a,a)& =& \{(q_1,\vacia)\}  & \qquad \qquad &
  \delta(q_1,b,b)& =& \{(q_1,\vacia)\} \\
  \delta(q_1,c,c) &=& \{(q_1,\vacia)\}  & \qquad \qquad &
  \delta(q_1,\vacia,Z_0) & = &\{(q_2,Z_0)\}
 \end{array}
\]

\subsection{De un Aut\'omata de Pila a una Gram\'atica Libre de Contexto}
Dado un aut\'omata de pila~$M$ que acepta por pila vac\'ia, vamos a construir 
una gram\'atica libre de contexto~$G_M$ tal que $L(M)=L(G_M)$.
Este m\'etodo genera gram\'aticas bastante complejas, con un gran n\'umero 
de variables y producciones. Adem\'as de que pueden generarse variables 
in\'utiles, pero ya se han estudiado los m\'etodos de transformaci\'on de 
gram\'aticas para simplificarlas o normalizarlas.

\vspace*{10pt}

\noindent Las variables de la gram\'atica ser\'an $S$ como s\'imbolo inicial y 
las expresiones de la forma:
$$[p,X,q] \quad \text{ donde } p,q\in Q,\;X\in\Gamma$$
La idea b\'asica es simular con derivaciones los c\'omputos de~$M$:
la variable $[p,X,q]$ debe generar todas las cadenas que llevan al 
aut\'omata de $p$ a $q$ al eliminar $X$ de la pila.

Las producciones de la gram\'atica son:
\bi
  \item Las producciones iniciales: $S\imp [q_0,Z_0,p]$ para todo $p\in Q$.

  \item Si $(p,\vacia)\in\delta(q,a,X)$ entonces leyendo~$a$ el aut\'omata pasa 
    de $q$ a $p$ y elimina a $X$ de la pila.
    Se agrega $ [q,X,p]\imp a $, esto incluye el caso $a=\vacia$.

 \item Si $(p,Y_1Y_2\ldots Y_m)\in\delta(q,a,X)$ entonces leyendo $a$ el 
  aut\'omata pasa de $q$ a $p$ y sustituye a $X$ por $Y_1Y_2\ldots Y_m$ en 
  la pila. 
  Se agrega $ [q,X,p_m]\imp a[p,Y_1,p_1][p_1,Y_2,p_2]\ldots[p_{m-1}Y_mp_m]$
  para todas las elecciones posibles de $p_1,\ldots,p_m\in Q$. 
  Esto incluye el caso $a=\vacia$.
\ei

\paragraph{Ejemplo:}
Considere el lenguaje $ L=\{a^ib^i \mid i\geq 0\} $ cuyo aut\'omata de pila es:
\[
 \begin{array}{rclcrcl}
  \delta(q_0,a,Z_0) & =& \{(q_0,XZ_0)\}& \qquad \qquad &
  \delta(q_0,a,X)& = &\{(q_0,XX)\} \\
  \delta(q_0,\vacia,Z_0) & = & \{(q_0,\vacia)\} & \qquad \qquad &
  \delta(q_0,b,X)&=& \{(q_1,\vacia)\} \\
  \delta(q_1,b,X) & = &\{(q_1,\vacia)\} & \qquad \qquad &
  \delta(q_1,\vacia,Z_0) & =&\{(q_1,\vacia)\}
 \end{array}
\]
La gram\'atica correspondiente tiene las siguientes producciones iniciales:
 \[
  S \imp [q_0,Z_0,q_0] \qquad \qquad S \imp [q_0,Z_0,q_1]
 \]
Adem\'as de las producciones correspondientes a las transiciones
\[
 \begin{array}{rclcrcl}
 \delta(q_0,\vacia,Z_0) & = & \{(q_0,\vacia)\} &  \qquad \qquad &
  [q_0,Z_0,q_0] & \imp & \vacia \\
 \delta(q_0,b,X) & = & \{(q_1,\vacia)\}  &  \qquad \qquad &
  [q_0,X,q_1]  & \imp & b \\
 \delta(q_1,b,X) & = & \{(q_1,\vacia)\} &  \qquad \qquad &
  [q_1,X,q_1]  & \imp & b \\
 \delta(q_1,\vacia,Z_0) & = & \{(q_1,\vacia)\} &  \qquad \qquad &
  [q_1,Z_0,q_1]  & \imp & \vacia 
 \end{array}
\]
Para la transici\'on $\delta(q_0,a,Z_0) = \{(q_0,XZ_0)\}$ se tienen 
las siguientes producciones:
\[
 \begin{array}{rcl}
 [q_0,Z_0,q_1] & \imp & a[q_0,X,q_0][q_0,Z_0,q_1] \\
 
 [q_0,Z_0,q_0] & \imp & a[q_0,X,q_1][q_1,Z_0,q_0] \\
 
 [q_0,Z_0,q_0]  & \imp & a[q_0,X,q_0][q_0 ,Z_0,q_0 ] \\

 [q_0,Z_0,q_1]  & \imp & a[q_0,X,q_1][q_1 ,Z_0,q_1 ] 
 \end{array}
\]
Y finalmente para $\delta(q_0,a,X)=\{(q_0,XX)\}$ las producciones:
\[
 \begin{array}{rcl}
 [q_0,X,q_0] & \imp &  a[q_0,X,q_0][q_0 ,X,q_0 ] \\
 
 [q_0,X,q_0]  & \imp & a[q_0,X,q_1][q_1 ,X,q_0 ] \\
 
 [q_0,X,q_1]  & \imp & a[q_0,X,q_0][q_0 ,X,q_1 ] \\
 
 [q_0,X,q_1]  & \imp & a[q_0,X,q_1][q_1 ,X,q_1 ] 
 \end{array}
\]
Veamos la derivaci\'on de la cadena~$aabb$:
 \[
  \begin{array}{rll}
  S & \imp & [q_0,Z_0,q_1] \\ 
    & \imp & a[q_0,X,q_1][q_1,Z_0,q_1] \\ 
    & \imp & aa[q_0,X,q_1][q_1,X,q_1][q_1,Z_0,q_1] \\
    & \imp & aab[q_1,X,q_1][q_1,Z_0,q_1] \\ 
    & \imp & aabb[q_1,Z_0,q_1] \\
    & \imp & aabb 
  \end{array}
  \]

\end{document}
