

\documentclass[letterpaper,11pt]{article}
\usepackage[includeheadfoot,margin=1.3in]{geometry}
\usepackage{anysize}

\usepackage[utf8]{inputenc}
% \usepackage[latin1]{inputenc}
\usepackage[english,spanish]{babel}
\usepackage{lmodern}   % font shapes...
\usepackage[T1]{fontenc} % join the compound symbols as a single symbol

\usepackage{amssymb,amsmath}
\usepackage{mathrsfs}
\usepackage{epsfig}

\usepackage{fancyhdr}
\usepackage{hyperref}
\usepackage{url}

\usepackage{import}
\usepackage{comment}
\usepackage[autostyle=true,spanish=mexican]{csquotes}

\usepackage{alltt}

\usepackage[section]{placeins}

\usepackage{pgf}
\usepackage{tikz}
\usetikzlibrary{automata,arrows,trees}
\usetikzlibrary{babel}
%\pagestyle{fancyplain}
\input{../macrosAyLF}

\title{Aut\'omatas y Lenguajes Formales 2017-1 \\ 
% Posgrado en Ciencia e Ingeniería de la Computación UNAM \\ 
Facultad de Ciencias UNAM \\ 
Nota de Clase 10,  Gramáticas libres de contexto} 
\author{Favio E. Miranda Perea \and A. Liliana Reyes Cabello \and
Lourdes Gonz\'alez Huesca}
\date{\today}

\begin{document}
\maketitle


La normalizaci\'on de gram\'aticas  consiste en transformar todas las 
producciones de una gram\'atica de manera que tengan cierta forma sint\'actica 
en particular.
As\'i la normalizaci\'on de gram\'aticas libres de contexto es \'util para    
homogeneizar la forma de las producciones adem\'as para optimizar los
procesos de derivaci\'on de cadenas.
Con las \textbf{formas normales} se facilita una soluci\'on al problema de 
la pertenencia, es decir decidir si una cadena pertenece o no a un lenguaje:
\bi
 \item[] \begin{center} Dada una gram\'atica~$G$ y una palabra $w$, 
  ?`se cumple que $w\in L(G)$ ? \\Es decir, pertenece $w$ al lenguaje generado 
  por~$G$.\end{center}
 \item Si la palabra $w$ es generada por $G$, el proceso para la 
construcci\'on 
  de un \'arbol de derivaci\'on terminar\'a eventualmente.
 \item En caso contrario \textbf{no} podemos saber cu\'ando parar en la 
  construcci\'on  de un \'arbol de derivaci\'on.
\ei

\section{Normalizaci\'on de Gram\'aticas}

Para obtener gram\'aticas normales, son necesarias varias transformaciones 
que simplificar\'an y reorganizar\'an las producciones.

\subsection{Eliminaci\'on de variables in\'utiles}

\defin{
Decimos que una variable~$A$ es \textbf{accesible} o \textbf{alcanzable} si  
existen $u,v\in(V\cup T)^\star$ tales que $S\imp^\star uAv$. 
Obs\'ervese que seg\'un esta definic\'on~$S$ siempre es alcanzable.
}

\defin{
Una variable~$A$ es \textbf{productiva} o \textbf{terminable} si existe 
$w\in T^\star$ tal que $A\imp^\star w$. En particular si $A\imp\vacia$ es una 
producci\'on entonces $A$ es productiva.
}

\defin{
Una variable~$A$ es \textbf{in\'util} si no es alcanzable o no es productiva.
}  

\newpage

\noindent A continuaci\'on damos un algoritmo para hallar variables 
productivas, mediante la transformaci\'on de una gram\'atica:
\bi
 \item[] \textbf{Iniciar}
  \begin{small}
    el conjunto $Prod$ con todas las producciones que contienen cadenas de 
    terminales a la derecha de $\imp$:   
  \end{small}
  $$ Prod :=\{A\in V\mid A\imp w\;\in P,\;w\in T^\star\}$$
 \item[] {\bf Repetir}
  \begin{small}
    la incorporaci\'on de variables cuyas producciones contienen variables 
    productivas y s\'imbolos terminales a la derecha de   $\imp$:
  \end{small}
  $$ Prod:=Prod\cup\{A\in V \mid A\imp w,\;w\in(T\cup Prod)^\star\} $$
 \item[] {\bf Hasta} 
   \begin{small} que no se a\~naden nuevas variables a $Prod$\end{small}
 \ei 

 
\paragraph{Ejemplo:}
Calculemos las variables productivas de la gram\'atica:
\[
\ba{rl}
  S \imp & ACD \mid bBd \mid ab \\
  A\imp & aB \mid aA \mid  C \\
  B\imp & aDS \mid aB \\
  C\imp & aCS \mid CB \mid CC \\
  D\imp & bD \mid ba \\
  E\imp & AB \mid aDb
\ea 
\]
Iniciamos con $Prod=\{S,D\}$.
Las iteraciones nos llevan a que $C$ es la \'unica variable improductiva, se 
elimina esta variable junto con todas las reglas donde figure:
\[
  \ba{rl}
  S \imp & bBd \mid ab \\
  A\imp & aB \mid aA \\
  B\imp & aDS \mid aB \\
%  C\imp & aCS \mid CB \mid CC \\
  D\imp & bD \mid ba \\
  E\imp & AB \mid aDb
  \ea
\]

\vspace*{10pt}

\noindent El siguiente algoritmo permite hallar variables accesibles:
\bi
 \item[] \textbf{Iniciar} $$Acc:=\{S\}$$
 \item[] \textbf{Repetir} 
  \begin{small}
   la incorporaci\'on de variables que aparecen a la derecha de $\imp$ en  
   producciones de variables accesibles:
  \end{small}
  \[
    Acc :=  Acc\cup\{A\in V \mid \exists\;B \imp uAv\in P,\;B\in Acc,
        \;u,v\in(V\cup T)^\star\}
    \]
 \item[] \textbf{Hasta}
 \begin{small} que no se a\~naden nuevas variables a $Acc$ \end{small}
\ei 

\paragraph{Ejemplo:}
Calculemos las variables accesibles de la gram\'atica:
\[
 \ba{rlcrl}
  S \imp & aS \mid AaB \mid ACS & \qquad \qquad & D\imp & aD \mid DD \mid ab \\
  A\imp & aS \mid AaB \mid  AC &  & E\imp & FF \mid aa \\
  B\imp & bB \mid DB \mid BB & & F\imp & aE \mid EF \\
  C\imp & aDa \mid ABD \mid ab 
  \ea
\]
Iniciamos con $Acc = \{S\}$. 
El resultado es: $  Acc=\{S,A,B,C,D\} $  ya que  $E$ y $F$ son variables 
inaccesibles, se eliminan junto con todas las reglas donde figuren:
\[
  \ba{rl}
  S \imp & aS \mid AaB \mid ACS \\
  A\imp & aS \mid AaB \mid  AC \\
  B\imp & bB \mid DB \mid BB \\
  C\imp & aDa \mid ABD \mid ab \\
  D\imp & aD \mid DD \mid ab \\
%  E\imp & FF \mid aa \\
%  F\imp & aE \mid EF
  \ea
\]


\noindent Para eliminar variables in\'utiles se aplican los dos 
algoritmos  anteriores \textbf{en el siguiente orden}:
\be
 \item Eliminar variables no productivas.
 \item Eliminar variables no accesibles.
\ee

La importancia del orden de los algoritmos radica en que si se aplican los 
algoritmos en orden inverso el resultado puede ser una gram\'atica que a\'un 
contenga variable in\'utiles. Veamos un ejemplo:
\paragraph{Ejemplo:}
Considere la siguiente gram\'atica
\[
  S\imp a \mid AB \qquad \qquad A\imp aA \mid \vacia
\]
Al eliminar primero las variables no accesibles se obtiene la misma 
gram\'atica, 
al ser todas las variables accesibles.\\
Posteriormente al eliminar variables improductivas resulta
$$S\imp a \qquad \qquad A\imp aA \mid \vacia$$
y claramente $A$ es in\'util por ser inaccesible.

\subsection{Eliminaci\'on de \texorpdfstring{$\vacia$}--producciones}
Las gram\'aticas libres de contexto permiten el uso de producciones de la 
forma 
$A\imp \vacia$. La eliminaci\'on de estas producciones llamadas 
$\cv$-producciones genera una transformaci\'on.

\defin{
Una variable $A$ se llama \textbf{anulable} si $A\imp^\star\vacia$, es decir si 
una derivaci\'on que empieza en $A$ genera la cadena vac\'ia.
}

\noindent Veamos un algoritmo para hallar variables anulables:
\bi
 \item[] \textbf{Iniciar} 
  \begin{small}
   el conjunto $Anul$ con las variables que tienen $\vacia$ como producci\'on
  \end{small}
  $$Anul:=\{A\in V \mid A\imp \vacia \;\in P\}$$
 \item[] \textbf{Repetir} 
  \begin{small}
   la incorporaci\'on de variables que tienen producciones cadenas de variables 
   anulables
  \end{small}
  \[
    Anul:=Anul\cup\{A\in V \mid \exists\;A\imp w\,\in P,\;\;w \in 
    Anul^\star\}
  \]
 \item[] \textbf{Hasta}
  \begin{small} que no se a\~naden nuevas variables a $Anul$\end{small}
\ei 

Una vez que se han identificado las variables anulables, la siguiente 
transformaci\'on de una gram\'atica libre de contexto elimina exactamente las 
$\cv$-producciones:
\bi
 \item[] Para cada producci\'on en la gram\'atica que tenga la 
  forma~$A\imp w_1\ldots w_n$ se deben agregar las producciones  
  $A\imp v_1\ldots v_n$ que son resultantes de los cambios de s\'imbolos donde:
  \bi
   \item $v_i = w_i$ si $w_i \notin Anul$, se respetan las variables no 
    anulables
   \item $v_i = w_i$ \'o $v_i = \vacia$ si $w_i\in Anul$, las variables 
    anulables pueden dejarse o eliminarse
  \ei
 \item[] Verificando que no se anulen todos los $v_i$ al mismo tiempo.
\ei
Es decir, se van a respetar las producciones existentes y si alguna 
contiene las variables anulables se agregar\'an las producciones que resulten 
de eliminar las variables anulables en esa producci\'on. Las 
$\vacia$-producciones desaparecer\'an.

\paragraph{Ejemplo:}
Eliminaci\'on de $\vacia$-producciones de la gram\'atica
\[
  \ba{rl}
  S \imp & AB \mid ACA \mid ab \\
  A\imp & aAa \mid B \mid  CD \\
  B\imp & bB \mid bA \\
  C\imp & cC \mid \cv \\
  D\imp & aDc \mid CC \mid ABb \\
%  E\imp & FF \mid aa \\
%  F\imp & aE \mid EF
  \ea
\]
Primero obtenemos las variables anulables, iniciando con $ Anul = \{C\}$:
$$  Anul=\{C,D,A,S\}$$
El proceso de anulaci\'on de variables hace que se elimine la 
producci\'on~$C\imp\vacia$, se dejan las producciones existentes y se agregan 
las producciones que eliminan los elementos de $Anul$. Se obtiene la siguiente 
gram\'atica:
\[
 \ba{rl}
  S \imp & AB \mid ACA \mid ab \mid \mathbf{B} \mid \mathbf{CA} \mid 
  \mathbf{AA} \mid \mathbf{AC} \mid \mathbf{A} \mid \mathbf{C} \\
  A\imp & aAa \mid B \mid  CD \mid \mathbf{aa} \mid \mathbf{C} \mid \mathbf{D}\\
  B\imp & bB \mid bA \mid \mathbf{b} \\
  C\imp & cC \mid \mathbf{c} \\
  D\imp & aDc \mid CC \mid ABb \mid \mathbf{ac} \mid \mathbf{C} \mid\mathbf{Bb}
%  E\imp & FF \mid aa \\
%  F\imp & aE \mid EF
  \ea
\]

\paragraph{Acerca de la palabra vac\'ia}
Si originalmente se ten\'ia $\vacia\in L(G)$ la eliminaci\'on de
$\vacia$-producciones genera una gram\'atica que \textbf{no} genera a $\vacia$.
Es posible saber si se pierde la palabra vac\'ia al eliminar 
$\vacia$-producciones verificando si $S\in Anul$. \\
Si se quiere recuperar a $\vacia$ debe agregarse un nuevo s\'imbolo inicial 
$S'$ 
as\'i como las producciones $S'\imp S$ y  $S'\imp\cv$.
De esta forma, $S'\imp\cv$ es la \'unica $\vacia$-producci\'on permitida.

    
\subsection{Eliminaci\'on de producciones unitarias} 
\defin{
Una producci\'on de la forma $A\imp B$ donde $A$ y $B$ son ambas variables se 
llama \textbf{producci\'on unitaria}. 
El \textbf{conjunto unitario} de $A$ se define como sigue:
\[
  Unit(A) = \{B\in V\mid A\imp^\star B \text{ usando s\'olo 
  producciones unitarias }\} 
\]
}
Obs\'ervese que por definici\'on se tiene $A\in Unit(A)$ si existe $A\imp B$.

\noindent Ahora veamos un algoritmo para hallar el conjunto $Unit(A)$ para 
cualquier variable o s\'imbolo no-terminal:
\bi
 \item[] \textbf{Iniciar} $$Unit(A):=\{A\}$$
 \item[] \textbf{Repetir} 
 \begin{small}
  la incorporaci\'on de variables que tengan producciones unitarias 
 \end{small}

  \[
    Unit(A):=Unit(A)\cup\{B\in V \mid \exists\;C\imp B,\;\;C \in
    Unit(A)\;\}
  \]
 \item[] \textbf{Hasta} que no se a\~naden nuevas variables a $Unit(A)$.
\ei 

\noindent Para la eliminaci\'on de producciones unitarias en una gram\'atica 
se debe realizar la siguiente transformaci\'on, para cada variable se debe 
calcular su conjunto unitario y realizar las siguientes acciones:
\bi
 \item Para cada $B\in Unit(A)$ y cada producci\'on $A\imp w$ agregar
  la producci\'on $$ B\imp w $$
 \item Eliminar todas las producciones unitarias
\ei 

\paragraph{Ejemplo:}
Eliminaci\'on de producciones unitarias de la gram\'atica:
\[
 \ba{rl}
  S \imp & AS \mid AA \mid BA \mid \cv \\
  A\imp & aA \mid a \\
  B\imp & bB \mid bC \mid C \\
  C\imp & aA \mid bA \mid B \mid ab \\
 \ea
\]
Los conjuntos unitarios para cada variable son:
\[
  \ba{cc}
   Unit(S)=\{S\} & \qquad \qquad Unit(A)=\{A\}\\
   Unit(B)=\{B,C\} & \qquad \qquad  Unit(C)=\{C,B\}
  \ea
\]
As\'i la gram\'atica obtenida al eliminar producciones unitarias es:
\[
 \ba{rl}
  S \imp & AS \mid AA \mid BA \mid \cv \\
  A\imp & aA \mid a \\
  B\imp & bB \mid bC \mid aA \mid bA \mid ab \\
  C\imp & aA \mid bA \mid ab \mid bB \mid bC \\
 \ea
\]

\subsection{Remplazo simple de producciones}
Esta transformaci\'on eliminar\'a reglas de la forma $A \imp u B v$ donde 
$ B\imp w_1 \mid w_2 \mid \dots \mid w_n$.
Esto se logra al redefinir el conjunto de reglas $P$ en la gram\'atica~$G$ con 
\[ 
 P' = \big(P -\{A \imp u B v\}\big) \cup \{A \imp u w_1 v \mid u w_2 
  v\mid\dots u w_n v  \} 
\]

\paragraph{Ejemplo:}
Reemplazar la regla $D \imp ABF $ de la gram\'atica siguiente:
\[
 \ba{rl}
  S \imp & DA \mid aF \\
  D \imp & ABF \mid a\\
  A \imp & a\\
  B \imp & b \mid ba \\
  F \imp & AF \mid A
 \ea
\]
Se obtiene:
\[
 \ba{rl}
  S \imp & DA \mid aF \\
  D \imp & AbF \mid AbaF \mid a\\
  A \imp & a\\
  B \imp & b \mid ba \\
  F \imp & AF \mid A
 \ea
\]


\subsection{Remoci\'on de producciones recursivas por la izquierda}
Una producci\'on recursiva por la izquierda es de la forma $A \imp A w$ con 
$w\in (V\cup T)^*$.\\
Ellas ser\'an reemplazadas por producciones cuya recursi\'on es por la 
derecha de la siguiente forma, 
para cada variable o s\'imbolo no-terminal $A$ crear dos categor\'ias de reglas:
\be
 \item $ A_{izq} = \{ A\imp A u_i \in P \mid u_i \in (V\cup T)^* \}$
 \item $ A_{der} = \{ A\imp v_j \in P \mid v_j \in (V\cup T)^* \}$
\ee
A continuaci\'on se obtendr\'an las reglas para la variable $A$ como sigue:
$$ A\imp v_1 \mid \dots \mid v_m \mid v_1 Z \mid \dots \mid v_n Z$$
Adem\'as se incorporar\'an las siguientes reglas a $P$:
$$ Z \imp u_1 Z \mid \dots \mid u_n Z \mid u_1 \mid \dots \mid u_n$$


\paragraph{Ejemplo:}
Considere las siguientes producciones:
$$ A \imp Aa \mid Aab \mid bb \mid b$$
Se eliminar\'an las producciones recursivas por la izquierda:
$$A_{izq}=\{A\imp Aa,\;A\imp Aab\}\qquad\qquad A_{der}=\{A\imp bb,\;A\imp b\}$$
generando la siguiente gram\'atica:
\[
 \ba{rl}
 A \imp &  bb \mid b\mid bbZ \mid bZ\\
 Z \imp & aZ \mid abZ \mid a \mid ab
 \ea
\]

\subsection{Problema de la pertenencia}
\noindent Dada una gram\'atica~$G$ \textit{sin $\cv$-producciones ni 
producciones unitarias} y una palabra~$w$ de longitud~$n$, 
?`se cumple que~$w\in L(G)$ ?
% Es decir, pertenece $w$ al lenguaje generado por~$G$.\\
Analicemos esta situaci\'on:
\bi
 \item Si la palabra $w$ es generada por $G$ la construcci\'on de un \'arbol 
  de derivaci\'on terminar\'a eventualmente.
 \item En caso contrario basta con construir un \'arbol hasta el nivel 
  $\mathbf{2n-1}$ para concluir que~$w\notin L(G)$, es decir un \'arbol que 
  pueda tener como hojas a la cadena~$w$.
 \item En cada paso de la construcci\'on de un \'arbol se obtiene un nuevo 
  terminal (a lo m\'as $n$ pasos) o se aumenta la longitud de la palabra en 
  $1$ (a lo m\'as $n-1$ pasos)
\ei


\section{Formas Normales}

Despu\'es de haber introducido varias transformaciones de gram\'aticas libres 
de contexto, veamos ahora dos formas normales de gran utilidad.

\subsection{Forma Normal de Chomsky}
\defin{
Una gram\'atica libre de contexto~$G$ est\'a en \textbf{forma normal de 
Chomsky (FNC)}~si:
\bi
 \item $G$ no contiene variables in\'utiles.
 \item $G$ no contiene producciones unitarias ni $\cv$-producciones (salvo 
  $S\imp\vacia$)
 \item Todas las producciones son binarias con variables o terminales, es decir 
  de la forma:
  \[ 
   A\imp BC\quad\text{ \'o }\quad A\imp a\qquad\text{ donde }B,C\in V,\,a\in T
  \]
\ei
}
 
\noindent Cualquier gram\'atica libre de contexto es equivalente a una 
gram\'atica en \textbf{FNC}, lo cual se logra parcialmente como sigue:
\be
 \item Eliminar las variables in\'utiles.
 \item Eliminar las $\cv$-producciones, salvo cuando la cadena 
  $\vacia$ pertenece al lenguaje original $L(G)$ y en ese caso se 
  agrega un nuevo s\'imbolo inicial $S'$ y $S' \imp\vacia \mid S$ .
 \item Eliminar producciones unitarias.
 \item Las producciones restantes son todas de la forma $A\imp a$ con 
  $a\in T$ \'o $A\imp w$ con $|w|\geq 2$      
\ee

El proceso restante es la eliminaci\'on de producciones $A\imp w$ donde 
$|w|\geq 2$. A este proceso le llamaremos \textbf{simulaci\'on de 
producciones}, para ello basta hacer lo siguiente para cada producci\'on $P$ 
de la forma $A\imp \alpha_1\alpha_2\ldots\alpha_n$ con $\alpha_i\in V\cup T$ y
$n\geq 2$ (obs\'ervese que si $n=2$, al menos uno de $\alpha_1,\alpha_2$ debe 
ser terminal, pues si no la producci\'on ya es v\'alida para FNC)
\bi
 \item Si $\alpha_i\in T$, digamos $\alpha_i=a$, entonces
  \be
   \item Agregar la producci\'on $T_a\imp a$, donde $T_a$ es una nueva 
  variable.
   \item Cambiar $\alpha_i$ por $T_a$ en la producci\'on $P$ 
    %$A\imp\alpha_1\alpha_2\ldots\alpha_n$
  \ee    
 \item Para cada producci\'on $P$ de la forma $A\imp B_1B_2\ldots B_m$
   con $B_i\in V,\;m\geq 3$
   \be
    \item Agregar $(m-2)$ nuevas variables $D_1,\ldots, D_{m-2}$ y reemplazar 
      a  $P$ %$A\imp B_1\ldots B_m$ 
      con las siguientes producciones:
    \[
      A\imp B_1D_1 \quad D_1\imp B_2D_2\quad \ldots\quad D_{m-2}\imp B_{m-1}B_m
    \]
   \ee
\ei

\paragraph{Ejemplo:}
Simulaci\'on de producciones $A\imp w_1w_2\ldots w_n,\;n\geq 2$.\\
La producci\'on $A\imp abBaC$ se simula con producciones simples y binarias 
como sigue:
\bi
 \item Agregamos las nuevas variables $T_a,T_b$ y las producciones
  \[
  A\imp T_aT_bBT_aC \qquad T_a\imp a \qquad T_b\imp b
  \]
 \item Para simular la producci\'on $A\imp T_aT_bBT_aC$ agregamos
   nuevas variables $D_1,D_2,D_3$ y las producciones binarias
   necesarias obteniendo finalmente la gram\'atica:
   \[
    A\imp T_aD_1 \qquad D_1\imp T_bD_2 \qquad D_2\imp BD_3\qquad D_3\imp T_aC
   \]
   \[
    T_a\imp a \qquad T_b\imp b
  \]
\ei


\paragraph{Ejemplo:}
Transformar la siguiente gram\'atica a \textbf{FNC}:
\[
  \ba{rlcrl}
  S \imp & AB \mid aBC \mid SBS &\qquad \qquad &
  A\imp & aA \mid C \\ 
  B\imp & bbB \mid b &\qquad \qquad &
  C\imp & cC \mid \vacia 
  \ea
\]
La gram\'atica equivalente es:
\[  
  \ba{rl}
  S \imp & AB \mid T_aD_1 \mid SD_2 \mid T_aB \mid T_bD_3 \mid b \\
  A\imp & T_aA \mid T_cC \mid a \mid c \\
  B\imp & T_bD_3 \mid b \\
  C\imp & T_cC \mid c 
  \ea
  \hspace*{20pt}
  \ba{rlcrl}
  D_1\imp & BC &\qquad \qquad &  T_a\imp & a \\
  D_2\imp & BS &\qquad \qquad &  T_b\imp & b \\
  D_3\imp & T_bB &\qquad \qquad &  T_c\imp & c 
 \ea
\]


\subsection{Forma Normal de Greibach}
Otra forma normal es la llamada de Greibach, esta forma normal es \'util en 
algoritmos de parsing ya que se da prioridad a los prefijos terminales de las 
cadenas generadas por las reglas de producci\'on de la gram\'atica.
Esto asegura que no existe recursi\'on por la izquierda de ninguna forma. 

\defin{
Una gram\'atica libre de contexto~$G$ est\'a en \textbf{forma normal de 
Greibach (FNG)}~si:
\bi
 \item La variable inicial $S$ no es recursiva, es decir, no figura en
  el lado derecho de las producciones.
 \item $G$ no tiene variables in\'utiles ni $\vacia$-producciones salvo 
  $S\imp\vacia$.
 \item Todas las producciones son de la forma $A\imp a\alpha$ con $a\in T$ y
  $\alpha \in V^\star$.
\ei
}

Veamos c\'omo transformar una gram\'atica en una equivalente en forma normal de 
Greibach, obs\'ervese que cualquier gram\'atica libre de contexto es 
equivalente a una gram\'atica en \textbf{FNG}.

\vspace*{10pt}

El proceso requiere primero que la gram\'atica est\'e en Forma Normal de 
Chomsky, de esta forma se eliminan las producciones in\'utiles y las 
$\vacia$-producciones.\\
Despu\'es, se aplica el siguiente m\'etodo para transformar a \textbf{FNG}:
\be
 \item Verificar que el s\'imbolo inicial no sea recursivo, si lo es entonces 
  agregar un nuevo s\'imbolo inicial con las producciones del anterior.
  
 \item Eliminar las producciones recursivas por izquierda.
 
 \item Enumerar u ordenar las variables o s\'imbolos no-terminales de la 
  gram\'atica: 
  iniciar con el s\'imbolo inicial y el resto de las variables siguen 
  cualquier orden.
 
 \item Reemplazar las producciones que contengan variables de orden mayor en el 
  lado derecho de~$\imp$, es decir todas las reglas deben tener la forma:
  \bi
  \item $ A \imp  B w $ donde la variable $A$ debe estar por debajo de la 
    variable   $B$ en el orden propuesto.
  \item $ A\imp aw$ con $a\in T$
  \ei
\ee 


Finalmente la gram\'atica resultante est\'a en forma normal de Greibach, esta 
transformaci\'on permite simplificar el problema de la pertenencia:
dada una gram\'atica $G$ \textbf{en forma normal de Greibach} y una 
palabra $w$ de longitud $n$, ?`Se cumple $w\in L(G)$ ? Es decir, 
pertenece $w$ al lenguaje generado por $G$.
\bi
 \item Si $w$ es generada por $G$ la construcci\'on de un \'arbol 
  de derivaci\'on entonces debe terminar\'a eventualmente.
 \item En caso contrario basta con construir el \'arbol hasta el nivel 
  $\mathbf{n}$ para concluir que $w\notin L(G)$.
 \item En cada paso se obtiene un nuevo terminal (a lo m\'as~$n$ pasos) y por 
  lo general hay una menor ramificaci\'on.
\ei

\paragraph{Ejemplo:}
Transformar la siguiente gram\'atica a \textbf{FNG}.
$$ S \imp S a B \mid a B \qquad \qquad B \imp b B \mid \vacia$$
\be
 \item Eliminaci\'on de variables in\'utiles\\
 $$ Prod = \{ B \} \qquad \qquad Acc = \{S, B\}$$
 \item Eliminaci\'on de $\vacia$-producciones\\
 \[
  \ba{lrl}
   Anul = \{ B \}  & \qquad S \imp & SaB \mid aB \mid Sa \mid a\\
   & \qquad B \imp & bB \mid b
  \ea
 \]
 \item Eliminaci\'on de producciones unitarias\\
 $$ Unit(S) = \{S\} \qquad \qquad Unit(B) = \{B\} $$
 \item Simulaci\'on de producciones \\
 S\'imbolos terminales:\\
 \[
  \ba{rlcrl}
   S \imp & ST_a B \mid T_a B \mid ST_a \mid a & \qquad \qquad T_a \imp& a\\
   B \imp & T_b B \mid b & \qquad \qquad T_b\imp & b
  \ea
 \]
 Producciones binarias:\\
 \[
  \ba{rlcrl}
   S \imp & SD_1 \mid T_a B \mid ST_a \mid a & \qquad \qquad T_a \imp& a\\
   B \imp & T_b B \mid b & \qquad \qquad T_b\imp & b \\
   D_1 \imp & T_a B & &   
  \ea
 \]

 \item S\'imbolo inicial no-recursivo\\
 \[
  \ba{rlcrl}
   S' \imp & SD_1 \mid T_a B \mid ST_a \mid a & \qquad \qquad T_a \imp& a\\
   S \imp & SD_1 \mid T_a B \mid ST_a \mid a & \qquad \qquad T_b \imp& b\\
   B \imp & T_b B \mid b & & \\
   D_1 \imp & T_a B & &   
  \ea
 \]

 \item Eliminaci\'on de producciones recursivas por la izquierda\\
 \[
  \ba{ccc}
   S'_{izq} = \vacio & \hspace{20pt} & S_{izq} = \{S\imp SD_1,\; S\imp ST_a\} \\
   B_{izq} = \vacio & & {D_1}_{izq} = \vacio \\
   {T_b}_{izq} = \vacio & & {T_b}_{izq} = \vacio 
   \ea
 \]
 
 \[
  \ba{rlcrl}
   S' \imp & SD_1 \mid T_a B \mid ST_a \mid a & \qquad \qquad T_a \imp& a\\
   S \imp & SD_1 \mid T_a B \mid ST_a \mid a \mid T_a BZ & 
    \qquad \qquad T_b \imp& b\\
   B \imp & T_b B \mid b &  &  \\
   D_1 \imp & T_a B & &    \\
   Z \imp & D_1 Z \mid T_a Z \mid D_1\mid T_a
  \ea
 \]
 
 \item Reemplazo simple de producciones\\
 \begin{small}
 \[
  \ba{c|rl|cl}
   orden &  & &reemplazar\\
   0  & S' \imp & SD_1 \mid T_a B \mid ST_a \mid a & T_a 
    & \text{ ya que se debe empezar con un terminal } \\    
   1  & S \imp & SD_1 \mid T_a B \mid ST_a \mid a \mid T_a BZ & T_a 
    & \text{ ya que se debe empezar con un terminal } \\
   2 & B \imp & T_b B \mid b & T_b 
   & \text{ ya que se debe empezar con un terminal } \\
   3 & D_1 \imp & T_a B  & T_a 
   & \text{ ya que se debe empezar con un terminal } \\
   4 & Z \imp & D_1 Z \mid T_a Z \mid D_1\mid T_a & D_1 & 
   \text{ ya que el orden de } D_1 \text{ es menor que el de } Z\\
   &&&T_a & \text{ ya que se debe empezar con un terminal } \\
   5 & T_a \imp& a & --\\
   6 & T_b \imp& b & --\\
  \ea
 \]


 \[
  \ba{rlcrl}
   S' \imp & aBD_1 \mid aD_1 \mid aBZD_1 \mid aZD_1\mid aB \mid aBT_a \mid 
    aT_a \mid aBZT_a     \mid aZT_a \mid a \\
   S \imp & aB \mid a \mid aBZ \mid aZ \\
   B \imp & bB \mid b \\
   D_1 \imp & aB \\
   Z \imp & aBZ \mid aZ \mid aB\mid a \\
   T_a \imp& a \\
   T_b \imp& b
  \ea
 \]
  \end{small}
\ee

\end{document}
