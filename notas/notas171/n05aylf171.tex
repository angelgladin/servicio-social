\documentclass[letterpaper,11pt]{article}
\usepackage[includeheadfoot,margin=1.3in]{geometry}
\usepackage{anysize}

\usepackage[utf8]{inputenc}
% \usepackage[latin1]{inputenc}
\usepackage[english,spanish]{babel}
%\usepackage{lmodern}   % font shapes...
\usepackage[T1]{fontenc} % join the compound symbols as a single symbol

\usepackage{amssymb,amsmath}
\usepackage{mathrsfs}
\usepackage{epsfig}

\usepackage{fancyhdr}
\usepackage{hyperref}
\usepackage{url}

\usepackage{import}
\usepackage{comment}
\usepackage[autostyle=true,spanish=mexican]{csquotes}

\usepackage{alltt}
\usepackage[section]{placeins}

\usepackage{pgf}
\usepackage{tikz}
\usetikzlibrary{automata,arrows,trees}
\usetikzlibrary{babel}
%\pagestyle{fancyplain}
\input{../macrosAyLF}

\title{Aut\'omatas y Lenguajes Formales 2017-I \\ 
% Posgrado en Ciencia e Ingeniería de la Computación UNAM \\ 
Facultad de Ciencias UNAM \\ 
Nota de Clase 5} 
\author{Favio E. Miranda Perea \and A. Liliana Reyes Cabello \and
Lourdes Gonz\'alez Huesca}
\date{\today}

\begin{document}
\maketitle

Como hemos visto, la relaci\'on entre aut\'omatas finitos y lenguajes 
regulares parece de equivalencia, las nociones detr\'as de ambos conceptos la 
suguieren. En esta nota demostraremos que ese es el caso utilizando el m\'etodo 
propuesto por Kleene~\footnote{Stephen Cole Kleene fue un matem\'atico 
estadounidense, alumno de Alonzo Church. Tambi\'en es conocido por iniciar la 
teor\'ia de la recursi\'on que fue usada para los fundamentos de la Teor\'ia de 
la Computaci\'on como la noci\'on de computabilidad.} 


\section{Teorema de Kleene}

\teo{ Un lenguaje es regular si y s\'olo si es aceptado por un aut\'omata 
finito.
}
\vspace*{-20pt}
\begin{proof}
 La prueba es en dos partes:
 \bi
  \item[I] S\'intesis: Dado un lenguaje regular~$L$, existe un aut\'omata 
    finito~$M$ tal que $L=L(M)$.
  \item[II] An\'alisis: Dado un aut\'omata finito~$M$, existe una      
    expresi\'on regular~$\alpha$ tal que $L(M)=L(\alpha)$. 
    Es decir, $L(M)$ es regular.
 \ei
\end{proof}

A continuaci\'on se abordar\'an dos teoremas que demuestran al anterior. 
Para ello recordemos que las expresiones regulares est\'an en correspondencia 
con los lenguajes regulares y de esta forma podremos empatar tambi\'en a las 
expresiones regulares con los aut\'omatas finitos. 


\subsection{Teorema de S\'intesis de Kleene}

En esta secci\'on se demostrar\'a una parte de la doble implicaci\'on del 
teorema de Kleene: se pasar\'a de expresiones regulares a aut\'omatas finitos.

Este teorema se denomina de s\'intesis ya que se proporcionar\'a una m\'aquina 
para reconocer un lenguaje regular dado. 
Es decir, se sintetizar\'a un aut\'omata finito analizando la forma de una 
expresi\'on regular que genera al lenguaje dado.

\teo{
Dada una expresi\'on regular~$\alpha$ existe un aut\'omata 
finito~$M$ tal que $L(\alpha) = L(M)$.
}
\vspace*{-20pt}
\begin{proof}
La demostraci\'on es \textit{constructiva} y se har\'a mediante 
inducci\'on sobre las expresiones regulares, es decir proporcionando un 
aut\'omata que \textit{reconzca} cada caso de una expresi\'on regular. 
\begin{description}
\item{\textbf{Base de la Inducción}} \hfill\\
El caso en que $\alpha = \vacio$, el siguiente autómata reconoce a $L(\alpha)$: 
\hspace*{10pt}
\begin{tikzpicture}[node distance=2cm,every node/.style={scale=0.8},semithick]
  \node[state,initial,initial text=] (q) {$q$};
\end{tikzpicture}

Caso en que $\alpha=\vacia$. Entonces el siguiente aut\'omata reconoce a 
$L(\alpha)$: 
\hspace*{10pt}
\begin{tikzpicture}[node distance=2cm,every node/.style={scale=0.8},semithick]
  \node[state,initial,initial text=,accepting by double] (q) {$q$};
\end{tikzpicture}

Para $\alpha = a$, con $a\in \Sigma$ se tiene el siguiente aut\'omata que 
reconoce a $L(\alpha)$:
\hspace*{7pt}
\begin{tikzpicture}[node distance=2cm,every node/.style={scale=0.8},semithick]
   \node[state,initial,initial text=,] (q) {$q$};
   \node[state,accepting by double] (p) [right of=q] {$p$};
   \path[->] (q) edge [bend left] node [above] {a} (p);
\end{tikzpicture}    

\item{\textbf{Hip\'otesis de inducci\'on}}\\\hfill
  Sean $M_1,\,M_2$ dos aut\'omatas que reconocen los lenguajes $L(\alpha_1)$ y 
  $L(\alpha_2)$ respectivamente.

\item{\textbf{Paso Inductivo}} \\\hfill
 Caso en que $\alpha=\alpha_1+\alpha_2$.
 El siguiente aut\'omata reconoce a $L(\alpha)$:
 \hspace{10pt}
 \begin{tikzpicture}[node distance=2cm,every node/.style={scale=0.8},semithick]
   \node[state,initial,initial text=] (q) {$q$};
   \node[state] (p1) [rectangle,above right of=q] {$M_1$};
   \node[state] (p2) [rectangle,below right of=q] {$M_2$};
   \path[->] (q) edge [bend left] node [above] {$\vacia$} (p1);
   \path[->] (q) edge [bend right] node [below] {$\vacia$} (p2);
 \end{tikzpicture}
 
donde $M_1,\,M_2$ son aut\'omatas dados por la hip\'otesis de inducci\'on y las 
transiciones $\vacia$ van hacia los estados iniciales de cada 
uno de ellos. 



Para el caso en que $\alpha=\alpha_1\alpha_2$ entonces el siguiente aut\'omata 
reconoce a $L(\alpha)$: 
\begin{center}
\begin{tikzpicture}[node distance=3cm,every node/.style={scale=0.8},semithick]
   \node[state,initial,initial text=] (p1) [rectangle] {$M_1$};
   \node[state] (p2) [rectangle,right of=p1] {$M_2$};
   \path[->] (p1) edge [bend left] node [above] {$\vacia$} (p2);
   \path[->] (p1) edge [left] node [above] {$\vdots$} (p2);
   \path[->] (p1) edge [bend right] node [above] {$\vacia$} (p2);
\end{tikzpicture}    
\end{center}
\noindent donde $M_1,\,M_2$ son aut\'omatas que reconocen a 
$L(\alpha_1),L(\alpha_2)$ dados por la hip\'otesis de inducci\'on, el estado 
inicial es el de $M_1$ y las transiciones $\vacia$ van de los estados finales de 
$M_1$ hacia el inicial en $M_2$.

Finalmente el caso $\alpha=\alpha_1^\star$ tiene al siguiente aut\'omata que 
reconoce a $L(\alpha)$:
\begin{center}
\begin{tikzpicture}[node distance=3cm,every node/.style={scale=0.8},semithick]
   \node[state,initial,initial text=,accepting by double] (q) {$q$};
   \node[state] (p1) [rectangle,right of=q] {$M_1$};
   \path[->] (q) edge [bend left] node [above] {$\vacia$} (p1);
   \path[->] (p1) edge [bend left] node [above] {$\vacia$} (q);
\end{tikzpicture}   
\end{center}
\noindent donde $M_1$ es un aut\'omata que reconoce a $L(\alpha_1)$ dado por la 
hip\'otesis de inducci\'on y las transiciones $\vacia$ conectan el estado 
inicial y los finales con el nuevo estado $q$.
\end{description}
\end{proof}

\subsection{Teorema de An\'alisis de Kleene}

Ahora demostraremos la segunda parte en donde un aut\'omata finito implica  
una expresi\'on regular, la noci\'on detr\'as de esta parte es que analizaremos 
los lenguajes acumulados de cada estado para generar una expresi\'on regular.

\teo{
Dado un aut\'omata finito~$M$ existe una expresi\'on regular~$\alpha$ tal que 
$L(M)=L(\al)$. Es decir, $L(M)$ es regular.
}
\vspace*{-20pt}
\begin{proof}
Existen diversas demostraciones, nosotros usaremos el m\'etodo de ecuaciones 
caracter\'isticas usando el Lema de Arden.
\end{proof}

\subsubsection{Lema de Arden}
Este lema extrae un conjunto de ecuaciones para determinar el lenguaje de 
aceptaci\'on de una m\'aquina.
Primero veamos la definic\'on de dichas ecuaciones y despu\'es el m\'etodo para 
obtener las ecuaciones dado un aut\'omata. 

\defin{
Sean $A,B\inc \sest$ y $X$ una variable:
\bi
 \item Una ecuaci\'on lineal derecha para $X$ es una expresi\'on de la forma:
  $$X = AX+B$$
 \item An\'alogamente, una ecuaci\'on lineal izquierda es una expresi\'on de 
  la forma:
  $$X = XA+B $$
  Donde el s\'imbolo $+$ denota a la uni\'on de lenguajes.
\ei
}

\lema{\textbf{\!\!: Lema de Arden}
Sean $A,B\inc\sest$ dos lenguajes y $X=AX+B$ una ecuaci\'on lineal derecha. 
Entonces
\be
 \item $A^\star B$ es una soluci\'on de la ecuaci\'on, es decir, 
  $A^\star B=A(A^\star B)+B$.
 \item Si $C$ es otra soluci\'on entonces $A^\star B\inc C$, es decir, 
  $A^\star B$ es la soluci\'on m\'inima.
 \item Si $\cv\notin A$ entonces $A^\star B$ es la \'unica soluci\'on.
\ee 
}

Esta parte mostrar\'a que un aut\'omata finito implica una expresi\'on regular, 
esto se har\'a por medio de un sistema de ecuaciones a partir de un 
AFN, para ello se abstraer\'a la noci\'on del lenguaje aceptado desde un estado 
particular del aut\'omata y no necesariamente del inicial.

\defin{
Dado un AFN~$M=\pt{Q,\Sigma,\delta,q_0,F}$ tal que $Q=\{q_0,\ldots, q_n\}$.
Definimos los siguientes conjuntos:
\bi
 \item El conjunto de cadenas que se aceptan desde el estado $q_i$,
   para cualquier $1\leq i\leq n$:
   $$ L_i=\{w\in\sest \mid \dest(q_i,w)\cap F\neq\vacio\}$$
 \item $L_0$ es el lenguaje aceptado por $M$, es decir, $L_0=L(M)$.
\ei
}

En general no es sencillo calcular directamente los conjuntos $L_i$. Para 
obtener una expresi\'on regular completa respecto al aut\'omata, se 
obtendr\'a un sistema de ecuaciones a partir de un AFN. Al resolverlo, $L_0$ 
ser\'a el lenguaje que reconoce el aut\'omata como se mencion\'o arriba. 

\noindent El sistema de ecuaciones se define usando:
\be
 \item El conjunto de s\'imbolos de $\Sigma$ tal que existe una transici\'on 
  del estado $q_i$ al estado $q_j$, para cualesquiera $1\leq i,j\leq n$, con 
  $n$ el total de estados:
      $$ X_{i,j}=\{a\in\Sigma \mid q_j\in\delta(q_i,a)\}$$
 \item El conjunto auxiliar $Y_i$ que indica si $\vacia$ es aceptada desde $q_i$
  \[
   Y_i = \left\{\ba{cl}
          \{\vacia\} & \mbox{si}\;q_i\in F \\
          \vacio & \mbox{en otro caso}
         \ea
         \right.
   \]
\ee
  
Por tanto, las ecuaciones de los lenguajes de cada estado est\'an dadas 
por la siguiente propiedad que es fácil de demostrar para cualquier 
$1\leq i\leq n$:
$$ L_i=\sum_{j=0}^n X_{i,j}L_j + Y_i$$
\noindent Dicha propiedad genera el llamado sistema de ecuaciones 
caracter\'isticas 
de un AFN.

\vspace*{20pt}

Finalmente, el Lema de Arden nos indica c\'omo calcular los conjuntos $L_i$. 
Esta ser\'a la idea a seguir para demostrar el Teorema de Análisis de Kleene:
\be
 \item Dado el aut\'omata~$M$ construir los conjuntos $ X_{i,j},\;Y_i $.
 \item Resolver el sistema de ecuaciones caracter\'isticas mediante el Lema 
  de Arden.
 \item La soluci\'on para $L_0$ genera una expresi\'on regular para $L(M)$.
\ee

\subsection{Minimizaci\'on de aut\'omatas}

Este proceso constructivo nos lleva a tener m\'aquinas demasiado grandes, con 
$\vacia$-transiciones. 
Por lo cual presentamos a continuaci\'on m\'etodos para reducir el tama\~no 
de aut\'omatas. Estudiaremos la forma en que se pueden minimizar los 
aut\'omatas para obtener m\'aquinas m\'as eficientes.

\subsubsection{Eliminaci\'on de Estados Inaccesibles}
Sea $M=\pt{Q,\Sigma,\delta,q_0, F}$ un AFD. Decimos que un estado $q\in Q$ es 
accesible si y sólo si existe $w\in\sest$ tal que  $\dest(q_0,w)=q$. 
Es decir, $q$ es accesible si y s\'olo si el procesamiento de alguna cadena 
termina en el estado $q$.

El conjunto de estados accesibles de un aut\'omata $M$ se denota   $Acc(M)$.
Si un estado no es accesible decimos que es inaccesible.

\vspace{10pt}
Es claro que el conjunto $Acc(M)$ puede construirse de manera algor\'itmica, 
por ejemplo como sigue:
\begin{alltt}
 \(A_N\) := \(\{q_0\}\)  \% estados accesibles
 \(A_V\) := \(\vacio\)     \% estados verificados
 while \( A_N \neq A_V\) do
   \(A_V\) := \(A_N\)
   \(A_N\) := \(A_N\cup\{q \in Q \mid \delta(p,a)=q,\,a\in\Sigma,\;p\in\!A_N\}\)
 return \(A_N\)
\end{alltt}
Los estados inaccesibles en un aut\'omata son in\'utiles y pueden ser
eliminados sin afectar el lenguaje de aceptaci\'on como vemos a continuaci\'on:

\prop{
Dado $M=\pt{Q,\Sigma,\delta,q_0, F}$ un AFN, existe un AFD 
$M'=\pt{Q',\Sigma,\delta',q_0, F'}$ equivalente a $M$ que contiene 
\'unicamente a los estados accesibles de $M$, es decir, $Q'=Acc(M)$ y por lo 
tanto no contiene estados inaccesibles. 
Para lo anterior basta definir $M'$ como sigue:
\bi
 \item $Q'=Acc(M)$
 \item $\delta' = \restr{\delta}{ _{Q'}}$
 \item $F'=F\cap Q'$
\ei
}
La prueba de la equivalencia $L(M)=L(M')$ es inmediata y se deja como ejercicio.
 
Debido a este resultado de ahora en adelante podemos suponer que un aut\'omata 
no tiene estados inaccesibles.

\subsubsection{Equivalencia de estados y el Aut\'omata cociente}
Puede ser el caso que unas partes de un aut\'omata sean redundantes, es decir 
que las cadenas que son aceptadas por una parte del aut\'omata tambi\'en pueden 
ser procesadas y aceptadas por otra parte.
Veamos c\'omo abstraer y generalizar partes de los aut\'omatas para minimizar 
el n\'umero de estados sin afectar al lenguaje de aceptaci\'on.

\defin{
Decimos que dos estados $q,q'\in Q$ de un AFD son equivalentes $q\equiv q'$
si y s\'olo si:
$$ \fa w\in\sest\;(\dest(q,w)\in F\Iff\dest(q',w)\in F)$$
Es decir, si $\dest(q,w),\dest(q',w)$ son ambos finales o ambos no finales.
}

La relación $\equiv$ entre estados es una relación de equivalencia, es decir 
cumple lo siguiente:
\bi
 \item Reflexividad: $q\equiv q$.
 \item Simetria: si $q\equiv q'$ entonces $q'\equiv q$.
 \item Transitividad: si $q\equiv q'$ y $q'\equiv q''$ entonces $q\equiv q''$.
\ei
Adicionalmente la funci\'on de transici\'on $\delta$ es compatible con
$\equiv$, en el siguiente sentido:
\begin{center}
 Si $q\equiv q'$ entonces 
 $\fa a\in\Sigma\;\big(\delta(q,a)\equiv\delta(q',a)\big)$
\end{center}

La relaci\'on de equivalencia $\equiv$ genera una \textbf{partici\'on} del
conjunto de estados dada por las clases de equivalencia de cada estado 
definidas como:
$$ [q] := \{p\in Q\;|\;q\equiv p\} $$
Es decir, los conjuntos de estados $[q]$ cumplen lo siguiente:
\bi
 \item $\fa q\in Q\;([q]\neq\vacio\;)$.
 \item $\fa p,q\in Q\;([q]=[p]$ ó $[q]\cap[p]=\vacio)$.
 \item $\bigcup_{q\in Q}[q]=Q$.
\ei

Al agrupar por clases de equivalencia a los estados de un aut\'omata, se puede 
calcular otro aut\'omata llamado \textbf{aut\'omata cociente} que tiene un 
n\'umero m\'inimo de estados. 

\defin{
Dado un AFD $M=\pt{Q,\Sigma,\delta,q_0,F}$ existe el aut\'omata 
cociente~$M/_{\equiv}$ tambi\'en conocido como $M^{min}$ que es la 
minimizaci\'on 
de $M$ y se define como $M^{min}=\pt{Q_m,\Sigma,\delta_m,[q_0],F_m}$ donde:
\bi
 \item $Q_m:=\{[q] \mid q\in Q\}$ los estados son las clases de 
  equivalencia
 \item la clase $[q_0]$ es el estado inicial.
 \item $F_m:=\{[q] \mid q\in F\}$
 \item $\delta_m:Q_m\times\Sigma\imp Q_m$ se define como         
  $\delta_m([q],a)=[\delta(q,a)]$
\ei
}

La definici\'on anterior indica que dado un AFD 
$M=\pt{Q,\Sigma,\delta, q_0, F}$ el aut\'omata cociente $M/_{\equiv}$ es el 
aut\'omata m\'inimo equivalente a $M$. 
Es decir, se tiene $L(M)=L(M/_{\equiv})$ y no existe un aut\'omata 
equivalente a~$M$ con menos estados que $M/_{\equiv}$.
La equivalencia entre $M$ y $M^{min}$ se sigue de la siguiente propiedad:
\lema{
Sean $M=\pt{Q,\Sigma,\delta,q_0,F}$ un AFD y $M^{min}$ su aut\'omata cociente. 
Para cualesquiera $q\in Q,\;w\in\sest$ se cumple    
$\delta^\star_m([q],w)=[\dest(q,w)]$
}
\vspace*{-20pt}
\begin{proof}
 La prueba es por inducción sobre $w$.
\end{proof}

\paragraph{\texorpdfstring{$k$}--equivalencia}
La demostraci\'on anterior requiere de una relaci\'on de equivalencia que 
depende de la longitud de una cadena. \\
Definimos la relación de $k$-equivalencia para cualquier $k\in\N$ como sigue:
$$ \fa w\in\sest,|w|\leq k \imp (\dest(q,w)\in F\Iff\dest(q',w)\in F)$$
Es decir, para cualquier cadena $w$ de longitud menor o igual que~$k$, los 
estados $\dest(q,w)$ y $\dest(q',w)$ son ambos finales o ambos no finales.\\
As\'i $\equiv_k$ es una relaci\'on de equivalencia cuyas clases se denotan
con $[q]_k$, es decir $$[q]_k=\{p\in Q\;|\;q\equiv_k p\}$$

\noindent La relaci\'on de $k$-equivalencia cumple las siguientes 
propiedades:
\bi
 \item[P1] $q\equiv q'$ si y s\'olo si $\fa k\in\N(q\equiv_k q')$.
 \item[P2] $q\equiv_0 q'$ si y s\'olo si $q,q'\in F$ ó $q,q'\in Q-F$.
 \item[P3] $[q]_0=F$ si y s\'olo si $q\in F$.
 \item[P4] Si $q\equiv_k q'$ entonces $q\equiv_{k-1} q'$.
 \item[P5] $[q]_k\inc[q]_{k-1}$
 \item[P6] Si $q\equiv_kq'$ entonces 
  $\fa a\in\Sigma(\delta(q,a)\equiv_{k-1}\delta(q',a))$
 \item[P7]$q\equiv_k q'$ si y s\'olo si $q\equiv_{k-1}q'$ y
    $\fa a\in\Sigma(\delta(q,a)\equiv_{k-1}\delta(q',a))$
 \item[P8] Sea $P_k=\{[q]_k\;|\;q\in Q\}$ la partici\'on dada por la      
  relaci\'on $\equiv_k$ para cualquier $k\in\N$.\\ 
  Si $P_k=P_{k-1}$ para alguna $k$ entonces $P_k=P_m$ para toda $m\geq k$.
\ei

Con las definiciones anteriores podemos construir el aut\'omata m\'inimo 
equivalente:

\defin{
Dado un AFD $M=\pt{Q,\Sigma,\delta,q_0,F}$ el AFD m\'inimo asociado puede 
construirse como sigue:
\begin{alltt}
 \(Q:= Acc(M)\qquad \) \% estados accesibles
 
 \(P_0:=\{F,Q-F\}\;\) \% construir la particion inicial:
               estados finales y no-finales 
               
 \( k:=0 \)
 
 repeat \{
 
 \(\qquad k:=k+1\)
   
 \(\qquad 
 P_k:= \{ q\in P_{k-1} \mid \forall a\in\Sigma,[\delta(q,a)]=[\delta(q',a)] \}\)
  
 \}
 until \(P_k=P_{k-1}\)
 
 return \(P_k\)
\end{alltt}
La partici\'on $P_k$ se construir\'a a partir de $P_{k-1}$ manteniendo a dos     
estados $q,q'$ en la misma clase si y s\'olo si para toda $a\in\Sigma$, los 
estados $\delta(q,a)$ y $\delta(q',a)$ estaban en la  misma clase en 
$P_{k-1}$.\\
Es decir que $P_k$ es la partición generada por $\equiv$:
$P_k= Q/_{\equiv}=\{[q] \mid q\in Q\}$.
}

La definci\'on anterior est\'a dada a trav\'es de un algoritmo el cual es 
correcto respecto a la especificaci\'on, ya que es consecuencia de la siguiente 
propiedad: 
\prop{
 Si $M$ es un AFD entonces la sucesi\'on de particiones 
 $P_0,P_1,\ldots, P_k $ generadas por las clases de $k$-equivalencia de 
 estados se estaciona, es decir existe un $n\in \N$ tal que para toda 
 $k\geq n$ se tiene que $P_k=P_n$. 
 Más aún $n\leq |Q|$, es decir $n$ es a lo más el número de estados de $M$.
}

\section{Ejemplo}
Sea $M$ el siguiente aut\'omata finito no-determin\'istico con transiciones 
$\vacia$:
\begin{center}
 \begin{tikzpicture}[node distance=3cm,every node/.style={scale=0.8},semithick]
    \node[state,initial,initial text=,accepting by double] (q0) {$q_0$};
    \node[state] (q1) [right of=q0] {$q_1$};
    \node[state] (q3) [below of=q1] {$q_3$};
    \node[state] (q2) [right of=q1] {$q_2$};
    \path[->] (q0) edge node [above] {$\vacia$} (q1);
    \path[->] (q1) edge [bend right] node [above] {a} (q2);
    \path[->] (q2) edge [bend right] node [above] {b} (q1);
    \path[->] (q1) edge node [right] {b} (q3);
    \path[->] (q3) edge node [above] {$\vacia$} (q0);
 \end{tikzpicture}
\end{center}
Usando los m\'etodos descritos antes se obtendr\'a un aut\'omata m\'inimo 
determinista de la siguiente forma:
\newpage
\be
\item Se eliminan las $\vacia$-transiciones calculando los conjuntos 
$Cl_{\vacia}$ de cada estado:
\begin{figure}[!ht]
 \centering
 \begin{small}
 \begin{minipage}{.5\textwidth}
  \centering
  \[
 \begin{array}{c|ccc|c|c }
  Q & Cl_{\vacia} & \hspace{30pt}& Q & a & b \\\hline
  q_0 & \{q_0, q_1\} && q_0 & \{q_2\} & \{q_3,q_0,q_1\} \\
  q_1 & \{q_1\} && q_1 & \{q_2\} & \{q_3,q_0,q_1\} \\
  q_2 & \{q_2\} && q_2 & \vacio & \{q_1\}\\
  q_3 & \{q_3,q_0,q_1\} && q_3 & \{q_2\} & \{q_3,q_0,q_1\}\\
 \end{array}
\]
 \end{minipage}
 \end{small}
\begin{minipage}{.4\textwidth}
 \centering
 \begin{tikzpicture}[node distance=3cm,every node/.style={scale=0.8},semithick]
    \node[state,initial,initial text=,accepting by double] (q0) {$q_0$};
    \node[state] (q1) [right of=q0] {$q_1$};
    \node[state] (q3) [below of=q1] {$q_3$};
    \node[state] (q2) [right of=q1] {$q_2$};
    \path[->] (q0) edge [loop above] node [above] {b} (q0);
    \path[->] (q0) edge [bend left] node [above] {b} (q1);
    \path[->] (q0) edge [bend right=45] node [above] {b} (q3);
    \path[->] (q0) edge [bend left=55] node [above] {a} (q2);
    \path[->] (q1) edge [bend right] node [above] {a} (q2);
    \path[->] (q1) edge [bend left] node [above] {b} (q0);
    \path[->] (q1) edge [loop above] node [above] {b} (q1);
    \path[->] (q1) edge [bend right] node [right] {b} (q3);
    \path[->] (q2) edge [bend right] node [above] {b} (q1);
    \path[->] (q3) edge [bend right=45] node [right] {a} (q2);
    \path[->] (q3) edge node [above] {b} (q0);
    \path[->] (q3) edge [loop below] node [above] {b} (q3);
    \path[->] (q3) edge [bend right] node [right] {b} (q1);
 \end{tikzpicture}
\end{minipage}
\end{figure}\FloatBarrier

\item Se transforma el aut\'omata anterior en determinista:
\begin{figure}[!ht]
 \centering
 \begin{small}
 \begin{minipage}{.4\textwidth}
  \centering
  \begin{tikzpicture}
  \tikzstyle{level 1}=[sibling distance=30mm]
  \tikzstyle{level 2}=[sibling distance=15mm]
 
  \node {$\{q_0\}$}
    child {node {$\{q_2\}$}
      child {node {$\vacio$}}
      child {node {$\{q_1\}$}
	child {node {$\{q_2\}$}
	}
	child {node {$\{q_1,q_0,q_3\}$}}
% 	  child {node {$\{q_2\}$}}
% 	  child {node {$\{q_1,q_0,q_3\}$}}
% 	}
      }
     }
      child {node {$\{q_0,q_1,q_3\}$}
        child {node {$\{q_2\}$}}
        child {node {$\{q_0,q_1,q_3\}$}}
    };
 \end{tikzpicture}
 \end{minipage}
 \end{small}
\begin{minipage}{.5\textwidth}
 \centering
\begin{tikzpicture}[node distance=3.5cm,every node/.style={scale=0.8},semithick]
    \node[state,initial,initial text=,accepting by double] (q0) {$q_0$};
    \node[state] (q2) [right of=q0] {$q_2$};
    \node[state] (q1) [right of=q2] {$q_1$};
%     \node[state,accepting by double] (q01) [below of=q1] {$q_{01}$};
    \node[state,accepting by double] (q013) [below of=q2] {$q_{013}$};
    \path[->] (q0) edge node [above] {b} (q013);
    \path[->] (q0) edge node [above] {a} (q2);
    \path[->] (q1) edge [bend right] node [above] {a} (q2);
    \path[->] (q1) edge node [right] {b} (q013);
    \path[->] (q2) edge [bend right] node [above] {b} (q1);
%     \path[->] (q01) edge  node [right] {a} (q2);
%     \path[->] (q01) edge node [above] {b} (q013);
    \path[->] (q013) edge [loop below] node [below] {b} (q013);
    \path[->] (q013) edge [bend right] node [right] {a} (q2);
 \end{tikzpicture}
 \end{minipage}
\end{figure}
\FloatBarrier
 

\item Finalmente se minimiza:
\begin{figure}[!ht]
 \centering
 \begin{small}
 \begin{minipage}{.4\textwidth}
  \centering
\[
 \begin{array}{l|cc|cc}
  & A && \hspace{20pt}B & \\\hline
  & \{q_0\} &  \{q_0,q_1,q_3\} & \{q_2\} & \{q_1\}\\
  a & B  & B & -- & B \\
  b & A & A & B & A \\
\end{array}
\]
\[
\begin{array}{l|cc|c|c}
  && A & B & C\\\hline
  & \{q_0\} & \{q_0,q_1,q_3\} & \{q_2\} & \{q_1\}\\
  a & B & B & -- & B \\
  b & A & A & C & A 
 \end{array}
\]
 \end{minipage}
 \end{small}
\begin{minipage}{.5\textwidth}
 \centering
\begin{tikzpicture}[node distance=3.5cm,every node/.style={scale=0.8},semithick]
    \node[state,initial,initial text=,accepting by double] (qA) {$q_A$};
    \node[state] (qB) [right of=qA] {$q_B$};
    \node[state] (qC) [right of=qB] {$q_C$};
    \path[->] (qA) edge [loop above] node [above] {b} (qA);
    \path[->] (qA) edge [bend left] node [above] {a} (qB);
    \path[->] (qB) edge [bend left] node [above] {b} (qC);
    \path[->] (qC) edge [bend left] node [above] {a} (qB);
    \path[->] (qC) edge [bend left] node [above] {b} (qA);
 \end{tikzpicture}
 \end{minipage}
\end{figure}
\FloatBarrier
\ee

\end{document}
