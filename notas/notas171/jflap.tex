
\documentclass[letterpaper,11pt]{article}
\usepackage[includeheadfoot,margin=1.3in]{geometry}
\usepackage{anysize}

\usepackage[utf8]{inputenc}
% \usepackage[latin1]{inputenc}
\usepackage[english,spanish]{babel}
\usepackage{lmodern}   % font shapes...
\usepackage[T1]{fontenc} % join the compound symbols as a single symbol

\usepackage{amssymb,amsmath}
\usepackage{mathrsfs}
\usepackage{epsfig}

\usepackage{fancyhdr}
\usepackage{hyperref}
\usepackage{url}

\usepackage{import}
\usepackage{comment}
\usepackage[autostyle=true,spanish=mexican]{csquotes}

\usepackage{alltt}
\usepackage[section]{placeins}

\usepackage{url}

\usepackage{pgf}
\usepackage{tikz}
\usetikzlibrary{automata,arrows,trees}
\usetikzlibrary{babel}
%\pagestyle{fancyplain}
\input{../macrosAyLF}

\title{Aut\'omatas y Lenguajes Formales 2017-1 \\ 
% Posgrado en Ciencia e Ingeniería de la Computación UNAM \\ 
Facultad de Ciencias UNAM \\ 
JFLAP: una herramienta para experimentar con lenguajes formales} 
\author{Lourdes Gonz\'alez Huesca}
\date{\today}

\begin{document}
\maketitle

El paquete \textit{Java Formal Language and Automata Package} mejor conocido 
como JFLAP es un software desarrollado en la universidad de Duke bajo la 
coordinaci\'on de Susan Rodger.
Este software permite experimentar con lenguajes formales dados por diferentes 
tipos de gram\'aticas, aut\'omatas finitos, aut\'omatas de pila o m\'aquinas de 
Turing.
Ofrece una forma visual e interactiva de crear, modificar m\'aquinas y 
de procesar cadenas. 

Usaremos esta herramienta para practicar el dise\~no de m\'aquinas as\'i como 
sus transformaciones y procesamiento de cadenas en ellas.

\section{Instalaci\'on}
Para descargar el software se debe acceder a la p\'agina oficial 
\url{http://www.jflap.org/} y en la parte \textbf{Get JFLAP} se deber\'a llenar 
un formulario para poder usar el software. 

Se debe instalar la versi\'on 7.0 de 2009 que incluye el manejo de SVG.
Tambi\'en se debe asegurar la instalaci\'on de una versi\'on del
\textsc{Java Runtime Environment} o JRE compatible~\footnote{Para un sistema 
operativo basado en Debian (Ubuntu, LMint, etc.) se pueden seguir las 
instrucciones que aparecen en
\url{https://community.linuxmint.com/tutorial/view/1091}.}\\

Una vez que el archivo \texttt{JFLAP.jar} est\'a disponible s\'olo basta con 
hacer doble click para iniciar una sesi\'on o lanzar su ejecuci\'on desde la 
l\'inea de comandos con la siguiente:
\begin{center}
 \texttt{java -jar JFLAP.jar}
\end{center}

\section{Uso}
El sitio oficial provee un tutorial detallado para el uso de JFLAP, en esta 
nota introductoria s\'olo se dar\'a una muy breve explicaci\'on de su uso.
\begin{center}
\textbf{Se deja al alumno la experimentaci\'on y pr\'actica \\para dominar 
moderadamente este paquete.}
\end{center}

A pesar de que JFLAP es muy intuitivo e interactivo, se debe tener en cuenta 
la definici\'on formal de las m\'aquinas vistas en clase y de los m\'etodos de 
transformaci\'on para entender las herramientas del sistema adem\'as de los 
procesos en ella. 


El paquete incluye una interfaz gr\'afica o GUI (\textit{Graphical User 
Interface}) para la interacci\'on con el usuario, la ventana principal de 
\'este despliega un men\'u para seleccionar la m\'aquina deseada:
\begin{figure}[!ht]
\centering
\includegraphics{machineSelection.png}
\end{figure}
\FloatBarrier

Usando la ventana principal se pueden ajustar las preferencias como lo es usar 
la cadena $\vacia$ para denotar la cadena vac\'ia\footnote{El caracter para 
representar a la cadena vac\'ia por default en el sistema y usada por varios 
autores suele ser $\lambda$.}, selecccionando del men\'u \textbf{Preferences} 
la opci\'on \textbf{Set the Empty String Character}.

Una vez seleccionada la m\'aquina se procede a construirla dentro del canvas o 
\'area para ello, utilizando las facilidades de la barra de herramientas o 
toolbar. 

\begin{figure}[!ht]
\centering
\includegraphics{NewWindow.png}
\end{figure}
\FloatBarrier

\newpage

La barra de herramientas tiene los siguientes botones, para usarlos es 
necesario seleccionar el bot\'on:
\bi
 \item \includegraphics[scale=0.75]{state.jpeg} Creador de estados, basta con 
  pulsar para crear un estado.
 \item \includegraphics[scale=0.75]{transition.jpeg} Creador de transiciones, 
  se debe arrastrar el puntero desde el estado origen al destino para crear una 
  transici\'on, adem\'as de escribir el s\'imbolo.
 \item \includegraphics[scale=0.75]{delete.jpeg} Eliminador de elementos, basta 
  con pulsar sobre el elemento a borrar.
 \item \includegraphics[scale=0.75]{arrow.jpeg} Editor de transiciones, basta 
  con seleccionar la transici\'on a modificar con doble pulsaci\'on y se 
  convertir\'a en un elemento sensible a movimiento y/o cambio de s\'imbolo.
\ei

La disposici\'on de los estados de la m\'aquina puede ser reorganizado por 
algunos algoritmos que provee JFLAP, para seleccionar alguno se puede usar la 
opci\'on \textbf{Apply A Specific Layout Algorithm} del men\'u \textbf{View}.
Tambi\'en es posible moverlos de formas m\'as simples selecccionando una 
opci\'on de las ofrecidas por \textbf{Move Vertices} del mismo men\'u.

\vspace*{10pt}

Cuando se requiera guardar un desarrollo se debe seleccionar del men\'u 
\textbf{File} la opci\'on \textbf{Save} o \textbf{Save as}. El archivo 
resultante tendr\'a la extensi\'on \texttt{.jff} cuyo contenido es la 
descripci\'on de la m\'aquina en XML:
\bi
 \item cada estado tiene un identificador y una etiqueta, adem\'as contiene 
  las coordenadas del estado en el canvas y una etiqueta para indicar si es 
  inicial y/o final:
  \begin{alltt}
   <state id=``0'' name=``q0''>
       <x>122.0</x>
       <y>214.0</y>
       <initial/>
       <final/>
  \end{alltt}

 \item y las transiciones tienen tres atributos: inicio, destino y s\'imbolo(s) 
  leido(s):
  \begin{alltt}
   <transition>
       <from>1</from>
       <to>0</to>
       <read>a,b</read>
  \end{alltt}

\ei
Se puede reescribir o crear un archivo \texttt{.jff} que describa un aut\'omata 
sin usar JFLAP y abrirlo usando la opci\'on \textbf{Open} del men\'u 
\textbf{File}.

\vspace*{10pt}

Para procesar una cadena en una m\'aquina se debe selccionar la opci\'on 
deseada del men\'u \textbf{Input}. Se puede seguir un procesamiento detallado 
\textbf{Step with Closure}, paso a paso \textbf{Step by State} o r\'apido para 
saber si la cadena es aceptada o rechazada \textbf{Fast Run}.
Tambi\'en es posible procesar varias cadenas en paralelo \textbf{Multiple Run}. 


\end{document}
