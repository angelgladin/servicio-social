\documentclass[xcolor=svgnames]{beamer}
% \documentclass[xcolor=svgnames,handout]{beamer}

\usepackage{../packages}
\usepackage{../beameropt}
\usepackage{../options}

%%%%% Macros para las notas del curso de Automatas y Lenguajes Formales. 

%%%%% Shortcuts
\newcommand{\bc}{\begin{center}}
\newcommand{\ec}{\end{center}}
\newcommand{\be}{\begin{enumerate}}
\newcommand{\ee}{\end{enumerate}}
\newcommand{\bi}{\begin{itemize}}
\newcommand{\ei}{\end{itemize}}

%%%%%
\newcommand{\imp}{\rightarrow}
\newcommand{\Imp}{\Rightarrow}
\renewcommand{\iff}{\leftrightarrow}
\newcommand{\Iff}{\Leftrightarrow}
\newcommand{\D}{\Delta}

\renewcommand{\L}{\mathcal{L}}
\newcommand{\Q}{\ensuremath{\mathbb{Q}}}
\newcommand{\Z}{\ensuremath{\mathbb{Z}}}
\newcommand{\N}{\ensuremath{\mathbb{N}}}
\newcommand{\R}{\ensuremath{\mathbb{R}}}
\newcommand{\I}{\mathcal{I}}
\newcommand{\fa}{\forall}
\newcommand{\ex}{\exists}
\newcommand{\inc}{\subseteq}
\newcommand{\lb}{\lambda}
\newcommand{\al}{\alpha}
\newcommand{\ga}{\gamma}

\newcommand{\sest}{\Sigma^\star}
\newcommand{\dest}{\ensuremath{\delta^\star}}

\newcommand{\vacio}{\varnothing}

\newcommand{\vacia}{\boldsymbol{\varepsilon}}

\newcommand{\cv}{\varepsilon}

\newcommand{\pt}[1]{\langle #1 \rangle}
\newcommand{\Pe}{\mathcal{P}}
\newcommand{\izq}{\leftarrow}
\newcommand{\der}{\rightarrow}
\newcommand{\blanks}{\textbf{\textvisiblespace}}

\definecolor{mycolor}{RGB}{230,230,255}

\title[\color{mpurple4!80!black}Nota 15]{Aut\'omatas y Lenguajes Formales\\
Facultad de Ciencias UNAM}
\subtitle{Nota 15 sobre lenguajes, m\'aquinas de Turing, \textit{Halting 
Problem} \dots }

\author[Lourdes Gonz\'alez Huesca]{
\texorpdfstring{Favio E. Miranda Perea \and A. Liliana Reyes Cabello \and \\
Lourdes del Carmen Gonz\'alez Huesca \newline
\scriptsize{
\href{mailto:luglzhuesca@ciencias.unam.mx}{luglzhuesca@ciencias.unam.mx}}}
{Lourdes del Carmen Gonz\'alez Huesca}}

\date[23/11/2016\hspace{10pt}]{23 y 24 noviembre 2016}

% \titlegraphic{\includegraphics[height=1.5cm]{fc2.png}}
  
\begin{document}

{
\setbeamertemplate{headline}[default]
\setbeamertemplate{footline}[default]
\begin{frame}
  \titlepage
\end{frame}
}

\section{Lenguajes}
\begin{frame}[fragile]
 \titulos{Lenguajes recursivos}{}
  \begin{center}
    Un lenguaje $L$ es \textbf{recursivo} si es reconocido por \\ una 
    m\'aquina de Turing \textbf{total}, es decir,\\ si existe 
    una m\'aquina $M$ que se detiene\\ con todas las cadenas de 
    entrada y $L=L(M)$.
   \end{center}
   \pause
   \begin{itemize}
   \item Los lenguajes recursivos tambi\'en se conocen como Turing-decidibles.
%    \item El adjetivo \enquote{recursivo} se debe a la siguiente propiedad 
%     de estos lenguajes: 
%    \item[] {\small Un lenguaje $L$ es recursivo si y s\'olo si su funci\'on 
%     caracter\'istica $\chi_L$ es recursiva.}
    \item Existe un algoritmo para decidir si cualquier cadena pertenece a $L$.
  \end{itemize}
\end{frame}


\begin{frame}
 \titulos{Lenguajes recursivamente enumerables}{}
  \begin{center}
    Un lenguaje $L$ es \textbf{recursivamente enumerable} \\si es reconocido
    por una m\'aquina de Turing, es decir, \\si existe una m\'aquina de Turing
    $M$ tal que $L=L(M)$. 
  \end{center}
  \pause
  \begin{itemize}
   \item Los lenguajes recursivamente enumerables tambi\'en se conocen como 
    Turing-reconocibles o semidecidibles.
%    \item El adjetivo \enquote{recursivamente} se debe a la siguiente propiedad 
%     de estos lenguajes: 
%     \begin{itemize}
%     \item[] Un lenguaje $L$ es recursivamente enumerable si y s\'olo si $L$ es 
%       la imagen de una funci\'on recursiva \textbf{total}.
%     \end{itemize}
   \item Existe un procedimiento para decidir si una cadena pertenece a $L$, si 
    no pertenece puede suceder que la m\'aquina se cicle.
  \end{itemize}
\end{frame}


\begin{frame}
 \titulos{Lenguajes recursivos}{Enumeraci\'on}
  \begin{itemize}
   \item Si un lenguaje $L$ es recursivo entonces existe un proceso de 
    enumeraci\'on para $L$. 
   \item Es decir, existe un proceso que genera \textbf{todas} las cadenas de 
    $L$.
   \item Una m\'aquina que enumera a $L$ se construye componiendo dos 
    m\'aquinas:
    \begin{enumerate}
     \item una m\'aquina $M$ que genera a todas las cadenas del alfabeto de 
    entrada
     \item una m\'aquina $N$ que reconozca a $L$
    \end{enumerate}
    \item El proceso de enumeraci\'on consiste en repetir los siguientes pasos:
     \begin{enumerate}
      \item $M$ genera una cadena $w$
      \item $N$ verifica si $w\in L$, en caso positivo se imprime $w$, en otro 
	caso se ignora a $w$.
      \end{enumerate}
    \item Este procedimiento funciona pues $N$ siempre se detiene.
    \end{itemize}
\end{frame}


\begin{frame}
  \titulos{Lenguajes recursivamente enumerables}{Enumeraci\'on}
   Del proceso anterior podemos decir que:
    \begin{center}
    Un lenguaje $L$ es recursivamente enumerable si y s\'olo si \\existe un 
    proceso de enumeraci\'on para $L$. 
    \end{center}
    \pause
   \begin{itemize}
    \item El proceso de enumeraci\'on para un lenguaje recursivo enumerable no 
    funciona de la misma forma.
    Se debe combinar la ejecuci\'on de $M$ y $N$ de otra manera, dado que $N$ 
    puede ciclarse. 
    \item El proceso de enumeraci\'on consiste en repetir los siguientes pasos:
     \begin{enumerate}
      \item $M$ genera a la $i$-\'esima cadena~$w_i$
      \item $N$ ejecuta un paso (transici\'on) en $w_i$, dos pasos en 
	$w_{i-1}$, tres pasos en $w_{i-2}$, y as\'i sucesivamente.
      \item Si en alg\'un momento $N$ acepta a $w_i$, se imprime dicha cadena.
     \end{enumerate}
    \item Este procedimiento funciona pues $N$ va procesando en tiempo finito 
      fragmentos de cada cadena $w_i$
 \end{itemize}
\end{frame}

\section{Algoritmos}
\begin{frame}
  \titulos{?`Qu\'e es un algoritmo?}{Introducci\'on}
  \bi
   \item ?`Qu\'e es un algoritmo? 
   \item Una receta, una serie de pasos a seguir, etc., podemos reconocer 
    cuando vemos un algoritmo. 
   \item Pero, ?`podemos dar una definici\'on precisa del concepto de  
    algoritmo? 
   \item ?`Por qu\'e es importante tener una definici\'on precisa
    (matem\'atica) de algoritmo?
  \ei
\end{frame}

\begin{frame}
  \titulos{?`Qu\'e es un algoritmo?}{Introducci\'on}
  \bi
   \item Un algoritmo es una colecci\'on de instrucciones simples para
    realizar una tarea o problema particular (procedimientos o recetas).
   \item Si se tiene un algoritmo para un problema dado~$P$ significa que 
    se tiene una manera para resolver $P$ o calcular efectivamente su resultado.
   \item Un algoritmo es un proceso \textbf{potencialmente} realizable ya que:
    \be
     \item las operaciones del proceso se pueden realizar inequ\'ivocamente y
     \item el n\'umero de operaciones o pasos del proceso es finito.
    \ee
  \ei
\end{frame}

%\subsection{Existencia y Formalizaci\'on}
\begin{frame}
 \titulos{Existencia de algoritmos}{Introducci\'on}
  \bi
  \item D\'ecimo problema de Hilbert: 
    \begin{center}
     \begin{small}
      Hallar un proceso de acuerdo al cual pueda determinarse en un n\'umero 
      finito de pasos (un algoritmo) si un polinomio dado tiene una raiz 
      entera. 
     \end{small}
    \end{center}
  \item Se cre\'ia que todo problema $P$ tenia una soluci\'on algor\'itmica. 
  \item M\'as a\'un se pensaba en la existencia de un algoritmo universal~$U$ 
   que pudiera resolver todos los problemas matem\'aticos.
  \ei
\end{frame}


\begin{frame}
 \titulos{Existencia de algoritmos}{Introducci\'on}
  \bi
   \item Los intentos por hallar el algoritmo universal $U$ fallaron. 
   \item Tal vez $U$ no exist\'ia. 
   \item ?`C\'omo probar la no existencia de $U$? 
   \item Era necesario definir el concepto de algoritmo de una manera precisa y 
    hallar un formalismo para poder probar propiedades de los mismos. 
   \item Un formalismo de algoritmos deber\'ia ser preciso y libre de 
    ambig\"uedades, simple y general.
  \ei
\end{frame}


\begin{frame}
 \titulos{Formalizaci\'on del concepto de algoritmo}{Diferentes modelos}
  \bi
   \item M\'aquinas de Turing (Alan Turing, Cambridge 1936) \medskip
   \item C\'alculo Lambda (Alonzo Church, Princeton 1936) \medskip
   \item Sistemas de Post (Emile Post) \medskip
   \item Funciones $\mu$-recursivas (G\"{o}del, Herbrand, Kleene) \medskip
   \item L\'ogica Combinatoria (Curry, Sch\"{o}nfinkel) \medskip
   \item M\'aquinas de registro (Sheperdson, Sturgis)
 \ei
\end{frame}


\begin{frame}
 \titulos{Equivalencia de los formalismos}{}
  \bi
   \item Sorprendentemente todos los formalismos han resultado equivalentes. 
    \medskip 
   \item Un problema tiene soluci\'on en un formalismo si y s\'olo si
    tiene soluci\'on en cualquiera de los otros. \medskip
   \item Tal afirmaci\'on es un teorema, o una serie de teoremas   
    rigurosamente demostrados. \medskip
   \item Esta coincidencia nos lleva a conjeturar que existe una \'unica 
    noci\'on de computabilidad. 
  \ei
\end{frame}

\section{La Tesis de Church-Turing}
\begin{frame}
 \titulos{La Tesis de Church-Turing}{}
  \bc
   Un problema es soluble algor\'itmicamente si y s\'olo si \\es
   soluble mediante una m\'aquina de Turing.
  \ec  
  \pause
  \bi
  \item La tesis de Church-Turing afirma que la noci\'on intuitiva de
   algoritmo es capturada de manera exacta por la noci\'on matem\'atica
   de m\'aquina de Turing. 
  \item Es decir, las m\'aquinas de Turing implementan a cualquier algoritmo. 
  \item Equivalentemente, una funci\'on es computable si y s\'olo si es 
    soluble mediante una m\'aquina de Turing.
  \item Tambi\'en se usa el t\'ermino \textbf{Turing-completo} para referirse a 
    un sistema que puede usarse para simular a una m\'aquina de Turing.
  \ei
\end{frame}

\begin{frame}
 \titulos{La Tesis de Church-Turing}{}
  \bc
    Un problema es soluble algor\'itmicamente si y s\'olo si\\ es
    soluble mediante una m\'aquina de Turing.
   \ec
   \pause
   \bi
    \item La afirmaci\'on es una tesis indemostrable pues la noci\'on de 
    algoritmo es intuitiva. 
    \item Por otro lado la tesis es refutable y se destruir\'a mostrando un 
    algoritmo que no pudiera ser implementado en una m\'aquina de Turing.
  \ei
\end{frame}

\begin{frame}
 \titulos{Tesis de Church-Turing}{Evidencias a favor}
  \bi
   \item Existen fuertes evidencias a favor de la tesis. 
   \item Intuitivamente cualquier algoritmo detallado para el c\'alculo
    manual puede programarse en una M\'aquina de Turing. 
   \item La equivalencia con otros formalismos m\'as modernos. 
   \item Existen demasiados ejemplos a favor y por supuesto ning\'un
    contraejemplo. 
   \item La comunidad tanto en matem\'aticas como en ciencias de la
    computaci\'on acepta ampiamente la tesis.
  \ei
\end{frame}

\begin{frame}
 \titulos{M\'aquinas de Turing vs. Computadoras}{} %{Introducci\'on}
  \bi
   \item Una computadora es capaz de interpretar algoritmos arbitrarios
    y obtener la misma respuesta que cada algoritmo particular.\medskip
   \item Entonces una computadora es una m\'aquina \'util para
    prop\'ositos generales.\medskip
   \item Las M\'aquina de Turing en cambio son dise\~nadas para prop\'ositos
    particulares.\medskip
   \item Conclusi\'on: el poder computacional de las M\'aquina de Turing  
    no puede ser equiparable al de las computadoras actuales.\medskip
   \item Las computadoras son programables, las M\'aquina de Turing no.
  \ei
\end{frame}

\begin{frame}
 \titulos{La M\'aquina Universal de Turing}{}
  \bi
   \item ?`Ser\'ia factible pensar en la existencia de una M\'aquina de Turing 
    que se comporte de la misma forma que una computadora real? 
   \item Es decir, una m\'aquina que sea \'util para prop\'ositos m\'ultiples. 
   \item Dicha m\'aquina ser\'ia capaz de programar y ejecutar m\'aquinas
    de Turing.
   \item Tal m\'aquina existe y se conoce como m\'aquina universal de 
    Turing~$\mathcal{M}$.%(MUT)
   \item Esta m\'aquina recibe como entrada una descripci\'on de una M\'aquina 
    de Turing~$M$ y una cadena~$w$ y simula el comportamiento de~$M$ sobre $w$. 
   \item Los datos de entrada $M$ y $w$ deben ser codificados de manera 
    adecuada.
  \ei
\end{frame}


\begin{frame}
 \titulos{Codificaci\'on de M\'aquina de Turing}{Observaciones}
  \bi
   \item La codificaci\'on de una M\'aquina de Turing no es \'unica, puesto que 
    el orden de las transiciones no importa y un orden distinto genera una 
    codificaci\'on  distinta. 
   \item De hecho si $M$ tiene $n$ transiciones, existen $n!$    
    codificaciones distintas para $M$. 
   \item El proceso de codificaci\'on puede revertirse, no es d\'ificil       
    definir un algoritmo que decida si una secuencia binaria
    representa un c\'odigo v\'alido para M\'aquina de Turing y en tal caso 
    lo decodifique.
   \item[] \vspace*{20pt}
   \begin{scriptsize}
      N.B. La codificaci\'on y el uso de esta m\'aquina se estudiaron en la     
      nota anterior.
   \end{scriptsize}
  \ei
\end{frame}

%\subsection{Enumerabilidad de las M\'aquina de Turing}
\begin{frame}
  \titulos{Enumerabilidad de las M\'aquina de Turing}{}
  \bi
  \item En conclusi\'on toda M\'aquina de Turing puede representarse como una 
    cadena binaria.\medskip
  \item No todas las cadenas binarias representan M\'aquina de Turing 
   v\'alidas, por ejemplo, las cadenas que empiezan o terminan con $1$ o las 
  que tienen m\'as de dos ceros consecutivos.\medskip
  \item Cada cadena binaria, representa por otra parte un n\'umero
   natural y viceversa. Es decir, hay tantas cadenas binaria como n\'umeros  
   naturales.\medskip
  \item De lo anterior se concluye que hay s\'olo un n\'umero numerable de 
   M\'aquina de Turing.
 \ei
\end{frame}


\begin{frame}
 \titulos{Enumerabilidad de las M\'aquina de Turing}{}
  \bi
   \item Las cadenas binarias pueden enumerarse en orden lexicogr\'afico
    considerando $0<1$:
    $$ 0,1,00,01,10,11,000,001,010,011,100,101,110,111,\ldots$$
   \item Cada M\'aquina de Turing figura varias veces en esta lista.\medskip
   \item Por motivos pr\'acticos si una cadena no codifica
    directamente a una M\'aquina de Turing entonces acordamos que codifica 
    a la m\'aquina sin transiciones que acepta el lenguaje vac\'io.\medskip
   \item De esta manera cada cadena binaria codifica a una M\'aquina de 
    Turing.
  \ei
\end{frame}


\begin{frame}
 \titulos{Existencia de funciones no computables}{Enumerabilidad de
    M\'aquinas de Turing}
  \bi
  \item Dado que las cadenas binarias son tantas como los n\'umeros
    naturales y cada cadena codifica a una M\'aquina de Turing concluimos que 
    s\'olo hay un n\'umero infinito numerable de M\'aquinas de Turing.
  \item Por otro lado si consideramos las funciones $f:\N\imp \N$ es
    bien sabido que son un n\'umero {\bf no} numerable (tantas como
    n\'umeros reales).
  \item De donde se concluye que existen funciones, que {\bf no}
    pueden calcularse mediante una M\'aquina de Turing.
  \item Lo cual bajo la tesis de Church-Turing equivale a que
   existen funciones que no pueden ser calculadas mediante una computadora.
  \ei
\end{frame}


%\subsection{Lenguaje universal y diagonal}
\begin{frame}
  \titulos{El lenguaje diagonal $\mathcal{L}_D$}{}
  \bi
  \item Consideremos la enumeraci\'on de las m\'aquinas de Turing
    $M_1,M_2,\ldots$ as\'i como la enumeraci\'on de todas las cadenas de
    $\sest$, digamos $w_1,w_2,\ldots,w_n,\ldots$\medskip
  \item Podemos entonces dar como entrada la $i$-\'esima palabra $w_i$ a
    la $i$-\'esima m\'aquina $M_i$.\medskip
  \item Definimos el lenguaje diagonal se define como:
   $$  \mathcal{L}_D=\{w_i \mid w_i\text{ no es aceptada por }M_i\}$$
  \item Es decir $\mathcal{L}_D$ contiene a la $i$-\'esima cadena
        si y s\'olo si \'esta \textbf{no} es aceptada por la $i$-\'esima 
m\'aquina. 
  \ei
\end{frame}



\begin{frame}
 \titulos{$\mathcal{L}_D$ \textbf{no} es recursivamente enumerable}{}
  \bi
   \item Si $\mathcal{L}_D$ fuera recursivamente enumerable ser\'ia aceptado 
    por una M\'aquina de Turing, digamos la $k$-\'esima m\'aquina 
    $M_k$.\medskip
   \item En tal caso $\mathcal{L}_D=L(M_k)$.\medskip
   \item Podemos preguntarnos entonces si $w_k\in \mathcal{L}_D$.
   \bi
    \item $w_k\in\mathcal{L}_D\Imp w_k$ no es aceptada por
      $M_k\Imp w_k\notin L(M_k)=\mathcal{L}_D$.
    \item $w_k\notin\mathcal{L}_D\Imp$ $w_k$ es aceptada por 
      $M_k\Imp w_k\in L(M_k)=\mathcal{L}_D$.
    \ei
  \item Por lo que se tendr\'ia
  $$w_k\in\mathcal{L}_D\text{ si y s\'olo si }\;w_k\notin\mathcal{L}_D$$
  \item Lo cual es absurdo.
          
  \ei
\end{frame}



\begin{frame}
\titulos{El lenguaje universal $\mathcal{L}_{\mathcal{M}}$}{Definici\'on}
\begin{itemize}
\item El lenguaje aceptado por la m\'aquina universal $\mathcal{M}$ se conoce
  como lenguaje universal, denotado $\mathcal{L}_{\mathcal{M}}$.
\item[] 
  $$\mathcal{L}_{\mathcal{M}}=\{\pt{M}\#\pt{w}\mid M\text{ acepta a }w\}$$
  \begin{small}
    $\pt{M}\#\pt{w}$ es la cadena de entrada con la m\'aquina y la 
    cadena   codificadas.   
  \end{small}
\item Y es recursivamente enumerable, analicemos a continuaci\'on porqu\'e no 
es recursivo.
\end{itemize}
\end{frame}


% \begin{frame}
%   \titulos{Lenguajes Recursivos y recursivamente enumerable}{}
%   \bi
%   \item Todo lenguaje recursivo es recursivamente enumerable. 
%     \item  Existen lenguajes recursivamente enumerables que no son
%       recursivos. En particular el lenguaje universal 
% $\mathcal{L}_{\mathcal{M}}$
%       no es recursivo. 
%       \item Existen lenguajes no recursivamente enumerables como
%         $\mathcal{L}_D$.
%       \ei
% \end{frame}

\begin{frame}
\titulos{El lenguaje universal $\mathcal{L}_{\mathcal{M}}$ no es recursivo}{}
  Supongamos lo contrario y sea~$M$ una m\'aquina de Turing que siempre se 
  detiene y tal que $\mathcal{L}_{\mathcal{M}}=L(M)$.\\
  
  Veamos que a partir de $M$ podemos construir una m\'aquina de Turing $M'$ 
  que acepte al lenguaje diagonal $\mathcal{L}_D$, lo cual es absurdo.\\
  $M'$ funciona como sigue para la entrada $w\in\sest$:
  \bi
   \item Enumerar las cadenas de $\sest$ hasta encontrar $k\in\N$ tal que
    $w=w_k$.
  \item Llamar a $M$ con entrada $\pt{M_k}\#\pt{w_k}$.
  \item Como $M$ siempre se detiene entonces decide si $M_k$ acepta a $w_k$.
  \item En tal caso forzamos a que $M'$ acepte tambi\'en a $w_k$. 
  \item Esto implica que $L(M')=\mathcal{L}_D$ que es un absurdo.
  \ei
\end{frame}


\section{Computabilidad}

\begin{frame}
  \titulos{Preguntas y Problemas}{}
  \bi
  \item Preguntas:
    \bi
     \item ?`Qu\'e $x$ n\'umero real cumple que $2x^2-3x+5=0$?\medskip
     \item Dadas las ciudades $a,b,c,d$ ?`Cual es la forma \'optima de
        visitarlas sin pasar dos veces por la misma ciudad?
    \ei\medskip
    \item Una problema es una clase de preguntas:
      \bi
      \item ?`Cuales son las soluciones de ?`$ax^2+bx+c=0$?\medskip
      \item ?` Dados $n$ v\'ertices en un grafo existir\'a un camino 
      hamiltoniano?
      \ei\medskip
    \item Cada caso particular de un problema es un ejemplo, ejemplar o
      instancia de \'este. 
  \ei
\end{frame}


\begin{frame}
 \titulos{Tipos de Problemas}{Introducci\'on}
   La mayor\'ia de los problemas de inter\'es en ciencias de la
   computaci\'on son de dos tipos.
  \be
   \item Problemas de c\'omputo: obtener el valor de una funci\'on en un
    argumento dado.
    \bi
    \item[] Por ejemplo obtener la ra\'iz cuadrada de un n\'umero dado $x$
      con exactitud de mil\'esimas.
    \ei
    \item Problemas de decisi\'on: problemas cuya respuesta es si o no.
      \bi
    \item[] Por ejemplo, decidir la existencia de un camino \'optimo en costos 
      para recorrer varias ciudades.
    \ei
  \ee
  \pause
  Si bien los problemas de decisi\'on tambi\'en podr\'ian considerarse 
  problemas de c\'omputo, es \'util hacer la distinci\'on.
\end{frame}

\subsection{Algoritmos y problemas}
\begin{frame}
  \titulos{Algoritmos y problemas}{}
  \bi
  \item Un algoritmo para un problema consiste de una serie de
    instrucciones capaces de responder cualquier instancia del
    problema dado.\medskip
  \item Un problema~$P$ se dice soluble si existe un algoritmo para~$P$. 
   En otro caso se dice insoluble.\medskip
  \item Si un problema de decisi\'on~$P$ es soluble entonces decimos que
   $P$ es \textbf{decidible}. En caso contrario el problema se dice
   \textbf{indecidible}.\medskip
  \item Un proceso o algoritmo para un problema de decisi\'on se conoce como un 
    proceso o algoritmo de decisi\'on.
  \ei
\end{frame}


\begin{frame}
 \titulos{Algoritmos y funciones}{}
  \bi
  \item Cualquier algoritmo puede verse como una funci\'on:
    $$\text{ entrada}\;x \imp \;\text{ algoritmo}\; \imp \; \text{ salida}\; y$$
  \item La salida est\'a en funci\'on de la entrada $f(x)=y $
  \item Una funci\'on es computable si existe un algoritmo que la calcule.
  \item La teor\'ia de la computabilidad se encarga esencialmente
   a contestar la pregunta \textbf{?`Qu\'e funciones son computables?}
  \item Lo cual equivale entonces a responder que problemas son solubles.
  \ei
\end{frame}

%\subsection{Problemas no computables}
\begin{frame}
  \titulos{Problemas no computables}{Introducci\'on}
  \bi
  \item ?`Por qu\'e nos interesa averiguar qu\'e problemas no son
    computables mediante un algoritmo o proceso?\medskip
  \item Tales problemas son aquellos que {\bf no} podemos resolver.\medskip
  \item Son resultados fundamentales y debemos conocerlos para
    tener una visi\'on general de las ciencias de la computaci\'on.\medskip
  \item Debemos conocerlos para evitar intentar resolverlos.\medskip
  \item Debemos entender que tales problemas son insolubles
   independientemente del desarrollo futuro del hardware.
  \ei
\end{frame}



\begin{frame}
  \titulos{Recursividad y decidibilidad}{Problemas insolubles}
  \bi
  \item Consid\'erese una propiedad $\Pe$ acerca de m\'aquinas de 
    Turing.\medskip
    \item$\Pe$ genera el problema de decisi\'on siguiente:\\
      \bc
      ?`Satisface la m\'aquina $M$ la propiedad $\Pe$ ?
      \ec
      \item Asumiendo la tesis de Church-Turing, tal problema de
        decisi\'on ser\'a decidible (soluble) si y s\'olo si el lenguaje
    \[
      L=\{\pt{M}\mid \pt{M} \text{\small{ es el c\'odigo de
       una M\'aquina de Turing que satisface} }\Pe\}
    \]
    es recursivo.
  \ei
\end{frame}


%\subsection{El problema de la detenci\'on}
\begin{frame}
 \titulos{El problema de la detenci\'on}{\emph{Halting problem}}
 \bc
 Dada una m\'aquina $M$ y una cadena $w$ \\
 ?`Se detendr\'a $M$ al procesar $w$?
 \ec 
 \pause
 \bi
  \item El problema ser\'a soluble si pudieramos hallar una m\'aquina $H$
   tal que al recibir como entrada a cualquier cadena $\pt{M}\#\pt{w}$ se
   detuviera si y s\'olo si $M$ se detiene al procesar $w$. \medskip
  \item Podr\'ia pensarse que una m\'aquina universal puede hacer el trabajo.
 \ei
\end{frame}


\begin{frame}
 \titulos{El problema de la detenci\'on}{\emph{Halting problem}}
  \bi
  \item El problema de la detenci\'on resulta indecidible, es decir, no existe 
   tal m\'aquina~$H$. 
  \item Adem\'as es quiz\'as el problema indecidible m\'as relevante en la 
   teor\'ia de la computabilidad. 
  \item Una consecuencia inmediata de su indecidibilidad es que no puede
   existir un programa que verifique si cualquier programa dado se cicla.
  \ei
\end{frame}


\begin{frame}
  \titulos{El problema universal}{Problemas no computables}
  \bc
  Dada una m\'aquina de Turing cualquiera $M$ y una cadena $w$\\ ?`Acepta
  $M$ a $w$?
  \ec 
  \pause
  \bi
  \item El problema universal equivale a que el lenguaje universal
   $$\mathcal{L}_{\mathcal{M}}=\{\pt{M}\#\pt{w}\mid M\;\text{ acepta a}\;w\}$$
   sea recursivo.
  \item Ya demostramos que $\mathcal{L}_{{\mathcal{M}}}$ no es recursivo.
  \item Por lo tanto el problema universal es indecidible.
  \ei
\end{frame}


\subsection{Reducibilidad de problemas}

\begin{frame}
 \titulos{Reducibilidad de problemas}{}
  \bi
  \item La reducibilidad de problemas es una t\'ecnica de gran utilidad
    para probar la indecidibilidad de un problema dado a partir de la
    indecidibilidad de un problema conocido. 
  \item Dados dos problemas $\Pe_1,\Pe_2$ decimos que $\Pe_1$ se
   reduce a $\Pe_2$ si un algoritmo para decidir $\Pe_2$ puede
   emplearse para decidir $\Pe_1$.
  \ei
\end{frame}


\begin{frame}
 \titulos{Reducibilidad de problemas}{}
   Formalmente decimos que $\Pe_1$ se reduce a $\Pe_2$,
    denotado $\Pe_1\prec\Pe_2$ si existe una M\'aquina de Turing $M$ tal 
    que: 
    \pause
  \bi
  \item $M$ recibe como entrada una instancia $I_1$ de $\Pe_1$\medskip
  \item $M$ devuelve como salida una instancia $I_2$ de $\Pe_2$.\medskip
  \item $M$ decide a $I_1$ de la misma manera que $I_2$.\medskip
  \item Es decir, $M$ responde con \emph{s\'i} a $I_1$ si y s\'olo si
   responde con \emph{s\'i} a $I_2$.\medskip
  \ei
  \pause 
  De esta manera se tiene que $\Pe_1$ es decidible
   si y s\'olo si $\Pe_2$ es decidible.
\end{frame}

\begin{frame}\titulos{PD $\prec$ PU}{Reducibilidad}
  \begin{small}
  \textbf{PD:} Dada una m\'aquina $M$ y una cadena $w$ ?`Se detendr\'a $M$ al 
  procesar $w$?\\
  \textbf{PU:} Dada una m\'aquina de Turing cualquiera $M$ y una cadena $w$
    ?`$M$ acepta~$w$?
    
  \vspace*{10pt}
  \pause
  El problema universal puede reducirse al problema de la detenci\'on:
  \vspace*{-6pt}
  \bi
   \item Supongamos que existe una m\'aquina $H$ que decide el
    problema de la detenci\'on.
   \item En tal caso $H$ decide tambi\'en al problema universal.
   \item Al recibir una entrada $\pt{M}0\pt{w}$, $H$ decide si $M$ se
    detiene o no con entrada $w$.
   \item Si $M$ no se detiene con $w$ entonces $M$ no acepta a $w$.
   \item Si $M$ se detiene con $w$ entonces $M$ procesa a $w$
    y decide si la acepta o no.
   \item De manera que el problema universal se decide, lo cual es absurdo.
  \ei
  \end{small}
\end{frame}

\begin{frame}
 \titulos{Otros problemas indecidibles}{Reducibilidad}
  \bi
   \item Detenci\'on con cinta en blanco: ?` se detiene la m\'aquina $M$ al
    iniciar con la cinta en blanco?\medskip
   \item Impresi\'on de un s\'imbolo: Dada $M$ y $s\in\Sigma$ 
    ?`Escribir\'a $M$ en alg\'un momento a $s$ sobre la cinta?\medskip
   \item Dada una gram\'atica libre de contexto $G$  ?` $G$ es ambigua?\medskip
   \item Determinar si dos GLC son equivalentes.\medskip
   \item Cualquier propiedad no trivial acerca de M\'aquina de Turing 
  (Teorema de Rice).
  \ei
\end{frame}


\section{Complejidad}
\begin{frame}
 \titulos{Complejidad}{Introducci\'on}
  \bi
  \item Las M\'aquinas de Turing son el formalismo m\'as \'util en el 
    an\'alisis de algoritmos, debido a que modelan de manera precisa los 
    conceptos centrales de c\'omputo, almacenamiento, espacio y tiempo.
   \item La noci\'on formal de c\'omputo permite precisar sin 
    ambig\"uedades el tiempo de computaci\'on de un problema.
   \item Las celdas de la cinta formalizan de manera clara la noci\'on de 
    espacio de almacenamiento (memoria)
   \item As\'i las nociones de tiempo y espacio se modelan de forma muy 
    realista mediante una M\'aquina de Turing, lo cual permite analizar 
    la complejidad computacional de un problema. 
  \ei
\end{frame}


\begin{frame}
 \titulos{Complejidad}{Introducci\'on}
  \bi
   \item La teor\'ia de complejidad se interesa por el estudio de la
      complejidad necesaria para resolver un problema.
   \item En particular por el consumo de recursos (espacio y tiempo).
   \item \textbf{No le conciernen los problemas insolubles.}
   \item Pero tampoco todos los problemas solubles, en la pr\'actica 
    aquellos problemas solubles que no pueden resolverse en un tiempo razonable 
    son tan despreciables como un problema insoluble.
   \item Los problemas se clasifican en clases de complejidad de acuerdo a 
    qu\'e tan dif\'icil es resolverlos.
  \ei
\end{frame}


\begin{frame}
 \titulos{La funci\'on de tiempo de ejecuci\'on de una M\'aquina 
  de Turing}{Complejidad}
  \bi
  \item[] La funci\'on tiempo de ejecuci\'on de una M\'aquina de Turing $M$ se 
  define como
   $$ t_M:\N\imp\N\cup\{\infty\}$$
  \item[] $t_M(n):=\;$ m\'aximo n\'umero de pasos de ejecuci\'on de $M$ para 
  una entrada de longitud $n$.
  \ei
\end{frame}

\begin{frame}
 \titulos{M\'aquinas que corren en tiempo polinomial}{Complejidad}
  Una m\'aquina de Turing $M$ corre en tiempo polinomial si:
  \bi
   \item[] Existe un polinomio con coeficientes enteros no negativos
     $$ p(x)=a_nx^n+a_{n-1}x^{n-1}+\ldots + a_1x+a_0 $$
   \item[] tal que $t_M(n)\leq p(n)$ para toda $n\in\N$.
   \item Es decir, si la funci\'on de tiempo de ejecuci\'on de $M$
     est\'a \textbf{acotada superiormente} por un polinomio.
   \item Una m\'aquina que corre en tiempo polinomial siempre se detiene.
  \ei
\end{frame}


\subsection{Clases de complejidad}
\begin{frame}
  \titulos{Clases de complejidad}{{\bf P}}
  \bi
   \item Clase de problemas que pueden ser resueltos eficientemente.
   \item Eficientemente significa que existe un algoritmo que corre
    en \textbf{tiempo polinomial}.
   \item {\bf P} se conoce tambi\'en como la clase de 
    problemas \textbf{tratables}.
   \item En contraste, un problema es \textbf{intratable} si es
     soluble pero cualquier soluci\'on algor\'itmica corre en tiempo
    exponencial en el peor de los casos.
   \item Un problema intratable es pr\'acticamente insoluble,
    excepto para entradas muy peque\~nas, a no ser que el caso
   promedio sea mucho mejor que el peor caso.
  \ei
\end{frame}


\begin{frame}
 \titulos{Clases de complejidad}{{\bf NP}}
  \bi
  \item Clase de problemas que pueden ser resueltos en tiempo
    polinomial pero s\'olo por un algoritmo \textbf{no determinista}.
  \item Es decir son los problemas solubles por una M\'aquina de Turing
   no-determinista en tiempo polinomial.
  \item Sabemos que las M\'aquina de Turing no-deterministas equivalen a 
   las M\'aquina de Turing  deterministas.
  \item Sin embargo no se sabe en general si un problema soluble 
  mediante un algoritmo no-determinista tendr\'a una soluci\'on determinista.
  \ei
\end{frame}


\begin{frame}
  \titulos{{\bf P} vs {\bf NP}}{}
  \bi
  \item Dado un problema $\Pe\in\;${\bf NP} en general no sabemos si
    existir\'a una soluci\'on determinista.
  \item Si {\bf P} = {\bf NP} entonces la respuesta es afirmativa.
  \item Aun no se ha probado ni refutado si {\bf P}={\bf NP}.
  \item En caso positivo siempre habr\'i un algoritmo determinista para un 
    problema cuya soluci\'on no-determinista se conoce.
  \item Se sospecha que {\bf P}$\neq${\bf NP} y esta es la pregunta 
    m\'as importante en la teoria de la complejidad (la respuesta vale 1 
    mill\'on de d\'olares).
  \ei
\end{frame}

\subsection{Problemas {\bf NP}-completos}
\begin{frame}
 \titulos{Problemas {\bf NP}-completos}{}
  \begin{itemize}
  \item Los problemas {\bf NP}-completos son los m\'as dif\'iciles de la
    clase {\bf NP}. 
  \item Cualquier problema de la clase {\bf NP} puede reducirse a
   cualquier problema {\bf NP}-completo. 
  \item Existen muchos problemas {\bf NP}-completos, el m\'as
   relevante es probablemente el problema {\bf SAT} (Teorema de Cook).
\end{itemize}  
\end{frame}


\begin{frame}
 \titulos{Problemas {\bf NP}-completos}{}
  \bi
   \item Un problema $\Pe$ es {\bf NP}-completo ($\Pe\in\;${\bf NPC}) si:
    \bi
    \item $\Pe\in\;${\bf NP}.
    \item $\mathcal{Q}\prec\Pe$ en tiempo polinomial para cualquier problema
      $\mathcal{Q}\in{\bf NP}$.
    \ei  
  \item Todos los problemas en la clase {\bf NPC} son equivalentes, es
   decir, si $A,B\in\;${\bf NPC} entonces $A\prec B$ y $B\prec A$. 
  \item La segunda condici\'on de la definici\'on se puede intercambiar
    por: $C\prec \Pe$ para alg\'un problema {\bf NP}-completo $C$.
  \ei
\end{frame}

\begin{frame}
 \titulos{El problema de satisfacci\'on {\bf SAT}}{Algunos problemas {\bf 
NP}-completos}
  \bi
  \item Dada una f\'ormula de la l\'ogica proposicional en forma normal 
    conjuntiva:
  $$P:= C_1\lor C_2\lor C_3\ldots C_n$$
  \item ?`Existe una asignaci\'on de verdad que satisfaga a $P$? 
  \item Teorema de Cook (1971): {\bf SAT} es {\bf NP}-completo. 
  \item Este es el primer problema {\bf NP}-completo.
  \ei
\end{frame}

\begin{frame}
 \titulos{Circuito Hamiltoniano {\bf PCH}}{Algunos problemas {\bf NP}-completos}
  \bi
  \item Un circuito hamiltoniano es un camino que inicia y termina en un
  mismo v\'ertice de un grafo conexo y que visita a todos los v\'ertices 
  exactamente una vez (el v\'ertice de inicio y fin cuenta solo una vez). 
  \medskip
  
  \item {\bf PCH}: ?`Dado un grafo conexo $G$, tiene $G$ un circuito 
  hamiltoniano?
\ei
\end{frame}

\begin{frame}
 \titulos{Problema del agente viajero {\bf PAV}}{Algunos problemas {\bf 
NP}-completos}
  \bi
  \item Se tienen dadas $m$ ciudades, las distancias entre cualesquiera dos de
  ellas y un entero $N$. 
  \medskip
  \item {\bf PAV}: ?`Existe un recorrido que visite cada ciudad
    exactamente una vez y cuya longitud sea menor o igual que $N$ ?
\ei
\end{frame}



\end{document}
