\documentclass[letterpaper,11pt]{article}
\usepackage[includeheadfoot,margin=1.3in]{geometry}
\usepackage{anysize}

\usepackage[utf8]{inputenc}
% \usepackage[latin1]{inputenc}
\usepackage[english,spanish]{babel}
\usepackage{lmodern}   % font shapes...
\usepackage[T1]{fontenc} % join the compound symbols as a single symbol

\usepackage{amssymb,amsmath}
\usepackage{mathrsfs}
\usepackage{epsfig}

\usepackage{fancyhdr}
\usepackage{hyperref}
\usepackage{url}

\usepackage{import}
\usepackage{comment}
\usepackage[autostyle=true,spanish=mexican]{csquotes}

\usepackage{alltt}
\usepackage[section]{placeins}

\usepackage{url}

\usepackage{pgf}
\usepackage{tikz}
\usetikzlibrary{automata,arrows,trees}
\usetikzlibrary{babel}
%\pagestyle{fancyplain}
\input{../macrosAyLF}

\title{Aut\'omatas y Lenguajes Formales 2017-1 \\ 
% Posgrado en Ciencia e Ingeniería de la Computación UNAM \\ 
Facultad de Ciencias UNAM \\ 
Nota de Clase 6} 
\author{Favio E. Miranda Perea \and A. Liliana Reyes Cabello \and
Lourdes Gonz\'alez Huesca}
\date{\today}

\begin{document}
\maketitle

\section{Lenguajes regulares o ?`no?}

Los lenguajes que hemos tratado han sido caracterizados por ser sencillos, es 
decir son conjuntos de cadenas cuyas propiedades son f\'aciles de abstraer 
mediante lenguajes b\'asicos y operaciones como la concatenaci\'on, uni\'on y 
cerradura de Kleene.
A partir de ellos, las propiedades de cerradura nos permiten construir 
nuevos lenguajes regulares:
% a partir de lenguajes ya conocidos por medio de algunas operaciones entre 
% lenguajes.\\
Si $L,M$ son lenguajes regulares entonces:
\begin{center}
\begin{tabular}{ccc}
$L\cup M$ es regular  & \hspace*{50pt} & $LM$ es regular\\
$L^\star$ es regular & & $L^+$ es regular\\
$\overline{L}$ es regular && $L\cap M$ es regular\\
$L-M$ es regular & &
\end{tabular}
\end{center}

As\'i mismo, hemos considerado la equivalencia entre lenguajes regulares y los 
aut\'omatas finitos a trav\'es del teorema de Kleene. 
Pero ?`c\'omo es posible decidir si un lenguaje dado, sin una representaci\'on 
de expresi\'on regular o aut\'omata que lo reconoce, es regular?

Un caso conocido de un lenguaje \enquote{sencillo} no regular es 
$$ L = \{a^i b^i \mid i\in \N \} $$
Veamos dos ejemplos de lenguajes no regulares al tratar de dar una expresi\'on 
que los genere:

\paragraph{Ejemplo:}
Considere el lenguaje~$L=\{a^ib^j \mid i\neq j, \; i,j\in \N\}$. 
Decida si es regular.\\
Para la demostraci\'on, supongamos primero que~$L$ es regular.
Ahora procedemos a construir una expresi\'on que lo represente:
partimos del lenguaje~$a^\star b^\star$ que claramente es regular.\\
Por propiedades de cerradura, el lenguaje $a^\star b^\star - L$ tambi\'en debe 
ser regular.\\
!`Pero $a^\star b^\star-L=\{a^ib^i \mid i\in\N\}$ no es regular!\\
Por lo tanto $L$ no puede ser regular.

\paragraph{Ejemplo:}
Ahora considere el lenguaje~$L=\{wb^n \mid |w|=n, n\geq 1\}$. 
Decida si es regular.\\
Nuevamente, supongamos que $L$ es regular.
Y partimos otra vez de $a^\star b^\star$ que claramente es regular.\\
Usando las propiedades de cerradura $L\cap a^\star b^\star$ tambi\'en debe ser 
regular.
Pero $L\cap a^\star b^\star = \{a^ib^{\,i} \mid \in\N\}$ no es regular.\\
Por lo tanto $L$ no puede ser regular.


\subsection{?`Cu\'antos lenguajes regulares hay?}

Los lenguajes son variados y requerimos de una forma eficiente para 
decidir si un lenguaje es regular y entonces atrevernos a proporcionar una 
m\'aquina que lo reconozca.

Consideremos el conjunto de todos los lenguajes regulares, dado un alfabeto:
$$REG=\{L\inc\sest\mid L\mbox{ es regular }\}$$
?`Cu\'al es la cardinalidad de $REG$? Analicemos el conjunto:
\bi
 \item Dado que cualquier lenguaje~$L$ es un subconjunto de $\sest$, existen 
  tantos lenguajes como elementos en $\Pe(\sest)$.
 \item Puesto que $\sest$ es infinito numerable, es decir es del tama\~no del 
  conjunto $\mathbb{N}$ de los n\'umeros naturales, entonces $\Pe(\sest)$ es 
  del tama\~no del conjunto de los n\'umeros reales $\mathbb{R}$.
 \item Existen s\'olo tantos lenguajes regulares como n\'umeros naturales, 
  $|REG|=|\mathbb{N}|$:
  \bi
   \item La idea de la prueba es enumerar lexicogr\'aficamente \textit{todos} 
    los AFD posibles con alfabeto de entrada $\Sigma$, es decir, primero los 
    aut\'omatas con un s\'olo estado, luego los de dos estados, 
    etc.
   \item Esto implica que el n\'umero de lenguajes regulares es a lo m\'as 
    numerable.
   \item Adem\'as, claramente es numerable pues hay una infinidad numerable de 
    lenguajes regulares, por ejemplo 
    $$\{a\},\{aa\},\{aaa\},\ldots $$
  \ei
 \item De manera que el conjunto $\Pe(\sest)-REG$ no puede ser numerable, pues 
  la uni\'on de numerables sigue siendo numerable.
 \item Es decir, hay tantos lenguajes \textbf{no} regulares como n\'umeros 
  reales.
\ei

\vspace*{10pt}

En esta nota nos dedicaremos a discutir dos m\'etodos para probar que un 
lenguaje \textbf{no} es regular.

\section{Lema de Bombeo para lenguajes regulares}
\lema{[Lema del Bombeo]
Si $L$ es un lenguaje regular infinito entonces existe un n\'umero $n\in\N$,
llamado constante de bombeo para $L$, tal que para cualquier cadena $w$
de $L$ con $|w|\geq n$ existen cadenas $u,v,x$ tales que:
\be
 \item $w=uvx$
 \item $v\neq\cv$
 \item $|uv|\leq n$
 \item $\fa i\geq 0,\;uv^ix\in L$.
\ee
}
\begin{proof}
Informalmente, la idea del lema se basa en que la caracter\'istica de un 
lenguaje regular consiste en \textit{repetir} o \textit{ciclar} una subcadena 
y de ah\'i que exista un ciclo en un aut\'omata que reconoce a $L$.
Recordemos que un aut\'omata es una m\'aquina con una memoria corta, es decir 
s\'olo recuerda lo que est\'a abstraido en un estado y no tiene memoria 
auxiliar. 
El siguiente diagrama resume el lema:
\begin{center}
\begin{tikzpicture}[node distance=3.5cm,every node/.style={scale=0.8},thick]
    \node[state,initial,initial text=,accepting by double] (q0) {$q_0$};
    \node[state] (qi) [right of=q0] {$q_i$};
    \node[state] (qf) [right of=qi] {$q_f$};
    \path[->,style=loosely dashed] (q0) edge [bend left] node [above] {$u$} 
(qi);
    \path[->,style=loosely dashed] (qi) edge [out=130,in=45,loop] node [above] 
{$v$} (qi);
    \path[->,style=loosely dashed] (qi) edge [bend left] node 
[above] {$x$} (qf);
 \end{tikzpicture}
\end{center}
\end{proof}

\vspace*{15pt}

Para probar que un lenguaje $L$ \textbf{no} es regular se procede por 
\textit{contradicci\'on} usando del Lema del Bombeo como sigue:
\be
 \item Si $L$ fuera regular entonces existir\'ia una constante de bombeo $n$.
 \item Y cualquier palabra $w\in L$ con longitud mayor o igual a $n$ se 
  descompone como  $w=uvx$ donde $v\neq\vacia,\;\;|uv|\leq n$.
 \item Se llega a una contradicci\'on como sigue:
 \item[] por el lema del bombeo la cadena~$uv^ix$ debe pertenecer a $L$,
   para toda $i\geq 0$. Pero por la definici\'on particular de $L$, se 
   puede mostrar alguna $i$ tal que $uv^ix\notin L$.
\ee
Debemos observar que encontrar la $i$ adecuada depende del problema particular 
y no hay un m\'etodo general, pero usualmente basta con valores peque\~nos de  
$i$.

\paragraph{Ejemplo:}
Veamos que $L=\{a^ib^i\;|\;i\in\N\}$ \textbf{no} es regular usando el Lema del 
bombeo.\\
Sup\'ongase que $L$ es regular y sea $n$ una constante de bombeo.
Ahora considere una palabra cualquiera del lenguaje, es decir tiene la 
forma $w=a^nb^n$.
Una descomposici\'on de $w$ en $uvx$ es la siguiente:
$u,v$ contienen s\'olo $a$'s, digamos 
$$u=a^k,\;v=a^\ell,\;k\geq 0,\ell\geq 1$$
Con esto aseguramos que se cumple que $v\neq\cv$, $|uv|\leq n$ y adem\'as
$x=a^{n-k-\ell}b^n$. \\
Si tomamos $i=2$, por el lema del bombeo la cadena~$uv^2x$ debe pertenecer a 
$L$. Pero, realmente sucede que
$$  uv^2x=a^ka^\ell a^\ell a^{n-k-\ell}b^n=a^{n+\ell}b^n\notin L$$
Por lo tanto, $L$ no es regular.

\paragraph{Ejemplo:}
Demuestre que $L_1=\{w\in\{a,b\}^\star \mid w=w^R\}$ \textbf{no} es regular.\\
Suponer que $L$ es regular y sea $n$ una constante de bombeo.
Ahora considere que la palabra $w=a^nb^na^n$ en $L_1$ tiene una 
descomposici\'on en $w=uvx$ con $v\neq\cv,\;|uv|\leq n$ y en donde $u,v$ tienen 
s\'olo $a$'s digamos $u=a^k,\;v=a^\ell,\;\ell\geq 1$.\\
Por tanto $x=a^{n-k-\ell}b^na^n$.
Al hacer $i=2$ debemos tener a $uv^2x\in L$, por el lema del bombeo.
Pero por otra parte se tiene que:
$$uv^2x=a^ka^\ell a^\ell a^{n-k-\ell}b^na^n=a^{n+\ell}b^na^n\notin L$$
Y por lo tanto $L$ no es regular.

\vspace*{15pt}

As\'i el lema anterior \textit{no} permite demostrar que un lenguaje es 
regular, generalmente se usa como m\'etodo de demostraci\'on para probar que un 
lenguaje \textbf{no} es regular.
Veamos a continuaci\'on otro m\'etodo, el cual permite decidir si un lenguaje es 
regular.


\section{Lema de Myhill-Nerode}

La estrecha relaci\'on entre lenguajes regulares y aut\'omatas finitos invita a 
reflexionar de otra forma un m\'etodo para identificar lenguajes regulares.
Una manera de hacer lo anterior fue estudiada a trav\'es de la minimizaci\'on 
de aut\'omatas finitos al abstraer partes de una m\'aquina que reconcen cierto 
tipo de subcadenas, las clases de equivalencia de estados. 

El m\'etodo expuesto en esta secci\'on utiliza tambi\'en relaciones de 
equivalencias pero sin considerar \textit{a priori} una m\'aquina que acepte 
cierto lenguaje, sino que se analizar\'an las cadenas de un lenguaje para 
decidir si \'este es regular o no. 

\subsection{Relaciones de equivalencia sobre cadenas}

Consid\'erense las siguientes relaciones de equivalencia sobre $\Sigma^\star$
relacionadas a un lenguaje dado $L$ y a un aut\'omata finito determinista dado 
$M$, sean $u,v$ dos cadenas:
\bi
 \item $u\equiv_L v$ si y s\'olo si  \[\fa w\in\sest(uw\in L\Iff vw\in L)\]
  Se dice que $u,v$ son cadenas \textbf{indistinguibles} para $L$.

 \item $u\equiv_M v$ si y s\'olo si \[\dest(q_0,u)=\dest(q_0,v)\]
  Es decir, $u,v$ son cadenas \textbf{indistinguibles} seg\'un $M$.
\ei    

\paragraph{Ejemplo:}
Dado el lenguaje $L=\{a^i b^i \mid i\in\N\}$ sobre $\Sigma = \{a,b\}$ se tiene 
que $$ x\equiv_L y \text{ si y s\'olo si } \fa z\in\sest(xz\in L \Iff yz\in L)$$
\bi
 \item $ a^4b^3 \equiv_L a^3b^2$ pues 
  \[\fa z\in\sest ( a^4b^3 z \in L \Iff a^3b^2z\in L)\] 
  En particular con $z=b$, las concatenaciones est\'an en $L$. En otro caso no 
  lo est\'an pero se cumple la equivalencia.
 \item $a^2b^2\not\equiv_La^3b^2$ pues para $z=\vacia$ se tiene 
  \[a^2b^2z\in L \text{ y } a^3b^2z\notin L\]
 \item $a^4b \not\equiv_L a^3b^2$ pues para $z=b$, se tiene 
  \[a^4b^2z\notin L \text{ y }a^3b^2z\in L\]
 \item $abb \equiv_L baba$ pues ambas cadenas no pertenecen a $L$.
 \item La relaci\'on $\equiv_L$ tiene una infinidad de clases de equivalencia, 
  por ejemplo:  $$[\vacia],[a],[a^2],\ldots,[a^n],\ldots $$
  Todas estas clases son diferentes pues si $i\neq j$ entonces 
  $a^i\not\equiv_L a^j$ ya que si consideramos a $z=b^i$ entonces 
  $a^iz\in L$ pero $a^jz\notin L$.
\ei

\paragraph{Observaci\'on:} Por lo general no hay relaci\'on alguna entre un 
lenguaje~$L$ y un aut\'omata~$M$. M\'as a\'un, la relaci\'on $\equiv_L$ puede 
definirse para cualquier lenguaje $L$ a\'un cuando este no sea regular.
Sin embargo, en el caso particular en que $L=L(M)$ se cumple que $\equiv_M$ es 
un \textit{refinamiento} de $\equiv_L$:
$$\textbf{Proposici\'on:} \;\;\fa x,y\in\sest(x\equiv_My\to x\equiv_L y)$$
Esta proposici\'on nos deja ver la m\'as importante limitaci\'on de los 
aut\'omatas finitos, el hecho de que carecen de memoria m\'as all\'a de 
lo que recuerde el estado actual:
\begin{center}
si $x\equiv_M y$ entonces $x\equiv_{L(M)}y$, por lo que ninguna cadena $w$ 
procesada despu\'es de $x$ o $y$\\ permitir\'a que $M$ determine cu\'al de $x$ o 
$y$ se proces\'o anteriormente.
\end{center}

\vspace*{10pt}

Prepar\'emonos ahora para enunciar el siguiente m\'etodo:
\defin{
Una relaci\'on de equivalencia $\equiv$ sobre $\sest$ es invariante por la
derecha si y s\'olo si \[\fa x,y,w\in\sest(x\equiv y\to xw\equiv yw).\]
}
As\'i la relaci\'on $\equiv_L$ es invariante por la derecha.

\lema{[Lema de continuaci\'on]
Sean $x,y\in\sest$. Si $\dest(q_0,x)=\dest(q_0,y)$ entonces para cualquier 
$z\in\sest$, se cumple que \[\dest(q_0,xz)=\dest(q_0,yz)\]
}
Del lema anterior se sigue que la relaci\'on $\equiv_M$ es invariante por 
la derecha.

\paragraph{Propiedades de $\equiv_M$.}
Recordemos que el \'indice de una relaci\'on de equivalencia $\equiv$ es el 
n\'umero de clases de equivalencia generadas por $\equiv$. 
As\'i, dado un AFD $M=\pt{Q,\Sigma,q_0,\delta,F}$ se cumple lo siguiente:
\bi
\item La relaci\'on $\equiv_M$ es invariante por la derecha.
\item La relaci\'on $\equiv_M$ es de \'indice finito.
\item $L(M)$ es la uni\'on de algunas de las clases de equivalencia de la
  relaci\'on $\equiv_M$.
\ei

\teo{[Myhill-Nerode]
Sea $L\inc\sest$. Las siguientes condiciones son equivalentes:
\be
 \item $L$ es regular.
 \item Existe una relaci\'on de equivalencia $\equiv$ sobre $\sest$, invariante
  por la derecha y de \'indice finito, tal que $L$ es la uni\'on de algunas de 
  las clases de equivalencia de $\equiv$.
 \item La relaci\'on de equivalencia $\equiv_L$ tiene \'indice finito.
\ee
}

\vspace*{20pt}

El teorema anterior permite demostrar que un lenguaje $L$ \textbf{no} es 
regular al mostrar que $L$ no es de \'indice finito.
Es decir, basta ver que $\equiv_L$ tiene una infinidad de clases de 
equivalencia. Esto se hace expl\'icito mediante el siguiente lema que es una 
consecuencia directa del teorema de Myhill-Nerode.

\lema{[Lema del \'indice finito]
Sea $L\inc\sest$ un lenguaje regular infinito. Cualquier conjunto $S\inc\sest$
suficientemente grande contiene al menos dos cadenas distintas, $x,y\in S$
tales que $x\equiv_L y$. 
}

Las pruebas de no regularidad se sirven de la contrapositiva del lema del 
\'indice finito. Es decir, hallando un conjunto $S\inc\sest$ que no cumpla la 
propiedad del lema, habremos probado que $L$ no es regular. 
La siguiente definici\'on de conjuntos estafadores\footnote{En ingl\'es 
\textit{fooling set}.} servir\'a para encontrar los conjuntos que no cumplen 
la propiedad, as\'i para mostrar que un lenguaje $L$ \textbf{no} es regular 
basta construir un conjunto estafador para $L$.

\defin{
Un conjunto infinito $S\inc\sest$ es un conjunto estafador para $L$ si y s\'olo 
si para cualesquiera $x,y\in S$ existe una cadena $z\in\sest$ tal que una y 
s\'olo una de $xz$ y de $yz$ pertenece a $L$. 
Es decir, $S$ es un conjunto estafador para $L$ si y s\'olo si  
\[\fa x,y\in S\, (x\not\equiv_L y).\]
}

\paragraph{Ejemplo:}
Demostrar que $L=\{a^ib^i\;|\;i\in\N\}$ \textbf{no} es regular usando el 
Teorema de Myhill-Nerode.\\
Como dijimos antes, basta hallar un conjunto estafador para  
$L=\{a^nb^n\;|\;n\in\N\}$, es decir un conjunto de cadenas 
$S\subseteq\{a,b\}^\star$ tal que para las cadenas~$x,y\in S$ y $z\in\sest$ se 
tiene que $xz\in L$ y $yz \notin L$.
Sea $S=\{a^k\;|\;k\in\N\}$, veamos que $S$ es un conjunto estafador:
\begin{center}
si $a^i,a^j\in S$ con $i\neq j$ entonces claramente $a^ib^i\in 
L$ y $a^ib^j\notin L$. 
\end{center}
Por lo tanto $a^i\not\equiv_L a^j$ con $S$ es un conjunto estafador para $L$.

\paragraph{Ejemplo:}
Decidir si el lenguaje~$L=\{a^{i^2}\;|\;i\in\N\}$ es regular usando el 
Teorema de Myhill-Nerode.\\
Analizando informalmente, se puede ver que este lenguaje requiere 
memoria m\'as all\'a de la que ofrece un aut\'omta finito: se necesita 
contabilizar $i^2$.\\
Tratemos de encontrar un  conjunto estafador para el lenguaje.
Sea $S$ exactamente $L$, entonces sean $a^{i^2},a^{j^2}\in S$ con $j>i$. 
\bi
\item Por un lado tenemos que $a^{i^2}a^{2i+1}=a^{i^2+2i+1}=a^{(i+1)^2}\in
  L$
\item Por otra parte, $a^{j^2}a^{2i+1}=a^{j^2+2i+1}\notin L$ puesto que
  $j^2<j^2+2i+1<j^2+2j+1=(j+1)^2$.
\ei
Por lo tanto $a^{i^2}\not\equiv_La^{j^2}$ y $S$ es un conjunto estafador para 
$L$.
Podemos concluir que el lenguaje $L$ no es regular.

\vspace*{10pt}

Usar el teorema anterior para demostrar que un lenguaje es regular tambi\'en 
es posible al proporcionar una relaci\'on de equivalencia $\equiv$ invariante 
por la derecha y de \'indice finito, es decir, se debe dar $\equiv_L$ tal 
que es la uni\'on de algunas clases de equivalencia sobre $\Sigma^\star$.
Para ello se siguen los siguientes pasos:
\be
 \item Proponer unas clases de equivalencia que describan las cadenas en el 
  lenguaje, una de las clases debe ser la clase que contiene a la 
  cadena vac\'ia: $[\vacia]$.
  
 \item Mostrar que todas las clases antes propuestas se distinguen y son 
  invariantes por la derecha siguiendo la definici\'on.\\
  Para mostrar que dos clases se distinguen, se toman dos cadenas~$u,v$ una 
  en cada clase y se suponen equivalentes~$u\equiv_L v$. 
  Basta encontrar una cadena $w$ tal que $uw\in L$ y $vw\notin L$ para romper 
  la definici\'on y decir que las dos clases se distinguen.

 \item Crear un aut\'omata finito usando las clases anteriores de la siguiente 
  forma:
  \bi
   \item Los estados son las clases de equivalencia.
   \item El estado inicial es la clase $[\vacia]$.
   \item Los estados finales son aquellas clases que contienen a las cadenas 
    en el lenguaje.
   \item La funci\'on de transic\'on est\'a dada por:
    $$ \forall a \in \Sigma, \delta(q_{[w]}, a) = [w'] $$
    donde $w$ es el representante de la clase $[w]$ y $wa$ pertenece a la 
    clase~$[w']$.
  \ei
\ee

\paragraph{Ejemplo:}
Decidir si el lenguaje~$L=\{w\in\{a,b\}^\star \mid bb 
\text{ no es subcadena de } w \text{ y } |w|=2k, k\in\N\}$ es regular.\\
Despu\'es de un an\'alisis informal, podemos pensar que $L$ es regular dado 
que se deben identificar cadenas de longitud par y deshechar las cadenas que 
contengan $bb$.\\
Veamos c\'omo usar el Teorema de Myhill-Nerode para obtener un aut\'omata que 
reconozca este lenguaje.\\ 
Se debe mostrar que $\equiv_L$ es la uni\'on de algunas clases de equivalencia 
sobre $\{a,b\}^\star$. Se proponen las siguientes clases:
\bi
 \item La clase~$[\vacia]$ representa a las cadenas de longitud par sin la 
  subcadena~$bb$.
 \item La clase~$[ab]$ que representa a las cadenas de longitud par con 
  potencial para tener la subcadena~$bb$.
 \item La clase~$[bb]$ representa a las cadenas que contienen a la 
  subcadena~$bb$.
 \item La clase~$[a]$ engloba a las cadenas de longitud impar sin la 
  subcadena~$bb$.
 \item La clase~$[b]$ contiene a las cadenas de longitud impar con potencial 
  para tener a la subcadena~$bb$.
\ei
Estas cuatro clases separan a las cadenas en $\{a,b\}^\star$, de ellas s\'olo 
la clase $[\vacia]$ es la que contiene las cadenas en $L$.

\vspace*{10pt}

Para mostrar que las cuatro clases se distiguen se comparan dos a dos:
\bi
 \item Sean $\vacia \in [\vacia]$ y $bb\in [bb]$, suponer que 
  $\vacia \equiv_L bb$. Tomar $w= \vacia$, entonces $\vacia\vacia = \vacia$ 
  est\'a en el lenguaje pero $bb\vacia = bb$ no lo est\'a.
 \item Sean $ab \in [ab]$ y $bb\in [bb]$, suponer que 
  $ab \equiv_L bb$. Tomar $w= \vacia$, entonces $ab\vacia = ab$ 
  est\'a en el lenguaje pero $bb\vacia = bb$ no lo est\'a.
 \item Sean $ab \in [ab]$ y $b\in [b]$, suponer que 
  $ab \equiv_L b$. Tomar $w=\vacia$, entonces $ab\vacia = ab$ 
  est\'a en el lenguaje pero $b\vacia = b$ no lo est\'a.
 \item Sean $a\in [a]$ y $b \in [b]$, suponer que $a \equiv_L b$. 
  Sea $w=b$ entonces $ab$ est\'a en el lenguaje pero $bb$ no est\'a.
 \item Sean $\vacia \in [\vacia]$ y $a \in [a]$, suponer que 
  $\vacia \equiv_L a$. Tomar $w=b$ entonces $\vacia b = b$ no est\'a en $L$ 
  y $ab$ s\'i est\'a en $L$.
 \item Sean $b\in [b]$ y $bb\in [bb]$ y suponer que $b\equiv_L bb$. 
  Sea $w=a$ entonces $ba$ que pertenece a $L$ y $bba$ que no pertenece a $L$.
\ei 
Ahora crearemos el aut\'omata a trav\'es de la tabla de transiciones $\delta$, 
abajo est\'a su representaci\'on en diagrama:
\[
 \begin{array}{rrcc|c|c}
  wa \in w'& \hspace{10pt} wb \in w''& &\delta & a & b \\\hline
  \vacia a \in [a] & \vacia b \in [b] & \hspace{10pt} inicial,\;final & 
    q_{[\vacia]} & q_{[a]} & q_{[b]} \\
  aa \in [\vacia] & ab \in [ab] &  & 
    q_{[a]} & q_{[\vacia]} & q_{[ab]}\\
    aba \in [a] & abb \in [bb] & final & q_{[ab]} & q_{[a]} & q_{[bb]} \\
  ba \in [\vacia] & bb \in [bb] & & q_{[b]} & q_{[\vacia]} & q_{[bb]} \\
  bba \in [bb] & bbb \in [bb] & & q_{[bb]} & q_{[bb]} & q_{[bb]}
 \end{array}
\]

\begin{center}
\begin{tikzpicture}[node distance=3.5cm,every 
node/.style={scale=0.8},semithick]
    \node[state,initial,initial text=,accepting by double] (qe) 
      {$q_{[\vacia]}$};
    \node[state] (qa) [right of=qe] {$q_{[a]}$};
    \node[state] (qb) [below of=qe] {$q_{[b]}$};
    \node[state,accepting by double] (qab) [right of=qa] {$q_{[ab]}$};
    \node[state] (qbb) [right of=qb] {$q_{[bb]}$};
    \path[->] (qe) edge [bend right] node [above] {a} (qa);
    \path[->] (qe) [bend right] edge node [left] {b} (qb);
    \path[->] (qa) edge [bend right] node [above] {a} (qe);
    \path[->] (qa) edge [bend right] node [above] {b} (qab);
    \path[->] (qab) edge [bend right] node [above] {a} (qa);
    \path[->] (qab) edge  node [above] {b} (qbb);
    \path[->] (qb) edge [bend right] node [left] {a} (qe);
    \path[->] (qb) edge node [above] {b} (qbb);
    \path[->] (qbb) edge [loop right] node [right] {a,b} (qbb);
 \end{tikzpicture}
\end{center}


\end{document}
