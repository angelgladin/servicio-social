
\documentclass[letterpaper,11pt]{article}
\usepackage[includeheadfoot,margin=1.3in]{geometry}
\usepackage{anysize}

\usepackage[utf8]{inputenc}
% \usepackage[latin1]{inputenc}
\usepackage[english,spanish]{babel}
\usepackage{lmodern}   % font shapes...
\usepackage[T1]{fontenc} % join the compound symbols as a single symbol

\usepackage{amssymb,amsmath}
\usepackage{mathrsfs}
\usepackage{epsfig}

\usepackage{fancyhdr}
\usepackage{hyperref}
\usepackage{url}

\usepackage{import}
\usepackage{comment}
\usepackage[autostyle=true,spanish=mexican]{csquotes}

\usepackage{alltt}

\usepackage[section]{placeins}

\usepackage{pgf}
\usepackage{tikz}
\usetikzlibrary{automata,arrows,trees}
\usetikzlibrary{babel}
%\pagestyle{fancyplain}
\input{../macrosAyLF}

\title{Aut\'omatas y Lenguajes Formales 2017-1 \\ 
% Posgrado en Ciencia e Ingeniería de la Computación UNAM \\ 
Facultad de Ciencias UNAM \\ 
Nota de Clase 12} 
\author{Favio E. Miranda Perea \and A. Liliana Reyes Cabello \and
Lourdes Gonz\'alez Huesca}
\date{\today}

\begin{document}
\maketitle

Como hemos discutido, los lenguajes libres de contexto engloban lenguajes m\'as 
complejos o expresivos que los lenguajes regulares, sin embargo esta 
caracter\'istica hace que algunos problemas de decisi\'on sean m\'as 
dif\'iciles de resolver o incluso que no sean decidibles. \\
En esta nota abordaremos algunos problemas de decisi\'on sobre lenguajes libres 
de contexto.

\section{Ambig\"{u}edad}
Est\'a probado que no puede existir un algoritmo que determine con certeza si 
una gram\'atica es ambigua o no, y que en tal caso elimine dicha ambig\"{u}edad 
produciendo una gram\'atica no ambigua equivalente a la original. 

Es decir, el problema de ambig\"{u}edad es indecidible. Lo m\'as que se puede 
saber es que hay ciertas condiciones que determinan ambig\"{u}edad pero en
caso de no cumplirse estas nada puede decirse a cerca de la gram\'atica en
cuesti\'on.
En algunos casos, dada una gram\'atica ambigua, se puede encontrar otra
gram\'atica equivalente no ambigua, por ejemplo agregando precedencia de 
s\'imbolos u operadores y asociatividad. Sin embargo existen lenguajes cuya 
ambig\"{u}edad es inevitable.

\paragraph{Ejemplo:}
Un lenguaje compacto para expresiones aritm\'eticas es generado por la 
siguiente gram\'atica ambigua
$$ S \imp S + S \mid S*S \mid a $$
donde $a$ es un terminal que representa a los identificadores y constantes.\\

\noindent Podemos demostrar la ambig\"{u}edad de la gram\'atica con la cadena 
$a + a * a$ que tiene dos \'arboles de derivaci\'on: si se empieza con la 
producci\'on 
$S\imp S + S$ o con la producci\'on $S \imp S * S$.\\

\noindent Una gram\'atica no ambigua equivalente se obtiene modelando la 
precedencia de operadores como sigue:
\[
 \ba{rllc}
  S & \imp & S+T \mid T & \\
  T & \imp & T*F \mid F & \text{ la multiplicaci\'on tiene mayor precedencia }\\
  F & \imp & (S) \mid a & \text{ agregar par\'entesis limita la precedencia }
 \ea
\]


\subsection{Gram\'aticas libres de contexto en lenguajes de programaci\'on}

\noindent El estudio formal de los lenguajes de programaci\'on se divide en 
sintaxis, pragm\'atica y sem\'antica:
\bi
\item La sem\'antica se encarga de definir el significado de las expresiones, 
enunciados y unidades de programa.
\item La pragm\'atica define la implementaci\'on del lenguaje basada en la 
metodolog\'ia y estrategias de programaci\'on deseadas.
\item La sintaxis se encarga de definir la forma de las expresiones y 
enunciados de un lenguaje y se sirve fundamentalmente de los conceptos y 
herramientas de nuestro curso. 
\ei

Antes del proceso de evaluaci\'on, un compilador e int\'erprete necesita 
realizar los procesos de an\'alisis l\'exico y sint\'actico, los describimos a
continuaci\'on a grandes rasgos:
\bi
 \item El an\'alisis l\'exico se encarga de transformar el programa fuente en 
  una lista de unidades sint\'acticas de bajo nivel llamadas \textit{lexemas}, 
  los cuales se clasifican en distintas categor\'ias llamadas \textbf{tokens}, 
  como pueden ser identificadores, constantes, separadores, etc. \\
  El an\'alisis l\'exico se sirve fundamentalmente de expresiones regulares 
  para su definici\'on y reconocimiento.
 
 \item El an\'alisis sint\'actico o \textit{parsing} se encarga de transformar 
  la lista de lexemas en un programa objeto, el cual es una expresi\'on 
  v\'alida de la llamada sintaxis abstracta del lenguaje.\\
  El programa objeto es esencialmente un \'arbol de derivaci\'on dictado por 
  una gram\'atica libre de contexto que define al lenguaje de programaci\'on. 
\ei
Por lo tanto el an\'alisis sint\'actico es esencialmente una forma del problema 
de la pertenencia en gram\'aticas libres de contexto.


\paragraph{Forma de Backus-Naur} 
Las gram\'aticas libres de contexto para lenguajes de programaci\'on suelen
escribirse en la forma de Backus-Naur o \textbf{BNF}.
Este m\'etodo de definici\'on de gram\'aticas fue introducido por John 
Backus para el lenguaje \textsc{Algol} 58 en 1959 y fue mejorado por Peter 
Naur para la definici\'on de \textsc{Algol} 60.
Este sistema notacional para definir lenguajes libres de contexto sigue las 
siguientes convenciones:
\bi
 \item El s\'imbolo de reescritura $\imp$ se reemplaza con $::=$.
 \item El s\'imbolo $\mid$ significa \textit{\'o} y se usa para abreviar las 
  producciones de una misma variable.
 \item Las variables se escriben entre par\'entesis triangulares y por lo 
  general utilizan nombres largos que ayuden a la descripci\'on de las 
  categor\'ias del lenguaje.
\ei

\paragraph{Ejemplo:}
Lenguaje de par\'entesis balanceados 
\[
 \ba{rll}
  <parent\_balanc> & ::= & \vacia \mid  \\
                 & & (<parent\_balanc>) \mid  \\
                 & & <parent\_balanc><parent\_balanc> 
 \ea
\]
Por ejemplo la cadena~$()()()$ tiene dos \'arboles de derivaci\'on. \\
Una gram\'atica no ambigua equivalente es:
$\;\;S \imp \vacia \mid (S)S\;\; $ y en forma \textbf{BNF}:
\[
  <parent\_balanc>\; ::= \;\vacia \mid (<parent\_balanc>) <parent\_balanc> 
\]



\paragraph{Ejemplo:}
El lenguaje de las expresiones aritm\'eticas 
\[
 \ba{rll}
  <expr> & ::= & <expr>\,<op>\,<expr>\mid (<expr>)\mid <id>\\
  <op> & := & + \mid - \mid * \mid / \\
  <id> & := & a \mid b \mid c
 \ea
\]
Extendido para generar bloques de asignaci\'on de la forma:
\begin{center}
{{\tt begin}\;a {\tt :=} b/c\; ; \; b {\tt :=} a*(b+c)\;{\tt end}} 
\end{center}
\[
\ba{rll}
<programa> & ::= & {\tt begin}\;<sec\_enunc>\;{\tt end}\\ \\
<sec\_enunc> & ::= & <enunc>\mid<enunc>\,;\,<sec\_enunc>\\ \\
<enunc> & ::= & <id>\;{\tt :=}\;<expr>\\ \\
<expr> & ::= & <expr>\,<op>\,<expr>\mid (<expr>)\mid<id>\\ \\
<op> & := & +\mid - \mid * \mid/\\ \\
<id> & := & a\mid b\mid c
\ea
\]


\paragraph{Ejemplo:}
El lenguaje de las expresiones condicionales:
\[
 \ba{rll}
  <enunc> & ::= & <condicional> \mid \dots \\
  <condicional> & := & {\tt if}\;<expr>\;{\tt then}\;<enunc> \mid \\
		&  & {\tt if}\;<expr>\;{\tt then}\;<enunc> {\tt else}\;<enunc>
 \ea
\]
La ambig\"{u}edad de este lenguaje se presenta con problema llamado del 
\texttt{if} colgante:
\begin{verbatim}
if false then if false then 0 else 1
\end{verbatim}
Este c\'odigo tiene dos significados:
\begin{verbatim}
if false then (if false then 0) else 1
if false then (if false then 0 else 1)
\end{verbatim}

\noindent Una gram\'atica equivalente no ambigua es:
\[
 \ba{rll}
  S & \imp & C \mid O  \\
  C & \imp & C1\mid C2 \\
  C1 & \imp & {\tt if}\;E\;{\tt then}\;C1\;{\tt else}\;C1  \\
  C2 & \imp   & {\tt if}\;E\;{\tt then}\;S \mid 
    {\tt if}\;E\;{\tt then}\;C1\;{\tt else}\;C2
 \ea
\]
La idea detr\'as de esta gram\'atica es separar los posibles condicionales 
anidados:
\bi
 \item $C1$ genera condicionales dobles (if-then-else) balanceados 
 \item $C2$ representa condicionales simples (if-then) y condicionales dobles 
  pero de forma que un $\mathtt{if-then}$ s\'olo figura colgando al final es 
  decir en el $\mathtt{else}$.
\ei


\section{Problema de pertenencia}

\begin{center}
Dado un lenguaje libre de contexto $L$ y una cadena~$w \in \Sigma^\star$, 
?`c\'omo decidir si $w\in L$?
\end{center}

Para resolver esta pregunta, por un lado se puede generar un aut\'omata 
no-determinista que reconozca el lenguaje, en este caso las derivaciones que 
pueden generarse son exponenciales. 
 
Por otro lado si se utilizara una gram\'atica libre de contexto, se 
puede optar por normalizarla y obtener una equivalente en forma normal de 
Chomsky o de Greibach. As\'i se pueden generar todas la derivaciones de 
longitud $2|w| -1 $ y obtener una derivaci\'on que produzca exactamente a la 
cadena~$w$.

Pero en general, si se quiere resolver el problema de decisi\'on anterior de 
manera eficiente las opciones anteriores no son las mejores.

\vspace*{10pt}

En la d\'ecada de 1960 se desarrollaron varias aportaciones a los lenguajes de 
programaci\'on y a los compiladores de los mismos, entre ellas se encuentra un 
algoritmo dise\~nado por J. Cocke, T. Kasami y D.H. Younger para realizar el 
an\'alisis sint\'actico de una cadena dado un lenguaje. 

Este algoritmo llamado \textbf{CYK} pertenece a la programaci\'on din\'amica 
cuyo lema es \enquote{divide y vencer\'as}: utiliza una gram\'atica libre de 
contexto normalizada y realiza un an\'alisis de abajo hacia arriba 
(\textit{bottom-up}) para revisar la cadena y las subcadenas de ella.

Este algoritmo es sobresaliente dada la eficiencia bajo ciertas
caracter\'isticas: en el peor de los casos tiene un tiempo  
de~$\Theta(n^3\cdot |V|)$ donde $n$ es la longitud de la cadena a procesar y 
$|V|$ es el n\'umero de variables en la gram\'atica. 

\vspace*{10pt}

Formalemente, el algoritmo recibe como entrada una gram\'atica en forma 
normal de Chomsky y una cadena $w =a_1a_2\cdots a_n$. El procedimiento analiza 
la cadena por partes, es decir, las subcadenas y reglas que pueden generar 
esas subcadenas.\\
Para este an\'alisis se forman conjuntos $X_{i,j}$, que continen s\'imbolos no 
terminales correspondientes a la parte izquierda de una producci\'on que genera 
la subcadena:
\bc
  $A \in X_{i,j}$ si $A \imp^+ a_ia_{i+1}\cdots a_{i+j-1}$ 
\ec
Si el conjunto $X_{1,n}$ correspondiente a la longitud de la cadena contiene al 
s\'imbolo inicial de la gram\'atica entonces se puede decir que la cadena $w$ 
pertenece al lenguaje generado por la gram\'atica.

\vspace*{10pt}

\noindent El pseudo-algoritmo para generar los conjuntos $X_{i,j}$ es el 
siguiente:
\be
 \item Formar una cuadr\'icula o una matriz de $n\times n$:
 \bi
  \item cada espacio de la matriz ser\'a el conjunto $X_{i,j}$ correspondiente 
  a sus coordenadas, los renglones ($j$) est\'an enumerados de abajo hacia 
  arriba y las columnas ($i$)de izquierda a derecha ambos desde $1$ hasta 
  $n$.
  \item en la parte inferior de la cuadr\'icula colocar por cada columna un 
    s\'imbolo de la cadena 
  \item el rengl\'on $1$ (el que est\'a m\'as abajo) es llenado con los 
    s\'imbolos no terminales que puedan generar el s\'imbolo que se encuentra 
    debajo de cada celda.
  \ei
\item Para cada una de las celdas de los siguientes renglones ($X_{i,j}$ con 
  $i,j >1$) agregar los s\'imbolos $A$ de las producciones $A \imp BC$ donde
  $B \in X_{i,k}$ y $C \in X_{i+k,j-k}$ y $k < j-1$.
\ee


\paragraph{Ejemplo:}
Dada la cadena $w = bbab$ y la gram\'atica $G$ en \textbf{FNC}, decir si 
$w\in L(G)$.
\[
 \ba{rll}
  S & \imp & BA \mid AC \\
  A & \imp & CC \mid b \\
  B & \imp & AB \mid a \\
  C & \imp & BA \mid a \\
 \ea
\]
\be
\item Inicializamos la tabla con la longitud de la cadena y el rengl\'on 
$j=1$ con las variables que generan los s\'imbolos
\bc
\begin{tabular}{c||c|c|c|c}
$j=4$ & & & & \\
\hline
$j=3$ & & & & \\
\hline
$j=2$ &  & & & \\  
\hline
$j=1$ & A &A &B, C &A \\
\hline
\hline
& $i=1$ & $i=2$ &$i=3$ &$i=4$ \\
\hline
 & b & b & a & b 
\end{tabular}
\ec

\item Para $j=2$ tenemos las producciones para subcadenas de longitud dos 
\bc
\begin{tabular}{c||c|c|c|c}
$j=4$ & & & & \\
\hline
$j=3$ & & & & \\
\hline
$j=2$ & - &B, S &S, C & \\  
\hline
$j=1$ & A &A &B, C &A \\
\hline
\hline
& $i=1$ & $i=2$ &$i=3$ &$i=4$ \\
\hline
 & b & b & a & b 
\end{tabular}
\ec

\item Para $j=3$ tenemos las producciones para subacadenas de longitud tres, en 
el caso particular del conjunto $X_{2,3}$ se utilizan los conjuntos por debajo 
de \'el que generan las subcadenas:
\bc
\begin{tabular}{c||c|c|c|c}
$j=4$ & & & & \\
\hline
$j=3$ & B& \textbf{S, C} & & \\
\hline
$j=2$ & - &\textbf{B, S} & \textbf{S, C} & \\  
\hline
$j=1$ & A &\textbf{A} &B, C &\textbf{A} \\
\hline
\hline
& $i=1$ & $i=2$ &$i=3$ &$i=4$ \\
\hline
 & b & b & a & b 
\end{tabular} 
\ec


\item Y finalmente la tabla completa
\bc
\begin{tabular}{c||c|c|c|c}
$j=4$ & \textbf{S, C} & & & \\
\hline
$j=3$ & \textbf{B }&\textbf{S, C} & & \\
\hline
$j=2$ & \textbf{-} &B, S &\textbf{S, C} & \\  
\hline
$j=1$ & \textbf{A} &A &B,  C &\textbf{A} \\
\hline
\hline
& $i=1$ & $i=2$ &$i=3$ &$i=4$ \\
\hline
 & b & b & a & b 
\end{tabular}
\ec

\ee
En el conjunto $X_{1,4}$, que representa la cadena completa, est\'a contenido 
el s\'imbolo inicial de la gram\'atica y por lo tanto $w$ pertenece a 
$L(G)$.
Para poder obtener una derivaci\'on de $w$ s\'olo es necesario tomar los 
s\'imbolos correspondientes que generan las subcadenas:
$$ S \imp  AC  \imp aC \imp  bBA \imp bABA\imp  bbBA \imp  bbaA  \imp  bbab $$

\vspace*{10pt}

\noindent Ahora analicemos la cadena $w= bbaa$, tenemos las siguientes tablas:
\be
\item Inicializamos la tabla con $j=1$
\bc
\begin{tabular}{c||c|c|c|c}
$j=4$ & & & & \\
\hline
$j=3$ & & & & \\
\hline
$j=2$ &  & & & \\  
\hline
$j=1$ & A &A &B, C &B, C \\
\hline
\hline
& $i=1$ & $i=2$ &$i=3$ &$i=4$ \\
\hline
 & b & b & a & a 
\end{tabular}
\ec

\item Para $j=2$ tenemos las producciones para subcadenas de longitud dos 
\bc
\begin{tabular}{c||c|c|c|c}
$j=4$ & & & & \\
\hline
$j=3$ & & & & \\
\hline
$j=2$ & - &B, S &A & \\  
\hline
$j=1$ & A &A &B, C &B, C \\
\hline
\hline
& $i=1$ & $i=2$ &$i=3$ &$i=4$ \\
\hline
 & b & b & a & a 
\end{tabular}
\ec


\item Para $j=3$ tenemos las producciones para subacadenas de longitud tres 
\bc
\begin{tabular}{c||c|c|c|c}
$j=4$ & & & & \\
\hline
$j=3$ & B& - & & \\
\hline
$j=2$ & - &B, S &A& \\  
\hline
$j=1$ & A &A &B, C &B, C \\
\hline
\hline
& $i=1$ & $i=2$ &$i=3$ &$i=4$ \\
\hline
 & b & b & a & a
\end{tabular} 
\ec

\item Y finalmente la tabla completa
\bc
\begin{tabular}{c||c|c|c|c}
$j=4$ & - & & & \\
\hline
$j=3$ & B&- & & \\
\hline
$j=2$ & - &B, S &A & \\  
\hline
$j=1$ & A &A &B,  C &B,C \\
\hline
\hline
& $i=1$ & $i=2$ &$i=3$ &$i=4$ \\
\hline
 & b & b & a & a
\end{tabular}
\ec
\ee
Como podemos ver el conjunto $X_{1,4}$ es vac\'io, por lo tanto la cadena no es 
aceptada por la gram\'atica.




\section{Lenguajes que no son libres de contexto}

An\'alogamente al caso de lenguajes regulares, las dos formas principales para 
demostrar que un lenguaje no es libre de contexto son el uso de las propiedades 
de cerradura y el lema del bombeo, el cual enunciamos m\'as adelante.\\
Recordemos las propiedades de cerradura de lenguajes libres de contexto:
\bi
 \item Uni\'on: si $L_1,L_2$ son lenguajes libres del contexto entonces 
  $L_1\cup L_2$ es libre del contexto.
 \item Concatenaci\'on: si $L_1,L_2$ son lenguajes libres del contexto entonces 
  $L_1L_2$ es libre del contexto.
 \item Estrella de Kleene: si $L_1$ es un lenguaje libres del contexto entonces 
  $L_1^\star$ es libre del contexto.
\ei


Estas propiedades permiten construir lenguajes libres de contexto y tambi\'en 
proveen un mecanismo para demostrar si alguno lo es. El siguiente lema es 
\'util al \textbf{refutar} que un lenguaje sea libre de contexto.

\lema{[Lema del Bombeo para Lenguajes libres de contexto]
Si $L$ es un lenguaje libre de contexto entonces existe un n\'umero $n \in \N$ 
llamado constante de bombeo para $L$, tal que para cualquier cadena
$z\in L$ con $|z| \geq n$ existen cadenas $u,\; v,\; w, \;x, \;y$ tales que
$z = uvwxy$ y adem\'as
\bi
 \item $|vwx| \leq n$
 \item $|vx| \geq 1$, es decir $v \neq \vacia$ \'o $x \neq \vacia$ pero no ambas
 \item para todo $i \geq 0$, $uv^iwx^iy$ debe pertenecer al lenguaje
\ei
}
\vspace*{-10pt}
\begin{proof}
 La demostraci\'on se basa en un an\'alisis de \'arboles de derivaci\'on de 
  gram\'aticas en forma normal de Chomsky.
\end{proof}

Veamos algunos ejemplos del uso del lema del bombeo:

\paragraph{Ejemplo:}
Sea $L = \{a^i b^i c^i \mid i \in \N\}$. Demostrar que $L$ \textbf{no} es libre 
de contexto. \\
Sup\'ongase que $L$ es libre del contexto y sea $n$ una constante de bombeo.
Entonces sea $z = a^n b^n c^n$ una palabra en $L$ y damos su descomposici\'on 
como sigue: $z = uvwxy$ con $vx \neq \vacia$ y $ |vwx| \leq n$.
Analicemos la cadena $z$:
\bi
 \item Como $|vwx| \leq n$ entonces en $vwx$ \textbf{no} pueden figurar los 
  tres terminales $a,\,b,\,c $ simult\'aneamente.
 \item Por otro lado como $v \neq \vacia$ \'o $x\neq \vacia$ se distinguen dos 
  casos:
  \bi
   \item En alguna de $v$ \'o $x$ figuran dos tipos de terminales.\\
   En tal caso en $uv^2wx^2 y $ figuran algunas $b$'s seguidas de $a$'s \'o
   algunas $c$'s seguidas de $b$'s rompiendo con la forma de las cadenas 
   en~$L$. 
  \item Cada una de las cadenas $v$ y $x$ contiene s\'olo un tipo de terminal. 
   Dado que en $vwx$ no figuran los tres terminales, en la cadena $uv^2 wx^2y$ 
   se altera el n\'umero de dos de los terminales pero nunca de los tres, por 
   lo que $uv^2 wx^2y$ no est\'a en $L$.
  \ei
\ei
De cualquier forma al \enquote{bombear} la cadena, el resultado no es parte del 
lenguaje. Por lo tanto $L$ no es libre de contexto.

\begin{comment}
 L = {ai2
 | i ∈ N } no es libre de contexto
Supongamos lo contrario y sea n una constante de bombeo.
Considérese la cadena z = an2
 . Por el lema del bombeo tenemos z = uvwxy tal que |vx| = q ≥ 1 (pues
v 6= ε ó x 6= ε).
Por el lema del bombeo se tiene uv n+1 wxn+1y ∈ L, es decir uvn+1wxn+1 y = ap2
 con p ∈ N. Pero
obsérvese que
p2 = |uv n+1wxn+1y| = |uvwxy| + |vn xn| = n2 + nq = n(n + q)
lo cual es absurdo pues q ≥ 1. Por lo tanto L no es libre de contexto.
2.0.3.
 L = {ww | w ∈ {0, 1}? } no es libre de contexto
Supongase que L es libre del contexto y sea n una constante de bombeo.
Considérese la palabra z = 0n+1 1n+1 0n+1 1n+1
Por el lema del bombeo tenemos z = uvwxy con |vwx| ≤ n, v 6= ε ó x 6= ε
Además z0 = uv0 wx0 y = uwy ∈ L
Basta mostrar que z0 no puede ser escrita en la forma ss.
• Al borrar v y x se está borrando al menos un sı́mbolo pues v 6= ε ó x 6= ε.
• Como |vwx| ≤ n la distancia entre v y x es a lo más n.
• Analizando los tres casos posibles es imposible escribir a z0 en la forma ss. 
(El lector puede
convencerse mediante casos particulares, por ejemplo con n = 3, la 
generalizacion es clara a
partir de este caso.)

\end{comment}

\end{document}