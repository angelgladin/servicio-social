
\documentclass[letterpaper,11pt]{article}
\usepackage[includeheadfoot,margin=1.3in]{geometry}
\usepackage{anysize}

\usepackage[utf8]{inputenc}
% \usepackage[latin1]{inputenc}
\usepackage[english,spanish]{babel}
\usepackage{lmodern}   % font shapes...
\usepackage[T1]{fontenc} % join the compound symbols as a single symbol

\usepackage{amssymb,amsmath}
\usepackage{mathrsfs}
\usepackage{epsfig}

\usepackage{fancyhdr}
\usepackage{hyperref}
\usepackage{url}

\usepackage{import}
\usepackage{comment}
\usepackage[autostyle=true,spanish=mexican]{csquotes}

\usepackage{alltt}

\usepackage[section]{placeins}

\usepackage{pgf}
\usepackage{tikz}
\usetikzlibrary{automata,arrows,trees}
\usetikzlibrary{babel}
%\pagestyle{fancyplain}
\input{../macrosAyLF}

\title{Aut\'omatas y Lenguajes Formales 2017-1 \\ 
% Posgrado en Ciencia e Ingeniería de la Computación UNAM \\ 
Facultad de Ciencias UNAM \\ 
Nota de Clase 8} 
\author{Favio E. Miranda Perea \and A. Liliana Reyes Cabello \and
Lourdes Gonz\'alez Huesca}
\date{\today}

\begin{document}
\maketitle


Los \textbf{lenguajes libres de contexto} especializan o refinan a los 
lenguajes regulares, es decir se pueden describir cadenas m\'as expresivas que 
en su conjunto establecen lenguajes m\'as complejos como lenguajes de 
programaci\'on de alto nivel u otros lenguajes formales.

Estos lenguajes ser\'an estudiados comenzando con las gram\'aticas que los 
definen para estudiar las caracter\'isticas que los distinguen de los 
lenguajes regulares. 

Hacemos \'enfasis en que la descripci\'on de un lenguaje por medio de una 
gram\'atica permite que la descripci\'on y an\'alisis del lenguaje sea claro.


\section{Gram\'aticas Libres de Contexto}

\defin{
Una \textbf{gram\'atica es libre o independiente del contexto} si todas sus 
producciones 
son de la forma
$$ A\imp\alpha$$
con $A\in V,\;\alpha \in(V\cup T)^\star$.
Se incluye a la regla $S\imp\vacia$.
}
 
\paragraph{Ejemplo:} 
Las siguientes son gram\'aticas libres de contexto, en forma compacta est\'an 
representadas por expresiones regulares:
\bi
 \item $L = a^\star$  $$ S\imp aS \mid \vacia$$
 \item $L = a^\star b^\star$ 
  \[
    \ba{rll}
      S & \imp & aS \mid bA \mid \vacia\\
      A & \imp & bA \mid b \mid \vacia
    \ea
  \]
 \item $L = 0^+1^+$
  \[
   \ba{rll}
   S & \imp & CU\\
   C & \imp & 0C \mid 0 \\
   U & \imp & 1U \mid 1
   \ea
  \]
\ei

\paragraph{Ejemplo:}
Los siguientes ejemplos son lenguajes con una gram\'atica libre de contexto que 
los genera:
\bi
 \item $L=\{a^nba^m \mid n,m\geq 1\} = a^+ba^+$
  \[
   \ba{rll}
   S & \imp & aS \mid aB\\
   B & \imp & bC \\
   C & \imp & aC \mid a
   \ea
   \]
 \item $L = \{a^n b^n \mid n\in\N\}$  (no es regular)
   $$ S\imp aSb \mid \vacia $$
 \item $L = \{w\in\{a,b\}^\star \mid w=w^R\}$ (no es regular)
   $$S\imp aSa \mid bSb \mid a \mid b \mid \vacia$$
\ei

De los ejemplos anteriores podemos atestiguar que los lenguajes libres 
de contexto permiten expresar lenguajes m\'as refinados como $a^nb^n$. 
Tambi\'en podemos observar que en particular toda gram\'atica regular es libre 
de contexto.

\subsection{Derivaciones y \'arboles}

Una derivaci\'on formal usando gram\'aticas libres de contexto puede ser de 
dos tipos, por la izquierda o por la derecha, dependiendo de la forma en que se 
reescriben los s\'imbolos no terminales.

\defin{
Una derivaci\'on $S\imp^\star w$ es una \textbf{derivaci\'on por la izquierda} 
si en cada paso se reescribe la variable m\'as a la izquierda en la palabra.
}

\defin{
Una derivaci\'on $S\imp^\star w$ es una \textbf{derivaci\'on por la derecha} 
si en cada paso se reescribe la variable m\'as a la derecha en la palabra.
}
\paragraph{Ejemplo:} En la gram\'atica 
$$  S\imp aAs \mid a\qquad \qquad A\imp SbA \mid SS \mid ba $$
\be
 \item Tenemos la siguiente derivaci\'on por la izquierda de la cadena~$aabbaa$
\[
 \mathbf{S}\imp a\mathbf{A}S\imp a\mathbf{S}bAS\imp aab\mathbf{A}S\imp 
  aabba\mathbf{S}\imp aabbaa 
\]
  \item Tenemos la siguiente derivaci\'on por la derecha de $aabbaa$
\[
\mathbf{S}\imp aA\mathbf{S}\imp a\mathbf{A}a\imp aSb\mathbf{A}a\imp 
a\mathbf{S}bbaa\imp aabbaa   
\]
\ee

Por otro lado tambi\'en se puede verificar que una cadena es generada por una 
grama\'atica usando \'arboles de derivaci\'on o \'arboles sint\'acticos.
Este mecanismo es muy \'util para representar las derivaciones de gram\'aticas 
libres de contexto.

El uso de estas estructuras tienen beneficios tales como su utilizaci\'on en 
compiladores, en espec\'ifico para el an\'alisis sint\'actico de 
programas fuente o \textit{parsing} y sirven de base para la generaci\'on de 
c\'odigo. 

\subsection{Constucci\'on de \'arbol es de derivaci\'on}
Dada una gram\'atica libre de contexto~$G=\pt{V,T,S,P}$, un \'arbol de  
derivaci\'on en~$G$ se construye como sigue:
\be
 \item La ra\'iz contiene al s\'imbolo inicial~$S$.
 \item Cada nodo interior contiene una variable.
 \item Cada hoja contiene un s\'imbolo de $V\cup T\cup\{\vacia\}$.
 \item Si un nodo interior contiene una variable~$A$ entonces sus hijos 
  contienen s\'imbolos (de izquierda a derecha)~$a_1,\ldots,a_n$ si y s\'olo 
  si~$A\imp a_1a_2\ldots a_n$ est\'a en~$P$.
 \item La palabra generada se puede reconstriur al leer las hojas de izquierda 
  a derecha.
\ee

\section{Ambig\"{u}edad}
Puede ser que dos derivaciones distintas tengan un mismo \'arbol de 
derivaci\'on y tambi\'en puede suceder que haya m\'as de un \'arbol de 
derivaci\'on para una cadena. Lo ideal es que cada cadena tenga s\'olo un 
\'arbol asociado. Si sucede lo anterior, implica que el lenguaje 
\textbf{no es ambiguo}. Desafortunadamente existen lenguajes ambiguos.
Veamos una definici\'on formal de este fen\'omeno:

\defin{
Una gram\'atica se dice \textbf{ambigua} si existe una palabra~$w$ con dos o 
m\'as \'arboles de derivaci\'on distintos.
En general una palabra puede tener m\'as de una derivaci\'on, pero  
\textnormal{un s\'olo \'arbol} y en tal caso no hay ambig\"{u}edad.
}

Algunas veces se puede suprimir la ambig\"{u}edad directamente. Sin embargo 
\textbf{no} hay un algoritmo para remover ambig\"{u}edad.
As\'i tambi\'en hay algunos lenguajes cuya ambig\"{u}edad es inevitable.

\vspace*{-7pt}

\paragraph{Ejemplo:}
Considere la siguiente gram\'atica: 
$$ \qquad \qquad S\imp AA \quad A\imp aSa \mid a $$

\noindent La palabra $a^5$ tiene las siguientes derivaciones diferentes y por 
la izquierda:
\bi
 \item[] $S\imp AA\imp aA \imp aaSa\imp aaAAa\imp aaaAa\imp aaaaa$
 \item[] $S\imp AA\imp aSaA \imp aAAaA\imp aaAaA\imp aaaaA\imp aaaaa $
\ei
\noindent Las dos derivaciones generan \'arboles distintos:


\begin{small}
\begin{minipage}{.4\textwidth}
  \centering
  \begin{tikzpicture}[level distance=12mm]
  \tikzstyle{level 1}=[sibling distance=20mm]
  \tikzstyle{level 2}=[sibling distance=10mm]
 
  \node {$S$}
    child {node {$A$}
      child {node {$a$}}
     }
      child {node {$A$}
        child {node {$aSa$}}
        child {node {$AA$}
	  child {node {$a$}}
	  child {node {$a$}}
	}
    };
 \end{tikzpicture}
 \end{minipage}
\begin{minipage}{.4\textwidth}
 \centering
 \begin{tikzpicture}[level distance=12mm]
  \tikzstyle{level 1}=[sibling distance=20mm]
  \tikzstyle{level 2}=[sibling distance=10mm]
 
  \node {$S$}
    child {node {$A$}
      child {node {$aSa$}
	child {node {$AA$}
	  child {node {$a$}}
	  child {node {$a$}}
	}
     }
    }
      child {node {$A$}
        child {node {$a$}}
    };
 \end{tikzpicture}
 \end{minipage}
\end{small}


\subsection{Lenguajes Ambiguos}

\defin{
Un lenguaje~$L$ es ambiguo si existe una gram\'atica ambigua $G$ que 
genera a $L$.\\
Un lenguaje es \textbf{inherentemente} ambiguo si \textit{todas} las 
gram\'aticas que lo generan son ambiguas.
}

\paragraph{Ejemplo:}
El lenguaje~$L = \{a^{2+3i} \mid i\geq 0\}$ es ambiguo por ser generado por la 
siguiente gram\'atica ambigua:
$$ S\imp AA \qquad \qquad A\imp aSa \mid a $$
Sin embargo este lenguaje tambi\'en es generado por una gram\'atica no ambigua:
$$ S\imp aa \mid aaU \qquad \qquad U\imp aaaU \mid aaa $$
En este caso la derivaci\'on de $a^5$ es:
$$S\imp aaU\imp aaaaa$$
y por lo tanto $L$ no es un lenguaje inherentemente ambiguo.

\paragraph{Ejemplo:}
El lenguaje 
$$L = \{a^nb^nc^md^m \mid n,m\geq 1\}\cup\{a^nb^mc^md^n \mid n,m\geq 1\}$$
es generado por la gram\'atica:
\[
 \ba{lll}
  S\imp AB \mid C & \qquad A\imp aAb \mid ab &\qquad B\imp cBd \mid cd \\
  C\imp aCd \mid aDd &\qquad  D\imp bDc \mid bc & 
 \ea
\]
La cadena~$aabbccdd$ tiene dos derivaciones por la izquierda:
\bi
\item[] $S\imp AB\imp aAbB\imp aabbB\imp aabbcBd\imp aabbccdd$
\item[] $S\imp C\imp aCd\imp aaDdd\imp aabDcdd\imp aabbccdd$
\ei

Como se coment\'o antes, la ambig\"{u}edad no puede ser eliminada con un 
m\'etodo y en general, probar la ambig\"{u}edad inherente es complicado.  


\section{Propiedades de Cerradura}

Estudiemos ahora a los lenguajes libres de contexto, la clase de estos
lenguajes es cerrada bajo las siguientes operaciones:
\bi
 \item Uni\'on: si $L_1,L_2$ son lenguajes libres del contexto entonces 
  $L_1\cup L_2$ es un lenguaje libre del contexto.
 \item Concatenaci\'on: si $L_1,L_2$ son lenguajes libres del contexto entonces 
  $L_1L_2$ es un lenguaje libre del contexto.
 \item Estrella de Kleene: si $L_1$ es un lenguaje libres del contexto entonces 
  $L_1^\star$ es un lenguaje libre del contexto.
\ei

Veamos m\'as a detalle estas operaciones:
\bi
 \item Cerradura bajo la uni\'on\\
  Si  $G_1=\pt{V_1,T,S_1,P_1},\; G_2=\pt{V_2,T,S_2,P_2}$ son dos 
  gram\'aticas libres de contexto donde $L_1=L(G_1)$ y $L_2=L(G_2)$ 
  entonces $L_1\cup L_2 = L(G)$ y $G$ es la gram\'atica
  $$ G=\pt{V_1\cup V_2\cup\{S\},T,S,P} $$
  y las reglas de producci\'on est\'an dadas por $P_1\cup P_2$ m\'as las 
  producciones:
  $$ S\imp S_1 \qquad S\imp S_2 $$

 \item Cerradura bajo la concatenaci\'on\\
  Si  $G_1=\pt{V_1,T,S_1,P_1},\; G_2=\pt{V_2,T,S_2,P_2}$ son dos gram\'aticas 
  libres de contexto donde $L_1=L(G_1)$ y $L_2=L(G_2)$ entonces $L_1L_2=L(G)$ 
  donde $G$ es la gram\'atica 
  $$ G=\pt{V_1\cup V_2\cup\{S\},T,S,P} $$
  y $P$ est\'a dado por $P_1\cup P_2$ m\'as la producci\'on: 
  $$ S\imp S_1S_2$$

 \item Cerradura bajo la estrella de Kleene\\
  Si $G_1=\pt{V_1,T,S_1,P_1}$ es una gram\'atica libre de contexto con  
  $L_1=L(G_1)$ entonces $L_1^\star=L(G)$ donde $G$ es la gram\'atica
  $$G=\pt{V_1\cup\{S\},T,S,P}$$
  y $P$ est\'a dado por $P_1$ m\'as la producciones:
  $$S\imp S_1S_1 \qquad S\imp\vacia $$
\ei

La intuici\'on puede traicionar y algunas propiedades de cerradura no son 
v\'alidas. En general las siguientes propiedades \textbf{no son v\'alidas} para 
lenguajes libres del contexto.
\bi
 \item Cerradura bajo la intersecci\'on.
 \item Cerradura bajo el complemento.
 \item Cerradura bajo la diferencia.
\ei
Estudiemos a detalle estas cerraduras:
\bi
 \item Intersecci\'on\\
  La intersecci\'on de dos lenguajes libres de contexto puede ser un lenguaje 
  que \textbf{no} es libre del contexto.\\
  Por ejemplo consid\'erense $L_1=\{a^ib^ic^j \mid i,j\geq 1\}$ libre del 
  contexto:
$$  S\imp AB,\;A\imp aAb \mid ab,\;B\imp cC \mid c $$
y $L_2=\{a^ib^jc^j \mid i,j\geq 1\}$ tambi\'en libre del contexto:
$$  S\imp AB,\;A\imp aA \mid a,\;B\imp bBc \mid bc$$ 
Pero $L_1\cap L_2=\{a^ib^ic^i \mid i\geq 1\}$ no es independiente del 
contexto.

\item Complemento\\
Si el complemento de un lenguaje libre de contexto~$L$, $\bar{L}$ 
fuera tambi\'en libre del contexto entonces la intersecci\'on tambi\'en lo 
ser\'a pues:
$$  L_1\cap L_2 = \overline{\overline{L_1}\cup\overline{L_2}}$$
 
\item Diferencia\\
Si la diferencia fuera un lenguaje libre de contexto, entonces tambi\'en lo 
ser\'a el complemento pues:
$$  \overline{L}=\sest - L $$

\ei 

\end{document}
