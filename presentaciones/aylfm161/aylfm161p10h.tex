\documentclass[xcolor=dvipsnames]{beamer}
%\documentclass[xcolor=dvipsnames,handout]{beamer}

\input{macros2006}

\newcommand{\titulos}[2]{\frametitle{#1}\framesubtitle{#2}}
\newcommand{\espc}{\vspace{.5cm}}
\newcommand{\U}{\mathcal{U}}
\newcommand{\cv}{\ensuremath{\varepsilon}}
\newcommand{\sest}{\ensuremath{\Sigma^\star}}
\newtheorem{teorema}{Teorema}
\renewcommand{\S}{\Sigma}
\newcommand{\de}{\delta}
\newcommand{\izq}{\leftarrow}
\newcommand{\der}{\rightarrow}
\newcommand{\dest}{\ensuremath{\delta^\star}}


\mode<presentation>
{
\usetheme[secheader]{Boadilla}
%\useinnertheme{rounded}
\useoutertheme{infolines}
%\usepackage{BeamerColor}
%\usecolortheme[RGB={33,66,33}]{structure}
\usecolortheme[named=NavyBlue]{structure}
%\usefonttheme{}
\setbeamercovered{invisible}
%\setbeamertemplate{headline}{\insertshortinstitute}
%\setbeamertemplate{footline}[page number]{\insertshortinstitute}
%\insertshortinstitute
}

\usepackage[spanish]{babel}
\usepackage{times}
\usepackage[T1]{fontenc}


\title[AyLF MCIC 2016-1]{Autómatas y Lenguajes Formales 2016-1}
\subtitle
{Maestría en Ciencia e Ingeniería de la Computación UNAM\\
  Tema 10:  Ambigüedad y gramáticas libres de contexto en lenguajes de programación}

\author[Favio E. Miranda Perea]{Dr. Favio Ezequiel Miranda Perea \\ \texttt{favio@ciencias.unam.mx}}


\institute[FC UNAM]{Facultad de Ciencias UNAM}

\date{\today}

\subject{Theoretical Computer Science}

% \pgfdeclareimage[height=0.5cm]{LogoTecBueno}{LogoTecBueno.eps}
\logo{\includegraphics[height=1cm]{fc2.png}}




\begin{document}

\beamerdefaultoverlayspecification{<+->}

\begin{frame}
  \titlepage
\end{frame}



\begin{frame}
\titulos{Ambigüedad}{}
\bi
\item  Está probado que no puede existir un algoritmo que determine con certeza si una gramática es
ambigua o no, y que en tal caso elimine dicha ambigüedad produciendo una
gramática no ambigua equivalente a la original. 
\item Es decir, el problema de ambigüedad es indecidible. Lo más que se
sabe es que hay ciertas condiciones que determinan ambigüedad pero en
caso de no cumplirse éstas nada puede decirse de la gramática en
cuestión.
\item En algunos casos, dada una gramática ambigua, se puede encontrar otra
gramática equivalente no ambigua, por ejemplo agregando precedencia de
operadores y asociatividad. 
\item Sin embargo existen lenguajes cuya
ambigüedad es inevitable.
\ei
\end{frame}
 


\begin{frame}
\titulos{Gramatícas libres de contexto en lenguajes de programación}{}
\bi
\item  El estudio formal de los lenguajes de programación se divide en sintaxis, pragmática y semántica.
\item La semántica se encarga de definir el significado de las expresiones, enunciados
y unidades de programa.
\item La pragmática define la implementación del lenguaje basada en la metodología y estrategias de programación deseadas
\item La sintaxis se encarga de definir la forma de
las expresiones y enunciados de un lenguaje y se sirve fundamentalmente de los
conceptos y herramientas de nuestro curso. 
\item Antes del proceso de evaluación, un compilador e intérprete necesita realizar
los procesos de análisis léxico y sintáctico, describimos a
continuación a grandes rasgos.
\ei
\end{frame}


\begin{frame}
\titulos{Análisis léxico y sintáctico}{}
\bi
\item Análisis léxico: se encarga de transformar el programa fuente en una
  lista de unidades sintácticas de bajo nivel llamadas lexemas, los cuales se
  clasifican en distintas categorías llamadas {\em tokens}, como pueden ser
  identificadores, constantes, separadores, etc. 
\item El análisis léxico se sirve
  fundamentalmente de expresiones regulares para su definición y reconocimiento.
\item Análisis sintáctico: se encarga de transformar la lista de lexemas en un
  programa objeto, el cual es una 
  expresión válida de la llamada sintaxis abstracta del lenguaje. Este
  programa es esencialmente un árbol de derivación dictado por
  una gramática libre de contexto que define al lenguaje de programación. 
\item Por
  lo tanto el análisis sintáctico es esencialmente una forma del problema de
  la pertenencia en gramáticas libres de contexto.
\ei
\end{frame}

\begin{frame}
\titulos{Forma de Backus-Naur}{}  
\bi
\item Las gramáticas libres de contexto para lenguajes de programación suelen
escribirse en la forma de Backus-Naur o BNF.
\item Este método de definición de gramáticas fue introducido por John Backus para el lenguaje {\sc Algol} 58 en 1959 y fue mejorado por Peter Naur para la definición de {\sc Algol} 60.
\item Este sistema notacional para definir lenguajes libres de contexto 
 sigue las siguientes convenciones:

\bi
\item El símbolo de reescritura $\to$ se reemplaza con $::=$.\medskip
\item El símbolo $\mid$ significa {\em o} y abreviar la definición de producciones de una misma variable.\medskip
\item Las variables se escriben entre paréntesis triangulares y por lo general
  utilizan nombres largos que ayuden a la descripción de \\ las categorías del lenguaje.
\ei
\ei
\end{frame}


\begin{frame}
\titulos{Gramáticas en forma de Backus-Naur}{Ejemplos}

\bi
\item Lenguaje de paréntesis balanceados
\beqs
\ba{rll}
<parent\_balanc> & ::= & \cv \;|\; \\
                 & & (<parent\_balanc>)\;|\; \\
                 & & <parent\_balanc><parent\_balanc> 
\ea
\eeqs
\item Expresiones aritméticas:
\beqs
\ba{rll}
<expr> & ::= & <expr>\,<op>\,<expr>\;|\; \\
       & & (<expr>)\;|\;<id>\\
<op> & := & +\;|\;-\;|\;*\;|\;/\\
<id> & := & a\;|\;b\;|\;c
\ea
\eeqs
\ei
\end{frame}

\begin{frame}
\titulos{Bloques de asignación de expresiones aritméticas:}
{{\tt begin}\;a {\tt :=} b/c\; ; \; b {\tt :=} a*(b+c)\;{\tt end}}
%{Gramáticas en forma de Backus-Naur}
\beqs
\ba{rll}
<programa> & ::= & {\tt begin}\;<sec\_enunc>\;{\tt end}\\ \\
<sec\_enunc> & ::= & <enunc>\;|\;<enunc>\,;\,<sec\_enunc>\\ \\
<enunc> & ::= & <id>\;{\tt :=}\;<expr>\\ \\
<expr> & ::= & <expr>\,<op>\,<expr>\;|\; \\ \\
       & & (<expr>)\;|\;<id>\\ \\
<op> & := & +\;|\;-\;|\;*\;|\;/\\ \\
<id> & := & a\;|\;b\;|\;c
\ea
\eeqs


\end{frame}

\begin{frame}[fragile]
\titulos{Gramáticas en forma de Backus-Naur}{Ejemplos}
\bi
\item Expresiones condicionales:
\beqs
\ba{rll}
<enunc> & ::= & <condicional>\;|\;<otras>\\
<condicional> & := & {\tt if}\;<expr>\;{\tt then}\;<enunc> \;|\; \\
              &    & {\tt if}\;<expr>\;{\tt then}\;<enunc> \\
              &    & \;{\tt            else}\;<enunc>
\ea
\eeqs
\item Problema del {\tt if} colgante:
\begin{verbatim}
if false then if false then 0 else 1
\end{verbatim}
\item Dos significados:
\begin{verbatim}
if false then (if false then 0) else 1
if false then (if false then 0 else 1)
\end{verbatim}
%\ec
\ei
\end{frame}

\begin{frame}
\titulos{Paréntesis balanceados}{Eliminación de la ambigüedad}
  \bi 

\item Paréntesis balanceados:
\beqs
S \to \cv\;|\;(S)\;|\;SS
\eeqs
\item Ambigüedad: $()()()$ tiene dos árboles de derivación %es ambigua pero equivalente a la siguientes gramática no ambigua:
\item Gramática no ambigua equivalente:
\beqs
S \to \cv\;|\;(S)S
\eeqs
\ei
\end{frame}
\begin{frame}
\titulos{Expresiones aritméticas}{Eliminación de la ambigüedad}
\bi
\item Expresiones aritméticas: % este lenguaje es generado por la siguiente
  % gramática ambigua
\beqs
S \to S + S\;|\;S*S\;|\;a
\eeqs
donde $a$ es un terminal que representa a los identificadores y constantes.
\item Ambigüedad: a + a * a

\item Gramática no ambigua equivalente:  se obtiene modelando la precedencia de
operadores como sigue:
\beqs
\ba{rll}
S & \to & S+T\;|\;T \\
T & \to & T*F\;|\;F\\
F & \to & (S)\;|\;a
\ea
\eeqs
\ei
\end{frame}
\begin{frame}
\titulos{Expresiones condicionales}{Eliminación de la ambigüedad}
\bi
\item Expresiones condicionales:
\beqs
\ba{rll}
S & \to & C\;|\;O\\
C & \to & {\tt if}\;E\;{\tt then}\;S \;|\; %\\
              %&    & 
{\tt if}\;E\;{\tt then}\;S\;{\tt else}\;S
\ea
\eeqs
\item Una gramática equivalente no ambigua es:
\beqs
\ba{rll}
S & \to & C\;|\;O\\
C & \to & C1\;|\;C2\\
C1 & \to & {\tt if}\;E\;{\tt then}\;C1\;{\tt else}\;C1  \\
C2 & \to   & {\tt if}\;E\;{\tt then}\;S\;|\; %\\
%   & & 
{\tt if}\;E\;{\tt then}\;C1\;{\tt else}\;C2
\ea
\eeqs
\item Idea: $C1$ genera condicionales dobles (if-then-else) balanceados; $C2$ representa condicionales
simples (if-then) y condicionales dobles pero de forma que un if-then sólo figura colgando \\ al final (en el else). %único lugar donde puede quedar un condicional sencillo es en la condición else.
\ei
\end{frame}


\begin{frame}
  \titulos{Ejercicio presencial}{}

Muestre que la siguiente gramática es ambigua:

\[
\ba{rll}
S & \to & AS\mid\varepsilon \\ \\
A & \to & A1\mid 0A1\mid\varepsilon
\ea
\]

\end{frame}

\begin{frame}
  \titulos{Ejercicio del tema anterior}{}
Transformar la siguiente gramática a la forma normal de Chomsky:

\[
\ba{rll}
S & \to & 0S1\mid A\mid AB \\ \\
A & \to & 1A0\mid S\mid\varepsilon \\ \\
B & \to & 0B\mid 1C\\ \\
C & \to & 0C\mid 0\mid\varepsilon\\ \\
D & \to & 0C\mid 1D\mid F\\ \\
F & \to & 1F\mid\varepsilon\\ \\
\ea
\]
\end{frame}





\end{document}
