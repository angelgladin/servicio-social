%\documentclass[xcolor=dvipsnames]{beamer}
\documentclass[xcolor=dvipsnames,handout]{beamer}

\input{macros2006}

\newcommand{\titulos}[2]{\frametitle{#1}\framesubtitle{#2}}
\newcommand{\espc}{\vspace{.5cm}}
\newcommand{\U}{\mathcal{U}}
\newcommand{\cv}{\ensuremath{\varepsilon}}
\newcommand{\sest}{\ensuremath{\Sigma^\star}}
\newtheorem{teorema}{Teorema}
\renewcommand{\S}{\Sigma}
\newcommand{\de}{\delta}
\newcommand{\izq}{\leftarrow}
\newcommand{\der}{\rightarrow}
\newcommand{\dest}{\ensuremath{\delta^\star}}


\mode<presentation>
{
\usetheme[secheader]{Boadilla}
%\useinnertheme{rounded}
\useoutertheme{infolines}
%\usepackage{BeamerColor}
%\usecolortheme[RGB={33,66,33}]{structure}
\usecolortheme[named=NavyBlue]{structure}
%\usefonttheme{}
\setbeamercovered{invisible}
%\setbeamertemplate{headline}{\insertshortinstitute}
%\setbeamertemplate{footline}[page number]{\insertshortinstitute}
%\insertshortinstitute
}

\usepackage[spanish]{babel}
\usepackage{times}
\usepackage[T1]{fontenc}


\title[AyLF]{Autómatas y Lenguajes Formales 2019-II}
\subtitle
{%Maestría en Ciencia e Ingeniería de la Computación UNAM\\
  Homomorfismos, lenguajes de Dyck y el Teorema de Chomsky-Schuetzenberger,
  jerarquía de Chomsky}

\author[Favio E. Miranda Perea]{Dr. Favio Ezequiel Miranda Perea \\ \texttt{favio@ciencias.unam.mx}}


\institute[FC UNAM]{Facultad de Ciencias UNAM\footnote{Con el apoyo del proyecto PAPIME PE102117}}

\date{\today}

\subject{Theoretical Computer Science}

% \pgfdeclareimage[height=0.5cm]{LogoTecBueno}{LogoTecBueno.eps}
\logo{\includegraphics[height=1cm]{fc2.png}}




\begin{document}

\beamerdefaultoverlayspecification{<+->}

\begin{frame}
  \titlepage
\end{frame}

\begin{frame}
  \titulos{Homomorfismos de lenguajes}{}
\bi
\item Sean $\Sigma,\Delta$ dos alfabetos. Una función $h:\Sigma^\star\to\Delta^\star$ es un homomorfismo si y sólo si
\bi
\item $h(\varepsilon)=\varepsilon$
\item $h(xy)=h(x)h(y)$
\ei
\medskip
\item Un homomorfismo $h$ queda determinado de manera única por sus valores en $\Sigma$. \medskip
\item Más aún, cualquier función $f:\Sigma\to\Delta^\star$ puede extenderse de manera única a un homomorfismo $h:\Sigma^\star\to\Delta^\star$ tal que 
si $x\in\Sigma$ entonces $h(x)=f(x)$. Este homomorfismo se define como:
\[
h(s_1s_2\ldots s_n) =_{def} f(s_1)f(s_2)\ldots f(s_n)
\]
para cada $s_i\in\Sigma$
\ei



\end{frame}



\begin{frame}
  \titulos{Imágen e imagen inversa de una función}{}
Dados una función $f:A\to B$, $X\subseteq A$ y $Y\subseteq B$, definimos:
\begin{itemize}
\item La imagen de $X$ bajo $f$, denotada $f[X]$, como
\[
f[X]=\{f(x)\mid x\in A\}
\]
 \medskip
\item La imagen inversa de $Y$ bajo $f$, denotada $f^{-1}[Y]$, como
\[
f^{-1}[Y]=\{x\mid f(x)\in Y\}
\]
 \medskip
\item Obsérvese que $f[X]\subseteq B$ y $f^{-1}[Y]\subseteq A$ \medskip
\item Propiedad relevante: $f[X\cup Z] = f[X]\cup f[Z]$
\end{itemize}
\end{frame}


\begin{frame}
  \titulos{Propiedades de cerradura}{Homomorfismos de lenguajes}

Sean $h:\Sigma^\star\to\Delta^\star, L\subseteq \Sigma^\star, S\subseteq\Delta^\star$.
  \begin{itemize}
 
  \item Si $L$ es regular entonces $h[L]$ es regular. Es decir, la imagen de un lenguaje regular bajo un homomorfismo es regular. \medskip
   \item Si $S$ es regular entonces $h^{-1}[S]$ es regular. Es decir, la imagen inversa de un lenguaje regular bajo un homomorfismo es regular. \medskip
     \item Si $L$ es libre de contexto entonces $h[L]$ es libre de contexto. Es decir, la imagen de un lenguaje libre de contexto bajo un homomorfismo es libre de contexto. \medskip
\item Propiedad de cerradura importante: Si $R$ es regular y $L$ es libre de contexto entonces $R\cap L$ es libre de contexto.
  \end{itemize}

\end{frame}


% \begin{frame}
%   \titulos{Morfismo de Thue-Morse}{}
% \end{frame}


\begin{frame}
  \titulos{Homomorfismos de borrado}{}
\bi
\item Sean $\Sigma,B$ dos alfabetos ajenos, es decir, $\Sigma\cap B=\varnothing$.  \medskip
\item Definimos $f:\Sigma\cup B\to \Sigma^\star$ como 
\[
\ba{rlll}
f(s) & = & \sigma & \mbox{si }s\in \Sigma \\ \\
f(s) & = & \varepsilon & \mbox{si }s\in B 
\ea
\] \medskip
\item $f$ induce un único homomorfismo $h:(\Sigma\cup B)^\star\to\Sigma^\star$ de tal forma que $h(x)= y$ si y sólo si $y$ se obtiene a partir de $x$ al borrar todos los símbolos de $B$. \medskip
\item A este homomorfismo lo denotamos con $erase_B$ \medskip
\item Ejemplo: Si $\Sigma=\{0,1\}$ y $B=\{(,\;)\}$ entonces $erase_B$ es el homomorfismo que borra paréntesis, por ejemplo:
\[
erase_B\Big( 00(11)(0)(01) \Big) = 0011001
\]
\ei
\end{frame}


\begin{frame}
  \titulos{Gramática de borrado a partir de $G$}{}
\bi
\item Sean $G=\pt{V,T,S,\mathcal{P}}$ una gramática libre de contexto y $R\subseteq T$
\item Definimos la gramática que borra a (símbolos de) $R$ en $G$, denotada $Er_R(G)$, como
\[
Er_R(G)= \pt{V,T-R,S,\mathcal{P}'}
\]
\item Donde $\mathcal{P}'$ se define como sigue:
\bi
\item Si $X\to v$ es una producción en $\mathcal{P}$, entonces $X\to erase_R(v)$ es una producción de $\mathcal{P}'$.
\ei
\item Si $G$ es una gramática libre de contexto y $R\subseteq T$, entonces
\[
L(Er_R(G))= erase_R[L(G)]
\]
\item Corolario: Si $L\subseteq \Sigma^\star$ es libre de contexto y $R\subseteq \Sigma$ entonces $erase_R[L]$ es libre de contexto.
% \item Es decir, el lenguaje de la gramática que borra a $R$ en $G$ coincide con la imagen de $L(G)$ bajo el homomorfismo que borra a $R$.
% %\item En particular $Er_R(G)$ es libre de contexto.
\ei
\end{frame}

\begin{frame}
  \titulos{Gramática separadora de $G$}{}
\bi
\item Sean $G=\pt{V,T,S,\mathcal{P}}$ una gramática libre de contexto en forma normal de Chomsky.
\item Definimos la gramática separadora de $G$, denotada $G_{sep}$ como
\[
G_{sep}= \pt{V,T\cup B,S,\mathcal{P}'}
\]
\item El conjunto de producciones $\mathcal{P}'$ se define como sigue:
\bi
\item Las producciones de $\mathcal{P}$ de la forma $A\to a$, están en $\mathcal{P}'$
\item Para cada producción en $\mathcal{P}$ de la forma $A_i\to B_iC_i$, $\mathcal{P}'$ tiene la siguiente producción
\[
A_i\to p_iB_i\bar{p_i}C_i
\]
\ei
\item Donde $B=\{p_1,\bar{p}_1,\ldots, p_n,\bar{p}_n\}$ es un conjunto de símbolos nuevos, i.e., $B\cap T=\varnothing$
\item $B$ es esencialmente un conjunto de $n$ juegos distintos de paréntesis.
\ei
\end{frame}



\begin{frame}
  \titulos{Gramática separadora}{Ejemplo}

\bi
\item Sea $G$ dada por
\[
S\to XY\;\;\;\;\;S\to YX\;\;\;\;\; Y\to ZZ\;\;\;\;\; 
X\to a\;\;\;\;\; Z\to a
\]
\item $G$ es ambigua:
\[
S\to XY\to aY \to aZZ \to aaZ \to aaa 
\]
\[
S\to YX\to ZZX \to aZX \to aaX \to aaa 
\]
\item $G_{sep}$ está dada por
\[
S\to (X)Y\;\;\;\;\;S\to [Y]X\;\;\;\;\; Y\to \{Z\}Z\;\;\;\;\; 
X\to a\;\;\;\;\; Z\to a
\]
\item $G_{sep}$ separa y codifica las dos derivaciones en $G$:
\[
S\to (X)Y\to (a)Y \to a\{Z\}Z \to (a)\{a\}Z \to (a)\{a\}a 
\]
\[
S\to [Y]X\to [\{Z\}Z]X \to [\{a\}Z]X \to [\{a\}a]X \to [\{a\}a]a 
\]
\ei
\end{frame}




\begin{frame}
  \titulos{Lenguajes de paréntesis o lenguajes de Dyck}{}
\bi
\item Sean $A$ un conjunto finito y $B_n=\{p_1,\bar{p}_1,\ldots,p_n,\bar{p}_n\}$ tal que $A\cap B_n=\varnothing$.
\item $B_n$ puede pensarse como un alfabeto con $n$ juegos de paréntesis distintos. Por ejemplo
  \begin{itemize}
  \item $B_1=\{(,\,)\}$
  \item $B_2=\{(,\,),\, [ ,\,]\}$
  \item $B_3=\{(,\,),\,[,\;],\,\{,\;\}\}$
  \end{itemize}
\item Con $D_n(A)$ denotamos al lenguaje de cadenas de $A\cup B_n$ generadas por la siguiente gramática libre de contexto

  \begin{itemize}
  \item Para toda $a\in A,\;S\to a$
    \item Para toda $p_i\in P_n,\;\;S\to p_iS\bar{p}_i$
  \item $S\to SS$
    \item $S\to \varepsilon$ 
  \end{itemize}
\ei
\end{frame}


\begin{frame}
\titulos{Lenguajes de Dyck}{Propiedades}
\bi
\item $D_n(A)$ se llama el lenguaje (generalizado) de Dyck de orden $n$.\medskip
\item $D_n(\varnothing)$ se llama el lenguaje (simple) de Dyck de orden $n$, denotado simplemente $D_n$.\medskip
\item Los lenguajes de Dyck tienen las siguientes propiedades:
  \begin{itemize}
  \item $A^\star\subseteq D_n(A)$\medskip
  \item Si $x,y\in D_n(A)$ entonces $xy\in D_n(A)$\medskip
  \item Si $x\in D_n(A)$ entonces $p_ix\bar{p}_i\in D_n(A)$\medskip
  \item Si $x\in D_n(A)$ y $x\notin A^\star$ entonces existen $u\in A^\star,\,v,w\in D_n(A),\,i\leq n$ tales que 
        $x=up_iv\bar{p}_iw$
  \end{itemize}
\ei
\end{frame}



\begin{frame}
  \titulos{La gramática $J$}{}
\bi
\item A partir de $G_{sep}=\pt{V,T\cup B,S,\mathcal{P}'}$ definimos ahora una gramática $J$ como sigue:\medskip
\item $J=\pt{V,T\cup B,S,\,\mathcal{P}''}$ donde $\mathcal{P}''$ consta de las siguientes producciones:
\bi
\item Todas las producciones $V\to a$ de  $\mathcal{P}'$\medskip %entonces $V\to a$ está en $\mathcal{P}''$
\item Si $A_i\to p_iB_i\bar{p_i}C_i$ está en $\mathcal{P}'$ entonces $\mathcal{P}''$ tiene a:
\bi
\item $A_i\to p_iB_i$\medskip
\item $V\to a\bar{p}_iC_i$, para cada producción $V\to a$ que esté en $\mathcal{P}'$
\ei
\ei
\item Es claro que $J$ es equivalente a una gramática lineal derecha.\medskip
\item Por lo tanto $L(J)$ es un lenguaje regular.
\ei
\end{frame}




\begin{frame}
  \titulos{Gramática $J$}{Ejemplo}
\bi
\item $G_{sep}$ está dada por
\[
S\to (X)Y\;\;\;\;\;S\to [Y]X\;\;\;\;\; Y\to \{Z\}Z\;\;\;\;\; 
X\to a\;\;\;\;\; Z\to a
\]
\item $J$ se construye como sigue:
\bi
\item $S\to (X)Y$ genera a : $S\to (X\;\;\;\;\;\;X\to a)Y\;\;\;\;\;\; Z\to a)Y$\medskip
\item $S\to [Y]X$ genera a : $S\to [Y\;\;\;\;\;\;X\to a]X\;\;\;\;\;\; Z\to a]X$\medskip
\item $Y\to \{Z\}Z$ genera a : $Y\to \{Z\;\;\;\;\;\;X\to a\}Z\;\;\;\;\;\; Z\to a\}Z$
\ei
\medskip
\item Así $J$ es:
\[
\ba{l}
S\to (X\;\;\;\;\;\;X\to a)Y\;\;\;\;\;\; Z\to a)Y \\ \\
S\to [Y\;\;\;\;\;\;X\to a]X\;\;\;\;\;\; Z\to a]X \\ \\
Y\to \{Z\;\;\;\;\;\;X\to a\}Z\;\;\;\;\;\; Z\to a\}Z \\ \\
X\to a\;\;\;\;\;\;Z\to a
\ea
\]
\ei
\end{frame}



\begin{frame}
  \titulos{Propiedades de $G_{sep}$}{}

  \begin{itemize}
  \item[P1] $erase_B[L(G_{sep})]= L(G)$\medskip
  \item[P2] $L(G_{sep})\subseteq D_n(T)$\medskip
  \item[P3] $L(G_{sep})\subseteq L(J)$\medskip
  \item[P4] $L(J)\cap D_n(T)\subseteq L(G_{sep})$\medskip
  \item[P5] Si $G$ está en FNC entonces existe $R$ regular tal que \[ L(G_{sep})=R\cap D_n(T) \]
   
  \end{itemize}
\end{frame}


% \begin{frame}
%   \titulos{Lenguajes de }{}
% \end{frame}

\begin{frame}
  \titulos{Teorema de Chomsky-Schuetzenberger}{}

Sea $L\subseteq T^\star$. Las siguientes condiciones son equivalentes:

\begin{enumerate}
\item[I)] $L$ es un lenguaje libre de contexto
\item[II)] Existen un lenguaje regular $R$, un homomorfismo $h$ y un número natural $n\in\mathbb{N}$ tales que
\[
L=h[R\cap D_n(T)]
\]
\end{enumerate}

Es decir, todo lenguaje libre de contexto es imagen homomórfica de la intersección de un lenguaje regular y un lenguaje de Dyck.
\end{frame}







\begin{frame}
  \titulos{Teorema de Chomsky-Schuetzenberger}{Versión usual}
\bi
\item El teorema de Chomsky-Schuetzenberger suele enunciarse y demostrarse utilizando lenguajes de Dyck simples como sigue:\medskip
%Sea $L\subseteq T^\star$. Las siguientes condiciones son equivalentes:
\item[] {\em Si $L$ es un lenguaje libre de contexto entonces existen un lenguaje regular $R$, un homomorfismo $h$ y un número natural $n>0$ tales que:}

 \[
 L=h[R\cap D_n]
 \]

% \begin{enumerate}
% \item[I)] $L$ es un lenguaje libre de contexto
% \item[II)] Existen un lenguaje regular $R$, un homomorfismo $h$ y un número natural $n\in\mathbb{N}$ tales que
% \[
% L=h[R\cap D_n(T)]
% \]
% \end{enumerate}
% Es decir, todo lenguaje libre de contexto es imagen homomórfica de la intersección de un lenguaje regular y un lenguaje de Dyck.
\medskip
\item Este teorema resulta equivalente a la versión aquí presentada.
\ei
\end{frame}


\begin{frame}
  \titulos{Lenguajes recursivamente enumerables o tipo
    0}{Jerarquía de Chomsky}
  Son aquellos lenguajes generados por una gramática cuyas producciones no
  tienen restricciones.\\
  Tales gramáticas pueden incluir reglas de la forma
  \beqs
  \alert{\al\imp\cv},\;\mbox{donde }\al\in (V\cup T)^*,\;\al\neq\cv 
  \eeqs

  De manera que la gramática es capaz de borrar cadenas. \pause Tales
  gramáticas se conocen como \alert{contraibles}.\\
Ejemplo:
\beqs
aS\imp bSb,\; aSb\imp \cv,\; SbS\imp bcS
  \eeqs
Los lenguajes tipo $0$ son equivalentes a las máquinas de Turing. Es
  decir, $L$ es generado por una gramática tipo $0$ si y sólo si $L$ es
  reconocido por una máquina de Turing.
\end{frame}


\begin{frame}
  \titulos{Lenguajes recursivamente enumerables o tipo
    0}{Jerarquía de Chomsky}
\bi
\item La siguiente es una gramática de tipo 0:
\[
S\to AT\;\;\;\;\;\;A\to 0AU\;\;\;\;A\to 1AI\;\;\;\;U0\to 0U
\]
\[
U1\to 1U\;\;\;\;\;I0\to 0I\;\;\;\;\;I1\to1I\;\;\;\;\;UT\to 0T
\]
\[
IT\to 1T\;\;\;\;\;\alert{A\to\varepsilon}\;\;\;\;\;\alert{T\to\varepsilon}
\]
\medskip
\item La gramática es contraible debido a que se pueden borrar $A$ y $T$.
\item $L(G)=\{ww\mid w\in\{0,1\}^\star\}$

\medskip 
% \item La idea del diseño de esta gramática y la razón del nombre {\em recursivamente enumerable} se discutirán más adelante.
\ei
\end{frame}



\begin{frame}
  \titulos{Lenguajes dependientes del contexto o tipo 1}{Jerarquía de
    Chomsky}
  Son aquellos generados por gramáticas con todas sus producciones son de la forma
  \beqs
\alert{\al_1A\al_2\imp\al_1\beta\al_2}
  \eeqs
  con $\al_1,\al_2\in(V\cup T)^\star,\;A\in V,\;\beta\neq\cv$.\\
  
  Con la posible excepción de la regla $S\imp\cv$, en cuyo caso se
  prohibe la presencia de $S$ a la derecha de las producciones.\\
\pause  Por ejemplo la siguiente gramática dependiente del contexto genera al
lenguaje $L=\{a^ib^ic^i\;|\;i\geq 0\}$
  \beqs
  S\imp A\;\;\;
  A\imp aABC\;|\;abC\;\;\;
      CB\imp BC \eeqs
   \beqs bB\imp bb
         \;\;\;bC\imp bc
         \;\;\;cC\imp cc
         \eeqs
\end{frame}


\begin{frame}
  \titulos{Lenguajes dependientes del contexto o tipo 1}{Jerarquía de
    Chomsky}
\bi
\item Estos lenguajes también se llaman sensibles al contexto y tienen cierta
  relevancia para la teoría de lenguajes de programación.
\item Un lenguaje $L$ es sensible al contexto si y sólo si $L$ es reconocido
  por un autómata linealmente acotado.
\item Un autómata linealmente acotado es una clase especial de Máquina de
  Turing que sólo permite el uso de un pedazo de la cinta cuyo tamaño depende de la
  longitud de la cadena de entrada. 
\ei
%  Por ejemplo la siguiente gramática dependiente del contexto genera al
% lenguaje $L=\{a^ib^ic^i\;|\;i\geq 0\}$
%   \beqs
%   S\imp A\;\;\;
%   A\imp aABC\;|\;abC\;\;\;
%       CB\imp BC \eeqs
%    \beqs bB\imp bb
%          \;\;\;bC\imp bc
%          \;\;\;cC\imp cc
%          \eeqs
\end{frame}


\begin{frame}
  \titulos{Lenguajes libres del contexto o tipo 2}{Jerarquía de
    Chomsky}
\bi
\item   Son aquellos generados por gramáticas con todas sus producciones de
  la forma
  \beqs
  \alert{A\imp\al}
  \eeqs
  con $A\in V,\;\al\in(V\cup T)^\star$.\medskip
\item  Esta definición incluye a la regla $S\imp\cv$. \medskip
\item   La mayoría de las gramáticas para lenguajes de programación caen en
  esta categoría.
\item Un lenguaje $L$ es libre de contexto si y sólo si $L$ es reconocido
  por un autómata de pila.
\ei
\end{frame}

\begin{frame}
  \titulos{Lenguajes regulares o tipo 3}{Jerarquía de Chomsky}
  Son aquellos generados por una gramática de una de las siguientes
  formas:
  \bi
  \item Lineal por la derecha: todas las producciones de la forma
    \beqs
    \alert{A\imp aB\;\;\;A\imp a\;\;\;A\imp\cv}
    \eeqs
    %con $A,B\in V,\;a\in T$
  
  \item Lineas por la izquierda: todas las producciones de la forma
       \beqs
    \alert{A\imp Ba\;\;\;A\imp a\;\;\;A\imp\cv}
    \eeqs
    %con $A,B\in V,\;a\in T$
\item No se permite mezclar ambos tipos de producciones.
\item Tales gramáticas se conocen también como gramáticas regulares.
\item \item Un lenguaje $L$ es tipo 3  si y sólo si $L$ es reconocido
  por un autómata finito. 
  \ei
  \end{frame}

  \begin{frame}
    \titulos{Jerarquía de Chomsky}{Observaciones}
    \bi
    \item Decimos que un lenguaje es de tipo $i$ si y sólo si $i$ es el índice mas grande tal que existe una gramática de tipo $i$ que genera a $L$
% la gramática con
%       índice más mayor que lo genera es de tipo $i$.
      \item La jerarquía de gramáticas genera una jerarquía en los
        lenguajes generados:
        \beqs
        \alert{\mathcal{L}_3\subsetneq \mathcal{L}_2\subsetneq
          \mathcal{L}_1\subsetneq \mathcal{L}_0\subsetneq\sest
 }
        \eeqs
       \item $\{0^n1^n\mid n\in\mathbb{N}\}\in
         \mathcal{L}_2\setminus\mathcal{L}_3$ 
             \item $\{0^n1^n0^n\mid n\in\mathbb{N}\}\in
               \mathcal{L}_1\setminus\mathcal{L}_2$
       \item Un ejemplo en $\mathcal{L}_0\setminus\mathcal{L}_1$ puede
         obtenerse mediante un truco de diagonalización o bien mediante complicados
         resultados de la teoría de la complejidad
       \item El lenguaje diagonal $L_\mathcal{D}$ cumple $L_\mathcal{D}\in \sest\setminus\mathcal{L}_0$ 
        % \item La jerarquía de Chomsky permite refinar la teoría de la
        %   computación clasificando lenguajes en función de los
        %   recursos computacionales necesarios para reconocerlos.
    \ei
    
  \end{frame}



\end{document}


