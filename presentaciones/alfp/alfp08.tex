\documentclass[xcolor=dvipsnames]{beamer}
%\documentclass[xcolor=dvipsnames,handout]{beamer}

\input{macros2006}

\newcommand{\titulos}[2]{\frametitle{#1}\framesubtitle{#2}}
\newcommand{\espc}{\vspace{.5cm}}
\newcommand{\U}{\mathcal{U}}
\newcommand{\cv}{\ensuremath{\varepsilon}}
\newcommand{\sest}{\ensuremath{\Sigma^\star}}
\newtheorem{teorema}{Teorema}
\renewcommand{\S}{\Sigma}
\newcommand{\de}{\delta}
\newcommand{\izq}{\leftarrow}
\newcommand{\der}{\rightarrow}
\newcommand{\dest}{\ensuremath{\delta^\star}}


\mode<presentation>
{
\usetheme[secheader]{Boadilla}
%\useinnertheme{rounded}
\useoutertheme{infolines}
%\usepackage{BeamerColor}
%\usecolortheme[RGB={33,66,33}]{structure}
\usecolortheme[named=NavyBlue]{structure}
%\usefonttheme{}
\setbeamercovered{invisible}
%\setbeamertemplate{headline}{\insertshortinstitute}
%\setbeamertemplate{footline}[page number]{\insertshortinstitute}
%\insertshortinstitute
}

\usepackage[spanish]{babel}
\usepackage{times}
\usepackage[T1]{fontenc}


\title[AyLF]{Autómatas y Lenguajes Formales}
\subtitle
{%Maestría en Ciencia e Ingeniería de la Computación UNAM\\
  Tema 8: Gramáticas, jerarquía de Chomsky, ambigüedad}

\author[Favio E. Miranda Perea]{Dr. Favio Ezequiel Miranda Perea \\ \texttt{favio@ciencias.unam.mx}}


\institute[FC UNAM]{Facultad de Ciencias UNAM\footnote{Con el apoyo del proyecto PAPIME PE102117}}

\date{\today}

\subject{Theoretical Computer Science}

% \pgfdeclareimage[height=0.5cm]{LogoTecBueno}{LogoTecBueno.eps}
\logo{\includegraphics[height=1cm]{fc2.png}}

\begin{document}


\beamerdefaultoverlayspecification{<+->}

\begin{frame}
  \titlepage
\end{frame}


%\section{Introducción a las gramáticas}
%\subsection{Introducción}
\begin{frame}
  \titulos{Gramáticas}{Introducción}
    \bi
\item Un mecanismo relevante para generar un lenguaje es mediante el
   concepto de gramática formal.\medskip
   \item Las gramáticas formales fueron introducidas por Noam Chomsky en
     1956.\medskip
     \item La intención era tener un modelo para la descripción de
       lenguajes naturales.\medskip
       \item Posteriormente se utilizaron como herramienta para
         presentar la sintaxis de lenguajes de programación y para el
         diseño de analizadores léxicos de compiladores
        % \item Volveremos a éllas más tarde.
     \ei
\end{frame}

%\subsection{Definiciones básicas}
\begin{frame}
  \titulos{Gramáticas}{Definición}
  Una gramática es una cuaterna \alert{$G=\pt{V,T,S,P}$} tal que:\pause
  \bi
  \item $V$ es un alfabeto de \alert{variables} o \alert{símbolos
      no-terminales}, los cuales se denotan con mayúsculas $A,B,C,\ldots$\medskip
    \item $T$ es un alfabeto de \alert{símbolos terminales}, los  
      cuales se denotan con minúsculas $a,b,c,\ldots$. Además se
      requiere $T\cap V=\vacio$.\medskip
      
      \item $S\in V$ es una variable distinguida llamada el \alert{símbolo
        inicial}.\medskip
        \item $P$ es un conjunto finito de reglas de reescritura,
          llamadas \alert{reglas de producción} o producciones.
  \ei
\end{frame}


\begin{frame}
  \titulos{Reglas de Producción}{Gramáticas}
El conjunto de reglas de producción $P$ es un conjunto finito de
pares $\pt{\al,\beta}$ tales que\pause
\bi
\item $\al\in (V\cup T)^\star - T^\star$. Es decir, $\al$ es una
  cadena de símbolos terminales ó no terminales, con al menos un
  símbolo no-terminal.\medskip
  \item  $\beta\in (V\cup T)^\star$. Es decir, $\beta$ es una cadena de
    símbolos de $V\cup T$ , los cuales podrían ser todos terminales.\medskip
    \item Usualmente en vez de escribir $\pt{\al,\beta}\in P$
      escribimos
      \beqs
      \al \imp \beta
      \eeqs
   \item[] y decimos que $\al$ produce a $\beta$, o que $\al$ se reescribe
      en $\beta$.
\ei
\end{frame}


\begin{frame}
  \titulos{Derivaciones}{Generación de cadenas}
  Las reglas de producción sirven para generar cadenas, proceso que se
  formaliza mediante las derivaciones formales:\\ \pause
 
  Dadas dos palabras $w,v\in (V\cup T)^\star$ decimos que $v$ es
    \alert{derivable} a partir de $w$ en {un paso} ($w\imp v$) si y sólo si:\pause\medskip
    \bi
    \item Existe una regla $\al\imp\beta$ en $P$ y  cadenas $\gamma_1,\gamma_2\in(V\cup T)^\star$ tales
      que:
  \beqs
  w=\gamma_1\al\gamma_2\;\;\mbox{y}\;\;v=\gamma_1\beta\gamma_2
  \eeqs\medskip
  \item Algunos autores utilizan $\Imp$ en vez de $\imp$ para denotar
    la relación de derivación. Nosotros preferimos sobrecargar el
    operador $\imp$.
  \ei
\end{frame}

\begin{frame}
  \titulos{Derivaciones formales}{$\imp^\star$}
 Decimos que una cadena $v$ es \alert{derivable} a partir de $w$ si existen
 palabras $\ga_2,\ldots,\ga_n$ tales que
 \beqs
 w=\ga_1\imp \ga_2\ldots\ga_{n+1}\imp\ga_n=v
 \eeqs \pause
En tal caso escribimos \alert{$w\imp^\star v$}. 
\end{frame}

\begin{frame}
  \titulos{Lenguaje generado
   por una gramática}{$L(G)$}
  Dada una gramática $G=\pt{V,T,S,P}$ definimos al lenguaje generado
  por $G$, denotado $L(G)$, como el conjunto de palabras de símbolos
  \alert{terminales} derivables a partir del símbolo
  inicial $S$. Es decir,\pause
\beqs
L(G)=\{w\in T^\star\;|\;S\imp^\star w\}
\eeqs
\end{frame}


%\subsection{Ejemplos}



\begin{frame}
\titulos{$L = (a+b)^\star$}{Ejemplos}
\bi
  \item Cualquier cadena de aes y bes debe generarse.
  \item La cadena vacía debe generarse: \beqs S\imp\cv\eeqs
    \item Si $w\in L$ entonces $wa\in L$
      \beqs
      S\imp Sa\eeqs
    \item Si $w\in L$ entonces $wb\in L$
      \beqs
      S\imp Sb\eeqs
      \item Ejemplo de derivación: $w=ababb$
        \beqs
        S\imp Sb\imp Sbb\imp Sabb\imp Sbabb\imp Sababb\imp ababb
        \eeqs
  \ei
\end{frame}







\begin{frame}
  \titulos{$L=\{a^ib^j\;|\;i,j\in\N,i\leq j\}$}{Ejemplos}
  \bi
  \item La cadena vacía debe generarse ($i=j=0$): \beqs S\imp\cv\eeqs
    \item Debe haber al menos tantas bes como aes, primero aes y luego bes: \beqs S\imp aSb\eeqs
      \item Puede haber más bes al final:  \beqs S\imp Sb\eeqs
      \item Ejemplo de derivación: $w=aabbb$
        \beqs
        S\imp aSb\imp aaSbb\imp aaSbbb\imp aabbb
        \eeqs
  \ei
\end{frame}

\begin{frame}
  \titulos{$L=\{a^ib^ja^jb^i\;|\;i,j\in\N\}$}{Ejemplos}
  \bi
  \item Primero generamos el centro de la palabra, $b^ja^j$:
    \beqs
     S\imp B\;\;\; B\imp bBa\;\;\;B\imp \cv
    \eeqs
  \item Después los extremos $a^i,\;b^i$:
    \beqs
    S\imp aSb
    \eeqs
      \item Ejemplo de derivación: $w=aababb$
        \beqs
        S\imp aSb\imp aaSbb\imp aaBbb\imp aabBabb \imp aababb
\eeqs
  \ei
\end{frame}


\begin{frame}
  \titulos{$L=\{a^{i} b^{i} a^{j} b^{j}\;|\;i,j\in\N\}$}{Ejemplos}
  \bi
  \item El lenguaje $\{a^i b^i\;|\; i\in\N\}$ se genera mediante:
    \beqs
    P\imp \cv\;\;\;\;\;P\imp aPb
    \eeqs
    \item Para generar a $L$ simplemente agregamos:
      \beqs
      S\imp PP
      \eeqs
      \item Ejemplo de derivación: $w=aabbab$
        \beqs
        S\imp PP \imp aPbP\imp aPbaPb\imp aaPbbaPb \imp aaPbbab\imp aabbab
\eeqs
  \ei
\end{frame}


\begin{frame}
  \titulos{$L=\{a^{i} b^{i}\;|\;i\in\N\}\cup\{b^{i} a^{i}\;|\;i\in\N\}$}{Ejemplos}
  \bi
  \item El lenguaje $\{a^i b^i\;|\; i\in\N\}$ se genera mediante:
    \beqs
    P\imp \cv\;\;\;\;\;P\imp aPb
    \eeqs
    \item El lenguaje $\{b^i a^i\;|\; i\in\N\}$ se genera mediante:
    \beqs
    Q\imp \cv\;\;\;\;\;Q\imp bQa
    \eeqs
    
    \item Para generar a $L$ simplemente agregamos:
      \beqs
      S\imp P\;\;\; S\imp Q
      \eeqs
 \item Ejemplo de derivación: $w=bbbaaa$
        \beqs
        S\imp P \imp aPb\imp aaPbb\imp aaaPbbb \imp aaabbb
\eeqs
  \ei
\end{frame}



\begin{frame}
  \titulos{Correctud y completud}{Diseño de gramáticas}
\bi
\item Si bien muchas veces el diseño de una gramática $G$ para un lenguaje dado $L$  es intuitivamente claro y correcto, esto debe mostrarse formalmente, mostrando que $L=L(G)$. Esto se hace, por supuesto, mostrando lo siguiente:
\medskip
\item Correctud: la gramática $G$ genera únicamente cadenas de $L$, es decir, $L(G)\inc L$.
\medskip
\item Completud: toda cadena de $L$ es generada por $G$, es decir, $L\inc L(G)$.
\ei

\end{frame}



%\subsection{Jerarquía de Chomsky}

\begin{frame}
  \titulos{Lenguajes recursivamente enumerables o tipo
    0}{Jerarquía de Chomsky}
  Son aquellos lenguajes generados por una gramática sin restricciones
  adicionales.\\
  Tales gramáticas pueden incluir reglas de la forma
  \beqs
  \alert{\al\imp\cv}
  \eeqs
  De manera que la gramática es capaz de borrar cadenas. \pause Tales
  gramáticas se conocen como \alert{contraibles}.\\
Ejemplo:
\beqs
aS\imp bSb,\; aSb\imp \cv,\; SbS\imp bcS
  \eeqs
\end{frame}


\begin{frame}
  \titulos{Lenguajes recursivamente enumerables o tipo
    0}{Jerarquía de Chomsky}
\bi
\item La siguiente es una gramática de tipo 0:
\[
S\to AT\;\;\;\;\;\;A\to 0AO\;\;\;\;A\to 1AI\;\;\;\;O0\to 0O
\]
\[
O1\to 1O\;\;\;\;\;I0\to 0I\;\;\;\;\;I1\to1I\;\;\;\;\;OT\to 0T
\]
\[
IT\to 1T\;\;\;\;\;\alert{A\to\varepsilon}\;\;\;\;\;\alert{T\to\varepsilon}
\]
\medskip
\item $L(G)=\{ww\mid w\in\{0,1\}^\star\}$
\medskip
\item La idea del diseño de esta gramática y la razón del nombre {\em recursivamente enumerable} se discutirán más adelante.
\ei
\end{frame}



\begin{frame}
  \titulos{Lenguajes dependientes del contexto o tipo 1}{Jerarquía de
    Chomsky}
  Son aquellos generados por gramáticas con todas sus producciones son de la forma
  \beqs
\alert{\al_1A\al_2\imp\al_1\beta\al_2}
  \eeqs
  con $\al_1,\al_2\in(V\cup T)^\star,\;A\in V,\;\beta\neq\cv$.\\
  
  Con la posible excepción de la regla $S\imp\cv$, en cuyo caso se
  prohibe la presencia de $S$ a la derecha de las producciones.\\
\pause  Por ejemplo la siguiente gramática dependiente del contexto genera al
lenguaje $L=\{a^ib^ic^i\;|\;i\geq 0\}$
  \beqs
  S\imp A\;\;\;
  A\imp aABC\;|\;abC\;\;\;
      CB\imp BC \eeqs
   \beqs bB\imp bb
         \;\;\;bC\imp bc
         \;\;\;cC\imp cc
         \eeqs
\end{frame}


\begin{frame}
  \titulos{Lenguajes libres del contexto o tipo 2}{Jerarquía de
    Chomsky}
\bi
\item   Son aquellos generados por gramáticas con todas sus producciones de
  la forma
  \beqs
  \alert{A\imp\al}
  \eeqs
  con $A\in V,\;\al\in(V\cup T)^\star$.\medskip
\item  Esta definición incluye a la regla $S\imp\cv$. \medskip
\item   La mayoría de las gramáticas para lenguajes de programación caen en
  esta categoría.
\ei
\end{frame}

\begin{frame}
  \titulos{Lenguajes regulares o tipo 3}{Jerarquía de Chomsky}
  Son aquellos generados por una gramática de una de las siguientes
  formas:
  \bi
  \item Lineal por la derecha: todas las producciones de la forma
    \beqs
    \alert{A\imp aB\;\;\;A\imp a\;\;\;A\imp\cv}
    \eeqs
    con $A,B\in V,\;a\in T$
  
  \item Lineas por la izquierda: todas las producciones de la forma
       \beqs
    \alert{A\imp Ba\;\;\;A\imp a\;\;\;A\imp\cv}
    \eeqs
    con $A,B\in V,\;a\in T$
\item No se permite mezclar ambos tipos de producciones.
  \ei
  \end{frame}

  \begin{frame}
    \titulos{Jerarquía de Chomsky}{Observaciones}
    \bi
    \item Decimos que un lenguaje es de tipo $i$ si y sólo si $i$ es el índice mas grande tal que existe una gramática de tipo $i$ que genera a $L$
% la gramática con
%       índice más mayor que lo genera es de tipo $i$.
      \item La jerarquía de gramáticas genera una jerarquía en los
        lenguajes generados:
        \beqs
        \alert{\mathcal{L}_3\subsetneq \mathcal{L}_2\subsetneq \mathcal{L}_1\subsetneq \mathcal{L}_0}
        \eeqs
        \item La jerarquía de Chomsky permite refinar la teoría de la
          computación clasificando lenguajes en función de los
          recursos computacionales necesarios para reconocerlos.
    \ei
    
  \end{frame}


%\section{Gramáticas Regulares}

%\subsection{Introducción}
%\begin{frame}
%  \titulos{Gramáticas Regulares}{Definición}
%    \bi
%  \item Lineal por la derecha: todas las producciones de la forma
%    \beqs
%    \alert{A\imp aB\;\;\;A\imp a\;\;\;A\imp\cv}
%    \eeqs
%    con $A,B\in V,\;a\in T$
  
%  \item Lineas por la izquierda: todas las producciones de la forma
%       \beqs
%    \alert{A\imp Ba\;\;\;A\imp a\;\;\;A\imp\cv}
%    \eeqs
%    con $A,B\in V,\;a\in T$
%\ei
%No se permite mezclar ambos tipos de producciones.
%\end{frame}




\begin{frame}
  \titulos{Gramáticas Regulares}{}
    \bi
  \item Una gramática regular es una gramática lineal por la
    derecha o lineal por la izquierda.\espc
   \item  No se permite mezclar ambos tipos de producciones.\espc
     \item Se puede probar que toda gramática lineal por la izquierda
       es equivalente a una gramática lineal por la derecha.
       \ei
       \end{frame}
       \begin{frame}
  \titulos{Gramáticas Regulares}{Lenguajes Regulares}
       \bi
       \item Decimos que un lenguaje $L$ es regular si \alert{existe} una
         gramática regular $G$  que lo genere, es decir, si 
         $L=L(G)$.\espc
         \item Si $L$ es generado por una gramática de tipo $i$, no se puede
           asegurar de inmediato que $L$ sea un lenguaje de tipo
           $i$. Debe asegurarse que $i$ es máximo.
\ei
\end{frame}

%\subsection{Ejemplos}
\begin{frame}
  \titulos{$L=0^\star10^\star 10^\star$}{Ejemplos}
  $L$ es generado por:
  \beqs
 \ba{rll}
 S & \imp & A1A1A \\
 A & \imp & 0A\;|\;\cv
 \ea
\eeqs
esta gramática no es regular, \pause pero el lenguaje si lo es al existir una
gramática regular equivalente:
\beqs
 \ba{rll}
 S & \imp & 0S\;|\;1A \\
 A & \imp & 0A\;|\;1B \\
 B & \imp & 0B\;|\;\cv \\
 \ea
\eeqs
\end{frame}


\begin{frame}
  \titulos{$L=(a+b)^\star b$}{Ejemplos}
  $L$ es generado por:

\beqs
 \ba{rll}
 S & \imp & aS\;|\;bC \\
 C & \imp & bC\;|\;aS\;|\;\cv\\
 \ea
\eeqs
\end{frame}

%\subsection{AF $\;\Iff\;$ GR}
\begin{frame}
  \titulos{Lenguajes y gramáticas regulares}{AF$\;\Imp\;$ GR}
  Dado un AF  $M=\pt{Q,\S,\de,q_0,F}$ existe una gramática regular
  $G=\pt{V,T,S,P}$ tal que $L(M)=L(G)$. Es decir, todo lenguaje
  regular es generado por una gramática regular.\pause\\
  Definimos a $G$ como sigue:
  \bi
  \item Suponemos s.p.g. que no hay $\cv$-transiciones.
    \item $V=Q\;\;\;\;\;T=\S\;\;\;S=q_0$
             \item $P$ se define como sigue:
          \bi
          \item Si $p\in\de(q,a)$ entonces
            agregamos $q\imp ap$ a $P$.\espc
            \item Si $q_f\in\de(q,a)$ con $q_f\in F$ entonces
              agregamos $q\imp a$.
           % \item Si $q_F\in F$ entonces agregamos $q_F\imp\cv$ a $P$.
          \ei
  \ei
\end{frame}



%\begin{frame}
%  Por ejemplo, para el autómata $M$ dado por: 
%\beqs
%\VCDraw{
%\begin{VCPicture}{(0,0)(6,3)}  
%\State[q_0]{(0,3)}{q_0} \State[q_1]{(3,3)}{q_1}
%\FinalState[q_2]{(6,0)}{q_2} 
%\Initial{q_0} %\Final{q_3}
%\EdgeL{q_0}{q_1}{1}\ArcR{q_1}{q_2}{0}\ArcR{q_2}{q_1}{1}
%\ArcL{q_2}{q_0}{0} %\ArcL{q_2}{q_0}{a}
%\LoopN{q_0}{0} \LoopN{q_1}{1} %\LoopS{q_2}{a,b}
%\end{VCPicture}
%}
%\eeqs
%La gramática obtenida es:
%\beqs
%q_0\imp 0q_0\;|\;1q_1\;\;\;\;\;q_1\imp
%1q_1\;|\;0q_2\;|\;0\;\;\;\;\;\;q_2\imp 0q_0\;|\;1q_1
%\eeqs

%\end{frame}
  
%\begin{frame}


%Para el autómata $M$:


%  \beqs
%\VCDraw{
%\begin{VCPicture}{(0,0)(3,3)}  
%\FinalState[q_0]{(0,3)}{q_0} \FinalState[q_1]{(3,3)}{q_1}
%\State[q_2]{(3,0)}{q_2} 
%\Initial{q_0} %\Final{q_3}
%\ArcL{q_0}{q_1}{b}\ArcL{q_1}{q_0}{a}\ArcL{q_1}{q_2}{b}
%\ArcL{q_2}{q_1}{b}\ArcL{q_2}{q_0}{a}
%\LoopN{q_0}{a} %\LoopN{q_1}{b}\LoopS{q_2}{a,b}
%\end{VCPicture}
%}
%\eeqs

%la gramática generada es:


%\beqs
%q_0\imp aq_0\;|\;bq_1\;|\;a\;|\;b\;\;\;\;q_1\imp
%aq_0\;|\;bq_2\;|\;a\;\;\;\;q_2\imp aq_0\;|\;bq_1\;|\;a
%\eeqs
%\end{frame}




\begin{frame}
  \titulos{Lenguajes y gramáticas regulares}{GR$\;\Imp\;$ AF}
  Dada una gramática regular $G=\pt{V,T,S,P}$ existe un AF
  $M=\pt{Q,\S,\de,q_0,F}$ tal que $L(M)=L(G)$. Es decir todo lenguaje
  generado por una gramática regular es un lenguaje regular. \pause \\
  Definimos a $M$ como sigue:
  \bi
  \item Suponemos s.p.g. que $G$ es lineal derecha.
   \item  $Q=V\cup\{q_F\}\;\;\;\;\S=T\;\;\;\;q_0=S\;\;\;\;F=\{q_F\}$
          \item $\de$ se define como sigue:
            \bi
            \item Si $A\imp aB\in P$ entonces $B\in\de(A,a)$.
              \item Si $A\imp a\in P$ entonces $q_F\in\de(A,a)$.
                \item Si $A\imp\cv\in P$ entonces $q_F\in\de(A,\cv)$.
            \ei
  \ei
\end{frame}


%\section{Gramáticas Libres de Contexto}
\begin{frame}
  \titulos{Gramáticas libres de contexto}{Definición}
Una gramática es libre o independiente del contexto si 
todas sus producciones son de
  la forma
  \beqs
  \alert{A\imp\al}
  \eeqs
  con $A\in V,\;\al\in(V\cup T)^\star$.\\
\pause  Esta definición incluye a la regla $S\imp\cv$. \\

\pause Obsérvese que en particular toda gramática regular es libre de contexto.
\end{frame}

%\subsection{Ejemplos}

\begin{frame}
  \titulos{Gramáticas libres de contexto}{Ejemplos}
\bi
\item $L=a^\star$
  \beqs
  S\imp aS\;|\;\cv
  \eeqs
\item $L=a^\star b^\star$
     \beqs
   \ba{rll}
    S & \imp & aS\;|\;bA\;|\;\cv\\
    A & \imp & bA\;|\;b\;|\;\cv
   \ea
   \eeqs
   \item $L=0^+1^+$
   \beqs
   \ba{rll}
   S & \imp & CU\\
   C & \imp & 0C\;|\;0 \\
   U & \imp & 1U\;|\;1
   \ea
   \eeqs
  \ei
\end{frame}


\begin{frame}
  \titulos{Gramáticas libres de contexto}{Ejemplos}
\bi
\item $L=\{a^nba^m\;|\;n,m\geq 1\}=a^+ba^+$
    \beqs
   \ba{rll}
   S & \imp & aS\;|\;aB\\
   B & \imp & bC \\
   C & \imp & aC\;|\;a
   \ea
   \eeqs
\item $L=\{a^n b^n\;|\;n\in\N\}$  que no es regular
  \beqs
S\imp aSb\;|\;\cv
\eeqs
\item $L=\{w\in\{a,b\}^\star\;|\;w=w^R\}$ que no es regular
  \beqs
S\imp aSa\;|\;bSb\;|\;a\;|\;b\;|\;\cv
  \eeqs
  
  \ei
\end{frame}


%\subsection{Derivaciones y árboles}


\begin{frame}
  \titulos{Derivaciones por la izquierda}{GLC}
  Una derivación $S\imp^\star w$ es una derivación por la izquierda si
  en cada paso se reescribe la variable mas a la izquierda en la
  palabra.\\ \pause
  En la gramática
  \beqs
  S\imp aAs\;|\;a\;\;\;\;\;A\imp SbA\;|\;SS\;|\;ba
  \eeqs\pause
  tenemos la siguiente derivación por la izquierda de $aabbaa$.
  \beqs
  \alert{S}\imp a\alert{A}S\imp a\alert{S}bAS\imp aab\alert{A}S\imp aabba\alert{S}\imp aabbaa
  \eeqs
\end{frame}


\begin{frame}
  \titulos{Derivaciones por la derecha}{GLC}
  Una derivación $S\imp^\star w$ es una derivación por la derecha si
  en cada paso se reescribe la variable mas a la derecha en la
  palabra.\\ \pause
  En la gramática
  \beqs
  S\imp aAs\;|\;a\;\;\;\;\;A\imp SbA\;|\;SS\;|\;ba
  \eeqs \pause
  tenemos la siguiente derivación por la derecha de $aabbaa$.
  \beqs
  \alert{S}\imp aA\alert{S}\imp a\alert{A}a\imp aSb\alert{A}a\imp a\alert{S}bbaa\imp aabbaa
  \eeqs
  
\end{frame}


\begin{frame}
  \titulos{Árboles de derivación}{GLC}
  \bi
  \item Los árboles de derivación o árboles sintácticos son un
    mecanismo para representar las derivaciones de gramáticas libres
    de contexto.\espc
  \item  En compiladores se utilizan para el análisis sintáctico de
    programas fuente (parsing) y sirven de base para la generación de
    código.\espc
    \item Puede ser que dos derivaciones distintas tengan el mismo
      árbol.
\ei
    \end{frame}
\begin{frame}
  \titulos{Árboles de derivación}{GLC}
  \bi
    \item Puede haber mas de un árbol de derivación para una cadena.\espc
      \item Lo ideal es que cada cadena tenga sólo un árbol asociado,
        esto implica que el lenguaje no es ambiguo.\espc
        \item Desafortunadamente existen lenguajes ambiguos.
  \ei
\end{frame}


\begin{frame}
  \titulos{Construcción de árboles de derivación}{GLC}
  Dada una gramática libre de contexto $G=\pt{V,T,S,P}$, un árbol de
  derivación en $G$ se construye como sigue: \pause
\bi
\item La raiz contiene al símbolo inicial $S$.
\item Cada nodo interior contiene una variable
  \item Cada hoja contiene un símbolo de $V\cup T\cup\{\cv\}$.
    \item Si un nodo interior contiene una variable $A$ entonces sus
      hijos contienen símbolos (de izquierda a derecha)
      $a_1,\ldots,a_n$ si y sólo si $A\imp a_1a_2\ldots a_n$ está en
      $P$.
      \item La palabra generada se puede leer al leer las hojas de
        izquierda a derecha.
\ei
\end{frame}

%\subsection{Ambigüedad}

\begin{frame}
  \titulos{Ambigüedad}{GLC}
  \bi
  \item Una gramática se dice \alert{ambigua} si existe una palabra
    $w$ con dos o más árboles de derivación distintos.
    \item En general una palabra puede tener mas de una derivación,
      pero un sólo árbol, en tal caso no hay ambigüedad.
      \item Algunas veces se puede suprimir la ambigüedad
        directamente.
        \item Sin embargo no hay un algoritmo para remover ambigüedad.
          \item Pero aún, hay lenguajes cuya ambigüedad es inevitable.
  \ei
\end{frame}


\begin{frame}
  \titulos{Ejemplos}{Ambigüedad}
  \beqs
  S\imp AA\;\;\;A\imp aSa\;|\;a
  \eeqs
La palabra $a^5$ tiene las siguientes derivaciones:\pause
\bi
\item $S\imp AA\imp aA\imp aaSa\imp aaAAa\imp aaaAa\imp aaaaa$\espc
  \item $S\imp AA\imp aSaA\imp aAAaA\imp aaAaA\imp aaaaA\imp aaaaa $\espc
    \item Las dos derivaciones son por la izquierda y generan
      árboles distintos.
\ei
\end{frame}


\begin{frame}
  \titulos{Lenguajes Ambiguos}{Ambigüedad}
  \bi
  \item Un lenguaje $L$ es ambiguo si existe una gramática ambigua $G$
    que genera a $L$.
    \item $L=\{a^{2+3i}\;|\;i\geq 0\}$ es ambiguo.\espc
    \item Un lenguaje es inherentemente ambiguo si todas las
      gramáticas que lo generan son ambiguas.
      \item $L=\{a^nb^nc^md^m\;|\;n,m\geq
        1\}\cup\{a^nb^mc^md^n\;|\;n,m\geq 1\}$ es inherentemente
        ambiguo.
        \ei
\end{frame}

\begin{frame}
  \titulos{$L=\{a^{2+3i}\;|\;i\geq 0\}$}{Lenguajes ambiguos}
  \bi
\item[] $L$ es ambiguo por se generado por la gramática ambigua \beqs
  S\imp AA\;\;\;A\imp aSa\;|\;a
  \eeqs
\item[] Sin embargo este lenguaje también es generado por una gramática no
ambigua:
\beqs
S\imp aa\;|\;aaU\;\;\;\;\;\;U\imp aaaU\;|\;aaa
\eeqs
\item[] en este caso la derivación de $a^5$ es:
\beqs
S\imp aaU\imp aaaaa
\eeqs
\item[] por lo tanto $L$ no es un lenguaje inherentemente ambiguo
\ei
\end{frame}

\begin{frame}
  \titulos{$L=\{a^nb^nc^md^m\;|\;n,m\geq
        1\}\cup\{a^nb^mc^md^n\;|\;n,m\geq 1\}$}{Lenguajes
        inherentemente ambigüos}
\bi
\item[] $L$ es generado por la gramática:
\beqs
\ba{l}
S\imp AB\;|\;C\;\;\;\;A\imp aAb\;|\;ab\;\;\;\;B\imp cBd\;|\;cd \\
C\imp aCd\;|\; aDd\;\;\;\;\; D\imp bDc\;|\;bc
\ea
\eeqs
\item[] La cadena $aabbccdd$ tiene dos derivaciones por la izquierda:
\item[] $S\imp AB\imp aAbB\imp aabbB\imp aabbcBd\imp aabbccdd$
\item[] $S\imp C\imp aCd\imp aaDdd\imp aabDcdd\imp aabbccdd$
\item[] Probar la ambiguedad inherente es complicado.  
\ei
\end{frame}


\end{document}
  
%%% Local Variables: 
%%% mode: latex
%%% TeX-master: "fcem7"
%%% End: 
