\documentclass[letterpaper,11pt]{article}
\usepackage[includeheadfoot,margin=1.3in]{geometry}
\usepackage{anysize}

\usepackage[utf8]{inputenc}
% \usepackage[latin1]{inputenc}
\usepackage[spanish]{babel}
\usepackage{lmodern}   % font shapes...
\usepackage[T1]{fontenc} % join the compound symbols as a single symbol

\usepackage{amssymb,amsmath}
\usepackage{mathrsfs}
\usepackage{epsfig}

\usepackage{fancyhdr}
\usepackage{hyperref}
\usepackage{url}

\usepackage{import}
\usepackage{comment}


\usepackage{multicol}
\setlength{\columnsep}{1.2cm}
\input{macrosAyLF}

\title{Autómatas y Lenguajes Formales, Semestre 2020-2\\
Facultad de Ciencias UNAM}
\author{
% \small\texttt{\href{mailto: liliana.mar@gmail.com}{}}}
% \and A. Liliana Reyes Cabello 
Lourdes del Carmen Gonz\'alez Huesca 
\small\texttt{\href{mailto:luglzhuesca@ciencias.unam.mx}{
luglzhuesca@ciencias.unam.mx}}
}
\date{\today}

\begin{document}
\maketitle

\begin{comment}
\section{Prerrequisitos}

\begin{itemize}
\item Nociones elementales de teoría de conjuntos: operaciones con conjuntos, funciones
\item Relaciones binarias, sus propiedades y operaciones, en particular relaciones de equivalencia.
\item Inducción: definición de números naturales, inducción, inducción fuerte.
\item Programación: es deseable tener nociones de programación funcional.
\end{itemize}

Para cubrir los primeros tres puntos sugiero leer el  capítulo 1 del libro \cite{martin} así como resolver sus ejercicios, 
para el último punto sugiero el libro \cite{hutton}.
\end{comment}


\section*{Temario}
\begin{multicols}{2}
\begin{enumerate}
\item Introducción
  \begin{itemize}
  \item Cadenas y lenguajes.
  \item Definiciones inductivas.
  \item Inducción estructural.
  \end{itemize}
\item Lenguajes regulares
  \begin{itemize}
  \item Expresiones regulares
  \item Autómatas finitos (AFD, AFN, AFN$\varepsilon$)
  \item Teorema de Kleene (Derivadas de expresiones regulares y ecuaciones de 
  lenguajes) 
%   \item Homomorfismos
  \item Teorema de Myhill-Nerode (Minimización de autómatas)
  \end{itemize}
\item Lenguajes libres de contexto
  \begin{itemize}
  \item Gramáticas y formas normales
  \item Autómatas de Pila
  \item Ambigüedad
  \item Algoritmos y procedimientos de decisión.
%   \item Lenguajes de Dyck y Teorema de Chomsky-Schützenberger
  \end{itemize}
\item Máquinas de Turing
  \begin{itemize}
  \item Definición y diseño de máquinas de Turing.
  \item Lenguajes recursivos y recursivamente enumerables.
%   \item Equivalencias (programas {\sc While})
%   \item Decidibilidad (Teorema de Rice)
  \end{itemize}
% \item Otros formalismos
%   \begin{itemize}
%   \item Funciones $\mu$-recursivas
%   \item Cálculo lambda  
%   \end{itemize}
\end{enumerate}
\end{multicols}

\section*{Evaluaci\'on}
\noindent 
El curso se calificará mediante 6 exámenes parciales cuyas fechas de 
aplicaci\'on tentativas aparecen abajo. Los ex\'amenes ser\'an de los temas 
incluidos en las tareas y boletines correspondientes.
% y que tendr\'an eventualmente ejercicios pr\'acticos. 
% Adem\'as se aplicar\'an ejercicios semanales presenciales e individuales.
Los porcentajes son los siguientes: 
\begin{itemize}
\item Exámenes parciales: 60\%\\
 {\footnotesize Fechas: 
%  29 de agosto, 19 de septiembre, 15 de octubre, 12 de 
% noviembre y la primera fecha de examen final.
18 de febrero, 10 de marzo, 31 de marzo, 28 de abril, 19 de mayo y la segunda 
fecha de examen final.}
\item Tareas semanales: 40\% 
% \item Ejercicios presenciales: 20\%
\end{itemize}
\noindent
No habrá examen final ni reposiciones. 
Se debe tener promedio aprobatorio en exámenes y un promedio aprobatorio para 
obtener calificaci\'on final. 
Las dem\'as consideraciones para tener derecho a calificaci\'on final se 
encuentran en la p\'agina del curso:
\url{https://sites.google.com/ciencias.unam.mx/aylf/AyLF}


\section*{Material del curso}
\noindent 
No habrá un texto oficial para el curso, existen varias fuentes que 
pueden ser consultadas que se encuentran listadas a continuación. 
En su lugar habr\'a notas y presentaciones de clase. 
Los libros \cite{hmu},\cite{kozen},\cite{linz},\cite{martin} son básicos y con 
el enfoque clásico, el resto presentan un enfoque distinto o más general o 
avanzado.
\vspace*{-10pt}
\begin{thebibliography}{50}
\bibitem{anderson} 
  James A. Anderson. Automata Theory with Modern Applications. Cambridge 
  University Press 2006.
  
\bibitem{davis} 
  Martin Davis, Ron Sigal, Elaine J. Weyuker. Computability, Complexity
  and Languages: Fundamentals of Theoretical Computer Science. Academic Press, 
  1983.
  
\bibitem{fernandez} 
  Maribel Fernández, Models of Computation, An Introduction to Computability 
  Theory. Springer 2009.
  
\bibitem{gopalak} 
  Ganesh Gopalakrishnan. Computation Engineering, Applied Automata Theory and 
  Logic. Springer 2006.

\bibitem{hankin} 
  Chris Hankin. Lambda Calculi, A Guide for Computer Scientists, Clarendon 
  Press, Oxford 2004.

\bibitem{hmu} 
  J.E. Hopcroft, R. Motwani y J. Ullman. Introduction to Automata Theory, 
  Languages, and Computation. 3rd. Edition, Pearson, 2006
  
% \bibitem{hutton} 
% Graham Hutton. Programming in Haskell. Cambridge University Press. 2007.

\bibitem{jones} 
  Neil D. Jones. Computability and Complexity from a Programming Perspective. 
  MIT Press 1997.

\bibitem{kozen} 
  Dexter C. Kozen. Automata and Computability. Undergraduate Texts in Computer
  Science. Springer, 1997.
  
\bibitem{linz} 
  Peter Linz. An Introduction to Formal Languages and Automata. 5th. Edition. 
  Jones \& Bartlett. 2011.
  
\bibitem{martin} 
  John Martin. Introduction to Languages and the Theory of Computation, 4th 
  Edition, McGraw-Hill 2010.
  
\bibitem{rich} 
  Elaine Rich. Automata, Computability and Complexity, Theory and Applications. 
  Pearson Prentice Hall, 2008.
  
\bibitem{rosenberg} 
  Arnold L. Rosenberg. The Pillars of Computation Theory, State, Encoding, 
  Nondeterminism. Springer 2010.
  
\bibitem{shallit} 
  Jeffrey Shallit. A Second Course in Formal Languages and Automata Theory. 
  Cambridge University Press 2009.
  
\bibitem{sane}
  S.S.Sane. theory of Computer Science. Technical Publications, 2007
  
\end{thebibliography}


% %\vspace{-1cm}

% \be


% \item (1.5 pts) Para cada $n\geq 0$, se definen las cadenas $a_n$ y $b_n$ en $\{0,1\}^*$ como sigue:

% \[
% a_0=0; \quad b_0=1; \qquad \text{para } n>0,\; a_n=a_{n-1}b_{n-1}; \; \; b_n=b_{n-1}a_{n-1}
% \]

% Demuestre para cada $n\geq 0$ las cadenas $a_n$ y $b_n$ tienen la misma longitud.
% %\item Las cadenas $a_{2n}$ y $b_{2n}$ difieren en cada posición.

% \item (3 pts) A continuación se presenta una definición recursiva de
%   un subconjunto de $\{a,b\}^*$. Proporcione una descripción con
%   palabras y la expresión regular que
%   define a tal lenguaje. 
% % Suponga que la definición incluye una última expresión implícita: ``Un elemento no es parte de $L$, a menos que pueda obtenerse con las expresiones previas''

% \be
% \item $a \in L$ 
% \item Para cada  $x \in L$,  $ax\in L$ y $xb\in L$ 
% \item Son todas.
% % \text{ son parte
% %     de } $L$
% %\item $a \in L \text{; para cada } x \in L \text{, } xb \text{ y } xba \text{ son parte de }L$
% \ee

% % \item (2.5 pts) Para cada $n\geq 0$, se definen las cadenas $a_n$ y $b_n$ en $\{0,1\}^*$ como sigue:

% % \[
% % a_0=0; \quad b_0=1; \qquad \text{para } n>0,\; a_n=a_{n-1}b_{n-1}; \; \; b_n=b_{n-1}a_{n-1}
% % \]

% % Demuestre para cada $n\geq 0$ las cadenas $a_n$ y $b_n$ tienen la misma longitud.
% % %\item Las cadenas $a_{2n}$ y $b_{2n}$ difieren en cada posición.


% \item (2.5 pts) Sea $\al$ una expresión regular arbitraria en el alfabeto $\Sigma$. Encuentre una expresión regular más sencilla que corresponda al mismo lenguaje de la siguiente expresión
% \[
% 	\al(\al^*\al+\al^*)+\al^*
% \]


% \item (3 pts) Diseñe un AFD (Autómata Finito Determinista) que
%   reconozca al lenguaje 
% $$\{w\in\{a,b\}^*\;|\;\text{tal que } w \text{ {\bf no} tiene ni a } aa \text{ ni a } bb \text{ como subcadena} \}$$


% \item ({\bf Hasta 1.5 puntos extra}) Respecto al ejercicio 1,
%   demuestre que las cadenas $a_{2n}$ y $b_{2n}$ difieren en cada posición.
% % Escriba una expresión regular para cada uno de estos lenguajes

% % \be
% % \item El conjunto de cadenas binarias en las cuales el número de ceros es divisible entre 3.
% % \item Las cadenas sobre el alfabeto $\{a, b, c\}$ que contienen un número par de aes
% % \ee

% \ee





\end{document}

%%% Local Variables: 
%%% mode: latex
%%% TeX-master: t
%%% End: 
