\documentclass[letterpaper,11pt]{article}
\usepackage[includeheadfoot,margin=1.3in]{geometry}
\usepackage{anysize}

\usepackage[utf8]{inputenc}
% \usepackage[latin1]{inputenc}
\usepackage[english,spanish]{babel}
\usepackage{lmodern}   % font shapes...
\usepackage[T1]{fontenc} % join the compound symbols as a single symbol

\usepackage{amssymb,amsmath}
\usepackage{mathrsfs}
\usepackage{epsfig}

\usepackage{fancyhdr}
\usepackage{hyperref}
\usepackage{url}

\usepackage{import}
\usepackage{comment}
\usepackage[autostyle=true,spanish=mexican]{csquotes}

\usepackage{alltt}

\usepackage[section]{placeins}

\usepackage{pgf}
\usepackage{tikz}
\usetikzlibrary{automata,arrows,trees}
\usetikzlibrary{babel}
%\pagestyle{fancyplain}
\input{../macrosAyLF}

\title{Aut\'omatas y Lenguajes Formales  \\ 
Facultad de Ciencias UNAM \\ 
Nota de Clase 8
\thanks{\small{Material desarrollado bajo el proyecto UNAM-PAPIME PE102117 
2017--2018.}}} 
\author{Favio E. Miranda Perea \and A. Liliana Reyes Cabello \and
Lourdes Gonz\'alez Huesca}
\date{\today}

\begin{document}
\maketitle

\section{Gram\'aticas Regulares}

Recordemos que una gram\'atica regular es una gram\'atica lineal por la 
derecha o lineal por la izquierda y que no se permite mezclar ambos tipos de 
producciones:
$$\text{derecha} \qquad \qquad A\imp aB\;\;\;A\imp a\;\;\;A\imp\vacia$$
$$\text{izquierda} \qquad \qquad A\imp Ba\;\;\;A\imp a\;\;\;A\imp\vacia$$
con $A,B\in V,\;a\in T$, obs\'ervese que los s\'imbolos no-terminales est\'an 
a la derecha o a la izquierda respectivamente.

\defin{Decimos que un lenguaje~$L$ es regular si existe una gram\'atica 
regular~$G$ que lo genere, es decir, si $L=L(G)$.
}

Tambi\'en recordemos que un lenguaje tiene un tipo $i$ si $L$ es generado por 
una gram\'atica de ese tipo el cual debe asegurarse que $i$ es m\'aximo.
           
\paragraph{Ejemplo:}
Sea el lenguaje~$L=0^\star10^\star 10^\star$ y es generado por:
\[
 \begin{array}{rll}
 S & \imp & A1A1A \\
 A & \imp & 0A\;|\;\vacia
 \end{array}
\]
Esta gram\'atica no es regular, pero el lenguaje si lo es al existir una
gram\'atica regular equivalente:
\[
 \begin{array}{rll}
 S & \imp & 0S\;|\;1A \\
 A & \imp & 0A\;|\;1B \\
 B & \imp & 0B\;|\;\cv \\
 \end{array}
\]

\paragraph{Ejemplo:}
El lenguaje~$L=(a+b)^\star b$ es generado por la siguiente gram\'atica:
\[
 \begin{array}{rll}
 S & \imp & aS\;|\;bC \\
 C & \imp & bC\;|\;aS\;|\;\cv\\
 \end{array}
\]

\section{Lenguajes y gram\'aticas regulares}
Hay una correspondencia entre los lenguajes regulares y las gram\'aticas 
regulares, veamos las dos partes de esta equivalencia que involucran a los 
aut\'omatas finitos:

\be
\item Aut\'omatas Finitos $\;\Imp\;$ Gram\'aticas regulares\\
Dado una m\'aquina finita $M=\pt{Q,\Sigma,\delta,q_0,F}$ existe una 
gram\'atica regular $G=\pt{V,T,S,P}$ tal que $L(M)=L(G)$. 
Es decir, todo lenguaje regular es generado por una gram\'atica regular.

La construcci\'on de una gram\'atica regular~$G$ dado un aut\'omata~$M$ se 
obtiene como sigue:
\bi
 \item Suponemos sin p\'erdida de generalidad que no hay 
  $\vacia$-transiciones en $M$, es decir puede ser que $M$ sea no-determinista.
 \item Se definen las partes de $G$, $V=Q \qquad T=\Sigma \qquad S=q_0$. \\
  Es decir que los estados ser\'an los s\'imbolos no-terminales y los 
  terminales ser\'an exactamente los s\'imbolos del alfabeto.
 \item Las reglas de producci\'on~$P$ se definen como sigue:
  \bi
   \item Si $p\in\delta(q,a)$ entonces agregamos $q\imp ap$ a $P$.
   \item Adem\'as, si $q_f\in\delta(q,a)$ con $q_f\in F$ entonces agregamos 
    $q\imp a$.
   \footnote{No hay necesidad de agregar transiciones del tipo $q\imp \vacia$ 
    ya que se asegura con las opciones anteriores que los estados finales 
    est\'an representados por $q\imp a$ si $\delta(q,a) = q_f$.}
  \ei
\ei


\item Gram\'atica regular $\;\Imp\;$ Aut\'omata Finito\\
Dada una gram\'atica regular~$G=\pt{V,T,S,P}$ existe un aut\'omata 
finito~$M=\pt{Q,\Sigma,\delta,q_0,F}$ tal que $L(M)=L(G)$. Es decir todo 
lenguaje generado por una gram\'atica regular es regular.

Definimos al aut\'omata~$M$ en base a las reglas de producci\'on como sigue:
\bi
 \item Suponemos sin p\'erdida de generalidad que $G$ es lineal derecha.
 \item El alfabeto es exactamente el conjunto de s\'imbolos terminales 
  $\Sigma=T$ y los estados ser\'an $Q=V\cup\{q_F\}\qquad q_0=S\qquad F=\{q_F\}$
 \item La funci\'on de transici\'on~$\delta$ se define como sigue:
  \bi
   \item Si $A\imp aB \in P$ entonces $B\in\delta(A,a)$.
   \item Si $A\imp a \in P$ entonces $q_F\in\delta(A,a)$.
   \item Si $A\imp \vacia \in P$ entonces $q_F\in\delta(A,\vacia)$.
  \ei
\ei

\ee

\paragraph{Ejemplo:}
Considere el siguiente aut\'omata, la gram\'atica correspondiente obtenida con 
el m\'etodo anterior se encuentra a su derecha:

\begin{minipage}{.4\textwidth}
 \centering
  \begin{tikzpicture}[node distance=3.5cm,every 
    node/.style={scale=0.8},semithick]
    \node[state,initial,initial text=] (q0) {$q_0$};
    \node[state] (q1) [right of=q0] {$q_1$};
    \node[state,accepting by double] (q2) [below of=q0] {$q_2$};
    \path[->] (q0) edge node [above] {1} (q1);
    \path[->] (q1) edge [bend left] node [above] {0} (q2);
    \path[->] (q2) edge node [left] {0} (q0);
    \path[->] (q2) edge [bend left] node [left] {1} (q1);
    \path[->] (q0) edge [loop above] node [above] {0} (q0);
    \path[->] (q1) edge [loop above] node [above] {1} (q1);
 \end{tikzpicture}
 \end{minipage}
 \begin{minipage}{.4\textwidth}
 \centering
  \[
  \begin{array}{rcl}
    q_0 & \imp & 0q_0 \mid 1q_1\\
    q_1 & \imp & 1q_1 \mid 0q_2 \mid 0 \\
    q_2 & \imp & 0q_0 \mid 1q_1
  \end{array}
  \]
\end{minipage}

\subsection{Equivalencia entre gram\'aticas lineales}
Se puede probar que toda gram\'atica lineal por la izquierda es equivalente a 
una gram\'atica lineal por la derecha.
Veamos c\'omo convertir gram\'aticas lineales por la izquierda a lineales por 
la derecha y viceversa, para ello utilizamos las transformaciones a 
aut\'omatas finitos.
\be
 \item Dada una gram\'atica lineal por la derecha se debe construir un 
  aut\'omata finito que reconozca al mismo lenguaje.
 \item Para construir la gram\'atica lineal por la izquierda se debe obtener 
  un aut\'omata \textit{dual}:
  \bi
   \item intercambiar los estados inicial y final
   \item invertir las transiciones    
  \ei
 \item Obtener una nueva gram\'atica del aut\'omata nuevo y despu\'es  
  reordenar las reglas de producci\'on $A \imp aB$ en $ A \imp Ba$.
\ee
El m\'etodo anterior se puede usar cuando el aut\'omata tiene un s\'olo estado 
final.\\
Para la transformaci\'on de lineal izquierda a lineal derecha se siguen los 
mismos pasos que antes.

\paragraph{Ejemplo:}
Convertir la siguiente gram\'atica lineal por la derecha en una equivalente 
lineal por la izquierda:
$$ S \imp 0A \qquad \qquad \qquad A \imp 1A \mid \vacia $$
Se obtiene primero el aut\'omata: \hspace*{20pt}
\begin{tikzpicture}[node distance=3.5cm,every 
node/.style={scale=0.8},semithick]
    \node[state,initial,initial text=] (qS) {$q_{S}$};
    \node[state] (qA) [right of=qS] {$q_A$};
    \node[state,accepting by double] (q) [right of=qA] {$q$};
    \path[->] (qS) edge node [above] {0} (qA);
    \path[->] (qA) edge node [above] {$\vacia$} (q);
    \path[->] (qA) edge [loop above] node [right] {1} (qA);
 \end{tikzpicture}
 
\noindent Despu\'es el aut\'omata inverso o dual: \hspace*{20pt}
\begin{tikzpicture}[node distance=3.5cm,every 
node/.style={scale=0.8},semithick]
    \node[state,accepting by double] (qS) {$q_{S}$};
    \node[state] (qA) [right of=qS] {$q_A$};
    \node[state,initial,initial text=] (q) [right of=qA] {$q$};
    \path[->] (qA) edge node [above] {0} (qS);
    \path[->] (q) edge [bend right] node [above] {$\vacia$} (qA);
    \path[->] (qA) edge [loop above] node [right] {1} (qA);
 \end{tikzpicture}

\noindent Las transiciones del aut\'omata anterior son:
$$ q\imp\vacia q_A \qquad \qquad q_A\imp  1q_A \mid 0 $$

\noindent Y por \'ultimo se invierten los s\'imbolos y se elimina $\vacia$:
$$ q\imp q_A \qquad \qquad q_A\imp q_A 1 \mid 0 $$

\section{Aplicaciones pr\'acticas de los lenguajes regulares}
\bi
\item Expresiones regulares para filtrar cadenas en archivos. \\
Hay muchos usos de las expresiones regulares en Unix, por ejemplo en la 
b\'usqueda de patrones o las expresiones usadas por los comandos 
\verb=awk= o \verb=grep= 
\footnote{\url{
http://www.gnu.org/software/grep/manual/grep.html\#Regular-Expressions}}.
Veamos algunos ejemplos:
\bi
 \item \verb= $ ls -ld [[:upper:]]* = \\
  esta l\'inea obtiene todos los archivos que inician con may\'uscula en 
  cierto directorio.
 \item \verb= $ grep '\<c...h\>' /usr/share/dict/words= \\
  este comando obtiene las palabras del diccionario que empiezan en \verb=c=, 
  terminan en \verb=h= y tienen exactamente 5 caracteres (cada punto indica 
  que hay un caracter).
 \item \verb= $ grep -c "go" demo_text= \\
  obtiene el n\'umero de l\'ineas en el archivo \verb=demo_text= que 
  contienen la cadena \verb=go=.
\ei 


\item An\'alisis de programas booleanos\\
Por ejemplo programas que modelan sistemas de estados finitos como el 
comportamiento de un elevador o una m\'aquina despachadora. 
\begin{figure}[!ht]
 \centering
 \includegraphics[width=.6\textwidth]{elevator.png}
 \caption{Eventos de un elevador y puertas}
\end{figure}


\item An\'alisis l\'exico\\
Parte inicial de un compilador que se encarga de separar un programa en 
peque\~nos grupos de caracteres llamados \textit{tokens}, es decir en nombres 
de variables, n\'umeros, \textit{keywords}, etc.
Se describe el analizador l\'exico a trav\'es de una o varias expresiones
regulares para distinguir las clases de tokens.
Por ejemplo, las siguientes expresiones identifican los espacios en blanco, 
nombres y n\'umeros:
\begin{alltt}
 \(\alpha1\) = <hspace> + <newline>,
 \(\alpha2\) = [letter] ([letter] + [digit])*,
 \(\alpha3\) = [digit] [digit]*(\% + E [digit] [digit]*),
\end{alltt}

Se debe procesar un archivo y el analizador verifica subcadenas del programa 
que son precisamente los tokens, usualmente se considera al prefijo m\'as largo 
que sea un token como separador.\\
La forma m\'as sencilla de realizar un an\'alisis l\'exico o \textit{lexer} es 
mediante aut\'omatas no determin\'isticos, debe haber un aut\'omata por cada 
definici\'on de token y el aut\'omata principal es la uni\'on de los 
anteriores. 

\ei
\end{document}
