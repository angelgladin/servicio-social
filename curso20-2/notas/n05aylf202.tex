\documentclass[letterpaper,11pt]{article}
\usepackage[includeheadfoot,margin=1.3in]{geometry}
\usepackage{anysize}

\usepackage[utf8]{inputenc}
% \usepackage[latin1]{inputenc}
\usepackage[english,spanish]{babel}
\usepackage{lmodern}   % font shapes...
\usepackage[T1]{fontenc} % join the compound symbols as a single symbol

\usepackage{amssymb,amsmath}
\usepackage{mathrsfs}
\usepackage{epsfig}

\usepackage{fancyhdr}
\usepackage{hyperref}
\usepackage{url}

\usepackage{import}
\usepackage{comment}
\usepackage[autostyle=true,spanish=mexican]{csquotes}

\usepackage{url}

\usepackage{pgf}
\usepackage{tikz}
\usetikzlibrary{automata,arrows}
\usetikzlibrary{babel}
%\pagestyle{fancyplain}
\input{../macrosAyLF}

\title{Aut\'omatas y Lenguajes Formales  \\ 
Facultad de Ciencias UNAM \\ 
Nota de Clase 5
\thanks{\small{Material desarrollado bajo el proyecto UNAM-PAPIME PE102117 
2017--2018.}}} 
\author{Favio E. Miranda Perea \and A. Liliana Reyes Cabello \and
Lourdes Gonz\'alez Huesca}
\date{\today}



\begin{document}
\maketitle
\section{Aut\'omatas No-Deterministas}

El determinismo de un aut\'omata, deseable desde el punto de vista te\'orico, 
puede provocar complicaciones en la práctica.
Veamos las diferencias entre estos conceptos:
\bi
 \item \textbf{Determinismo:} dado un estado~$q$ y un s\'imbolo~$a$ existe una 
  \'unica transici\'on~$\delta(q,a)=p$, es decir $\delta$ es una 
  funci\'on total.
 \item \textbf{No-determinismo:} no hay una transici\'on \'unica al leer un
  s\'imbolo~$a$ en un estado dado~$q$.
\ei

El no-determinismo se traduce en que hay m\'as de una transici\'on al leer un 
s\'imbolo, es decir, $\delta(q,a)$ deja de ser función. 
O bien no hay transici\'on, es decir, $\delta(q,a)$ no est\'a definida 
($\delta$ se vuelve funci\'on parcial).
Sin embargo la m\'aquina funciona \'unicamente al leer un s\'imbolo y podemos 
decir que existe el no-determinismo sin lectura de s\'imbolos.

Veamos c\'omo se puede agregar el no-determinismo a la definci\'on de 
aut\'omata que vimos anteriormente:
\defin{
Un aut\'omata finito \textbf{no} determinista (AFN) es una quintupla
$M=\pt{Q,\Sigma,\delta,q_0,F}$ donde
\bi
 \item $Q$ es un conjunto finito de estados.
 \item $\Sigma$ es el alfabeto de entrada.
 \item $\delta:Q\times\Sigma\imp \Pe(Q)$ es la funci\'on de 
  transici\'on.
 \item $q_0\in Q$ es el estado inicial.
 \item $F\inc Q$ es el conjunto de estados finales.
\ei
}
Obs\'ervese que la imagen de~$\delta$ es ahora un elemento de~$\Pe(Q)$, es 
decir es un subconjunto de estados de $Q$.
Adem\'as~$\delta(q,a)=\{q_1,q_2,\ldots,q_n\}$ indica que al leer el 
s\'imbolo~$a$ en el estado~$q$ la m\'aquina puede pasar a cualquiera de los
estados $q_1,\ldots,q_n$.
Si $\delta(q,a)=\varnothing$ entonces no hay transici\'on posible desde 
el estado~$q$ al leer~$a$, es decir, la m\'aquina est\'a bloqueada.

\noindent Veamos como las nociones vistas para los AFD se modifican:

\defin{
Sea $M=\pt{Q,\Sigma,\delta,q_0,F}$ un AFN. La funci\'on de 
transici\'on~$\delta$ se extiende a cadenas mediante una funci\'on 
$\dest:Q\times\sest \imp \Pe(Q)$ definida recursivamente como sigue:
\bi
 \item $\dest(q,\vacia) = q $
 \item $\dest(q, wa) = \bigcup_{p\in\dest(q,w)}\delta(p,a)$ \\
  Alternativamente: $\dest(q,aw) = \bigcup_{p\in \delta(q,a)}\dest(p,w)$
\ei
}

\noindent El lenguaje de aceptaci\'on se define mediante~$\dest$ como sigue:
$$ L(M)=\{w\in\sest\;|\;\dest(q_0,w)\cap F\neq\vacio\} $$
Es decir, $w\in L(M)$ si y s\'olo si existe al menos un c\'omputo o un 
procesamiento de~$w$ que conduce a un estado final al iniciar la m\'aquina en 
$q_0$.

\subsection{Eliminaci\'on del no-determinismo}
\label{ssec:elimNdet}

Una pregunta natural es comparar dos aut\'omatas, uno determinista y el otro 
no. As\'i tambi\'en buscar una forma de eliminar el no-determinismo y conservar 
una m\'aquina que acepte un lenguaje dado. 

Todo AFD es a la vez un AFN con la particularidad de que~$\delta(p,a)$ consta 
de un \'unico estado.
La idea para transformar un AFN en un AFD es considerar a cada conjunto de 
estados~$\delta(p,a)$ del AFN como un \'unico estado del nuevo AFD.
A este m\'etodo se conoce como la \textit{construcci\'on de subconjuntos}.
\defin{
Dado un AFN~$M=\pt{Q,\Sigma,\delta_N,q_0,F}$ definimos un AFD
$M^d=\pt{Q^d,\Sigma,\delta,q^d_0,F^d}$ como sigue:
\bi
 \item $Q^d = \Pe(Q)$
 \item $\delta(S,a) = \delta_N(S,a)=\bigcup_{q\in S}\delta_N(q,a)$
 \item $q^d_0 = \{q_0\}$
 \item $F^d = \{S\inc Q\;|\;S\cap F\neq\vacio\}$
\ei  
Ambos aut\'omatas son equivalentes, es decir, $L(M)=L(M^d)$.
}


\section{AFN con \texorpdfstring{$\vacia$}--transiciones}

Otra de las m\'aquinas que son \'utiles para procesar cadenas son las que 
permiten procesar cadenas vac\'ias, no s\'olo al tener como estado final a 
$q_0$ sino que permiten procesar (sub)cadenas vac\'ias en una parte itermedia.
De esta forma se definen los aut\'omatas con transiciones etiquetadas con la 
cadena $\vacia$:
\defin{ 
Un aut\'omata finito \textbf{no} determinista con $\vacia$-transiciones 
(AFN$\cv$) es una quintupla $ M = \pt{Q,\Sigma,\delta,q_0,F}$ donde
\bi
 \item $Q$ es un conjunto finito de estados.
 \item $\Sigma$ es el alfabeto de entrada.
 \item $\delta:Q\times(\Sigma\cup\{\cv\})\imp \Pe(Q)$ es la funci\'on de 
  transici\'on.
 \item $q_0\in Q$ es el estado inicial.
 \item $F\inc Q$ es el conjunto de estados finales.
\ei
}

De la definici\'on de $\delta:Q\times(\Sigma\cup\{\cv\})\imp \Pe(Q)$ se observa 
que la \'unica diferencia est\'a en el dominio de $\delta$.
Es decir que las transiciones de la forma $\delta(q,\cv)=a$ est\'an permitidas 
e indican que la m\'aquina puede cambiar de estado \textit{sin} leer ning\'un 
símbolo.
Esto causa tambi\'en un no-determinismo, que es aun m\'as complicado de        
modelar matem\'aticamente pero tiene grandes ventajas:
\bi
 \item Se permiten m\'ultiples c\'omputos para una cadena de entrada.
 \item Pueden existir c\'omputos bloqueados.
 \item A diferencia de los AFD y AFN simples, pueden existir c\'omputos
  infinitos, es decir, surge la \textbf{no-terminaci\'on}.
 \item La presencia de $\vacia$-transiciones permite mayor libertad en el 
  dise\~no.
\ei

Para extender la definici\'on de transiciones a procesamiento de cadenas es 
necesario introducir un concepto previo que considera los estados a los que la 
m\'aquina puede llegar al incluir las cadenas vac\'ias.

\defin{
Dado un estado~$q$, definimos la~$\vacia$-cerradura de $q$ como el conjunto 
de estados alcanzables desde $q$ mediante cero o m\'as $\vacia$-transiciones.
Es decir 
\[
Cl_{\vacia} (q) =
\{s\in Q \mid \exists p_1,\ldots,p_n \text{ con }
  p_1 = q,\; p_n = s, \; p_i\in\delta(p_{i-1},\vacia)\}
\]
Recursivamente:
\bi
 \item $q\in Cl_{\vacia}(q)$
 \item Si $r\in Cl_{\vacia}(q)$ y $\delta(r,\vacia) = s$ 
  entonces $s\in Cl_{\vacia}(q)$
\ei
Esta definici\'on se extiende a conjuntos de estados como sigue:
$$ Cl_{\vacia}(S)=\bigcup_{q\in S} Cl_{\vacia}(q)$$
}

Con la $\vacia$-cerradura de estados se puede definir la extensi\'on de 
$\delta$:

\defin{
Sea $M=\pt{Q,\Sigma,\delta,q_0,F}$ un AFN$\vacia$. La funci\'on de 
transici\'on~$\delta$ se extiende a cadenas mediante una funci\'on
$$\dest:Q\times(\sest\cup\{\vacia\})\imp \Pe(Q) $$
definida recursivamente como sigue:
\bi
 \item $\dest(q,\vacia) = Cl_{\vacia}(q)$
 \item $\dest(q,wa) = 
    Cl_{\vacia}\Big(\bigcup_{q'\in\dest(q,w)}\delta(q',a)\Big)$
\ei
}

\subsection{Eliminaci\'on de \texorpdfstring{$\cv$}--transiciones}

Estudiaremos el proceso de eliminaci\'on de transiciones $\vacia$. Esta 
eliminaci\'on implica que un AFN es tambi\'en un AFN$\vacia$.
Es decir que cualquier AFN$\vacia$ es inmediatamente un AFN con la 
particularidad de que no existen $\vacia$-transiciones.

\defin{
Dado un AFN$\vacia$, $M_1=\pt{Q_1,\Sigma,\delta_1,q_{0},F_1}$ existe un $AFN$
equivalente $M=\pt{Q,\Sigma,\delta,p_0,F}$, definido mediante:
\bi
 \item $Q:=Q_1$
 \item $p_0:=q_{0}$
 \item $\delta(q,a)=\delta_1^\star(q,a)= 
      Cl_{\vacia}\Big(\bigcup_{p\in Cl_{\vacia}(q)}\delta_1(p,a)\Big)$
 \item $F:=F_1\cup\{q_{0}\}$ si $Cl_{\vacia}(q_{0})\cap F_1\neq\vacio$\\
 $F:=F_1$ en caso contrario.
\ei
}



\paragraph{Ejemplo:}
Consideremos el siguiente aut\'omata:
\begin{center}
\begin{tikzpicture}[node distance=2.5cm,every 
    node/.style={scale=0.8},semithick]
    \node[state,initial,initial text=] (q0) {$q_0$};
    \node[state] (q1) [right of=q0] {$q_1$};
    \node[state] (q2) [above of=q1] {$q_2$};
    \node[state,accepting by double] (q3) [right of=q1] {$q_3$};
    \path[->] (q0) edge node [above] {$\vacia$} (q1);
    \path[->] (q0) edge [loop above] node [above] {0} (q0);
    \path[->] (q1) edge node [above] {$\vacia$} (q3);
    \path[->] (q1) edge [bend right] node [right] {0} (q2);
    \path[->] (q2) edge [bend right] node [left] {1} (q1);
    \path[->] (q3) edge [loop above] node [above] {0} (q3);
 \end{tikzpicture}
\end{center}
Para eliminar las $\vacia$-transiciones calculamos las cerraduras de cada 
estado:
\[
\begin{array}[c]{rc|c|c|c|c|}
  &\delta &  0 & 1 &\vacia & Cl_{\vacia} \\\hline 
 inicial & q_0 & \{q_0\} & \vacio & \{q_1\} & \{q_0,q_1,q_3\} \\\hline 
         & q_1 & \{q_2\} & \vacio & \{q_3\} & \{q_1,q_3\}  \\\hline 
         & q_{2} & \vacio & \{q_1\} & \vacio &  \{q_2\}  \\\hline
  final  & q_{3} & \{q_3\} & \vacio & \vacio &  \{q_3\}  \\\hline
\end{array}
\]

Y calculamos la nueva funci\'on de transici\'on:
\[
\begin{array}[c]{rc|c|c|}
  &\delta' &  0 & 1 \\\hline 
 inicial,\,final & q_0 & 
    \delta^\star(q_0,0)= 
      Cl_{\vacia}\Big(\bigcup_{p\in Cl_{\vacia}(q_0)}\delta(p,0)\Big) =
      &  \vacio \\
      & & Cl_{\vacia}\Big(\bigcup_{p\in \{q_0,q_1,q_3 \}}\delta(p,0)\Big) = &\\
      & & Cl_{\vacia}\Big(\{q_0\} \cup \{q_2\} \cup \{q_3\}\Big) = &\\
      & & Cl_{\vacia}(\{q_0\}) \cup Cl_{\vacia}(\{q_2\}) \cup 
Cl_{\vacia}(\{q_3\}) = \{q_0,q_1,q_3,q_2\} &\\\hline
 & q_1 &  Cl_{\vacia}\Big(\bigcup_{p\in Cl_{\vacia}(q_1)}\delta(p,0)\Big) =
      & \vacio \\
      & &Cl_{\vacia}\Big(\bigcup_{p\in \{q_1,q_3\}}\delta(p,0)\Big) =
      Cl_{\vacia}\Big(\{q_2\} \cup \{q_3\}\Big) = \{q_2,q_3\} &\\\hline
    & q_{2} & \vacio & \{q_1,q_3\}  \\\hline
  final  & q_{3} & \{q_3\} & \vacio \\\hline
\end{array}
\]

Ahora $q_0$ es final dado que 
$\{q_3\} \cap Cl_{\vacia}(q_0) =\{q_0,q_1,q_2,q_3\} \neq \vacio$, 
es decir que los finales del aut\'omata original aparecen en la 
cerradura-$\vacia$ del estado $q_0$ (se acepta a $\vacia$).


\section{Equivalencias}
Nuevamente surgen preguntas como la equivalencia entre aut\'omatas 
deterministas, no deterministas y con transiciones $\vacia$. 

En la subsecci\'on~\ref{ssec:elimNdet} se estudi\'o la forma de eliminar el 
no-determinismo de un aut\'omata, creando un AFD. 
Es decir AFN $\Imp$ AFD. 

En la subsecci\'on pasada se revis\'o la forma de eliminar 
$\vacia$-transiciones y de esta forma obtener un AFN. As\'i tambi\'en se 
obsev\'o que un AFN es exactamente un AFN$\vacia$ ya que incluye el 
no-determinismo pero sin transiciones de la cadena vac\'ia.

Esta equivalencia, AFN $\;\Iff\;$ AFN$\vacia$, cierra el ciclo de equivalencias 
de aut\'omatas finitos. 
Cualquier tipo de aut\'omata finito puede convertirse en un AFD y viceversa, 
es decir:
\begin{center}
  AFD $\Iff\;$ AFN $\Iff\;$ AFN$\vacia$
\end{center}

\vspace*{15pt}

Nuestra siguiente meta es probar el Teorema de Kleene, el cual es uno de los 
resultados m\'as importantes en la Teor\'ia de la Computaci\'on pues asegura la 
equivalencia entre dos de nuestros tres conceptos fundamentales: los 
aut\'omatas finitos y los lenguajes regulares.

\end{document}
