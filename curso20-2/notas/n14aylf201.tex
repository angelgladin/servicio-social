\documentclass[letterpaper,11pt]{article}
\usepackage[includeheadfoot,margin=1.3in]{geometry}
\usepackage{anysize}

\usepackage[utf8]{inputenc}
% \usepackage[latin1]{inputenc}
\usepackage[english,spanish]{babel}
\usepackage{lmodern}   % font shapes...
\usepackage[T1]{fontenc} % join the compound symbols as a single symbol

\usepackage{amssymb,amsmath}
\usepackage{mathrsfs}
\usepackage{epsfig}

\usepackage{fancyhdr}
\usepackage{hyperref}
\usepackage{url}

\usepackage{import}
\usepackage{comment}
\usepackage[autostyle=true,spanish=mexican]{csquotes}

\usepackage{alltt}

\usepackage[section]{placeins}
\usepackage{wrapfig}
\usepackage{multicol}

\usepackage{pgf}
\usepackage{tikz}
\usetikzlibrary{automata,arrows,trees,positioning,calc}
\usetikzlibrary{babel}
%\pagestyle{fancyplain}
\input{../macrosAyLF}

\title{Aut\'omatas y Lenguajes Formales  \\ 
Facultad de Ciencias UNAM \\ 
Nota de Clase 14
\thanks{\small{Material desarrollado bajo el proyecto UNAM-PAPIME PE102117 
2017--2018.}}} 
\author{Favio E. Miranda Perea \and A. Liliana Reyes Cabello \and
Lourdes Gonz\'alez Huesca}
\date{\today}

\begin{document}
\maketitle

\section{Introducci\'on}
Las m\'aquinas de Turing como modelos de c\'omputo que tienen varios usos, como 
aceptadoras de lenguajes, como generadoras de lenguajes y como calculadoras de 
funciones, tienen una generalizaci\'on para realizar c\'omputos de forma 
\enquote{universal}.

La idea de proporcionar una m\'aquina capaz de realizar cualquier c\'omputo 
utilizando una memoria para almacenar datos fue propuesta por A. Turing en 1936 
en un art\'iculo donde describe estas m\'aquinas pero sobre todo resalta la 
idea de que las m\'aquinas pueden realizar los c\'alculos de cualquier 
algoritmo concebido.

La tesis de Church-Turing afirma que la noci\'on intuitiva de algoritmo es 
capturada de manera exacta por la noci\'on matem\'atica de m\'aquina de Turing:
\begin{center}
\textbf{
Tesis de Church-Turing: Un problema es soluble 
algor\'itmicamente \\si y s\'olo si es soluble mediante una m\'aquina de 
Turing.}
\end{center}

Es decir, las m\'aquinas de Turing implementan a cualquier algoritmo.      
Equivalentemente, una funci\'on es computable si y s\'olo si es soluble mediante 
una m\'aquina de Turing.

Estudiaremos en esta nota las m\'aquinas universales as\'i como sus 
codificiaciones. 

\subsection{La M\'aquina Universal}
Pensar en la existencia de una m\'aquina de Turing que se comporte de la 
misma forma que una computadora real es posible. 
Es decir, una m\'aquina que sea \'util para m\'ultiples prop\'ositos y que sea 
tambi\'en capaz de programar y ejecutar m\'aquinas de Turing.

Turing describi\'o a la m\'aquina universal~$\mathcal{M}$ como una m\'aquina 
que tiene programa para resolver un problema que es codificado como una 
m\'aquina espec\'ifica o de prop\'osito especial $M_1$ y un conjunto de datos a 
procesarse que ser\'a la entrada de $M_1$.

As\'i la m\'aquina universal~$\mathcal{M}$ simular\'a el comportamiento de 
$M_1$ que procesar\'a los datos de entrada. 


\section{Una codificaci\'on de m\'aquinas de Turing}
Para llevar a cabo la simulaci\'on es necesario que las m\'aquinas de 
prop\'osito especial junto con sus datos de entrada sean codificados en cadenas 
para que~$\mathcal{M}$ pueda procesarlos.
Veamos una forma de codificaci\'on de una m\'aquina espec\'ifica para obtener 
una cadena.

\noindent Se tienen las siguientes convenciones:
\bi
 \item Se fija un alfabeto de entrada $\Sigma$.
 \item Se asume $Q=\{q_0,\ldots,q_n\}$ siendo $q_0$ el estado inicial y     
  $q_1$ el \'unico estado final.
 \item El alfabeto de la cinta es de la forma $\Gamma=\{s_1,s_2,\ldots, s_p\}$
   siendo $s_1=\blanks$. 
\ei

\paragraph{Codificaci\'on de una cinta}
Las siguientes caracter\'isticas definen una codificaci\'on para la cinta de 
una m\'aquina dada:
\bi
 \item El s\'imbolo $s_i$ se codifica como $1^i$.
 \item Las cadenas $\Gamma^\star$ se codifican separando cada s\'imbolo con 
  un~$0$.
 \item En general, si $w=s_{n_1}s_{n_2}\ldots s_{n_k}$ la codificaci\'on de $w$ 
  es: $01^{n_1}01^{n_2}0\ldots 01^{n_k}0$
\ei 
Por ejemplo, si $\Gamma=\{\blanks,a,b\}$ entonces codificamos 
$\blanks:=1,\;a:=11,\;b:=111$.
As\'i, la palabra $bab\blanks aa$ se codifica como: $01110110111010110110$


\paragraph{Codificaci\'on de los estados}
Considerando que los estados de una m\'aquina est\'an dados por 
$Q=\{q_0,q_1,\ldots,q_n\}$ entonces se pueden codificar de la siguiente forma:
\bi
 \item El estado $q_i$ (en la posici\'on $i$ del conjunto~$Q$ visto como lista) 
  se codifica an\'alogamente a los s\'imbolos, es decir mediante cadenas de 
  $1$'s pero con un s\'imbolo m\'as que la posici\'on del estado:
   $$  q_i \text{ se codifica como } 1^{i+1}$$
  \item As\'i el estado inicial $q_0$ se codifica con $1$ y el estado final 
  $q_1$ con $11$.
\ei

\paragraph{Codificaci\'on de las direcciones de desplazamiento}
Consideramos que una m\'aquina puede tener tres desplazamientos 
$\{\der,\izq, -\}$ codificados respectivamente como $1$, $11$ y $111$.

\paragraph{Codificaci\'on de las transiciones}
Recordemos que una transici\'on en una m\'aquina de Turing es  
$\delta(q_i,s_k) = (q_j,b_\ell,D)$. 
Entones, los estados, s\'imbolos y direcci\'on de desplazamiento se codifican de 
la manera indicada anteriormente.
Y la transici\'on se codifica escribiendo en orden los c\'odigos respectivos 
separados por ceros:
$$ 01^{i+1}01^k01^{j+1}01^\ell01^n0\;\; \text{ donde } n=1,2,3$$
Por ejemplo $\delta(q_2,s_3)=(q_0,s_5,\izq)$ se codifica como 
$01^301^30101^501^20$

\vspace*{10pt}

Una m\'aquina de Turing queda completamente determinada mediante su funci\'on de
transici\'on~$\delta$.
Y se codifica mediante la sucesi\'on de los c\'odigos de sus transiciones 
sin separaciones. 
Es decir, si $C_1,\ldots,C_k$ son los c\'odigos de todas las transiciones 
de~$M$. Entonces~$M$ se codifica mediante $ C_1C_2\ldots C_k$

Obs\'ervese que no hay ambig\"uedad pues cada transici\'on tiene      
exactamente seis ceros, adem\'as dos ceros consecutivos indican que inicia otra 
transici\'on.


\paragraph{Ejemplo:}
Considere la m\'aquina~$M$ dada por:
\[
 \begin{array}{cc}
  \delta(q_0,a)=(q_2,b,\der) & \delta(q_2,b)=(q_3,c,\der)  \\
  \delta(q_3,a)=(q_1,c,\der) & \delta(q_1,b)=(q_3,\blanks,-) 
 \end{array}
\]
Se codifica mediante $a:=11,\;b:=111,c:=1111$ como sigue:
$$  0101101110111010011101110111101111010 $$
$$  01111011011011110100110111011110101110 $$

\paragraph{Ejemplo:}
Considere el siguiente macro que modifica la $a$ de m\'as a la izquierda de una 
cadena cambi\'andola por una $b$ y dejando la cabeza lectora al inicio de la 
cadena:
\begin{center}
\begin{tikzpicture}[node distance=3.3cm,every node/.style={scale=0.8},semithick]
    \node[state,initial,initial text=] (q0) {$q_{0}$};
    \node[state] (q2) [right of=q0] {$p$};
    \node[state,accepting by double] (q3) [right of=q2] {$q_1$};
    \path[->] (q0) edge node [above] {$a/b,\izq$} node
     [above=12pt]  {$\blanks/\blanks,\izq$} (q2);
    \path[->] (q2) edge node [above] {$\blanks/\blanks,\der$} (q3);
    \path[->] (q0) edge [loop above] node [above] {$b/b,\der$} (q0);
    \path[->] (q2) edge [loop above] node [above] {$b/b,\izq$} (q2);
\end{tikzpicture}
\end{center}
La codificaci\'on es la siguiente:
$$ 010111010111010 01011011101110110 $$
$$ 01010111010110 01110111011101110110 $$
$$ 01110101101010 $$ 
donde $\blanks:= 1$, $a:=11$, $b:= 111$ y los estados se definen como 
$q_0 := 1$, $q_1 := 11$ y $p:= 111$.

\vspace*{10pt}

\noindent Observaciones:
\bi
 \item La codificaci\'on de una m\'aquina de Turing no es \'unica, puesto que 
  el orden de las transiciones no importa y un orden distinto genera una 
  codificaci\'on distinta.
  \item De hecho si una m\'aquina~$M$ tiene $n$ transiciones, existen $n!$
   codificaciones distintas para $M$.
  \item El proceso de codificaci\'on puede revertirse, no es dif\'icil definir 
  un algoritmo que decida si una secuencia binaria representa un c\'odigo 
  v\'alido para una m\'aquina de Turing y en tal caso lo decodifique.
\ei


\section{Uso de la M\'aquina Universal}
Como mencionamos antes, la m\'aquina universal como dispositivo de 
programaci\'on y ejecuci\'on de programas es principalmente la simulaci\'on de 
una m\'aquina con un prop\'osito espec\'ifico.

Veamos una forma de c\'omo simular una m\'aquina de Turing~$M$ que se 
ejecutar\'a con cierta entrada~$w$ en la m\'aquina universal~$\mathcal{M}$:
\bi
 \item Codificar~$M$ y $w$, respectivamente las llamaremos $\mathcal{C}_M$ y 
  $\mathcal{C}_w$
 \item La m\'aquina universal~$\mathcal{M}$ tiene 5 cintas:
 \begin{itemize}
  \item[T1] la cinta de entrada que s\'olo almacenar\'a la cadena de entrada 
    codificada~$\mathcal{C}_w$.
  \item[T2] la cinta de descripci\'on que contiene la 
    codificaci\'on~$\mathcal{C}_M$ y la cual no se reescribir\'a.
  \item[T3] la cinta de trabajo que contendr\'a exactamente la cinta de~$M$.
  \item[T4] la cinta para saber el estado actual de~$M$.
  \item[T5] una cinta especial para guardar la direcci\'on en que se mover\'a 
    la cabeza de~$M$.
 \end{itemize}
 \item La fase de inicializaci\'on tiene las siguentes acciones:
  \begin{enumerate}
   \item copiar la cadena de entrada en la cinta T3
   \item inicializar T4 con el c\'odigo del estado inicial de~$M$
   \item almacenar los c\'odigos de las direcciones en T5
   \item mover las cabezas de las 5 cintas a la parte m\'as izquierda de las 
    cadenas que almacenan.
  \end{enumerate}
 \item La m\'aquina universal~$\mathcal{M}$ est\'a en un ciclo continuo para la 
  simulaci\'on, es decir que simular\'a cada transici\'on~$M$ en cada 
  iteraci\'on.
 \item El ciclo de la m\'aquina inicia siempre con las cabezas de las cintas lo 
  m\'as la izquierda de los datos almacenados excepto por la cinta T3 cuya 
  cabeza est\'a en el estado actual.
 \item El ciclo tiene varios pasos:
  \begin{enumerate}
   \item La m\'aquina universal~$\mathcal{M}$ se mueve en la 
    codificaci\'on~$\mathcal{C}_M$ para escoger una transici\'on adecuada al 
    comparar las entradas y los datos en las otras cintas correspondientes.
   \item Cuando encuentra la transici\'on deseada copia el nuevo estado en T4, 
    el s\'imbolo en T3 y mueve la cabeza de T3 seg\'un la direcci\'on. 
   \item Al final del ciclo mueve las cabezas de T2 y T4 a la izquierda.
  \end{enumerate}  
\ei
Esta codificaci\'on es exagerada en el n\'umero de cintas en~$\mathcal{M}$ ya 
que se pueden simplificar a menos e incluso se puede usar una s\'ola cinta. 
El hecho de tener los datos en diferentes cintas facilita la selecci\'on de la 
transici\'on dado que todo est\'a codificado.


La parte de simulaci\'on s\'olo se describir\'a dado que la m\'aquina universal 
es un aparato muy complejo pero se puede pensar que se tienen macros para 
cada una de las acciones de la m\'aquina. 

\end{document}
