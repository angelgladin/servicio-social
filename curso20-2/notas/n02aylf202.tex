\documentclass[letterpaper,11pt]{article}
\usepackage[includeheadfoot,margin=1.3in]{geometry}
\usepackage{anysize}

% \usepackage[utf8]{inputenc}
\usepackage[latin1]{inputenc}
\usepackage[spanish]{babel}
\usepackage{lmodern}   % font shapes...
\usepackage[T1]{fontenc} % join the compound symbols as a single symbol

\usepackage{amssymb,amsmath}
\usepackage{mathrsfs}
\usepackage{epsfig}

\usepackage{fancyhdr}
\usepackage{hyperref}
\usepackage{url}

\usepackage{import}
\usepackage{comment}
\usepackage[autostyle=true,spanish=mexican]{csquotes}

\usepackage{url}

%\pagestyle{fancyplain}
\input{../macrosAyLF}

\title{Aut\'omatas y Lenguajes Formales 2020-2 \\ 
Facultad de Ciencias UNAM \\ 
Nota de Clase 2
\thanks{\small{Material desarrollado bajo el proyecto UNAM-PAPIME PE102117 
2017--2018.}}} 
\author{Favio E. Miranda Perea \and A. Liliana Reyes Cabello \and
Lourdes Gonz\'alez Huesca}
\date{\today}

\begin{document}
\maketitle

Para entender los fundamentos de la computaci\'on es necesario estudiar otros 
conceptos como las funciones computables, los c\'omputos que dependen de las 
construcciones disponibles en los lenguajes de programaci\'on y tambi\'en en 
las 
propias computadoras. 

Estos conceptos ser\'an abstraidos por las m\'aquinas abstractas y el estudio 
de 
las propiedades de ellas. 
Como vimos anteriormente, las cadenas y los lenguajes son conceptos b\'asicos 
para estudiar los datos que procesar\'an los c\'omputos.
Con el concepto de lenguaje podemos reformular las preguntas acerca
de los datos de entrada y salida en una computadora:
\bi
 \item ?` Cu\'al es el lenguaje de entrada de una computadora dada? 
 \item ?` Cu\'al es el lenguaje de salida ? 
 \item ?` Ser\'an estos lenguajes describibles finitamente? 
\ei

Estas preguntas est\'an relacionadas a las preguntas que caracterizan a los 
modelos de c\'omputo y los aspectos que representan.
Los modelos de c\'omputo est\'an dise\~nados como un marco para estudiar el 
reconocimiento de lenguajes. 
En este curso se estudiar\'an tres clases de modelos de c\'omputo que se 
caracterizan por diferentes propiedades:
\bi
\item memoria finita: aut\'omatas finitos
\item memoria finita en un dispositivo de almacenamiento: aut\'omatas de pila
\item memoria sin restricci\'on: m\'aquinas de Turing, l\'ogica combinatoria, 
c\'alculo lambda.
\ei

\medskip

Estos modelos han sido estudiados por muchos, en particular por Noam Chomsky 
quien ha propuesto la noci\'on de lenguaje y de gram\'atica (este concepto se 
revisar\'a m\'as adelante en el curso) para establecer una equivalencia entre 
ellos.
En 1956, Chomsky propuso una jerarqu\'ia, que lleva su nombre, para clasificar 
los modelos de c\'omputo, los lenguajes que son reconocidos por estos modelos y 
las gram\'aticas que describen los lenguajes. 
Esta equivalencia se basa en las caracter\'isticas de los modelos para abstraer 
lenguajes, ya sea usando m\'aquinas o usando gram\'aticas.

\newpage

La jerarqu\'ia establece un orden en los lenguajes o gram\'aticas que depende 
de la cantidad y organizaci\'on de la memoria requerida para procesar un 
lenguaje:\\
% tabla de la jeraquia completa

\begin{tabular}{c|c|c|c}
 Tipo & Lenguajes & Modelo de C\'omputo & Gram\'aticas \\\hline\hline
 3 & Regulares & M\'aquinas de estados finitos & Regulares \\\hline
 2 & Libres de Contexto & Au\'omatas de Pila & Libres de contexto\\\hline
 1 & Sensibles al contexto & M\'aquina de Turing & No-contra\'ibles \\
 & & limitada en memoria  & \\\hline
 0 & Recursivamente enumerables o & M\'aquina de Turing & Sin restricci\'on \\
 & semidecidibles  & & \\
\end{tabular}
\smallskip

Existe otro tipo de lenguajes que se encuentra entre el tipo 0 y el 1, estos 
son los lenguajes recursivos pero no pertenecen a esta clasificaci\'on.\\
La relaci\'on de contenci\'on en la jerarqu\'ia es
\[
\mathcal{L}_3\subseteq \mathcal{L}_2\subseteq \mathcal{L}_1\subseteq 
\mathcal{L}_0
\]

\noindent Y al incluir los lenguajes recursivos, las contenciones son las 
siguientes: todo lenguaje regular es libre de contexto; todo lenguaje libre de 
contexto es sensible al contexto; todo lenguaje sensible al contexto es 
recursivo y todo lenguaje recursivo es recursivamente enumerable.

\medskip

La jerarqu\'ia de Chomsky permite refinar la teor\'ia de la computaci\'on 
clasificando lenguajes en funci\'on de los recursos computacionales 
necesarios para reconocerlos, es decir cada categor\'ia tiene 
asociado un tipo de m\'aquina que reconoce a sus lenguajes.
A continuaci\'on empezaremos a estudiar los lenguajes y las m\'aquinas que los 
reconocen comenzando por los regulares.




% revisar libro de elaine rich... cap 3.

\end{document}
