\documentclass[letterpaper,11pt]{article}
\usepackage[includeheadfoot,margin=1.3in]{geometry}
\usepackage{anysize}

\usepackage[utf8]{inputenc}
% \usepackage[latin1]{inputenc}
\usepackage[english,spanish]{babel}
\usepackage{lmodern}   % font shapes...
\usepackage[T1]{fontenc} % join the compound symbols as a single symbol

\usepackage{amssymb,amsmath}
\usepackage{mathrsfs}
\usepackage{epsfig}

\usepackage{fancyhdr}
\usepackage{hyperref}
\usepackage{url}

\usepackage{import}
\usepackage{comment}
\usepackage[autostyle=true,spanish=mexican]{csquotes}

\usepackage{alltt}

\usepackage[section]{placeins}
\usepackage{wrapfig}
\usepackage{multicol}

\usepackage{pgf}
\usepackage{tikz}
\usetikzlibrary{automata,arrows,trees,positioning,calc}
\usetikzlibrary{babel}
%\pagestyle{fancyplain}
\input{../macrosAyLF}

\title{Aut\'omatas y Lenguajes Formales  \\ 
Facultad de Ciencias UNAM \\ 
Nota de Clase 13
\thanks{\small{Material desarrollado bajo el proyecto UNAM-PAPIME PE102117 
2017--2018.}}} 
\author{Favio E. Miranda Perea \and A. Liliana Reyes Cabello \and
Lourdes Gonz\'alez Huesca}
\date{\today}

\begin{document}
\maketitle

\section{Máquinas de Turing}
Las máquinas de Turing (MT) son máquinas \textbf{idealizadas} capaces de 
realizar cómputos y corresponden a las m\'aquinas que reconocen lenguajes de 
tipo 0 en la Jerarquía de Chomsky. \\
Una m\'aquina de Turing consiste de una cinta \textit{infinita} dividida en 
sectores (cuadros) y una cabeza de lecto-escritura.
Cada sector de la cinta contiene un símbolo de cierto alfabeto de entrada o 
bien el símbolo blanco denotado por $\blanks$.
La cabeza lee el sector y puede escribir sobre él, as\'i como moverse a la
izquierda o a la derecha.
\begin{center}
\begin{tikzpicture}[every node/.style={block},
      block/.style={minimum height=2.5em,outer sep=0pt,draw,rectangle,node 
      distance=0pt}]
   \node (A) {$a$};
   \node (B) [left=of A] {$\blanks$};
   \node (C) [left=of B] {$\ldots$};
   \node (D) [right=of A] {$b$};
   \node (E) [right=of D] {$\blanks$};
   \node (G) [right=of E] {$\ldots$};
   \node (F) [minimum height=2.3em,below = 0.75cm of A,draw=red,thick] {\textsf 
q};
   \draw[-latex] (F) -- (A);
   \draw[-latex,blue] ($(F.east)!0.5!(A.east)$) -- ++(7mm,0);
   \draw (C.north west) -- ++(-1cm,0) (C.south west) -- ++ (-1cm,0) 
                 (E.north east) -- ++(1cm,0) (E.south east) -- ++ (1cm,0);
\end{tikzpicture}
\end{center}

\vspace*{-20pt}

\defin{ Una máquina de Turing es una s\'eptupla:
$$  M=\pt{Q,\Sigma,\Gamma,\delta,q_0, \blanks, F} $$
\bi
 \item $Q\neq\vacio$ es un conjunto finito de estados.
 \item $\Sigma$ es el alfabeto de entrada.
 \item $\Gamma$ es el alfabeto de la cinta, el cual incluye a $\Sigma$, es
   decir, $\Sigma\inc\Gamma$.
 \item $\delta:Q\times\Gamma\imp Q\times\Gamma\times\{\izq,\der\}$ 
  es la función de transición
 \item $q_0\in Q$ es el estado inicial.
 \item $\blanks\in\Gamma$ es el símbolo blanco tal que $\blanks\notin\Sigma$.
 \item $F\subseteq Q$ es el conjunto de estados finales. $F$ podría ser vacío.
\ei
}
La funci\'on de transici\'on est\'a determinada por:
$$\delta:Q\times\Gamma\imp Q\times\Gamma\times\{\izq,\der\}$$
es decir $\delta(q,a)=(p,b,D)$ donde
\bi
 \item El estado actual es $q$ y el símbolo a leer es $a$.
 \item La transición es hacia el estado $p$.
 \item $b$ es el símbolo escrito en el lugar de $a$ o en el sector donde se 
  encontraba $a$.
 \item La cabeza se mueve un sector según la dirección dada por 
  $D\in\{\izq,\der\}$. \\
 \textbf{Dichos movimientos se realizan después de leer~$a$ y escribir~$b$.}
\ei

\paragraph{Ejemplo:}
Si $\delta(q,a)=(p,b,\der)$ entonces
\bi
 \item Estado actual: $q$
 \item Símbolo a leer: $a$.
 \item La cabeza borra $a$ y escribe $b$.
 \item El nuevo estado es $p$.
 \item La cabeza se mueve un sector a la derecha.
\ei

\paragraph{Ejemplo:}  
Considere la siguiente m\'aquina 
$$M=\pt{\{q_0,q_1\},\{a,b\},\{a,b,\blanks\},\delta,q_0,\blanks,\{q_1\}}$$
donde la funci\'on de transici\'on est\'a definida por:
\[ 
\delta(q_0,a)=(q_0,b,\der) \qquad \qquad \delta(q_0,b)=(q_0,b,\der)
\qquad \qquad \delta(q_0,\blanks)=(q_1,\blanks,\izq)
\]

\paragraph{Ejemplo:}
Considere el lenguaje $L=\{a^nb^n\;|\;n\geq 1\}$, una MT que reconoce cadenas 
del lenguaje est\'a determinada por la siguiente funci\'on de transici\'on 
donde $F=\{q_4\}$:
\[
\begin{array}{c|c|c|c|c|c}
\delta & a & b & X & Y & \blanks \\ \hline
q_0 & (q_1,X,\der) & & & (q_3,Y,\der) & \\ \hline
q_1 & (q_1,a,\der) & (q_2,Y,\izq) & & (q_1,Y,\der) & \\ \hline
q_2 & (q_2,a,\izq) &  & (q_0,X,\der) & (q_2,Y,\izq) & \\ \hline
q_3 &  &  &  & (q_3,Y,\der) & (q_4,\blanks,\der) \\ \hline
q_4 &  &  &  &  &  \\ 
\end{array}
\]


\subsection{Representaci\'on gr\'afica de una M\'aquina de Turing}

Como hemos visto en otras m\'aquinas o aut\'omatas, existe una representaci\'on 
gr\'afica que modela la transici\'on entre estados:
\begin{itemize}
 \item cada nodo es un estado;
 \item las aristas dirigidas son las transiciones y est\'an etiquetadas por el 
  s\'imbolo que lee la cabeza, el s\'imbolo que escribir\'a y el movimiento que 
  realizar\'a;
 \item hay un estado inicial y uno o varios estados finales.
\end{itemize}


\paragraph{Ejemplo:}
Una m\'aquina de Turing que acepta al lenguaje 
$$L=\{w\in\{0,1\}^\star \mid w\;\text{ tiene un número par de ceros }\}$$
est\'a determinada por la funci\'on de transici\'on siguiente y se describe 
gr\'aficamente como:
\begin{center}
\begin{minipage}{.45\textwidth}
 \centering
\[
\begin{array}{c|c|c|c}
\delta & 0 & 1 & \blanks \\ \hline
q_0 & (q_1,0,\der) & (q_0,1,\der) & (q_f,\blanks,\der)  \\ \hline
q_1 & (q_0,0,\der) & (q_1,1,\der)&   \\ \hline
q_f &  &  &  
\end{array}
\]
\end{minipage}
\begin{minipage}{.4\textwidth}
 \centering
\begin{tikzpicture}[node distance=3.3cm,every node/.style={scale=0.8},semithick]
    \node[state,initial,initial text=] (q0) {$q_{0}$};
    \node[state] (q1) [right of=q0] {$q_1$};
    \node[state,accepting by double] (q2) [below of=q1] {$q_f$};
    \path[->] (q0) edge [loop above] node [above] {$1/1,\der$} (q0);
    \path[->] (q0) edge [bend left] node [above] {$0/0,\der$} (q1);
    \path[->] (q1) edge [loop above] node [above] {$1/1,\der$} (q1);
    \path[->] (q1) edge [bend left] node [above] {$0/0,\der$} (q0);
    \path[->] (q0) edge node [left] {$\blanks/\blanks,\der$} (q2);
 \end{tikzpicture}
\end{minipage}
\end{center}

% \paragraph{Ejemplo:}
% Considere la m\'aquina siguiente 
% $$M=\pt{\{q_0,q_1\},\{a,b\},\{a,b,\blanks\},\delta,q_0,\blanks,\vacio}$$
% \bi
%  \item[] $\delta(q_0,a)=(q_1,a,\der)$
%  \item[] $\delta(q_0,b)=(q_1,b,\der)$
%  \item[] $\delta(q_0,\blanks)=(q_1,\blanks,\der)$
%  \item[] $\delta(q_1,a)=(q_0,a,\izq)$
%  \item[] $\delta(q_1,b)=(q_0,b,\izq)$
%  \item[] $\delta(q_1,\blanks)=(q_0,\blanks,\izq)$
% \ei

\vspace*{10pt}

Existen diferentes versiones de m\'aquinas de Turing, cada una para ser 
utilizada en casos especiales. Estudiemos primero la formalizaci\'on 
est\'andar de ellas y enfatizaremos sus caracter\'isticas.

\subsection{Máquina Est\'andar de Turing}

Estas m\'aquinas tienen una cinta infinita en ambas direcciones. Se permite un 
número arbitrario de movimientos en cualquier dirección.

Una caracter\'istica importante es que estas máquinas son deterministas:
$\delta$ define a lo más un  movimiento para cada configuración posible.
No hay transiciones desde estados finales, es decir, $\delta(q,a)$ no está 
definida si $q\in F$.
Tampoco hay un archivo especial de entrada o salida, se asume que la máquina 
contiene algo al final y al principio del proceso.\\
Se considera que la cadena a procesar est\'a almacenada en la cinta: cada 
s\'imbolo de la cadena est\'a en un sector de la cinta.
La cabeza se encuentra al inicio de la cadena, o puede suponerse que se 
encuentra un sector antes de la cadena. Por esto se puede pensar que la cinta 
es un dispositivo de almacenamiento. 


\defin{
Una configuración o descripción instant\'anea
$$a_1 a_2\ldots a_{k-1}\textbf{q} a_k a_{k+1}\ldots a_n $$
está determinada por:
\bi
 \item El estado actual de la unidad de control (cabeza).
 \item El contenido de la cinta.
 \item La posición de la unidad de control.
 \item La configuración inicial es: $q_0 w$
\ei
}
  
\noindent Las configuraciones nos permiten formalizar la noci\'on de cómputos 
en las Máquinas de Turing:
\defin{
Un cómputo o paso de computación es el cambio de una descripción instant\'anea 
a otra mediante una transición dada por $\delta$:
$$u\textbf{q}av\vdash ub\textbf{p}v\;\; \text{ si y sólo si } \;\;
\delta(q,a)=(p,b,\der)$$
$$uc\textbf{q}av\vdash u\textbf{p}cbv \;\;\text{ si y sólo si } \;\;
\delta(q,a)=(p,b,\izq)$$
Adem\'as $\vdash^\star$ se define de la manera usual.
}

El s\'imbolo para el espacio en blanco tiene diferentes interpretaciones, 
algunos autores no hacen distinci\'on entre la cadena vac\'ia y el espacio en 
blanco.
Veamos algunos casos especiales para la cadena vac\'ia:
\bi
 \item $ u\textbf{q}a\vdash ub\textbf{p}\blanks $
  si y sólo si $\delta(q,a)=(p,b,\der)$

 \item $\textbf{q}av\vdash \textbf{p}\blanks bv $
 si y sólo si $\delta(q,a)=(p,b,\izq)$  
\ei
Algunas situaciones especiales en cuanto a c\'omputos:
\bi
 \item Cómputos bloqueados: \\
  el cómputo se bloquea porque la siguiente transición no está definida:
  $\quad u\textbf{q}v\not\vdash^\star $
 \item Cómputos infinitos: \\
  el cómputo entra en un ciclo infinito: $\quad u\textbf{q}v\vdash^\star\infty$
\ei

\defin{
El lenguaje de aceptación se define como todas aquellas cadenas de entrada con 
las cuales la máquina se detiene en un estado final:
$$L(M)=\{w\in\sest \mid q_0w\vdash^\star w_1q_fw_2 \quad q_f\in F\} $$
}

\vspace*{10pt}

\noindent Observaciones:
\textbf{Las m\'aquinas est\'andar se denominan m\'aquinas aceptadoras ya que 
determinan si una cadena pertenece a un lenguaje:}
\bi
 \item A diferencia con los autómatas se puede aceptar una cadena en el momento 
  en que el proceso llega a un estado final.
 \item \textbf{NO} es necesario consumir toda la cadena, es decir, se deben 
  contemplar $w_1$ y $w_2$ cadenas, inclusive pueden ser una subcadena de $w$.
 \item Si $\vacia \in L$ para alg\'un lenguaje entonces la cinta no contiene 
  informaci\'on al inicio y se acepta al avanzar s\'olo un sector vac\'io para  
  llegar al estado de aceptaci\'on.
 \item Una m\'aquina que acepta el lenguaje vac\'io debe aceptar al tener una 
  transici\'on hacia el estado final.
 \item Recordemos que se asume en el modelo est\'andar que no hay transici\'on 
  alguna desde un estado final, lo cual evita ambig\"uedades.
 \item Si la m\'aquina se detiene en un estado no final o se encuentra 
  bloqueada se puede concluir que la cadena a procesar no pertenece al lenguaje.
 \item Tambi\'en se puede considerar la existencia de un estado de rechazo y 
  en tal caso la m\'aquina se detiene al llegar a ese estado rechazando a la 
  cadena.
%  \item Si no hay estados finales se acepta una cadena en el momento en que la 
%   máquina se detiene.
\ei


\paragraph{Ejemplo:}
La siguiente m\'aquina acepta el lenguaje $L = \{ a^ib^ic^i \mid i\geq 0\}$
\begin{center}
\begin{tikzpicture}[node distance=3.5cm,every node/.style={scale=0.8},semithick]
    \node[state,initial,initial text=] (q0) {$q_{0}$};
    \node[state] (q1) [right of=q0] {$q_1$};
    \node[state] (q2) [right of=q1] {$q_2$};
    \node[state] (q3) [right of=q2] {$q_3$};
    \node[state] (q4) [below of=q0] {$q_4$};
    \node[state,accepting by double] (q5) [right of=q4] {$q_5$};
    \path[->] (q1) edge [in=130,out=90,loop] node [above] {$a/a,\der$} node 
      [above=10pt]  {$Y/Y,\der$} (q1);
    \path[->] (q2) edge [in=130,out=90,loop] node [above] {$b/b,\der$} node 
     [above=10pt]  {$Z/Z,\der$} (q2);
    \path[->] (q3) edge [loop right] node [right] {$a/a,\izq$} node 
     [above=5pt]  {$\qquad \qquad b/b,\izq$} node 
     [above=15pt]  {$\qquad \qquad Z/Z,\izq$} node 
     [above=25pt]  {$\qquad \qquad Y/Y,\izq$} (q3);
    \path[->] (q0) edge node [above] {$a/X,\der$} (q1);
    \path[->] (q1) edge node [above] {$b/Y,\der$} (q2);
    \path[->] (q2) edge node [above] {$c/Z,\izq$} (q3);
    \path[->] (q3) edge [bend left] node [above] {$X/X,\der$} (q0);
    \path[->] (q0) edge node [left] {$Y/Y,\der$} (q4);
    \path[->] (q4) edge node [above] {$\blanks/\blanks,\der$} (q5);
    \path[->] (q0) edge node [left] {$\blanks/\blanks,\der$} (q5);
 \end{tikzpicture}
\end{center}

\vspace*{10pt}

\section{Variaciones en MT}
Existen diversas variaciones en la definición de m\'aquinas de Turing. Todas 
ellas resultan equivalentes, es decir, el poder de computación de cualquier 
modelo resulta equivalente al de la máquina est\'andar.
Las variaciones son útiles para simplificar la presentación o programación de 
diversos problemas.

\begin{enumerate}
 \item MT con cabeza lectora estacionaria
  \bi
   \item Se permite que al leer y escribir un símbolo la cabeza no
    realice movimiento alguno.
   \item El conjunto de direcciones se amplia a $\{\izq,\der,-\}$.
   \item La transición
      $$ \delta(q,a)=(p,b,-)$$
    significa que la cabeza lee $a$, escribe $b$ y no se mueve.
   \item Tales transiciones pueden simularse mediante un nuevo estado y 
    movimientos consecutivos a la izquierda y a la derecha.
  \ei
 
 \newpage
 
 \item MT con múltiples pistas
  \bi
   \item Idea: la cinta se divide en multiples pistas.
   \item La función de transición es:
    $$\delta:Q\times\Gamma^n\imp Q\times\Gamma^n\times\{\izq,\der\}$$
    es decir $  \delta(q,\pt{a_1,\ldots,a_n})=(p,\pt{b_1,\ldots,b_n},D)$ 
    con $D\in \{\izq,\der\}$.
  \ei
  \begin{center}
\begin{tikzpicture}[every node/.style={block},
      block/.style={minimum height=6.3em,outer sep=0pt,draw,rectangle,node 
      distance=0pt}]
   \node (A) {$\begin{matrix}a\\\hline b\\\hline\vdots\\\hline c\end{matrix}$};
   \node (B) [left=of A] {$\begin{matrix}\blanks\\\hline 
    \blanks\\\hline\vdots\\\hline \blanks\end{matrix}$};
   \node (C) [left=of B] {$\ldots$};
   \node (D) [right=of A] {$\begin{matrix}b\\\hline b\\\hline\vdots\\\hline 
    d\end{matrix}$};
   \node (E) [right=of D] {$\begin{matrix}\blanks\\\hline 
    \blanks\\\hline\vdots\\\hline \blanks\end{matrix}$};
   \node (G) [right=of E] {$\dots$};
   \node (F) [minimum height=2.3em,below = 0.95cm of A,draw=red,thick] {\textsf 
q};
   \draw[-latex] (F) -- (A);
   \draw[-latex,blue] ($(F.east)!0.5!(A.east)$) -- ++(7mm,0);
   \draw (B.north west) -- ++(-2cm,0) (B.south west) -- ++ (-2cm,0) 
                 (E.north east) -- ++(2cm,0) (E.south east) -- ++ (2cm,0);
\end{tikzpicture}
\end{center}

 \item MT con múltiples cintas
  \bi
   \item Idea: se agregan más cintas a la máquina.
   \item La función de transición es:
     $$\delta: Q\times\Gamma^n\imp Q\times(\Gamma\times\{\izq,\der\})^n$$
    es decir   
    $\delta(q,\pt{a_1,\ldots,a_n})=(p,\pt{b_1,D_1},\ldots,\pt{b_n,D_n})$
    con $D_i\in \{\izq,\der\}$.
  \ei
  \begin{center}
\begin{tikzpicture}[every node/.style={block},
      block/.style={minimum height=2em,outer sep=0pt,draw,rectangle,node 
      distance=0pt}]
   \node (A) {$a$};
   \node (B) [left=of A] {$\blanks$};
   \node (C) [left=of B] {$\ldots$};
   \node (D) [right=of A] {$b$};
   \node (E) [right=of D] {$\blanks$};
   \node (G) [right=of E] {$\ldots$};
   \node (P) [draw=none,below = 0.3 of A] {$\vdots$};
   \node (A') [below = 0.5cm of P] {$c$};
   \node (B') [left=of A'] {$\blanks$};
   \node (C') [left=of B'] {$\ldots$};
   \node (D') [right=of A'] {$d$};
   \node (E') [right=of D'] {$\blanks$};
   \node (G') [right=of E'] {$\ldots$};
   \node (F) [minimum height=2.3em,below = 1.75cm of P,draw=red,thick] 
    {\textsf q};
   \draw[-latex] (F) -- (B);
   \draw[-latex] (F) -- (A');
   \draw[-latex,blue] ($(F.east)!0.5!(A'.east)$) -- ++(7mm,0);
   \draw (C'.north west) -- ++(-1cm,0) (C'.south west) -- ++ (-1cm,0) 
                 (E'.north east) -- ++(1cm,0) (E'.south east) -- ++(1cm,0);
   \draw (C.north west) -- ++(-1cm,0) (C.south west) -- ++ (-1cm,0) 
                 (E.north east) -- ++(1cm,0) (E.south east) -- ++ (1cm,0);
\end{tikzpicture}
\end{center}

  \item MT No-deterministas
   \bi
    \item La función de transición es:
      $$\delta:Q\times\Gamma^n\imp \Pe(Q\times\Gamma\times\{\izq,\der\})$$
    es decir
    $\delta(q,\pt{a_1,\ldots,a_n})=\{\pt{b_1,D_1},\ldots,\pt{b_n,D_n}\}$

  \item Las máquinas no-deterministas juegan un papel central en la
    teoría de la complejidad.
  \ei
\end{enumerate}

\section{MT Calculadoras}
Otro tipo de m\'aquinas pueden ser usadas como \textit{macros} cuyo objetivo 
es calcular sin necesidad de aceptar o generar cadenas en un lenguaje.
La máquina de Turing 
$$M=\pt{\{q_0,\dots,q_{\mathfrak{F}}\},\Sigma,\Gamma,\delta,q_0,\blanks,
\varnothing }$$
calcula una función $\mathfrak{F}:\sest\imp\Gamma^\star$ si
$$q_0w\vdash^\star q_{\mathfrak{F}}v \quad \mbox{donde } \mathfrak{F}(w)=v$$
Obs\'ervese que :
\be
 \item No hay estados finales, el estado $q_{\mathfrak{F}}$ se usa para detener 
  la máquina, es decir, no hay transiciones desde $q_{\mathfrak{F}}$.
 \item El proceso se termina en $q_{\mathfrak{F}}v$, es decir la cabeza debe 
  estar leyendo el primer símbolo de la salida $v$.
\ee

\paragraph{Ejemplo:} Un macro que copia una cadena se basa en la idea de 
recorrer la cadena de entrada almacenada en la cinta s\'imbolo por s\'imbolo,
para copiarlo en los sectores posteriores a la cadena pero usando un sector en 
blanco para separar las cadenas. \\
La siguiente m\'aquina copia cadenas sobre el alfabeto $\Sigma = \{a,b\}$:
\begin{center}
\begin{tikzpicture}[node distance=4cm,every node/.style={scale=0.8},semithick]
    \node[state,initial,initial text=] (q0) {$q_{0}$};
    \node[state] (q1) [above right of=q0] {$q_1$};
    \node[state] (q2) [right of=q1] {$q_2$};
    \node[state] (q3) [below right of=q2] {$q_3$};
    \node[state] (q4) [below right of=q0] {$q_4$};
    \node[state] (q5) [right of=q4] {$q_5$};
    \node[state] (q6) [below of=q0] {$q_6$};
    \path[->] (q1) edge [in=130,out=90,loop] node [above] {$a/a,\der$} node 
      [above=10pt]  {$b/b,\der$} (q1);
    \path[->] (q2) edge [in=130,out=90,loop] node [above] {$a/a,\der$} node 
     [above=10pt]  {$b/b,\der$} (q2);
    \path[->] (q3) edge [loop right] node [right] {$a/a,\izq$} node 
     [above=5pt]  {$\qquad \qquad b/b,\izq$} node 
     [above=18pt]  {$\qquad \qquad \blanks/\blanks,\izq$}  (q3);
    \path[->] (q4) edge [in=130,out=90,loop] node [above] {$a/a,\der$} node 
     [above=10pt]  {$b/b,\der$} (q4);
    \path[->] (q5) edge [in=130,out=90,loop] node [above] {$a/a,\der$} node 
     [above=10pt]  {$b/b,\der$} (q5);
    \path[->] (q6) edge [in=230,out=290,loop] node [below] {$X/a,\izq$} node 
     [below=10pt]  {$Y/b,\izq$} (q6);
    \path[->] (q0) edge node [left] {$a/X,\der$} (q1);
    \path[->] (q1) edge node [above] {$\blanks/\blanks,\der$} (q2);
    \path[->] (q2) edge node [right] {$\blanks/a,\izq$} (q3);
    \path[->] (q0) edge node [left] {$b/Y,\der$} (q4);
    \path[->] (q4) edge node [above] {$\blanks/\blanks,\der$} (q5);
    \path[->] (q5) edge node [right] {$\blanks/b,\izq$} (q3);
    \path[->] (q3) edge node [above] {$X/X,\der$}  
    node [above=13pt] {$Y/Y,\der $} (q0);
    \path[->] (q0) edge node [left] {$\blanks/\blanks,\izq$} (q6);
 \end{tikzpicture}
\end{center}

\section{MT Generadoras}

As\'i como hay variantes en las m\'aquinas aceptadoras, se puede hablar de 
m\'aquinas generadoras que tienen suficiente poder como para generar lenguajes 
y calcular funciones. 
Una m\'aquina de Turing~$M$ con m\'ultiples cintas genera al 
lenguaje~$L\inc\sest$ si 
\bi
 \item $M$ comienza a operar con las cintas en blanco en $q_0$.
 \item Cada vez que $M$ regresa a $q_0$ hay una cadena de $L$
  escrita sobre la cinta principal.
 \item Eventualmente se generan todas las cadenas de $L$ que se almacenar\'an 
  en la cinta principal separadas por un espacio en blanco o alg\'un otro 
  s\'imbolo para este efecto.
\ei

\paragraph{Ejemplo:}
Analicemos una m\'aquina que genera todas las cadenas del lenguaje:
$L = \{ a^ib^ic^i \mid i\geq 0\}$.
Utilizaremos dos cintas y la idea del dise\~no es la siguiente:
\begin{enumerate}
 \item se inicia con la cinta $C1$ vac\'ia y se dejan dos s\'imbolos $\#$ para 
  indicar que la cadena vac\'ia pertenece al lenguaje
 \item simult\'aneamente se escribe una $a$ en la cinta $C2$ y la cabeza 
  lectora retrocede
 \item a continuaci\'on se generar\'an las cadenas utilizando un ciclo que no 
  termina como sigue:
 \begin{enumerate}
  \item en la cinta $C1$ se escribe una $a$ por cada $a$ en la cinta $C2$
  \item la cabeza en $C2$ regresa a la izquierda para escribir $b$ en $C1$ por 
    cada $a$ que lea
  \item nuevamente se mueve la cabeza a la derecha en $C2$ para escribir 
  en $C1$ tantas $c$ como $a$ 
  \item se incrementa una $a$ en la cinta $C2$ para generar la siguiente cadena
  \item se escirbe el s\'imbolo $\#$ en $C1$ para separar las cadenas
 \end{enumerate}
\end{enumerate}
Veamos su diagrama:
\begin{center}
\begin{tikzpicture}[node distance=4cm,every node/.style={scale=0.8},semithick]
    \node[state,initial,initial text=] (q0) {$q_{0}$};
    \node[state] (q1) [below of=q0] {$q_1$};
    \node[state] (q2) [right of=q1] {$q_2$};
    \node[state] (q3) [right of=q2] {$q_3$};
    \node[state] (q4) [right of=q3] {$q_4$};
    \node[state] (q5) [right of=q4] {$q_5$};
    \path[->] (q2) edge [loop above] node {$[\blanks/a,\der;\,a/a,\der]$} (q2);
    \path[->] (q3) edge [loop above] node {$[\blanks/b,\der;\,a/a,\izq]$} (q3);
    \path[->] (q4) edge [loop above] node {$[\blanks/c,\der;\,a/a,\der]$} (q4);
    \path[->] (q5) edge [loop above] node {$[\blanks/\blanks,-;\,a/a,\izq]$} 
    (q5);
    \path[->] (q0) edge node [left] {$[\blanks/\#,\der;\,\blanks/a,-]$} (q1);
    \path[->] (q1) edge node [above] {$[\blanks/\#,\der;\,a/a,-]$} (q2);
    \path[->] (q2) edge node [above] 
    {$[\blanks/\blanks,-;\,\blanks/\blanks,\izq]$} (q3);
    \path[->] (q3) edge node [above] 
    {$[\blanks/\blanks,-;\,\blanks/\blanks,\der]$} (q4);
    \path[->] (q4) edge node [above] {$[\blanks/\blanks,-;\,\blanks/a,\izq]$} 
    (q5);
    \path[->] (q5) edge [bend left] node [above] 
    {$[\blanks/\#,\der;\,\blanks/\blanks,\der]$} (q2);
 \end{tikzpicture}
\end{center}



\section{Lenguajes Recursivos y Recursivamente enumerables}
La aceptación en una m\'aquina de Turing definen unas clase de lenguajes muy 
expresivos:
\bi
 \item Un lenguaje~$L$ es \textbf{recursivamente enumerable} si es reconocido
  por una máquina de Turing, es decir, si existe una máquina de Turing
  $M$ tal que $L=L(M)$.
 \item Un lenguaje $L$ es \textbf{recursivo} si es reconocido por una máquina
  de Turing que \textbf{siempre se detiene}, es decir, si existe una máquina
  de Turing $M$ que se detiene con todas las cadenas de entrada y $L=L(M)$.
\ei

Los lenguajes definidos gozan de algunas propiedades de cerradura:
\bi
 \item Si $L$ es recursivo entonces $\overline{L}$ es recursivo.
 \item Si $L,M$ son recursivos entonces $L\cup M$ es recursivo.
 \item Si $L,M$ son rec. enumerables entonces $L\cup M$ es rec. enumerable.
 \item $L$ es recursivo si y sólo si $L$ y $\overline{L}$ son rec. enumerables.
\ei
  
\section{M\'aquinas de Turing y gramáticas}
\noindent Recordando la Jerarquía de Chomsky y su relación con autómatas:
\begin{center}
\begin{tabular}{cc|c}
 & Tipo de Gram\'atica & M\'aquina equivalente \\\hline
 3 & Regulares & aut\'omatas finitos \\
 2 & Libres de Contexto & aut\'omatas de pila \\
 1 & Sensibles al Contexto & \textbf{??} \\
 0 & Irrestrictas & \textbf{??}
\end{tabular}

\vspace{10pt}

?`Qué pasa con las gramáticas tipo 1 (sensibles al contexto)?  \\
?`Qué pasa con las gramáticas tipo 0 (irrestrictas)?
\end{center}

\paragraph{Ejemplo:}
El lenguaje~$L=\{a^ib^ic^i\;|\;i\geq 0\}$ es generado por la 
siguiente gramática sensible al contexto:
\[
 \begin{array}{rcl}
  S &\imp & A \\
  A & \imp & AABC \mid aBC \\
  CB & \imp &BC \\
  bB & \imp & bb \\
  bC & \imp & bc \\
  cC & \imp & cc
 \end{array}
\]
Al restringir que $i\geq 1$ tenemos la siguiente gram\'atica:
\[
 \begin{array}{rcl}
  S & \imp & aSBc \mid abc\\
  cB & \imp & Bc \\
  bB & \imp & bb
 \end{array}
\]

\newpage

Recordemos que las gram\'aticas de tipo 1 permiten tener cadenas de s\'imbolos 
terminales y no terminales en ambos lados de $\imp$ de la forma: 
\begin{center}
$\alpha A \beta \imp \alpha \gamma \beta$ donde $\gamma \neq \vacia$, $A \in V$ 
y $\alpha,\beta,\gamma \in (V\cup \Sigma)^\star$ 
\end{center}
Se denominan sensibles al contexto dado que la reescritura de $A$ en $\gamma$ 
depende del contexto $\alpha\gamma$.
Y los au\'tomatas que reconocen estos lenguajes son los llamados acotados.

     
\subsection{Gram\'aticas Sensibles al Contexto y M\'aquinas de Turing}
La correspondencia entre los aut\'omatas linealmente acotados y las 
gramáticas es consecuencia de que todo lenguaje sensible al contexto es 
recursivo y est\'a descrita por:
\be
 \item Dada una gramática sensible al contexto $G$, existe un autómata 
  linealmente acotado $M$ tal que $L(G)=L(M)$. Es decir, los lenguajes 
  sensibles al contexto son reconocidos por autómatas linealmente acotados.
    
  \item Si $L=L(M)$ es un lenguaje reconocido por un autómata linealmente
   acotado $M$ entonces existe una gramática sensible al contexto
   $G$ tal que $L(M)=L(G)$. Es decir, los lenguajes reconocidos por
   autómatas linealmente acotados son sensibles al contexto.
\ee
A continuaci\'on describimos el proceso en el que dada una gram\'atica sensible 
al contexto~$G$ se puede obtener una m\'aquina de Turing que simule las 
derivaciones de~$G$:
\begin{itemize}
 \item Construir una m\'aquina de Turing no-determinista con 3 cintas tales que:
 \begin{itemize}
  \item la cinta $C1$ contiene la cadena a ser procesada
  \item la segunda cinta $C2$ tiene la forma generada por la simulaci\'on
  \item la \'ultima cinta $C3$ es la derivaci\'on
 \end{itemize}
 \item Los pasos a seguir son los siguientes:
 \begin{enumerate}
  \item $C3$ contiene $S\#$
  \item Una regla de producci\'on $\alpha \imp \beta$ se escoge desde $C2$
  \item Sea $\gamma\#$ la \'ultima cadena escrita en $C3$, as\'i se toma una 
instancia de $\alpha$ en $\gamma = \gamma'\alpha\gamma''$ existente, sino se va 
al estado de rechazo
  \item escribir $\gamma'\alpha\gamma''\#$ en $C3$ despues de $\gamma\#$, esto 
indicar\'a la aplicaci\'on de la regla
  \item Si $\gamma'\beta\gamma'' = w$ entonces la m\'aquina se detiene y acepta
  \item Si $\gamma'\beta\gamma''$ aparece en otra posici\'on dentro de $C3$ 
entonces es porque la m\'aquina se ha detenido pero en el estado de rechazo
  \item Si $|\gamma'\beta\gamma''| > |w|$ entonces la m\'aquina se detiene y 
tambi\'en rechaza
  \item Repetir los pasos 2. al 6.
  \end{enumerate}
\end{itemize}

\subsection{Autómatas Linealmente Acotados}

\defin{Un autómata linealmente acotado (ALA) es una máquina de Turing que 
satisface las siguientes condiciones:
\bi
 \item El alfabeto de entrada $\Sigma$ incluye dos símbolos especiales~$[,]$ 
  que sirven como marcas de fin de cinta izquierda y derecha respectivamente.
 \item La cabeza lectora no puede desplazarse más allá de dichos límites y 
  no puede sobreescribir tales sectores.
\ei
Formalmente tenemos la tupla:
$$ M=\pt{Q,\Sigma,\Gamma,\delta,q_0,\blanks,[,],F}$$
con $[,]\in\Sigma$.
Adem\'as el lenguaje de aceptación es 
$$L(M)=\{w\in\Sigma^\star-\{[,]\}\mid q_0[w]\vdash^\star w_1q_fw_2\;q_f\in F\}$$
}
Las marcas $[,]$ no son consideradas como parte de la cadena a procesar pero 
son las que limitan el movimiento de la cabeza. 
Esta versi\'on de m\'aquina no puede moverse fuera de la cadena de entrada de 
ah\'i que se le nombre acotado.

  
\subsection{MT y gramáticas irrestrictas}

Las gram\'aticas sin restricciones son aquellas que est\'an en 
correspondencia con las Máquinas de Turing:
\bi
 \item Para toda gramática~$G$ de tipo 0 existe una máquina de Turing~$M$ tal 
  que $L(M)=L(G)$. Es decir, los lenguajes tipo 0 son recursivamente 
  enumerables.
    
 \item Para toda máquina de Turing~$M$ existe una gramática~$G$ de tipo 0 
  tal que $L(G)=L(M)$. Es decir, los lenguajes recursivamente enumerables son 
  lenguajes tipo 0.
\ei
  
Revisemos nuevamente la Jerarquía de Chomsky:
\begin{center}
\begin{tabular}{cc|c}
 & Tipo de Gram\'atica & M\'aquina equivalente \\\hline
 3 & Regulares & aut\'omatas finitos \\
 2 & Libres de Contexto & aut\'omatas de pila \\
 1 & Sensibles al Contexto & autómatas linealmente acotados\\
 0 & Irrestrictas & máquinas de Turing
\end{tabular}
\end{center}

\newpage

\noindent Finalmente se muestra a continuaci\'on un 
diagrama~\footnote{Diagrama original de Anup Kumar y Hemanth Kumar del Indian 
Institute of Science, Bangalore}
que contiene la clasificaci\'on completa de los lenguajes:
\begin{center}
\includegraphics[scale=0.7]{jerarquiaCompleta.png}
\end{center}

\end{document}
