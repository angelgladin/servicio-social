\documentclass[letterpaper,11pt]{article}
\usepackage[includeheadfoot,margin=1.3in]{geometry}
\usepackage{anysize}

% \usepackage[utf8]{inputenc}
\usepackage[latin1]{inputenc}
\usepackage[spanish]{babel}
\usepackage{lmodern}   % font shapes...
\usepackage[T1]{fontenc} % join the compound symbols as a single symbol

\usepackage{amssymb,amsmath}
\usepackage{mathrsfs}
\usepackage{epsfig}

\usepackage{fancyhdr}
\usepackage{hyperref}
\usepackage{url}

\usepackage{import}
\usepackage{comment}
\usepackage[autostyle=true,spanish=mexican]{csquotes}

\usepackage{url}

%\pagestyle{fancyplain}
\input{../macrosAyLF}

\title{Aut\'omatas y Lenguajes Formales  \\ 
Facultad de Ciencias UNAM \\ 
Nota de Clase 3
\thanks{\small{Material desarrollado bajo el proyecto UNAM-PAPIME PE102117 
2017--2018.}}} 
\author{Favio E. Miranda Perea \and A. Liliana Reyes Cabello \and
Lourdes Gonz\'alez Huesca}
\date{\today}

\begin{document}
\maketitle

\section{Lenguajes Regulares}

El estudio de los lenguajes como conjuntos de cadenas debe ser susceptible a 
realizarse de manera formal mediante una abstracci\'on conveniente. 

Una forma para caracterizar lenguajes sencillos, de f\'acil descripci\'on, es 
utilizar operaciones sobre cadenas, \'estas permiten la \textit{generaci\'on} 
de nuevos lenguajes. 
Los lenguajes regulares son los conjuntos de cadenas m\'as simples.

\defin{
Un lenguaje~$L\inc\sest$ es regular si es generado a partir de los lenguajes 
b\'asicos $\vacio,\{\cv\},\{a\}$ (donde $a\in\Sigma$) y 
mediante las operaciones de uni\'on, concatenaci\'on y estrella de Kleene. 
Recursivamente:
\bi
 \item $\vacio$ y $\{\cv\}$ son lenguajes regulares.
 \item Si $a\in\Sigma$ entonces $\{a\}$ es regular.
 \item Si $L,M$ son regulares entonces $L\cup M, LM$ y $L^\star$ son regulares.
 \item Son todos.
\ei 
}
\paragraph{Ejemplos:}
\bi
 \item Cualquier lenguaje \textit{finito} es regular.
  Si
  $L=\{w_1,\ldots,w_n\}\inc\sest$ entonces $L=L_1\cup L_2\cup\ldots\cup
  L_n$ con $L_i=\{w_i\}$. Cada lenguaje $L_i$ es regular puesto que si
  $w_i=a_1\ldots a_{k_i}$ con $a_j\in\Sigma$  entonces
  $L_i=\{a_1\}\{a_2\}\ldots\{a_{k_i}\}$ y cada $\{a_j\}$ es regular por
  definici\'on.
  \item En particular cualquier alfabeto $\Sigma$ es un lenguaje regular.
  \item Sobre el alfabeto $\Sigma=\{a,b\}$ tenemos los siguientes lenguajes
   regulares \textit{infinitos}:
   \bi
    \item $L_1=\{w\in\sest \mid w \text{ empieza con }\;b\}=\{b\}\sest$
    \item $L_2=\{w\in\sest \mid w \text{ contiene exactamente una }\;a\}$
      $ = \{b\}^\star\{a\}\{b\}^\star$
    \item $L_3=\{w\in\sest \mid w \text{ contiene la subcadena } \;ba\} $ 
      $ = \sest\{ba\}\sest$
    \item $L_4=\{w\in\sest \mid w \text{ contiene exactamente dos } b \}$ 
      $ =\{a\}^\star\{b\}\{a\}^\star\{b\}\{a\}^\star$
    \item $L_5=\{w\in\sest \mid w \text{ contiene un n\'umero par de } a \} = $
    $\{b\}^\star\cup\big(\{b\}^\star\{a\}\{b\}^\star\{a\}\{b\}^\star\big)^\star$
  \ei
 \item No todo lenguaje infinito es regular% , mas adelante veremos que
  % el lenguaje $L=\{a^nb^n\;|\;n\in\N\}$ no es regular.
\ei   

\paragraph{Propiedades de cerradura}
Las propiedades de cerradura nos permiten construir nuevos lenguajes
regulares a partir de lenguajes ya conocidos por medio de algunas operaciones
entre lenguajes. \\
Si~$L,M$ son lenguajes regulares entonces:
\bi
\item $L\cup M$ es regular.
\item $LM$ es regular.
\item $L^\star$ es regular.
\item $L^+$ es regular.
\item $\overline{L}$ es regular
\item $L\cap M$ es regular.
\item $L-M$ es regular.
\ei


\ejem{
Sea $\Sigma = \{ 0,1,2 \}$ y los lenguajes $L_1$ y $L_2$ definidos como
el lenguaje de cadenas de ceros y unos con longitud impar y el lenguaje de 
cadenas de ceros y dos con longitud impar, respectivamente. \\
Estudiemos la relaci\'on entre $(L_1L_2)^\star$ y $L_1^\star L_2^\star$.

\[
 \begin{array}{rcl}
  L_1L_2 & = & \{w = w_1w_2 \; \mid \; w_1\in L_1,\; w_2\in L_2,\; |v| = 2j , 
j\geq 2\} \\\\
 (L_1L_2)^\star & = & \bigcup_{i=0}^\infty (L_1L_2)^i\\
    & =& \{ \vacia \} \cup L_1L_2 \cup (L_1L_2)^2 \cup \ldots \cup\\ \\
 L_1^\star  & = & \{ \vacia \} \cup L_1 \cup L_1L_2 \cup \ldots\\ 
 L_2^\star & = & \{ u_2 \;\mid \; u_2\in \{0,2\}^\star \} \\\\
 L_1^\star L_2^\star &= & \{ u = u_1u_2 \;\mid\; u_1 \in L_1^\star,\; u_2\in 
L_2^\star  \} \\
& =& \{ \vacia \} \cup L_2^\star \cup L_1L_2^\star \cup \ldots \cup
    L_1^\star \cup L_1^\star L_2 \cup \ldots \cup \ldots \\ 
 \end{array}
\]
Despu\'es de analizar los lenguajes, es evidente que 
$(L_1L_2)^\star \neq L_1^\star L_2^\star$.
Veamos un contraejemplo, consideremos la cadena $v = 2200$ la cual pertenece a 
$L_1^\star L_2^\star$ ya que en particular pertenece a $L_2^\star$,  pero no 
pertenece a $(L_1L_2)^\star$ dado que no es posible que una cadena en este 
lenguaje comience con un dos.
}


\paragraph{Problemas de inter\'es acerca de lenguajes regulares}
Dado un lenguaje regular~$L\inc\Sigma^\star$ existen diversos problemas de 
decisi\'on (es decir, problemas cuya respuesta es s\'i o no) acerca de~$L$, 
enunciamos algunos de ellos que tienen soluci\'on algor\'itmica.
\be
 \item Problema de vacuidad: ?`Es $L$ el lenguaje vac\'io?
 \item Problema de finitud: ?`Es $L$ finito ?
 \item Problema de la pertenencia: Dada $w\in\Sigma^\star$, ?`$w\in L$ ?   
 \item Problema de la equivalencia: Dado $M\inc\Sigma^\star$, ?`$M=L$ ?
\ee


\section{Expresiones Regulares}
Los lenguajes regulares se pueden describir por medio de una expresi\'on que 
generaliza la forma de las cadenas en el lenguaje, es decir una 
representaci\'on de las cadenas.
Un mecanismo de suma importancia para denotar lenguajes regulares es por
medio de las llamadas expresiones regulares, que son \textit{f\'ormulas} 
expl\'icitas de la cadena caracter\'istica del lenguaje.

\defin{ Las expresiones regulares est\'an definidas recursivamente como
sigue:
\bi
 \item $\vacio$ es una expresi\'on regular.
 \item $\cv$ es una expresi\'on regular.
 \item Si $a\in\Sigma$ entonces $a$ es una expresi\'on regular.
 \item Si $\al,\beta$ son expresiones regulares entonces 
  $\al\beta,\;\al+\beta,\;\al^\star$ son expresiones regulares.
\item Son todas.
\ei
}

\noindent Dada una expresi\'on regular $\al$, su significado~$L(\al)$ es un 
lenguaje regular definido como sigue:
\bi
\item $L(\vacio)=\vacio$
\item $L(\cv)=\cv$
\item $L(a)=\{a\}\;\;\;\mbox{ si }\al\in\Sigma$
\item $L(\al\beta)=L(\al)L(\beta)$
\item $L(\al+\beta)=L(\al)\cup L(\beta)$
\item $L(\al^\star)=L(\al)^\star$
\ei

De los significados de las expresiones, se puede ver a una expresi\'on regular 
como una representaci\'on alternativa de un lenguaje. 
As\'i podemos pensar en la equivalencia entre lenguajes regulares y 
expresiones regulares.

\newpage
\begin{proposition}
Un lenguaje~$M\inc\sest$ es regular si y s\'olo si existe una expresi\'on 
regular~$\al$ tal que $M = L(\al)$.
\end{proposition}
\vspace*{-15pt}
\begin{proof}
La direcci\'on $\Leftarrow$ es directa. 
Para la otra direcci\'on hacemos inducci\'on estructural.
\end{proof}

\medskip

\noindent Se puede estudiar la equivalencia de expresiones regulares:
\defin{
Dadas dos expresiones regulares~$\al,\beta$ decimos que son equivalentes, al 
escribir~$\al=\beta$, si y s\'olo si $\al$ y $\beta$ tienen el mismo 
significado, es decir, si y s\'olo si $L(\al)=L(\beta)$.
}


\paragraph{Propiedades de las Expresiones Regulares}
\begin{center}
 \begin{tabular}{ccc}
  $\al+(\beta+\ga)=(\al+\beta)+\ga$ & \hspace{10pt} & $\al+\beta=\beta+\al$ \\
  $\al+\vacio=\al$ & & $\al+\al=\al  $\\
  $\al\cv=\al$ & & $\al\vacio=\vacio$\\
  $\al(\beta\ga)=(\al\beta)\ga$ & & \\
  $\al(\beta+\ga)=\al\beta+\al\ga$ & & $(\beta+\ga)\al=\beta\al+\ga\al$\\
  $\cv^\star=\cv$ & & $\vacio^\star=\cv$ \\
  $\al\al^\star=\al^\star\al$ & & 
    $\al^\star=\al^\star\al^\star=(\al^\star)^\star$\\
  $\al^\star=\cv+\al\al^\star$ & & 
    $(\al+\beta)^\star=(\al^\star \beta^\star)^\star$\\
   Si $L(\al)\inc L(\beta)$ entonces $\al+\beta=\beta$&  &
 \end{tabular}
\end{center}

\ejem{
$L_1 = \{ w\in{a,b}^\star \mid w \text{ no tiene la subcadena } a 
    \text{ y termina en b }\}$ 
    le corresponde la expresi\'on regular $\alpha_1=(b+ab)^\star (b+ab)$.\\
$L_2 = \{ w\in\Sigma^\star \mid \#_a = 2n+1, \; n\in \N \}$ le corresponde la 
expresi\'on regular $\alpha_2=(b + a b^\star a)^\star a b^\star$.
}


\paragraph{Aplicaciones de las expresiones regulares}
\bi
\item B\'usqueda y sustituci\'on en editores de texto (vi, emacs, etc.)
\item Caza de patrones (pattern matching) y procesamiento de textos y
  conjuntos de datos, por ejemplo en miner\'ia de datos (grep, sed, awk, perl, 
etc.)
\item Implementaci\'on de lenguajes de programaci\'on (fase de an\'alisis 
l\'exico):
conversi\'on del programa fuente en una sucesi\'on de elementos l\'exicos 
(lexemas o tokens), en esta sucesi\'on se identifican las distintas componentes 
de un programa, como son palabras reservadas, identificadores, tipos de datos, 
etc.(lex, flex, etc.)
\ei

% ?`Cuando no es $L$ regular
\noindent No todos los lenguajes son regulares, por ejemplo:
$$ L=\{a^nb^n\;|\;n\geq 1\} $$
no es definible mediante una expresi\'on regular.
\bi
 \item ?`C\'omo decidir cuando un lenguaje no es regular?
 \item Mediante propiedades de cerradura.
 \item Mediante el lema del bombeo (pendiente).
\ei
\noindent En el siguiente tema discutiremos m\'as formas de decidir cu\'ando un 
lenguaje es regular o no. 


\end{document}
