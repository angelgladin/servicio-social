\documentclass[letterpaper,11pt]{article}
\usepackage[includeheadfoot,margin=1.3in]{geometry}
\usepackage{anysize}

\usepackage[utf8]{inputenc}
% \usepackage[latin1]{inputenc}
\usepackage[english,spanish]{babel}
\usepackage{lmodern}   % font shapes...
\usepackage[T1]{fontenc} % join the compound symbols as a single symbol

\usepackage{amssymb,amsmath}
\usepackage{mathrsfs}
\usepackage{epsfig}

\usepackage{fancyhdr}
\usepackage{hyperref}
\usepackage{url}

\usepackage{import}
\usepackage{comment}
\usepackage[autostyle=true,spanish=mexican]{csquotes}

\usepackage{alltt}
\usepackage[section]{placeins}

\usepackage{url}

\usepackage{pgf}
\usepackage{tikz}
\usetikzlibrary{automata,arrows,trees}
\usetikzlibrary{babel}
%\pagestyle{fancyplain}
\input{../macrosAyLF}

\title{Aut\'omatas y Lenguajes Formales  \\ 
Facultad de Ciencias UNAM \\ 
Nota de Clase 7
\thanks{\small{Material desarrollado bajo el proyecto UNAM-PAPIME PE102117 
2017--2018.}}} 
\author{Favio E. Miranda Perea \and A. Liliana Reyes Cabello \and
Lourdes Gonz\'alez Huesca}
\date{\today}

\begin{document}
\maketitle

\section{Gram\'aticas}
Un mecanismo relevante para generar un lenguaje es mediante el concepto de 
gram\'atica formal.
Las gram\'aticas formales fueron introducidas por Noam Chomsky en 1956.
La intenci\'on era tener un modelo para la descripci\'on de lenguajes 
naturales.
Posteriormente se utilizaron como herramienta para presentar la sintaxis 
de lenguajes de programaci\'on y para el dise\~no de analizadores l\'exicos 
de compiladores
         
\smallskip

\noindent Veamos el ejemplo del lenguaje de expresiones aritm\'eticas.\\
Una expresi\'on aritm\'etica $e\in ExprArit$ es una sucesi\'on de s\'imbolos 
definida recursivamente por:
\begin{enumerate}
 \item $c$ una constante es una expresi\'on arim\'etica. %$c\in ExprArit$ 
 \item Para cualesquiera expresiones aritm\'eticas $e_1$ y $e_2$, $e_1+e_2$ y 
 $e_1*e_2$ son expresiones aritm\'eticas.
 \item Para cualquier expresi\'on aritm\'eticas $e_1$ y $(e_1)$  es una 
expresic\'on aritm\'etica.
\item Son todas.
\end{enumerate}
La definici\'on anterior puede representarse de forma compacta usando una 
gram\'atica (usualmente se utilizan las gram\'aticas en forma Backus-Naur):
$$ e ::= c \;\mid\; e_1+e_2 \;\mid\; e_1*e_2 \;\mid\; (e_1) $$

Para este curso, utilizaremos la siguiente definici\'on de gram\'atica para 
describir un lenguaje:
\defin{
Una gram\'atica es una cuaterna $G=\pt{V,T,S,P}$ tal que:
\bi
 \item $V$ es un alfabeto de \textnormal{variables} o \textnormal{s\'imbolos    
 no-terminales}, los cuales se denotan con may\'usculas $A,B,C,\ldots$
 \item $T$ es un alfabeto de \textnormal{s\'imbolos terminales}, los cuales 
  se denotan con min\'usculas $a,b,c,\ldots$. Adem\'as se requiere que
  $T\cap V=\vacio$.
 \item $S\in V$ es una variable distinguida llamada el \textnormal{s\'imbolo    
  inicial}.
 \item $P$ es un conjunto finito de reglas de reescritura, llamadas 
  \textnormal{reglas de producci\'on} o producciones.
  \ei
El conjunto de reglas de producci\'on~$P$ es un conjunto finito de 
pares $\pt{\al,\beta}$ tales que
\bi
 \item $\al\in (V\cup T)^\star - T^\star$. Es decir, $\al$ es una
  cadena de s\'imbolos terminales o no terminales, con al menos un
  s\'imbolo no-terminal.
 \item $\beta\in (V\cup T)^\star$. Es decir, $\beta$ es una cadena de
 s\'imbolos de $V\cup T$ , los cuales podr\'an ser todos terminales.
\ei
En lugar de escribir $\pt{\alpha,\beta}\in P$, escribimos $$\alpha \imp \beta$$ 
y decimos que $\alpha$ \textbf{produce} a $\beta$, o que $\alpha$ se 
\textbf{reescribe} en $\beta$.
}

Las reglas de producci\'on sirven para generar cadenas, proceso que se
formaliza mediante las derivaciones formales:
 
\defin{[Derivaci\'on formal]
Dadas dos palabras $w,v\in (V\cup T)^\star$ decimos que $v$ es
\textnormal{derivable} a partir de $w$ en \textbf{un paso} ($w\imp v$) si y 
s\'olo si:
\bi
 \item[] Existe una regla $\al\imp\beta$ en $P$ y cadenas 
  $\gamma_1,\gamma_2\in(V\cup T)^\star$ tales que:
  $$  w=\gamma_1\al\gamma_2  \qquad \qquad v=\gamma_1\beta\gamma_2$$
  Es decir, se ha reescrito la cadena $\alpha$ en la cadena $\beta$.
\ei
}
Algunos autores utilizan $\Imp$ en vez de $\imp$ para denotar la relaci\'on 
de derivaci\'on. Nosotros preferimos sobrecargar el operador $\imp$.
  
  
\defin{
 Decimos que una cadena $v$ es \textnormal{derivable} a partir de $w$ si existen
 palabras $\ga_2,\ldots,\ga_n$ tales que
 $$ w=\ga_1\imp \ga_2\ldots\ga_{n+1}\imp\ga_n=v $$
 En tal caso escribimos $w\imp^\star v$.
}

\defin{[Lenguaje de una gram\'atica]
Dada una gram\'atica $G=\pt{V,T,S,P}$ definimos al lenguaje generado por $G$, 
denotado $L(G)$, como el conjunto de palabras de s\'imbolos \textbf{terminales} 
derivables a partir del s\'imbolo inicial $S$. 
Es decir, $$L(G)=\{w\in T^\star\;|\;S\imp^\star w\}$$
}

\paragraph{Ejemplo:}
Considere el siguiente lenguaje~$L = (a+b)^\star$, es el lenguaje en que 
cualquier cadena con $a$ y $b$ debe generarse. Veamos la gram\'atica 
correspondiente:
\bi
 \item La cadena vac\'ia debe generarse: $\quad S\imp\vacia$
 \item Si $w\in L$ entonces $wa\in L$: $\quad \qquad S\imp Sa $
 \item Si $w\in L$ entonces $wb\in L$: $\quad \qquad S\imp Sb$
\ei
La cadena~$w=ababb$ se deriva como sigue:
$$ S\imp Sb\imp Sbb\imp Sabb\imp Sbabb\imp Sababb\imp ababb $$

\paragraph{Ejemplo:}
Proporcionar una gram\'atica que genere las 
cadenas en $L=\{a^ib^j\;|\;i,j\in\N,i\leq j\}$.
\bi
 \item La cadena vac\'ia debe generarse ($i=j=0$): $\qquad S\imp\vacia$
 \item Debe haber al menos tantas $b$ como $a$, primero las $a$ y luego las 
  $b$: $$S\imp aSb$$
 \item Puede haber m\'as $b$ al final: $$S\imp Sb$$
\ei
Ejemplo de derivaci\'on para la cadena~$w=aabbb$
$$ S\imp aSb\imp aaSbb\imp aaSbbb\imp aabbb $$

\paragraph{Ejemplo:}
El lenguaje~$L=\{a^ib^ja^jb^i\;|\;i,j\in\N\}$ es generado por la siguiente 
gram\'atica:
\bi
 \item Primero generamos el centro de la palabra, $b^ja^j$:
  $$  S\imp B \qquad B\imp bBa \qquad B\imp \vacia$$
 \item Despu\'es los extremos $a^i,\;b^i$: $\qquad\qquad S\imp aSb$
\ei
Un ejemplo de derivaci\'on es la generaci\'on de la cadena~$w=aababb$
$$ S\imp aSb\imp aaSbb\imp aaBbb\imp aabBabb \imp aababb $$

\paragraph{Ejemplo:}
Dado~$L=\{a^{i} b^{i} a^{j} b^{j}\;|\;i,j\in\N\}$, la siguiente gram\'atica 
genera el lenguaje:
\bi
 \item Primero generaremos el lenguaje~$\{a^i b^i\;|\; i\in\N\}$ mediante:
  $$    P\imp \vacia \qquad \qquad P\imp aPb $$
 \item Para generar a $L$ simplemente agregamos:
  $$S\imp PP$$
\ei
Ejemplo de derivaci\'on, la cadena~$w=aabbab$ se genera como sigue:
$$ S\imp PP \imp aPbP\imp aPbaPb\imp aaPbbaPb \imp aaPbbab\imp aabbab $$

\paragraph{Ejemplo:}
El lenguaje~$L=\{a^{i} b^{i}\;|\;i\in\N\}\cup\{b^{i} a^{i}\;|\;i\in\N\}$ se 
genera mediante:
\bi
 \item Primero el lenguaje $\{a^i b^i\;|\; i\in\N\}$ se genera con la 
  siguiente gram\'atica:
  $$ P\imp \vacia \qquad \qquad P\imp aPb $$
 \item Despu\'es generamos el lenguaje $\{b^i a^i\;|\; i\in\N\}$ mediante:
  $$ Q\imp \vacia \qquad\qquad Q\imp bQa$$
 \item Finalmente, para generar a $L$ simplemente agregamos:
  $$S\imp P \qquad S\imp Q$$
\ei
Ejemplo de derivaci\'on para $w=bbbaaa$
$$ S\imp P \imp aPb\imp aaPbb\imp aaaPbbb \imp aaabbb $$

\subsection{Dise\~no de gram\'aticas}
Si bien muchas veces el dise\~no de una gram\'atica $G$ para un lenguaje dado 
$L$  es intuitivamente claro y correcto, esto debe mostrarse formalmente, 
mostrando que $L=L(G)$. Esto se hace, por supuesto, probando formalmente lo 
siguiente:
\be
 \item \textbf{Correctud}: la gram\'atica $G$ genera \'unicamente las cadenas 
  de $L$, es decir, $L(G)\inc L$.

 \item \textbf{Completud}: toda cadena de $L$ es generada por $G$, es decir, 
  $L\inc L(G)$.
\ee  

\section{Jerarqu\'ia de Chomsky}

Las gram\'aticas fueron clasificadas de acuerdo a sus propiedades por el 
ling\"uista Noam Chomsky.
Revisemos nuevamente la clasificaci\'on de lenguajes, ahora con las 
gram\'aticas que los definen.

\bi
 \item[Tipo 0] \textbf{Lenguajes recursivamente enumerables} \\
  Son aquellos lenguajes generados por una gram\'atica sin restricciones
  adicionales.\\
  Tales gram\'aticas pueden incluir reglas de la forma
  $\qquad \boxed{\alpha\imp\vacia}$
  
  \vspace*{10pt}
  
  De manera que la gram\'atica es capaz de borrar cadenas.
  Tales gram\'aticas se conocen como \textit{contra\'ibles}.  
  Por ejemplo: $$ aS\imp bSb,\; aSb\imp \cv,\; SbS\imp bcS $$

  As\'i tambi\'en la siguiente es una gram\'atica de tipo 0 donde 
  $L(G)=\{ww\mid w\in\{0,1\}^\star\}$
  \[
  \begin{array}{cccc}
   S\to AT & A\to 0AO & A\to 1AI & O0\to 0O \\
   O1\to 1O & I0\to 0I & I1\to1I & OT\to 0T\\
   IT\to 1T & A\to\vacia & T\to\vacia &
  \end{array}
  \]

  La idea del dise\~no de esta gram\'atica y la raz\'on del nombre 
  \textit{recursivamente enumerable} se discutir\'an m\'as adelante.

  Comentamos ahora que las m\'aquinas que aceptan este tipo de lenguajes son 
  las M\'aquinas de Turing que tambi\'en ser\'an estudiadas m\'as adelante.


\item[Tipo 1] \textbf{Lenguajes dependientes del contexto}\\
  Tambi\'en llamados sensibles al contexto, son aquellos lenguajes generados 
  por gram\'aticas con todas sus producciones son de la forma
  $$\boxed{\alpha_1 A\alpha_2 \imp\alpha_1\beta\alpha_2}$$
  con $\alpha_1,\alpha_2\in(V\cup T)^\star,\;A\in V,\;\beta\neq\vacia$, 
  adem\'as de que $|\alpha_1\beta\alpha_2| \geq |\alpha_1 A\alpha_2|$.\\  
  Con la posible excepci\'on de la regla $S\imp\vacia$, en cuyo caso se
  prohibe la presencia de $S$ a la derecha de las producciones.
  
  Los aut\'omatas que reconocen este tipo de lenguajes son los llamados 
  aut\'omatas acotados linealmente.
  
  Veamos un ejemplo, la siguiente gram\'atica dependiente del contexto 
  genera al lenguaje $$L=\{a^ib^ic^i\;|\;i\geq 0\}$$
  \[
  \begin{array}{rclcrcl}
    S & \imp & A & \hspace*{20pt} & A& \imp & aABC \mid abC \\
    CB&\imp& BC & \hspace*{20pt} & bB&\imp & bb \\
    bC&\imp& bc & \hspace*{20pt} &cC&\imp& cc
  \end{array}
  \]

% \newpage
  
\item[Tipo 2] \textbf{Lenguajes libres del contexto} \\
Son aquellos generados por gram\'aticas con todas sus producciones de la forma
$$\boxed{A\imp\al}$$
con $A\in V,\;\al\in(V\cup T)^\star$.
 
Esta definici\'on incluye a la regla $S\imp\vacia$.
La mayor\'ia de las gram\'aticas para lenguajes de programaci\'on caen en 
esta categor\'ia.
As\'i mismo las m\'aquinas que aceptan este tipo de lenguajes son los llamados 
aut\'omatas de pila que ser\'an estudiados en el siguiente tema.
         

\item[Tipo 3] \textbf{Lenguajes regulares}
  Son aquellos generados por una gram\'atica de una de las siguientes formas:
  \bi
  \item Lineal por la derecha: todas las producciones de la forma
    $$A\imp aB\;\;\;A\imp a\;\;\;A\imp\vacia$$
    con $A,B\in V,\;a\in T$
  \item Lineal por la izquierda: todas las producciones de la forma
   $$A\imp Ba\;\;\;A\imp a\;\;\;A\imp\vacia$$
    con $A,B\in V,\;a\in T$
  \item \textbf{No} se permite mezclar ambos tipos de producciones.
  \ei
Estos lenguajes son los que hemos estudiado a detalle hasta ahora y como hemos 
visto las m\'aquinas que los aceptan son los aut\'omatas finitos.
 \ei


Decimos que un lenguaje es de tipo $i$ si y s\'olo si $i$ es el \'indice
\textit{m\'as grande} tal que existe una gram\'atica de tipo $i$ que genera a 
$L$.
Es decir que si un lenguaje~$L$ es generado por una gram\'atica de tipo $i$, 
no se puede asegurar de inmediato que $L$ sea un lenguaje de tipo $i$. Debe 
asegurarse que $i$ es \textit{m\'aximo}.
Para ello es \'util la jerarqu\'ia generada por la clasificaci\'on de 
Chomsky:
\[
\mathcal{L}_3\subseteq \mathcal{L}_2\subseteq \mathcal{L}_1\subseteq 
\mathcal{L}_0
\]

\begin{tabular}{c|c|c|c}
 Tipo & Lenguajes & Modelo de C\'omputo & Gram\'aticas \\\hline\hline
 3 & Regulares & M\'aquinas de estados finitos & Regulares \\\hline
 2 & Libres de Contexto & Au\'omatas de Pila & Libres de contexto\\\hline
 1 & Sensibles al contexto & M\'aquina de Turing & No-contra\'ibles \\
 & & limitada en memoria  & \\\hline
 0 & Recursivamente enumerables o & M\'aquina de Turing & Sin restricci\'on \\
 & semidecidibles  & & \\
\end{tabular}

% La jerarqu\'ia de Chomsky permite refinar la teor\'ia de la computaci\'on 
% clasificando lenguajes en funci\'on de los recursos computacionales 
% necesarios para reconocerlos. Como vimos cada categor\'ia tiene asociado un 
% tipo de m\'aquina que reconoce a los lenguajes.


\end{document}
