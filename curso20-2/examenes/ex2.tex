\documentclass[letterpaper,12pt]{article}
\usepackage[includeheadfoot,margin=1in]{geometry}
\usepackage{anysize}

% \usepackage[utf8]{inputenc}
\usepackage[latin1]{inputenc}
\usepackage[english,spanish]{babel}
\usepackage{lmodern}   % font shapes...
\usepackage[T1]{fontenc} % join the compound symbols as a single symbol

\usepackage{amssymb,amsmath}
\usepackage{mathrsfs}
\usepackage{epsfig}

\usepackage{fancyhdr}
\usepackage{hyperref}
\usepackage{url}

\usepackage{import}
\usepackage{comment}
\usepackage[autostyle=true,spanish=mexican]{csquotes}

\usepackage{url}
\usepackage{array}

\usepackage{pgf}
\usepackage{tikz}

\input{../macrosAyLF}

\title{Aut\'omatas y Lenguajes Formales 2020-2 \\ 
Facultad de Ciencias UNAM \\
Examen 2}
\author{24 de marzo de 2020}
\date{}

\begin{document}
\maketitle

\begin{enumerate}
  
  \item El lenguaje $L\subseteq \{0,1\}^\star$ se define recursivamente como 
  sigue:
%   \[
%    \begin{array}{l}
%      0 \in L \\
%      \text{Si } x\in L  \text{ entonces } x001,\; 11x\in L\\
%      \text{Son todas} 
%    \end{array}
%   \]

  \[
   \begin{array}{l}
     \vacia \in L \\
     0 \in L \\
     \text{Si } w\in L \text{ entonces } w1, w10 \in L\\
     \text{Son todas} 
   \end{array}
  \]
  Realiza lo siguiente:
  \begin{enumerate}
    \item Dise\~na una expresi\'on regular que reconozca 
a~$L$.
    \item Dise\~na un aut\'omata finito~$M$ que acepte 
a~$L$, 
da la idea del dise\~no y su definici\'on formal mediante la funci\'on $\delta$ 
o un diagrama. 
\item \textbf{Punto extra} Muestra que la cadena $011001$ no es aceptada por 
el aut\'omta que dise\~naste. 
  \end{enumerate}
      
\item Dise\~ne un aut\'omata finito determinista para cadenas en 
$\{a, b\}^\star$ tal que \\
% terminan con la cadena $abb$.  \\
no tiene la cadena $aa$ ni la cadena $bb$ como subcadenas. \\
% tiene un n\'umero impar de $b$'s y un n\'umero par de $a$'s. \\
% empiezan o terminan en $ab$.


\item Para el siguiente aut\'omata, describir detalladamente pero sin usar 
expresiones regulares el lenguaje aceptado. Adem\'as muestra una cadena de 
longitud 6 que sea aceptada. 
\[
 \begin{array}{rc|c|c}
      &  & a & b \\\hline\hline
inicial& q_0 & q_1 & q_2 \\\hline
    & q_1 & q_3 & q_4 \\\hline
    & q_2 & q_5 & q_6 \\\hline
final & q_3 & q_3 & q_4 \\\hline
    & q_4 & q_5 & q_6 \\\hline
final & q_5 & q_3 & q_4 \\\hline
final & q_6 & q_5 & q_6 \\\hline
 \end{array}
\]
% 
% \includegraphics[width=0.5\textwidth]{ex2.png}
% 
% Es el lenguaje de w's que no acaban en $ab$.
% Recursivamente:
% \[
%    \begin{array}{l}
%      aa,\,bb,\,ba \in L \\
%      \text{Si } w\in L \text{ entonces } wa,\, aw,\, bw \in L\\
%      \text{Son todas} 
%    \end{array}
%   \]
% 
 

\end{enumerate}


\end{document}
