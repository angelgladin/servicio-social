\documentclass[letterpaper,12pt]{article}
\usepackage[includeheadfoot,margin=1in]{geometry}
\usepackage{anysize}

% \usepackage[utf8]{inputenc}
\usepackage[latin1]{inputenc}
\usepackage[english,spanish]{babel}
\usepackage{lmodern}   % font shapes...
\usepackage[T1]{fontenc} % join the compound symbols as a single symbol

\usepackage{amssymb,amsmath}
\usepackage{mathrsfs}
\usepackage{epsfig}

\usepackage{fancyhdr}
\usepackage{hyperref}
\usepackage{url}

\usepackage{import}
\usepackage{comment}
\usepackage[autostyle=true,spanish=mexican]{csquotes}

\usepackage{url}
\usepackage{array}


\usepackage{pgf}
\usepackage{tikz}
\usetikzlibrary{automata,arrows,trees}
\usetikzlibrary{babel}
%\pagestyle{fancyplain}
\input{../macrosAyLF}

\title{Aut\'omatas y Lenguajes Formales 2020-2 \\ 
Facultad de Ciencias UNAM \\
Tarea-Examen }
\author{\textbf{Fecha de entrega: 1 de abril de 2020}}
\date{Trabajo individual a registrar mediante un archivo o archivos en la 
plataforma GoogleClassroom}

\begin{document}
\maketitle

\begin{enumerate}

\item Dise\~na un AFN$_{\vacia}$, mediante el m\'etodo de S\'intesis de Kleene, 
que reconozca el lenguaje $L$ generado por la expresi\'on regular:
% \[\al= a(b+ba)^\star + a(a+ab)^\star bb\]
% \[\big((a^+ + ab)^\star(\vacio^\star+(aba)^\star)^\star\big)^\star\]
\[\alpha = (0+1)^*(01 + 110)\]
\[\alpha \;= \; 1^\star01\big((01)^\star + 1\big)^\star\]
\small{\textbf{Consideraciones:} 
Se debe mostrar el proceso de construcci\'on de los subaut\'omatas 
correspondientes a las subexpresiones regulares de $\alpha$.
No est\'a permitido simplificar $\alpha$ ni omitir estados en la 
construcci\'on. Se pueden omitir algunas transiciones $\vacia$. }

\item Transforma el siguiente aut\'omtata en uno AFN al eliminar las 
$\vacia$-transiciones  (s\'olo es necesario dar la tabla de 
transiciones). Muestra el c\'alculo de las $Cl_{\vacia}$ de 
cada estado.
\begin{center}
\begin{tikzpicture}[node distance=2.5cm,every node/.style={scale=0.8},semithick]
    \node[state,initial,initial text=] (q0) {$q_0$};
    \node[state] (q1) [above left of=q0] {$q_1$};
    \node[state] (q2) [above right of=q0] {$q_2$};
    \node[state] (q3) [right of=q0] {$q_3$};
    \node[state] (q4) [right of=q3] {$q_4$};
    \node[state,accepting by double] (q5) [right of=q4] {$q_5$};
    \path[->] (q0) edge node [above] {$\vacia$} (q1);
    \path[->] (q1) edge [bend left] node [above] {a} (q2);
    \path[->] (q2) edge [loop above] node [above] {b} (q2);
    \path[->] (q2) edge node [above] {$\vacia$} (q0);
    \path[->] (q0) edge node [above] {$\vacia$} (q3);
    \path[->] (q0) edge [bend left] node [above] {$\vacia$} (q5);
    \path[->] (q3) edge node [below] {a} (q4);
    \path[->] (q4) edge node [below] {a} (q5);
 \end{tikzpicture}
\end{center}


% \item Considera los siguientes aut\'omatas:
% \begin{figure}[!ht]
% \centering
% \begin{footnotesize}
% \begin{minipage}{.4\textwidth}
% $M_1$: \\
\begin{tikzpicture}[node distance=2.5cm,every 
node/.style={scale=0.7},semithick]
    \node[state,initial,initial text=] (q0) {$q_0$};
    \node[state,accepting by double] (q1) [above right of=q0] {$q_1$};
    \node[state] (q2) [above right of=q1] {$q_2$};
    \node[state] (q3) [below right of=q2] {$q_3$};
    \node[state] (q4) [below right of=q0] {$q_4$};
    \node[state] (q5) [right of=q4] {$q_5$};
    \node[state] (q6) [right of=q5] {$q_6$};
    \path[->] (q0) edge [loop above] node [above] {b} (q0);
    \path[->] (q0) edge node [above] {$\vacia$} (q1);
    \path[->] (q0) edge node [below] {$\vacia$} (q4);
    \path[->] (q1) edge node [above] {a} (q2);
    \path[->] (q2) edge node [above] {b} (q3);
    \path[->] (q3) edge node [above] {b} (q1);
    \path[->] (q4) edge [loop above] node [above] {b} (q4);
    \path[->] (q4) edge node [above] {$\vacia$} (q5);
    \path[->] (q5) edge [bend left] node [above] {b} (q6);
    \path[->] (q6) edge [bend left] node [above] {a} (q5);
 \end{tikzpicture}
% \end{minipage}
% \end{footnotesize}
% \hspace{20pt}
% \begin{footnotesize}
% \begin{minipage}{.4\textwidth}
%  $M_2$:\\
% \begin{tikzpicture}[node distance=3cm,every node/.style={scale=0.8},semithick]
%     \node[state,initial,initial text=,accepting by double] (q0) {$q_0$};
%     \node[state] (q1) [above right of=q0] {$q_1$};
%     \node[state] (q2) [below right of=q1] {$q_2$};
%     \node[state] (q3) [below of=q0] {$q_3$};
%     \node[state] (q4) [above right of=q3] {$q_4$};
%     \node[state,accepting by double] (q5) [below right of=q4] {$q_5$};
%     \path[->] (q0) edge node [above] {a} (q1);
%     \path[->] (q0) edge node [left] {$\vacia$} (q3);
%     \path[->] (q1) edge node [above] {b} (q2);
%     \path[->] (q2) edge node [above] {b} (q0);
%     \path[->] (q2) edge [loop right] node [above] {a} (q2);
%     \path[->] (q3) edge node [above] {b} (q4);
%     \path[->] (q4) edge node [above] {b} (q5);
%     \path[->] (q5) edge node [above] {b} (q3);
%  \end{tikzpicture}
% \end{minipage}
% \end{footnotesize}
% \end{figure}
% \FloatBarrier
% \bi
% \item Calcula el aut\'omata que reconce $L_2L_1^\star$ donde $L_1 = L(M_1)$  
% y $L_2 = L(M_2)$.
% \item Transforma uno de los aut\'omatas del ejercicio en determinista 
% mostrando la cerradura $\vacia$ para cada estado. 
% \ei

% \item Dado el siguiente aut\'omata, da un diagrama de \'el, 
% una trasformaci\'on que lo convierta en determinista as\'i como un 
% aut\'omata  equivalente m\'inimo.
% \[
%  \begin{array}{cl|c|c|c}
%   & \delta & a & b & \vacia \\\hline
%   inicial, final & q_0 & q_1 & -- & --\\
%    & q_1 & q_4 & q_2 & -- \\
%    & q_2 & q_3 & -- & q_0 \\
%    & q_3 & -- & q_2 & -- \\
%    & q_4 & -- & -- & q_0
% \end{array}
% \]

\item Minimiza el siguiente aut\'omata:
\[
 \begin{array}{cl|c|c}
  & \delta & a & b \\\hline
inicial, final & q_0 & q_4 & q_2 \\
 final    & q_1 & q_2 & q_1  \\
  final   & q_2 & q_4 & q_3  \\
   final  & q_3 & q_2 & q_1 \\
     & q_4 & q_4 & q_4 \\ 
%      & q_5 & q_2 & q_6 \\
% final & q_6 & q_3 & q_6 \\
\end{array} 
\]
\[
 \begin{array}{cl|c|c}
  & \delta & a & b \\\hline
inicial, final & q_0 & q_1 & q_3 \\
    & q_1 & q_2 & q_3  \\
  final   & q_2 & q_5 & q_2  \\
    & q_3 & q_4 & q_1 \\
    final  & q_4 & q_5 & q_4 \\ 
     & q_5 & q_5 & q_5 \\
\end{array} 
\]

% \item \textbf{(Punto extra)} Contesta una y s\'olo una de las siguientes:
% 
% \item Eliminar el no-determinismo del aut\'omata del ejercicio 2.
% 
% \item Suponga que un lenguaje $L$ es finito, ?`cu\'al es el n\'umero (m\'inimo) 
% de estados que un AFD debe tener para aceptar el lenguaje $L$?
% 
% \item Suponga que $M$ es un AFN$\vacia$ que acepta a un lenguaje 
% $L\subseteq \Sigma^*$. Describa c\'omo modificar $M$ para obtener un 
% AFN$\vacia$ que reconozca $L(\al)^r$, es decir $rev(L) = \{w^r \mid x \in L\}$.

\end{enumerate}

\end{document}
