\documentclass[letterpaper,12pt]{article}
\usepackage[includeheadfoot,margin=1in]{geometry}
\usepackage{anysize}

% \usepackage[utf8]{inputenc}
\usepackage[latin1]{inputenc}
\usepackage[english,spanish]{babel}
\usepackage{lmodern}   % font shapes...
\usepackage[T1]{fontenc} % join the compound symbols as a single symbol

\usepackage{amssymb,amsmath}
\usepackage{mathrsfs}
\usepackage{epsfig}

\usepackage{fancyhdr}
\usepackage{hyperref}
\usepackage{url}

\usepackage{import}
\usepackage{comment}
\usepackage[autostyle=true,spanish=mexican]{csquotes}

\usepackage{url}
\usepackage{array}

\usepackage{pgf}
\usepackage{tikz}
\usetikzlibrary{automata,arrows,trees}
\usetikzlibrary{babel}

\input{../macrosAyLF}

\title{Aut\'omatas y Lenguajes Formales 2020-2 \\ 
Facultad de Ciencias UNAM \\
Tarea-Examen 2}
\author{30 de abril de 2020\\\textbf{Fecha de entrega: 5 de mayo de 2020}}
\date{Trabajo individual a registrar mediante un archivo o archivos en la 
plataforma GoogleClassroom.}

\begin{document}
\maketitle

\vspace*{-10pt}

\begin{enumerate}
\item (1.5 pts.) Encuentra una expresi\'on regular para el lenguaje aceptado 
por el 
siguiente aut\'omata siguiendo el m\'etodo de ecuaciones de lenguajes, muestra 
el sistema de ecuaciones y las soluciones de cada una de ellas:
\begin{center}
\begin{tikzpicture}[node distance=2.5cm,every node/.style={scale=0.8},semithick]
    \node[state,initial,initial text=] (q0) {$q_0$};
    \node[state,accepting by double] (q1) [right of=q0] {$q_1$};
    \node[state,accepting by double] (q2) [right of=q1] {$q_2$};
    \path[->] (q0) edge [loop above] node [above] {0} (q0);
    \path[->] (q0) edge [bend left] node [above] {1} (q1);
    \path[->] (q1) edge [bend left] node [above] {1} (q2);
    \path[->] (q2) edge [loop right] node [above] {1} (q2);
    \path[->] (q2) edge [bend left] node [above] {0} (q1);
    \path[->] (q1) edge [bend left] node [above] {0} (q0);
 \end{tikzpicture}
\end{center}

\item (1.5 pts.)  Obtener una gram\'atica regular que genere el mismo lenguaje 
que el 
aceptado por el siguiente aut\'omata:
\begin{center}
\includegraphics[width=.3\textwidth]{AFD}
\end{center}


 \item (2 pts.) Genera un aut\'omata finito (con transiciones $\vacia$) cuyo 
lenguaje 
de aceptaci\'on es el  mismo que el generado por la siguiente gram\'atica: 
  \[ 
   \begin{array}{rrlcrrl}
    & S \imp & aA \mid \vacia &\qquad\qquad& D \imp & bC \mid b \mid aF \mid a\\
    & A \imp & aB \mid bE & & E \imp & bE \mid aF \mid a \mid \vacia \\
    & B \imp & aA \mid bC \mid b &  & F \imp & aF \mid a \mid bF \mid b \\
    & C \imp & bD \mid aF \mid a \mid bS &&&
   \end{array}
  \]
 \textbf{Hasta 1 punto extra:} Transforma el aut\'omata dado en uno 
  \textbf{sin} transiciones $\vacia$.

  
  
\item (1.5 pts.) Dise\~na una GLC para
  $L=\{w\in\{a,b\}^\star\;|\;w=a^ib^{2i}c^k\;o \;  w=a^kb^ic^{2i},\;i,k\geq 
0\}$. \\
Muestre la derivaci\'on m\'as a la derecha, as\'i como el \'arbol 
correspondiente para la cadena $aabbbbccc$.
%   
% \item  La gram\'atica $G$ permite construir el siguiente \'arbol de 
% derivaci\'on para la cadena $abccc$:
% \begin{tikzpicture}[level distance=12mm]
%   \tikzstyle{level 1}=[sibling distance=20mm]
%   \tikzstyle{level 2}=[sibling distance=10mm]
%  
%   \node {$S$}
%     child {node {$A$}
%       child {node {$aAb$}
%         child {node {$\vacia$}}}
%      }
%     child {node {$B$}
%         child {node {$cBc$}
%             child {node {$c$}}}
%     };
%  \end{tikzpicture}
% \begin{enumerate}
% \item Dar la definici\'on formal de $G$, asumiendo que las variables, 
% terminales y producciones de $G$ son \'unicamente las involucradas en el \'arbol 
% anterior.
% \item Construya dos derivaciones para $abccc$ cuyo \'arbol de derivaci\'on sea 
% el anterior y de manera que una sea por la izquierda y la otra arbitraria.
% \item ?` Qui\'en es $L(G)$ ? justifique su respuesta.
% \end{enumerate}

\item (2 pts.) Considere la siguiente gram\'atica  $G$:
% \beqs
%   \ba{rll}
%   S & \imp & A b \, | \, aaB\\
%   A &\imp  & a \, | \, Aa\\
%   B &\imp  & a \, | \, b \, | \, c \\
%   \ea
%   \eeqs
%   
\[
    S\imp aSA\;|\;\cv\;\;\;\;\;\;\; A\imp bcA\;|\;\cv
\]   
\begin{enumerate}
\item Demuestre que $G$ es ambigua mostrando dos \'arboles distintos de 
derivaci\'on para una misma cadena $w$.
\item Defina una gram\'atica $G'$ no ambigua equivalente a $G$.
\item Muestre el \'unico \'arbol de derivaci\'on para la cadena $w$ empleada en 
el inciso a).
\item Justifique por qu\'e $G'$ no es ambigua.
\end{enumerate}


 \item (1.5 pts.) Proporciona una gram\'atica libre de contexto para el 
  siguiente lenguaje:
    $$ L = \{ a^i b^j c^k \mid i=j \text{ \'o } i=k \}$$
  \begin{small}
   \textbf{[Hint]} Utiliza operaciones de lenguajes que son cerradas para 
    lenguajes libres de contexto. 
\end{small}

 \item[] \textbf{Hasta 1 punto extra:}  Considera la siguiente gram\'atica:
  \[
     S \rightarrow AB \qquad \qquad 
     A \rightarrow aA \mid \vacia \qquad \qquad 
     B \rightarrow ab \mid bB \mid \vacia
  \]
  \begin{center}
  \begin{small}
  \textit{Cualquier derivaci\'on de una cadena en esta gram\'atica debe 	
  comenzar con \\ la producci\'on~$S\imp AB$. 
  Est\'a claro que toda cadena derivable desde $A$ \\
  s\'olo tiene una derivaci\'on usando $A$ y lo mismo ocurre con $B$.\\ 
  Por lo tanto, la gram\'atica \textbf{no} es ambigua.}
  \end{small}
  \end{center}
  ?`El razonamiento anterior es correcto? Es decir, ?`es cierto que la 
  gram\'atica no es ambigua? Justifica tus respuestas. 

    
 \item[] \textbf{Hasta 1 punto extra:}  Dada la gram\'atica cuyas producciones 
son:
  \[
    S \rightarrow aAbC \qquad \qquad 
    A \rightarrow aAb \mid ab \qquad \qquad
    C \rightarrow cC \mid c 
  \]
  Describe y justifica cu\'al es el lenguaje que genera. ?`Es un lenguaje 
  regular? \\Si tu respuesta es afirmativa proporciona una gram\'atica regular 
  para el mismo lenguaje, sino \textbf{explica a detalle} porqu\'e no es 
  regular.

% % \item \textbf{Hasta 1 punto extra:} Sean $G = (N,T,P,S)$ una gram\'atica 
% % lineal por la derecha y 
% %   $G' = (N,T,P',S)$ la gram\'atica que se obtiene a partir de $G$ de la 
% %   siguiente manera.
% %   \begin{itemize}
% %    \item Si $A \rightarrow a \in P$ entonces  $A \rightarrow a \in P'$
% %    \item Si $A \rightarrow aB \in P$ entonces $B \rightarrow aA \in P'$
% %   \end{itemize}
% %   ?`Qui\'en es $L(G')$? Demuestra tu respuesta.
% %   %Demuestra que $L(G') = (L(G))^{R}$

\end{enumerate}

\end{document}
