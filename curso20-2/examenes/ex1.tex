\documentclass[letterpaper,12pt]{article}
\usepackage[includeheadfoot,margin=1in]{geometry}
\usepackage{anysize}

% \usepackage[utf8]{inputenc}
\usepackage[latin1]{inputenc}
\usepackage[english,spanish]{babel}
\usepackage{lmodern}   % font shapes...
\usepackage[T1]{fontenc} % join the compound symbols as a single symbol

\usepackage{amssymb,amsmath}
\usepackage{mathrsfs}
\usepackage{epsfig}

\usepackage{fancyhdr}
\usepackage{hyperref}
\usepackage{url}

\usepackage{import}
\usepackage{comment}
\usepackage[autostyle=true,spanish=mexican]{csquotes}

\usepackage{url}
\usepackage{array}

\usepackage{pgf}
\usepackage{tikz}

\input{../macrosAyLF}

\title{Aut\'omatas y Lenguajes Formales 2020-2 \\ 
Facultad de Ciencias UNAM \\
Examen 1}
\author{25 de febrero de 2020}
\date{}

\begin{document}
\maketitle

\begin{enumerate}


%  \item Considere las expresiones regulares:
%  $$\alpha = 0^* + 1^* \qquad\qquad \beta = 01^* + 10^* + 1^*0 + (0^*1)^*$$
%  \begin{enumerate}
%   \item Encuentre una cadena en $L(\alpha)$ y no en $L(\beta)$.
%   \item Encuentre una cadena en $L(\beta)$ y no en $L(\alpha)$.
%   \item Encuentre una cadena en $L(\alpha)$ y en $L(\beta)$.
%   \item Encuentre una cadena en $\{0,1\}^*$ que no est\'e ni en $L(\alpha)$ 
%   ni en $L(\beta)$.
%  \end{enumerate}


\item Sea $L = \{ab, aa, baa\}$. Considerar las siguientes cadenas en 
$\{a,b\}^\star$:
\begin{enumerate}
 \item $abaaabab$
 \item $abaabaabaa$
 \item $aaaabaaaa$
 \item $baaaaabaaaab$
\end{enumerate}
\begin{itemize}
 \item (\textbf{2pts}) Indicar cu\'ales de las palabras pertenecen a $L^+$ 
mostrando las subcadenas correspondientes.
 \item (\textbf{1pt}) Proponer dos cadenas diferentes a las anteriores y a la 
cadena vac\'ia que pertenezcan a $L^\star$.
\end{itemize}


\item (\textbf{2pts}) Decida si los lenguajes $(L^\star)^R$ y $(L^R)^\star$ 
son iguales o no para cualquier $L$.\\ Justifique formalmente su respuesta. 

% \item (\textbf{2pts}) Proporcione una definici\'on \textit{no recursiva} del 
% lenguaje
%   $$L = \{w\in\Sigma^* \mid aa \text{ no es subcadena de } w\}$$
% Use lenguajes y operaciones entre ellos y \textbf{no} expresiones regulares.
%  
 \item Sean $L_1$ y $L_2$ los siguientes lenguajes sobre el alfabeto  
 $\Sigma = \{a,b\}$:
 \[
  \begin{array}{rlcrl}
   L_1: & &\hspace{20pt} & L_2: & \\
   & \vacia \in L_1 &  & & \vacia \in L_2\\
   & a \in L_1 & & & a \in L_2 \\
   & \text{Si } w\in L_1 \text{ entonces } wb, wba \in L_1 & & &
     \text{Si } w\in L_2 \text{ entonces } bw, abw \in L_2 
  \end{array}
 \]
 \begin{itemize}
  \item (\textbf{1pt}) Analizar ambos lenguajes mostrando algunas cadenas que 
pertenzcan a cada lenguaje respectivamente. 
  \item (\textbf{2pts}) Demostrar que \textbf{alguno} de ellos es equivalente 
al lenguaje
 $$L = \{w\in\Sigma^* \mid aa \text{ no es subcadena de } w\}$$
%  del ejercicio anterior.
 \end{itemize}

 
 \textbf{Hasta dos puntos extra:} Demostrar que los tres son equivalentes, 
  argumenta correctamente las equivalencias.
%  
%  \item Proporcione expresiones regulares que correspondan a los lenguajes 
%   $L_1$ y $L_2$ del ejercicio anterior.

%   \item Construir expresiones regulares para los siguientes lenguajes
%   sobre   $\Sigma^\star=\{a,b,c\}$.
%   \begin{enumerate}
%   \item $L=\{vwv\in\{a,b\}^\star\;|\;|v|=2\}$
% 
%   \item $L=\{a^nb^m\;|\;n,m\geq 1,\;nm\geq 3\}$
%   \end{enumerate}
% %   
% \item (\textbf{2pts}) Decidir si la siguiente equivalencia entre expresiones 
% regulares es v\'alida mediante una demostraci\'on formal, en caso contrario dar 
% un contraejemplo.
%  \begin{enumerate}
% % %   \item $\big((01+1)^\star(10+0)^\star +
% % %    \vacio^\star\big)^\star(10+01)=(0+1)^\star(01+10)$
% % %    \item[] $\big((11\vacio^\star)+1^\star)^\star 1^\star\big)^\star
% % %      01((01)^\star +1)^\star=1^\star 01 ((01)^\star+1)^\star$
%  \item[] $a(ba)^{*} = (ab)^{*} a$
% %  \item $(a^{*} b^*)^{*} = (a + b)^{*}$
% % \item[]  $(\alpha + \beta)^\star\beta \; = \; (\alpha^\star\beta)^\star$
%  \end{enumerate}

 
  
%   \item El lenguaje $L\subseteq \{0,1\}^\star$ se define recursivamente como 
%   sigue:
% %   \[
% %    \begin{array}{l}
% %      0 \in L \\
% %      \text{Si } x\in L  \text{ entonces } x001,\; 11x\in L\\
% %      \text{Son todas} 
% %    \end{array}
% %   \]
% 
%   \[
%    \begin{array}{l}
%      \vacia \in L \\
%      0 \in L \\
%      \text{Si } w\in L \text{ entonces } w1, w10 \in L\\
%      \text{Son todas} 
%    \end{array}
%   \]
%   Realiza lo siguiente:
%   \begin{enumerate}
%     \item (\textbf{1.5pts}) Dise\~na una expresi\'on regular que reconozca 
% a~$L$.
%     \item (\textbf{1.5pts}) Dise\~na un aut\'omata finito~$M$ que acepte a~$L$, 
% da la idea del dise\~no y su definici\'on formal.
%   \end{enumerate}
      
% \item Dise\~ne un aut\'omata finito determinista para cadenas en 
% $\{a, b\}^\star$ tal que \\
% terminan con la cadena $abb$.  \\
% no tiene la cadena $aa$ ni la cadena $bb$ como subcadenas. \\
% tiene un n\'umero impar de $b$'s y un n\'umero par de $a$'s. \\
% empiezan o terminan en $ab$.


\item (\textbf{2pts}) Demuestre que dado $LL = L$ para alg\'un lenguaje 
$L$ es porque $\vacia \in L$ \'o $L= \vacio$.


\end{enumerate}

\end{document}
