\documentclass[letterpaper,11pt]{article}
\usepackage[includeheadfoot,margin=1.9cm]{geometry}
\usepackage{anysize}

% \usepackage[utf8]{inputenc}
\usepackage[latin1]{inputenc}
\usepackage[spanish]{babel}
\usepackage{lmodern}   % font shapes...
\usepackage[T1]{fontenc} % join the compound symbols as a single symbol

\usepackage{amssymb,amsmath}
\usepackage{mathrsfs}
\usepackage{epsfig}

\usepackage{fancyhdr}
\usepackage{hyperref}
\usepackage{url}

\usepackage{import}
\usepackage{comment}
\usepackage[autostyle=true,spanish=mexican]{csquotes}

\usepackage{url}

\usepackage{pgf}
\usepackage{tikz}
\usetikzlibrary{automata,arrows}
\usetikzlibrary{babel}
%\pagestyle{fancyplain}
\input{../macrosAyLF}

\title{Aut\'omatas y Lenguajes Formales 2020-2 \\ 
Facultad de Ciencias UNAM \\
Bolet\'in de Ejercicios 7\\
\Large{Formas normales de Gram\'aticas Libres de contexto}}
\author{Favio E. Miranda Perea \and A. Liliana Reyes Cabello \and 
Lourdes Gonz\'alez Huesca }
\date{\today}


\begin{document}
\maketitle
\begin{enumerate}
 \item Transformar las siguientes gram\'aticas a \textbf{FNChomsky}, detallando 
los 
  pasos:
  \[
   \begin{array}{rrlcrrl}
    G1 : & & & \qquad \qquad & G2 : &  &  \\    
    & S \imp & X \mid Y &&& S \imp & 0S1 \mid A \mid AB\\
    & X \imp & aZb \mid bZa &&& A \imp & 1A0 \mid S \mid \vacia \\
    & Z \imp & aZb \mid bZa \mid \vacia &&& B \imp & 0B \mid 1C\\
    & A \imp & a \mid aA &&& C \imp & 0C \mid 0 \mid \vacia \\
    & B \imp & b &&& D \imp & 0C \mid 1D \mid F \\
    & & &&& F \imp & 1F\mid \vacia 
   \end{array}
  \]

  \item Transformar las siguientes gram\'aticas a \textbf{FNGreibach}, 
detallando los 
  pasos:
  \[
   \begin{array}{rrlcrrl}
    G1 : & & & \qquad \qquad &G2 : & &  \\    
    & S \imp & CA \mid DB  && & S \imp & AB \mid BC\\
    & A \imp & CB & & & A \imp & BA \mid a \\
    & B \imp & EB\mid b & & & B \imp & CC\mid b\\
    & C \imp & b & & & C \imp & AB \mid a \\
    & D \imp & c & & & &\\
    & E \imp & a & & & &
   \end{array}
  \]

\end{enumerate}
\end{document}
