\documentclass[letterpaper,11pt]{article}
\usepackage[includeheadfoot,margin=1.9cm]{geometry}
% \usepackage{anysize}

% \usepackage[utf8]{inputenc}
\usepackage[spanish]{babel}
\usepackage{lmodern}   % font shapes...
\usepackage[T1]{fontenc} % join the compound symbols as a single symbol

\usepackage{amssymb,amsmath}
\usepackage{mathrsfs}
\usepackage{epsfig}

\usepackage{fancyhdr}
\usepackage{hyperref}
\usepackage{url}

\usepackage{import}
\usepackage{comment}
\usepackage[autostyle=true,spanish=mexican]{csquotes}

\usepackage{url}

%\pagestyle{fancyplain}
\input{../macrosAyLF}

\title{Autómatas y Lenguajes Formales 2020-2 \\ 
Facultad de Ciencias UNAM \\
Boletín de Ejercicios 1\\
\Large{Introducci\'on: cadenas y lenguajes }}
\author{Favio E. Miranda Perea \and A. Liliana Reyes Cabello \and 
Lourdes Gonz\'alez Huesca }
\date{\today}

\begin{document}
\maketitle

% \section*{Cadenas y Lenguajes}
\begin{enumerate}
 \item Demostrar las propiedades de la concatenaci\'on de cadenas:
  \bi
   \item Asociatividad: $(uv)w = u(vw)$.
   \item Identidad: $v\varepsilon = \varepsilon v= v$.
   \item Longitud: $|vw| = |v| + |w|$.
  \ei
 
 \item Demostrar que para cualesquiera cadenas~$u,v,w$ y n\'umero natural $n$, 
se cumplen las siguientes propiedades:
  \be
   \item $|w^r| = |w|$
   \item $(w^r)^r = w$
   \item $(uv)^R=v^Ru^R$
   \item Para cada $n\geq 0$, $(w^n)^R = (w^R)^n $
  \ee
 
 \item Sea $\Sigma=\{a,b\}$, el lenguaje $L$ se define como:
  \be
   \item[i)] $\vacia \in L$
   \item[ii)] Si $w_1,w_2\in L$ entonces $a w_1 b w_2,\,b w_1 a w_2 \in L$.
   \item[iii)] Son todas las cadenas en $L$.
  \ee
  Demuestre que las cadenas en $L$ tienen el mismo n\'umero de $a$'s que de 
  $b$'s  ($\#_a = \#_b$).
 
 \item Sea $w$ una cadena sobre el alfabeto $\Sigma$ y $n,m \in  \mathbb{N}$ . 
  Demuestre que  $ w^{n} \cdot w^{m} = w^{n+m}$.
    \textit{Pista: usar la definici\'on de concatenacion repetida 
    $w^{n+1} = w^n\cdot w$ y $w^{1+n} = w\cdot w^n$.}
 \item Para cada~$n\geq 0$, se definen las cadenas $a_n$ y $b_n$ en $\{0,1\}^*$ 
  como sigue:
  \[ 
   a_0=0; \quad b_0=1; \qquad \text{para } n>0,\; a_n=a_{n-1}b_{n-1}; \quad
   b_n=b_{n-1}a_{n-1}
  \]
  Demuestre las siguientes propiedades de las cadenas descritas
  \be
   \item Para cada~$n\geq 0$ las cadenas $a_n$ y $b_n$ tienen la misma   
    longitud.
   \item Las cadenas $a_{2n}$ y $b_{2n}$ son pal\'indromos.
   \item Las cadenas $a_{n}$ y $b_{n}$ difieren en cada posición.
  \ee
  
 \item Sea $w = babbab$ una cadena sobre el alfabeto~$\Sigma = \{a, b\}$. 
  Describa los conjuntos de \textit{todos} los prefijos y sufijos de $w$.
 
 \item Proponga una definición no recursiva sencilla para el lenguaje~$L$ dado 
  por la siguiente definición recursiva:
  \be
   \item[i)]  $\vacia \in L$.
   \item[ii)] Si $x\in L$ entonces $0x1\in L,\;1x0\in L$.
   \item[iii)] Si $x,y\in L$ entonces $xy\in L$
   \item[iv)] Son todas.
  \ee

 \item Dado el alfabeto $\Sigma = \{1,2,3,a,b,c\}$ y los lenguajes
  $L_1 = \{1,2,3\}$ y $L_2 =\{a,b,c\}$, definir los siguientes lenguajes:
  \be
   \item $L_1^2$
   \item $L_1 \cup L_2$
   \item $L_1L_2$
   \item $(L_1L_2)^2$
  \ee 
 
 \item Demuestre que para todo lenguaje $L\subseteq\{0,1\}^\star$, si 
  $L^2\subseteq L$ entonces $L^+ \subseteq L$.

 \item Sean $L_1,L_2$ y $L_3$ lenguajes en un alfabeto $\Sigma$.
  Decida si los siguientes pares de lenguajes son equivalentes o no. 
  Justifique su respuesta mediante una demostraci\'on o un contraejemplo 
  seg\'un sea el caso:
  \begin{center}
  \begin{tabular}{rccc}
   $a)$ & $L_1 \big(L_2 \cap L_3\big)$ &\hspace{5pt}& $L_1 L_2 \cap  L_1 L_3$\\
   $b)$ & $L_1^\star \cap L_2^\star $ & & $\big(L_1\cap L_2\big)^\star$\\
   $c)$ & $L_1^\star L_2^\star$ & & $ \big(L_1 L_2\big)^\star$ \\
   $d)$ & $L_1^\star \big(L_2 L_1^\star\big)^\star$ & & 
    $\big(L_1^\star L_2\big)^\star L_1^\star$ \\
   $e)$ & $\big(L_1 \cup L_2\big)^\star$ & & 
    $\big(L_1^\star L_2^\star\big)^\star$
  \end{tabular}
  \end{center}

\end{enumerate}

\end{document}


