\documentclass[letterpaper,11pt]{article}
\usepackage[includeheadfoot,margin=1.9cm]{geometry}
\usepackage{anysize}

% \usepackage[utf8]{inputenc}
\usepackage[latin1]{inputenc}
\usepackage[spanish]{babel}
\usepackage{lmodern}   % font shapes...
\usepackage[T1]{fontenc} % join the compound symbols as a single symbol

\usepackage{amssymb,amsmath}
\usepackage{mathrsfs}
\usepackage{epsfig}

\usepackage{fancyhdr}
\usepackage{hyperref}
\usepackage{url}

\usepackage{import}
\usepackage{comment}
\usepackage[autostyle=true,spanish=mexican]{csquotes}

\usepackage{url}

\usepackage{pgf}
\usepackage{tikz}
\usetikzlibrary{automata,arrows}
\usetikzlibrary{babel}
%\pagestyle{fancyplain}
\input{../macrosAyLF}

\title{Aut\'omatas y Lenguajes Formales 2020-2 \\ 
Facultad de Ciencias UNAM \\
Bolet\'in de Ejercicios 6\\
\Large{Gram\'aticas}}
\author{Favio E. Miranda Perea \and A. Liliana Reyes Cabello \and 
Lourdes Gonz\'alez Huesca }
\date{\today}


\begin{document}
\maketitle
\begin{enumerate}
 \item Dise\~nar gram\'aticas para los siguientes lenguajes, indica si es 
regular, libre de contexto o de alg\'un otro tipo.
\bi
 \item $L_0 = \{w \in\{0,1\}^\star \mid w \text{ termina en } 10\}$
 \item $L_1 = \{ ww^R \mid w\in \{a,b \}^\star \}$
 \item $L_2 = \{ w \in \{a,b\}^\star \mid \#_{a} \leq \#_{b} \}$
 \item $L_3 = \{w \in\{0,1\}^\star \mid w \text{ \textbf{no} contiene la 
  subcadena } 110\}$
 \item $L_4 = \{ w\in \{0,1\}^\star \mid \text{en donde cada } 0 
  \text{ est\'a seguido de } 11\}$
 \item $L_5 = \{ a^n \mid n\text{ es un n\'umero primo } \}$
 \item $L_6 = \{ w_1cw_2 \mid w_1,w_2\in \{a,b\}^\star,\; w_1 \neq w_2 \}$
 \item $L_7 = \{ ab^n w \mid n\geq 3,\; w \text{ comienza con } aba \}$
 \item $L_8 = \{ a ^{n!} \mid n\in\N \}$
 \item $L_9 = \{ w\in\{0,1\}^\star \mid \text{cada bloque de 3 s\'imbolos 
     consecutivos de } w \text{ tiene la subcadena } 01\} $
 \item $L_{10} = \{a^n b^m \in \{a,b\}^\star \mid n + m \text{ es par } \}$
 \item $L_{11} = \{w \in \{a,b,c\}^\star \mid w \text{ contiene al menos una 
  vez cada s\'imbolo de } \{a,b,c\} \}$
 \item $L_{12} = \{ a^n b^m\in \{a,b \}^\star \mid n\neq m \}$
 \item $L_{13} = \{w \in \{0,1\}^\star \mid w \text{ \textbf{no} termina 
  con } 01\}$
 \item $L_{14} = \{ w \in \{0,1\}^\star \mid |w| = 4n \text{ y }
      w \text{ contiene ceros en las posiciones pares } \}$ 
 \item $L_{15} = \{ (ab)^n a^m\in \{a,b\}^\star \mid n>m,\,m\geq 0 \}$
 \item $L_{16} = \{a^n b^m \mid n \text{ es impar o } m \text{ es par }\}$
 \item $L_{17} = \{w\in \{a,b\}^\star \mid |w|\geq 2 \text{ tiene al menos una 
  } a \text{ y termina en } b \}$
 \item $L_{18} = \{w\in \{a,b\}^\star \mid w \text{ inicia y termina en } a 
  \text{ y no hay dos } b \text{ consecutivas }\}$
\ei

\item Demuestre las siguientes propiedades de cerradura para lenguajes libres 
  de contexto:
  \be
   \item Si $L$ es libre de contexto y $R$ es regular entonces $L\cap R$ es 
    libre de contexto.
   \item Si $L$ es libre de contexto y $R$ es regular entonces $LR$ es libre 
    de contexto.
  \ee

 \item Encuentre una gram\'atica libre de contexto para los siguientes 
  lenguajes:
  \bi
   \item $L_1=\{a^ib^jc^k \mid i=2j \text{ \'o } j=2k,\, i,j,k\in N\}$ \\
    Adem\'as de una derivaci\'on por la izquierda de la cadena~$aaaabbcc$.
   \item $L_2 = \{a^ibc^k \mid i > k \geq 1\}$ 
   \item $L_3 = \{a^ib^jc^k \mid i> i+k,\, i,j,k\in N\}$ 
  \ei
  
 \item Sean $L_1 = \{x^ny^m \mid 1<n\leq m \leq 3n\}$ y $L_2 = L(G)$ donde $G$ 
  es:  
  \[
   \begin{array}{rlcrl}
    S \imp & A & \hspace*{20pt} &A \imp & xAy \mid B \mid C\\
    B \imp & xByy \mid \vacia & &  C \imp & xCyyy \mid \vacia \\
   \end{array}
  \]
  ?`Qu\'e relaci\'on de contenci\'on existe entre $L_1$ y $L_2$ ?

  
 \item Demuestra que las siguientes gram\'aticas son ambiguas y trata de  
encontrar gram\'aticas equivalentes que no lo sean. 
  \be
   \item $G_1 : \;\; S\imp aS \mid aSbS \mid \vacia$
   \item $G_2 : \;\; S \imp 0S \mid 1S \mid 1A \qquad \qquad A\imp 0A\mid 1A 
    \mid \vacia $
  \ee
 
\end{enumerate}
\end{document}

