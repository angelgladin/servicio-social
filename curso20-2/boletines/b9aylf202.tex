\documentclass[letterpaper,11pt]{article}
% \usepackage{geometry}
\usepackage{anysize}

% \usepackage[utf8]{inputenc}
\usepackage[latin1]{inputenc}
\usepackage[spanish]{babel}
\usepackage{lmodern}   % font shapes...
\usepackage[T1]{fontenc} % join the compound symbols as a single symbol

\usepackage{amssymb,amsmath}
\usepackage{mathrsfs}
\usepackage{epsfig}

\usepackage{fancyhdr}
\usepackage{hyperref}
\usepackage{url}

\usepackage{import}
\usepackage{comment}
\usepackage[autostyle=true,spanish=mexican]{csquotes}

\usepackage{url}

\usepackage{pgf}
\usepackage{tikz}
\usetikzlibrary{automata,arrows}
\usetikzlibrary{babel}
%\pagestyle{fancyplain}

\marginsize{2cm}{2cm}{1cm}{1cm}
\input{../macrosAyLF}

\title{Aut\'omatas y Lenguajes Formales 2020-2 \\ 
Facultad de Ciencias UNAM \\
Bolet\'in de Ejercicios 9\\
\Large{M\'aquinas de Turing}}
\author{Favio E. Miranda Perea \and A. Liliana Reyes Cabello \and 
Lourdes Gonz\'alez Huesca }
\date{\today}


\begin{document}
\maketitle
\thispagestyle{empty}
\begin{enumerate}

\item Dise\~na m\'aquinas de Turing, de tu elecci\'on, para \textbf{reconocer} 
 cadenas en los siguientes lenguajes. No olvides explicar a detalle la idea 
 del dise\~no.
  \begin{itemize}
   \item $L_1 = \{wcw \mid w\in \{a,b\}^\star \}$
   \item $L_2 = \{www \mid w\in \{0,1\}^\star \}$
   \item $L_3 = \{w\in\{a,b,c\}^\star \mid \#_{a} = \#_{b} = \#_{c} \}$
   \item $L_4 = \{a^3 b^{3n} \mid n\geq 1 \}$
   \item $L_5 = \{a^m b^n c^\ell \mid m > n \text{ y } m < \ell \}$
   \item $L_6 = \{w\in \{1\}^\star \mid |w| = 2^i, i\geq 0 \}$
  \end{itemize}
  
 \item Considera la siguiente m\'aquina de Turing:
  \[
   \begin{array}{c|c|c|c|c|c}
    \delta & 0 & 1 & X & Y & \blanks \\ \hline
    q_0 & (q_1,X,\der) & (q_1,1,\der) & & & \\ \hline
    q_1 & (q_1,0,\der) & (q_2,Y,\izq) & & (q_1,Y,\der) & \\ \hline
    q_2 & (q_4,0,\izq) & & (q_3,X,\der) & (q_2,Y,\izq)  & \\ \hline
    q_3 &  & & (q_3,X,\izq) & (q_3,Y,\izq)  &  (q_f,\blanks\,,\der)\\\hline
    q_4 & (q_4,0,\izq) & & (q_0,X,\der) &   &  \\
   \end{array}
  \]
  \begin{enumerate}
  \item Procesa las cadenas $011,010,101,0110$, mediante configuraciones.
  \item ?`Qu\'e lenguaje reconoce esta m\'aquina?
  \end{enumerate}

 \item Considere la siguiente m\'aquina:
 \[
  \begin{array}{c|c|c|c}
   \delta &  0 & 1 & \blanks\\\hline
   q_0 & (q_1,\blanks,\der) & (q_5,\blanks,\der) &  \\ \hline
   q_1 & (q_1,0,\der) & (q_2,1,\der) &  \\  \hline
   q_2 & (q_3,1,\izq) & (q_2,1,\der) & (q_4,\blanks,\izq) \\ \hline
   q_3 & (q_3,0,\izq) & (q_3,1,\izq) & (q_0,\blanks,\der) \\ \hline
   q_4 & (q_4,0,\izq) & (q_4,\blanks,\izq) & (q_f,0,\der) \\ \hline
   q_5 & (q_5,\blanks,\der) & (q_5,\blanks,\der) & (q_f,\blanks,\der)\\
   \end{array}
  \]
  \begin{enumerate}
  \item Procese las cadenas $000100$ y $001000$.
  \item Considerando que un n\'umero~$n$ se representa con $n$ ceros, 
  ?`Qu\'e funci\'on calcula esta m\'aquina?
  \end{enumerate}
  
 \item Suponga dos m\'aquinas de Turing dadas $M_f$ y $M_g$ que calculan las
  funciones $f$ y $g$ respectivamente.  Describa c\'omo construir una m\'aquina 
  de Turing que calcule cada funci\'on siguiente:
  \begin{enumerate}
   \item $f + g $
   \item el m\'inimo de $f$ y $g$
   \item $f\circ g$
  \end{enumerate}
  
 \item Dise\~na una m\'aquina de Turing sobre el alfabeto $\Sigma = \{c,d\}$ 
  para el proceso de hallar el primer s\'imbolo no blanco a la izquierda de 
  una cadena de blancos. Por ejemplo se debe cumplir que:
   \[
    cd \blanks\blanks \dots\blanks q_0 \blanks 
    \vdash^\star c q_{\mathfrak{F}} d \blanks\blanks \dots\blanks
   \]

    
 \item Dise\~na una m\'aquina de Turing $M$ que calcule el doble de un n\'umero
  binario.
  Dar la definici\'on formal de $M$, la idea del dise\~no, el significado de 
  cada estado y el procesamiento formal de la entrada $101$ mediante 
  configuraciones

 \item Demuestra que todo lenguaje regular es recursivo. Para esto proceda como 
  sigue:
  \begin{enumerate}
  \item De un algoritmo \textit{informal} que transforme a un aut\'omata 
  finito~$A$ en una m\'aquina de Turing equivalente~$M_A$ que siempre se 
  detiene. 
  \item Defina formalmente el algoritmo del inciso anterior especificando 
  c\'omo transformar la definici\'on formal de $A$ en la definici\'on formal 
  de $M_A$.
  \item Justifique porqu\'e $M_A$ siempre se detiene.
  \item Muestre un ejemplo de la ejecuci\'on del algoritmo, obteniendo $M_A$ 
  para un aut\'omata finito $A$ que acepte al siguiente lenguaje :
  $L=\{w\in\{a,b\}^\star\mid \text{ toda } a\text{ tiene una } b \text{ a la 
    derecha }\}$
  \end{enumerate}
  
 \item Demuestra que todo lenguaje libre de contexto es recursivo simulando un 
  aut\'omata de pila con una m\'aquina de Turing.
  Para esto puede usar la siguiente idea: dise\~nar una m\'aquina de Turing 
  con dos cintas, la primera simula la entrada y la segunda simula la pila.


 \item  Encuentra una gram\'atica sensible de contexto que genere al lenguaje
  $$ L = \{ ww  \mid w \in \{a, b\}^\star \}$$ 

 \item Codifica las m\'aquinas resultantes del ejercicio 1.
  
 \item Describa lo que realiza la siguiente m\'aquina as\'i como escriba una 
  posible codificaci\'on para la siguiente m\'aquina de Turing:
\[
 \begin{array}{c|ccccc}
  Q & 0 & 1 & X & Y & B \\\hline
  q_0 & (q_0,X,\der) & - & - & (q_3,Y,\der) & - \\
  q_1 & (q_1,0,\der) & (q_2,Y,\izq) & - & (q_1,Y,\der) & - \\
  q_2 & (q_2,0,\izq) & - & (q_0,X,\der) & (q_2,Y,\izq) & - \\
  q_3 & - & - & - & (q_3,Y,\der) & (q_4,B,\der) \\
  q_4 & - & - & - & - & - 
 \end{array}
\]


 \item Decodifica las siguientes m\'aquinas de Turing y determina qu\'e 
  realiza cada una. 
  El alfabeto de entrada es $\Sigma=\{0,1\}$ codificado con 
  $\blanks := 1,\;0:=11,\;1:=111$
  \begin{enumerate}
   \item 
   \[
    0101^30101^30100101^201^301^20100101^201^501^201001^301^201^401^201001^4 
   \]
   \[
    01^30101^301001^501^30101^301001^50101^20101^30010101^20101^3
   \]
   \item 
   \[
    0101101110111010 0111011101011010 01110101101010   
    \]
    \[
    010110101010 010111011101010 01110111011101010 01110101101010
   \]

  \end{enumerate}


\end{enumerate}


\end{document}
