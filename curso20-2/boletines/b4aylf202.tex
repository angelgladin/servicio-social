\documentclass[letterpaper,11pt]{article}
\usepackage[includeheadfoot,margin=1.9cm]{geometry}
\usepackage{anysize}

% \usepackage[utf8]{inputenc}
\usepackage[latin1]{inputenc}
\usepackage[spanish]{babel}
\usepackage{lmodern}   % font shapes...
\usepackage[T1]{fontenc} % join the compound symbols as a single symbol

\usepackage{amssymb,amsmath}
\usepackage{mathrsfs}
\usepackage{epsfig}

\usepackage{fancyhdr}
\usepackage{hyperref}
\usepackage{url}

\usepackage{import}
\usepackage{comment}
\usepackage[autostyle=true,spanish=mexican]{csquotes}

\usepackage{url}

\usepackage{pgf}
\usepackage{tikz}
\usetikzlibrary{automata,arrows}
\usetikzlibrary{babel}
%\pagestyle{fancyplain}
\input{../macrosAyLF}

\title{Aut\'omatas y Lenguajes Formales 2020-2 \\ 
Facultad de Ciencias UNAM \\
Bolet\'in de Ejercicios 4\\
\Large{Lenguajes regulares y el teorema de s\'intesis Kleene}}
\author{Favio E. Miranda Perea \and A. Liliana Reyes Cabello \and 
Lourdes Gonz\'alez Huesca }
\date{\today}


\begin{document}
\maketitle



\be
\item Considere las siguientes expresiones regulares:
\bi
 \item $\alpha_1 = 1^\star 01\big( (01)^\star+1\big)^\star $
 \item $\alpha_2 = (a+b)^\star(aaa+bbb)(a+b)^\star$
  %Muestre, usando $\delta^\star$, la aceptaci\'on de la cadena $abbbb$.
 \item $\alpha_3 = (01)(01)^\star + (010)(010)^\star$
 \item $\alpha_4 = (0+1)^*(01 + 110)$
 \item $\alpha_5 = (0+1)^*(011 + 01010)(0+1)^*$
\ei
Para cada una de ellas realice las siguientes acciones:
\be
 \item Dise\~nar un AFN$\vacia$, $M_i$ para cada $\alpha_i$ mediante el 
  m\'etodo de s\'intesis de Kleene (se pueden omitir algunas transiciones 
$\vacia$).
 \item Calcular el conjunto~$Cl_{\vacia}(q)$ para cada estado $q$ de cada 
$M_i$.
 \item Encontrar el AFN equivalente a~$M_i$ que reconoce a $L(\alpha_i)$.
 \item Encuentre el AFD equivalente a~$M_i$ que reconoce a $L(\alpha_i)$.
 \item Encuentre el AFD {\bf m\'inimo} equivalente a~$M_i$ que reconoce a 
  $L(\alpha_i)$.
\ee

\ee

\end{document}
