\documentclass[letterpaper,11pt]{article}
\usepackage[includeheadfoot,margin=1.9cm]{geometry}
\usepackage{anysize}

% \usepackage[utf8]{inputenc}
\usepackage[latin1]{inputenc}
\usepackage[spanish]{babel}
\usepackage{lmodern}   % font shapes...
\usepackage[T1]{fontenc} % join the compound symbols as a single symbol

\usepackage{amssymb,amsmath}
\usepackage{mathrsfs}
\usepackage{epsfig}

\usepackage{fancyhdr}
\usepackage{hyperref}
\usepackage{url}

\usepackage{import}
\usepackage{comment}
\usepackage[autostyle=true,spanish=mexican]{csquotes}

\usepackage{url}

\usepackage{pgf}
\usepackage{tikz}
\usetikzlibrary{automata,arrows}
\usetikzlibrary{babel}
%\pagestyle{fancyplain}
\input{../macrosAyLF}

\title{Aut\'omatas y Lenguajes Formales 2020-2 \\ 
Facultad de Ciencias UNAM \\
Bolet\'in de Ejercicios 5\\
\Large{Teorema de an\'alisis de Kleene y Gram\'aticas de lenguajes}}
\author{Favio E. Miranda Perea \and A. Liliana Reyes Cabello \and 
Lourdes Gonz\'alez Huesca }
\date{\today}


\begin{document}
\maketitle
\begin{enumerate}


\item Encuentre la expresi\'on regular $\alpha_i$ tal que $L(M_i) = (\alpha_i)$ 
para cada aut\'omata $M_i$:
\[
   \begin{array}{r|c|c|c|r|c|c|c|}
    M_1: &&&&\hspace*{30pt} M_2: &&& \\
	& \delta & a & b & 
	& \delta & a & b  \\\hline 
	
    inicial,final & q_{0} & q_2 & q_0 & 
    inicial & q_{0} & q_1& q_1 \\\hline
    
	& q_{1} & q_0 & q_2 & 
	final & q_{1} & q_1 & q_2 \\\hline 
	
	& q_{2} & q_1 & q_3 & 
	& q_{2} & q_1 & q_0 \\\hline
	
	& q_{3} & q_3 & q_1 &
	& & &\\\hline
	\end{array}
  \]

  
  
%  \item Sean $G = (N,T,P,S)$ una gram\'atica lineal por la derecha y 
%   $G' = (N,T,P',S)$ la gram\'atica que se obtiene a partir de $G$ de la 
%   siguiente manera.
%   \begin{itemize}
%    \item Si $A \rightarrow a \in P$ entonces  $A \rightarrow a \in P'$
%    \item Si $A \rightarrow aB \in P$ entonces $B \rightarrow aA \in P'$
%   \end{itemize}
%   ?`Qui\'en es $L(G')$? Demuestra tu respuesta.
%   %Demuestra que $L(G') = (L(G))^{R}$

 \item Dada la gram\'atica $G= \pt{\{S,A,B,C\}, \{a,b,c\}, P, S}$ cuyas
  producciones son: 
   \[
     \begin{array}{rcl}
       S & \rightarrow & aABC \\
       A & \rightarrow & aA \mid a \\
       B & \rightarrow & bB \mid b \\
       C & \rightarrow & cC \mid c \\
     \end{array}
   \]
 contesta las siguientes preguntas.
 \begin{enumerate}
  \item ?`Qu\'e lenguaje genera, es decir qu\'en es $L(G)$?
  \item ?`Cu\'al es el tipo de gram\'atica m\'as restringido al que pertenece 
  $G$?
  \item ?`$L(G)$ es regular? En caso de que tu respuesta sea afirmativa, da una
   gram\'atica regular, en caso contrario justifica por qu\'e no lo es. 
 \end{enumerate}

%  \item Demuestra que las siguientes gram\'aticas son ambiguas y encuentra 
%   gram\'aticas equivalentes que no lo sean:
%   \bi
%    \item $G_1 : \;\; S\imp aS \mid aSbS \mid \vacia$
%    \item $G_2 : \;\; S \imp 0S \mid 1S \mid 1A \qquad \qquad A\imp 0A\mid 1A 
%     \mid \vacia $
%   \ei
%  
 \item Construya aut\'omatas que reconozcan los lenguajes generados por cada 
  una de las gram\'aticas siguientes:
  \[
   \begin{array}{rrlcrrl}
    G1 : & & & \qquad \qquad &G2 : & &  \\    
    & S \imp & aA \mid bC \mid b && & S \imp & bS \mid aA \mid \vacia\\
    & A \imp & aS \mid bB & & & A \imp & aA \mid bB \mid b\\
    & B \imp & aC \mid bA \mid a & & & B \imp & bS\\
    & C \imp & aB \mid bS & & & &
   \end{array}
  \]
  
 \item Transforma las gram\'aticas lineales por la derecha del ejercicio 
  anterior en gram\'aticas lineales por la izquierda.

 \item Defina gram\'aticas regulares que correspondan a los lenguajes 
  generados por las siguientes expresiones regulares:
  \bi
   \item $\alpha_1 = bba(ab)^\star$
   \item $\alpha_2 = (aab+bbaba)^\star$
  \ei
  
\end{enumerate}



\end{document}
