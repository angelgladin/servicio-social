\documentclass[letterpaper,11pt]{article}
\usepackage[includeheadfoot,margin=1.9cm]{geometry}
\usepackage{anysize}

% \usepackage[utf8]{inputenc}
\usepackage[latin1]{inputenc}
\usepackage[spanish]{babel}
\usepackage{lmodern}   % font shapes...
\usepackage[T1]{fontenc} % join the compound symbols as a single symbol

\usepackage{amssymb,amsmath}
\usepackage{mathrsfs}
\usepackage{epsfig}

\usepackage{fancyhdr}
\usepackage{hyperref}
\usepackage{url}

\usepackage{import}
\usepackage{comment}
\usepackage[autostyle=true,spanish=mexican]{csquotes}

\usepackage{url}

\usepackage{pgf}
\usepackage{tikz}
\usetikzlibrary{automata,arrows}
\usetikzlibrary{babel}
%\pagestyle{fancyplain}
\input{../macrosAyLF}

\title{Aut\'omatas y Lenguajes Formales 2020-2 \\ 
Facultad de Ciencias UNAM \\
Bolet\'in de Ejercicios 8\\
\Large{Aut\'omatas de Pila y Algoritmo CYK}}
\author{Favio E. Miranda Perea \and A. Liliana Reyes Cabello \and 
Lourdes Gonz\'alez Huesca }
\date{\today}


\begin{document}
\maketitle
\begin{enumerate}

\item Dise\~na aut\'omatas de pila \textbf{no deterministas} que reconozcan 
  los siguientes lenguajes, pueden ser aut\'omatas que acepten por estados 
  finales o por pila vac\'ia.
  \begin{itemize}
   \item El conjunto de cadenas en $\{a,b\}^\star$ que \textbf{no} contienen 
    cadenas de la forma $ww$.
   \item $L_1 = \{ a^i b^j c^k \mid i\neq j \text{ \'o } j\neq k \}$
   \item $L_2 = \{ a^i b^j c^k \mid i = j \text{ \'o } j = k \}$
  \end{itemize}

  \item Dise\~na aut\'omatas de pila \textbf{no deterministas} para los 
  siguientes lenguajes y muestra los pasos de c\'omputo para las
  cadenas en los correspondientes lenguajes, debes mostrar la secuencia de 
configuraciones obtenidas.
%   el \'arbol generado por las transiciones del aut\'omata. 
  \be
   \item $v = 010100$ con $L = \{ w\in \{0,1\}^\star \mid \#_0 = 2\#_1  \}$
   \item $v = aaaabab$ con 
    $L = \{ a^nw \mid n\geq 0,\; w\in \{a,b\}^\star, |w|\leq n\}$
   \item $v = 00111222$ con 
    $L = \{ 0^i 1^j 2^k \mid i,j,k\geq 0, j=i \text{ \'o } j=k \}$
   \item $v = aacbbb$ con 
    $L = \{ a^i c b^k \mid i\leq k,\;i,k\geq 1\} $
  \ee
  
%  \item Transforma el aut\'omata que acepta por estados finales en uno que 
%   acepta por pila vac\'ia.
% 
%   \item Transforma el aut\'omata que acepta por pila vac\'ia en uno que 
% acepta por estados finales. 
 
 \item Da aut\'omatas de pila \textbf{deterministas} que acepten los siguientes 
  lenguajes:
  $$L_1 = \{ 0^n1^m \mid n\leq m \} \qquad L_2 = \{ 0^n1^m \mid n\geq m 
\} \qquad 
L_3 = \{ 0^n1^m0^n \mid n,m\;\text{son arbitrarios} \}$$

 \item Convierte las siguientes gram\'aticas a aut\'omatas de pila que acepten 
  el mismo lenguaje:
  \begin{itemize}
   \item $G_1 : \;\;
      S \imp 0S1 \mid A \quad \quad A \imp 1A0 \mid S \mid \vacia$ 
   \item $G_2 : \;\;
      S \imp aA \qquad \qquad A\imp aABC \mid bB \mid a \quad \quad B\imp b
      \qquad \quad C\imp c $
   \item $G_3 : \;\;
      S \imp aAA  \qquad \quad A \imp aS \mid bS \mid a $
   \item $G_4 : \;\;
      S\imp abA \mid bB \mid aba \quad \quad A\imp b\mid aB \mid bA 
      \qquad \quad B\imp aB \mid aA$
  \end{itemize}
  
 \item Encuentre una gram\'atica libre de contexto que genere el lenguaje 
  aceptado por el aut\'omata\\ $M = \pt{\{q_0, q_1\}, \{a, b\}, \{A, Z_0\}, 
  \delta, q_0 , Z_0, \{q_1\}}$ con las siguientes transiciones:
  \[
    \delta(q_0,a,Z_0)  =  \{(q_0,AZ_0)\} \qquad 
    \delta(q_0,b,A)  =  \{(q_0, AA) \} \qquad 
    \delta(q_0,a,A)  = \{(q_1,\vacia) \}
  \]
%   $$ M = \pt{\{p,q\}, \{0,1\}, \{X,Z\_0},\delta, q, Z_0}$$
% \[
%  \begin{array}{rcl}
%   \delta(q,1,Z_0) & = & \{(q,XZ_0)\} \\
%   \delta(q,1,X) & = & \{(q,XX)\} \\
%   \delta(q,0,X) & = & \{(p,X)\} \\
%   \delta(q,\vacia,X) & = & \{(q,\vacia)\} \\
%   \delta(p,1,X) & = & \{(p,\vacia)\} \\
%   \delta(p,0,Z_0) & = & \{(q,Z_0)\} \\
%  \end{array}
% \]


 \item Convierte los aut\'omatas de pila del ejercicio 1. en gram\'aticas 
  libres de contexto.
  
  \item Si $A$ y $B$ son lenguajes definimos 
  $$ A\diamond B = \{ xy \mid x\in A,\; y\in B,\; |x| = |y| \}$$
  Muestra que si $A$ y $B$ son libres de contexto entonces $A\diamond B$ 
  tambi\'en lo es.


 \item Aplique el algoritmo \textsc{CYK} a las cadenas siguientes para decidir 
  si pertencen a $L(G)$ para cada una de las gram\'aticas dadas.
  \be
   \item $w = bbacb$, $w = caaab$ con $G$ dada por:\\
   $S\imp CA \mid DB \quad\quad C\imp b\quad\quad D\imp c
   \quad\quad A \imp CB \quad\quad B\imp EB \mid b \quad\quad E\imp a$
   \item $w = aaaaa$, $w= baba $ con $G$ dada por:\\
   $S\imp AB \mid BC \qquad\qquad A\imp BA \mid a \qquad \qquad B\imp CC\mid b
    \qquad \qquad C\imp AB\mid a$
   \item $w = 00111$ con $G$ dada por: \\
   $S \imp AB\mid BE\mid 0\mid 1 \quad\quad A \imp CB\mid CA\mid 0 
  \quad\quad B \imp BE\mid 1\quad \quad C \imp 0\quad \quad E \imp 1  $
  \ee  
  
\end{enumerate}


\end{document}
