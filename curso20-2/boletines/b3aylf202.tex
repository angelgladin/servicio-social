\documentclass[letterpaper,11pt]{article}
\usepackage[includeheadfoot,margin=1.9cm]{geometry}
\usepackage{anysize}

% \usepackage[utf8]{inputenc}
\usepackage[latin1]{inputenc}
\usepackage[spanish]{babel}
\usepackage{lmodern}   % font shapes...
\usepackage[T1]{fontenc} % join the compound symbols as a single symbol

\usepackage{amssymb,amsmath}
\usepackage{mathrsfs}
\usepackage{epsfig}

\usepackage{fancyhdr}
\usepackage{hyperref}
\usepackage{url}

\usepackage{import}
\usepackage{comment}
\usepackage[autostyle=true,spanish=mexican]{csquotes}

\usepackage{url}

\usepackage{pgf}
\usepackage{tikz}
\usetikzlibrary{automata,arrows}
\usetikzlibrary{babel}
%\pagestyle{fancyplain}
\input{../macrosAyLF}

\title{Aut\'omatas y Lenguajes Formales 2020-2 \\ 
Facultad de Ciencias UNAM \\
Bolet\'in de Ejercicios 3\\
\Large{Aut\'omatas Finitos No-Deterministas y con transiciones $\vacia$}}
\author{Favio E. Miranda Perea \and A. Liliana Reyes Cabello \and 
Lourdes Gonz\'alez Huesca }
\date{\today}


\begin{document}
\maketitle

%    
% \section*{Aut\'omatas finitos no-deterministas}

\be
\item Para cada uno de los lenguajes descritos abajo
\begin{enumerate}
 \item Dise\~ne un $AFN$ (sin $\vacia$-transiciones) que acepte $L_i$.
 \item Transforme el aut\'omata anterior a un $AFD$ mediante la 
  construcci\'on de subconjuntos.
\end{enumerate}
\bi
 \item $L_1 = \{ w\in \{0,1\}^* \mid w \text{ en donde cada } 0 
 \text{ est\'a seguido de } 11\}$
 \item $L_2 = \{a^n b^m \in \{a,b\}^* \mid n + m \text{ es par } \}$
 \item $L_3 = \{w \in \{a,b,c\}^* \mid w \text{ contiene al menos una vez 
cada s\'imbolo de } \{a,b,c\} \}$
\ei
 
\item Para cada uno de los aut\'omatas en la siguiente figura, encontrar un 
$AFD$ equivalente:
\begin{center}
\includegraphics[width=.75\textwidth]{afns}
\end{center}

\item Para cada uno de los $AFN_{\vacia}$ descritos por 
las siguientes tablas de transici\'on, de un diagrama y procese las cadenas  
$ba,\, ababa$ siguiendo la definici\'on recursiva de $\delta^*$ (y mostrando el 
c\'alculo de las cerraduras):
\[
\dest(q,\vacia) = Cl_{\vacia}(q)  \qquad  \qquad 
\dest(q,wa) = Cl_{\vacia}\Big(\bigcup_{q'\in\dest(q,w)}\delta(q',a)\Big)
\]
  
  
  \[
   \begin{array}{r|c|c|c|c|r|c|c|c|c|}
    M_1: &&&&&\hspace*{30pt} M_2: &&&& \\
	& \delta & a & b & \vacia & 
	& \delta & a & b & \vacia \\\hline 
    inicial & q_{0} & \vacio & \vacio & \{q_1\} & 
    inicial,final& q_{0} & \vacio & \vacio & \{q_1\} \\\hline
	& q_{1} & \{q_2\} & \vacio & \{q_4\} &
	& q_{1} & \{q_2\} & \vacio & \{q_4\}\\\hline 
	& q_{2} & \vacio & \{q_3\} & \vacio & 
	& q_{2} & \vacio & \{q_3\} & \vacio \\\hline
	& q_{3} & \{q_3\} & \vacio & \{q_1\} &
	& q_{3} & \{q_3\} & \vacio & \{q_1\}\\\hline
    final & q_{4} & \vacio & \{q_5,q_6\} & \vacio &
	& q_{4} & \vacio & \{q_5,q_6\} & \vacio \\\hline
	& q_{5} & \{q_4\} & \vacio & \vacio &
	& q_{5} & \{q_4\} & \vacio & \vacio \\\hline
	& q_{6} & \vacio & \vacio & \{q_0\} &
	& q_{6} & \vacio & \vacio & \{q_0\}\\\hline
   \end{array}
  \]
  
  \vspace*{20pt}
  
  \[
   \begin{array}{r|c|c|c|c|r|c|c|c|c|}
    M_3: &&&&&\hspace*{30pt} M_4: &&&& \\
	& \delta & a & b & \vacia & 
	& \delta & a & b & \vacia \\\hline 
	
    inicial,final & q_{0} & \vacio & \vacio & \{q_1\} & 
    inicial & q_{0} & \{q_1, q_2 \} & \vacio & \vacio \\\hline
    
	& q_{1} & \{q_2\} & \{q_3\} & \vacio &
	final & q_{1} & \{q_0\} & \vacio & \vacio \\\hline 
	
	& q_{2} & \vacio & \{q_1\} & \vacio & 
	& q_{2} & \vacio & \{q_3\} & \vacio \\\hline
	
	& q_{3} & \vacio & \vacio & \{q_0\} &
	& q_{3} & \{q_3\} & \vacio & \{q_1\}\\\hline
	\end{array}
  \]
\ee


\end{document}
